\chapter*{Introduction}
The main purpose of these notes is 
to prove a duality theorem 
for cohomology of quasi-coherent sheaves, 
with respect to a proper morphism of 
locally noetherian preschemes. 
Various such theorems are already known. 
Typical is the duality theorem for 
a non-singular complete curve $X$ 
over an algebraically closed field $k$, 
which says that
\[
    h^0(D)=h^1(K-D),
\]
where $D$ is a divisor, $K$ is the canonical divisor, 
and
\[
    h^i(D)=\dim_k H^i(X, L(D)),
\]
for any $i$, and any divisor $D$. 
(See e.g. [16,Ch. II] for a proof.)

Various attempts were made to generalize this theorem 
to varieties of higher dimension, 
and as Zariski points out in his report [20], 
his generalization of a lemma of Enriques-Severi [19] 
is equivalent to the statement that
for a normal projective variety $X$ of dimension n over $k$,
\[
    h^0(D)=h^n(K-D)
\]
for any divisor $D$. 
This is also equivalent to a theorem of 
Serre [FAC §76 Thm. 4] on the vanishing 
of the cohomology group $H^1(X,L(-m))$ 
for $m$ large and $L$ locally free. 
Using a related theorem [FAC §75 Thm. 3], 
Zariski shows how one can deduce on a non-singular 
projective variety the formula
\[
    h^i(D)=h^{n-i}(K-D)
\]
for $0 < i < n$. 
In terms of sheaves, this result corresponds to the fact that 
the $k$-vector spaces
\[
    H^i(X,F)=H^{n-i}(X,F^\vee\otimes\omega)
\]
are dual to each other, where $F$ is a locally free sheaf, 
$F^\vee$ is the dual sheaf \(\Hom(F,\mcal{O}_X)\) 
and $\omega$ is the sheaf $\omega = \Omega_{X/k}^n$ of 
$n$-differentials on $X$. 
Serre [15] gives a proof of this same theorem 
by analytic methods for a compact complex analytic manifold $X$.

Grothendieck [8] gave some generalizations 
of these theorems for non-singular projective varieties, 
and then in [9] announced the general theorem 
for schemes proper over a field, 
with arbitrary singularities, 
which is the subject of the present lecture notes.

To motivate the statement of our main theorem, 
let us consider the case of projective 
space \(X=\pp^n_k\) over an algebraically closed field $k$. 
Then there is a canonical isomorphism