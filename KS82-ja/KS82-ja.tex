%================================================
%    sheaves on manifolds ノート
%================================================

% -----------------------
% preamble
% -----------------------
% ここから本文 (\begin{document}) までの
% ソースコードに変更を加えた場合は
% 編集者まで連絡してください. 
% Don't change preamble code yourself. 
% If you add something
% (usepackage, newtheorem, newcommand, renewcommand),
% please tell it 
% to the editor of institutional paper of RUMS.

% ------------------------
% documentclass
% ------------------------
\documentclass[11pt, a4paper, dvipdfmx, leqno]{jsarticle}

% ------------------------
% usepackage
% ------------------------
\usepackage{algorithm}
\usepackage{algorithmic}
\usepackage{amscd}
\usepackage{amsfonts}
\usepackage{amsmath}
\usepackage[psamsfonts]{amssymb}
\usepackage{amsthm}
\usepackage{ascmac}
\usepackage{color}
\usepackage{enumerate}
\usepackage{fancybox}
\usepackage[stable]{footmisc}
\usepackage{graphicx}
\usepackage{listings}
\usepackage{mathrsfs}
\usepackage{mathtools}
\usepackage{otf}
\usepackage{pifont}
\usepackage{proof}
\usepackage{subfigure}
\usepackage{tikz}
\usepackage{verbatim}
\usepackage[all]{xy}
\usepackage{url}
\usetikzlibrary{cd}



% ================================
% パッケージを追加する場合のスペース 
%\usepackage{calligra}
\usepackage[dvipdfmx]{hyperref}
\usepackage{xcolor}
\definecolor{darkgreen}{rgb}{0,0.45,0} 
\definecolor{darkred}{rgb}{0.75,0,0}
\definecolor{darkblue}{rgb}{0,0,0.6} 
\hypersetup{
    colorlinks=true,
    citecolor=darkgreen,
    linkcolor=darkred,
    urlcolor=darkblue,
}
\usepackage{pxjahyper}
\usepackage{okumacro}

\usepackage{framed}
\definecolor{lightgray}{rgb}{0.75,0.75,0.75}
\renewenvironment{leftbar}{%
  \def\FrameCommand{\textcolor{lightgray}{\vrule width 0.4zw} \hspace{10pt}}% 
  \MakeFramed {\advance\hsize-\width \FrameRestore}}%
{\endMakeFramed}
\newenvironment{redleftbar}{%
  \def\FrameCommand{\textcolor{red}{\vrule width 1pt} \hspace{10pt}}% 
  \MakeFramed {\advance\hsize-\width \FrameRestore}}%
 {\endMakeFramed}

%=================================


% --------------------------
% theoremstyle
% --------------------------
\theoremstyle{definition}

% --------------------------
% newtheoem
% --------------------------

% 日本語で定理, 命題, 証明などを番号付きで用いるためのコマンドです. 
% If you want to use theorem environment in Japanece, 
% you can use these code. 
% Attention!
% All theorem enivironment numbers depend on 
% only section numbers.
\newtheorem{Axiom}{公理}[section]
\newtheorem{Definition}[Axiom]{定義}
\newtheorem{Theorem}[Axiom]{定理}
\newtheorem{Proposition}[Axiom]{命題}
\newtheorem{Lemma}[Axiom]{補題}
\newtheorem{Corollary}[Axiom]{系}
\newtheorem{Example}[Axiom]{例}
\newtheorem{Claim}[Axiom]{主張}
\newtheorem{Property}[Axiom]{性質}
\newtheorem{Attention}[Axiom]{注意}
\newtheorem{Question}[Axiom]{問}
\newtheorem{Problem}[Axiom]{問題}
\newtheorem{Consideration}[Axiom]{考察}
\newtheorem{Alert}[Axiom]{警告}
\newtheorem{Fact}[Axiom]{事実}
\newtheorem{com}[Axiom]{コメント}


% 日本語で定理, 命題, 証明などを番号なしで用いるためのコマンドです. 
% If you want to use theorem environment with no number in Japanese, You can use these code.
\newtheorem*{Axiom*}{公理}
\newtheorem*{Definition*}{定義}
\newtheorem*{Theorem*}{定理}
\newtheorem*{Proposition*}{命題}
\newtheorem*{Lemma*}{補題}
\newtheorem*{Example*}{例}
\newtheorem*{Corollary*}{系}
\newtheorem*{Claim*}{主張}
\newtheorem*{Property*}{性質}
\newtheorem*{Attention*}{注意}
\newtheorem*{Question*}{問}
\newtheorem*{Problem*}{問題}
\newtheorem*{Consideration*}{考察}
\newtheorem*{Alert*}{警告}
\newtheorem*{Fact*}{事実}
\newtheorem*{com*}{コメント}



% 英語で定理, 命題, 証明などを番号付きで用いるためのコマンドです. 
% If you want to use theorem environment in English, You can use these code.
%all theorem enivironment number depend on only section number.
\newtheorem{Axiom+}{Axiom}[section]
\newtheorem{Definition+}[Axiom+]{Definition}
\newtheorem{Theorem+}[Axiom+]{Theorem}
\newtheorem{Proposition+}[Axiom+]{Proposition}
\newtheorem{Lemma+}[Axiom+]{Lemma}
\newtheorem{Example+}[Axiom+]{Example}
\newtheorem{Corollary+}[Axiom+]{Corollary}
\newtheorem{Claim+}[Axiom+]{Claim}
\newtheorem{Property+}[Axiom+]{Property}
\newtheorem{Attention+}[Axiom+]{Attention}
\newtheorem{Question+}[Axiom+]{Question}
\newtheorem{Problem+}[Axiom+]{Problem}
\newtheorem{Consideration+}[Axiom+]{Consideration}
\newtheorem{Alert+}{Alert}
\newtheorem{Fact+}[Axiom+]{Fact}
\newtheorem{Remark+}[Axiom+]{Remark}

% ----------------------------
% commmand
% ----------------------------
% 執筆に便利なコマンド集です. 
% コマンドを追加する場合は下のスペースへ. 

% 集合の記号 (黒板文字)
\newcommand{\NN}{\mathbb{N}}
\newcommand{\ZZ}{\mathbb{Z}}
\newcommand{\QQ}{\mathbb{Q}}
\newcommand{\RR}{\mathbb{R}}
\newcommand{\CC}{\mathbb{C}}
\newcommand{\PP}{\mathbb{P}}
\newcommand{\KK}{\mathbb{K}}


% 集合の記号 (太文字)
\newcommand{\nn}{\mathbf{N}}
\newcommand{\zz}{\mathbf{Z}}
\newcommand{\qq}{\mathbf{Q}}
\newcommand{\rr}{\mathbf{R}}
\newcommand{\cc}{\mathbf{C}}
\newcommand{\pp}{\mathbf{P}}
\newcommand{\kk}{\mathbf{K}}

% 特殊な写像の記号
\newcommand{\ev}{\mathop{\mathrm{ev}}\nolimits} % 値写像
\newcommand{\pr}{\mathop{\mathrm{pr}}\nolimits} % 射影
\newcommand{\grad}{\mathop{\mathrm{grad}}\nolimits} % 射影



% スクリプト体にするコマンド
%   例えば {\mcal C} のように用いる
\newcommand{\mcal}{\mathcal}

% 花文字にするコマンド 
%   例えば {\h C} のように用いる
\newcommand{\h}{\mathscr}

% ヒルベルト空間などの記号
\newcommand{\F}{\mcal{F}}
\newcommand{\X}{\mcal{X}}
\newcommand{\Y}{\mcal{Y}}
\newcommand{\Hil}{\mcal{H}}
\newcommand{\RKHS}{\Hil_{k}}
\newcommand{\Loss}{\mcal{L}_{D}}
\newcommand{\MLsp}{(\X, \Y, D, \Hil, \Loss)}

% 偏微分作用素の記号
\newcommand{\p}{\partial}

% 角カッコの記号 (内積は下にマクロがあります)
\newcommand{\lan}{\langle}
\newcommand{\ran}{\rangle}



% 圏の記号など
\newcommand{\Set}{{\bf Set}}
\newcommand{\Vect}{{\bf Vect}}
\newcommand{\FDVect}{{\bf FDVect}}
\newcommand{\Mod}{\mathop{\mathrm{Mod}}\nolimits}
\newcommand{\CGA}{{\bf CGA}}
\newcommand{\GVect}{{\bf GVect}}
\newcommand{\Lie}{{\bf Lie}}
\newcommand{\dLie}{{\bf Liec}}



% 射の集合など
\newcommand{\Map}{\mathop{\mathrm{Map}}\nolimits}
\newcommand{\Hom}{\mathop{\mathrm{Hom}}\nolimits}
\newcommand{\End}{\mathop{\mathrm{End}}\nolimits}
\newcommand{\Aut}{\mathop{\mathrm{Aut}}\nolimits}
\newcommand{\Mor}{\mathop{\mathrm{Mor}}\nolimits}

% その他便利なコマンド
\newcommand{\dip}{\displaystyle} % 本文中で数式モード
\newcommand{\e}{\varepsilon} % イプシロン
\newcommand{\dl}{\delta} % デルタ
\newcommand{\pphi}{\varphi} % ファイ
\newcommand{\ti}{\tilde} % チルダ
\newcommand{\pal}{\parallel} % 平行
\newcommand{\op}{{\rm op}} % 双対を取る記号
\newcommand{\lcm}{\mathop{\mathrm{lcm}}\nolimits} % 最小公倍数の記号
\newcommand{\Probsp}{(\Omega, \F, \P)} 
\newcommand{\argmax}{\mathop{\rm arg~max}\limits}
\newcommand{\argmin}{\mathop{\rm arg~min}\limits}





% ================================
% コマンドを追加する場合のスペース 
\renewcommand\proofname{\bf 証明} % 証明
%\numberwithin{equation}{subsection}
\newcommand{\cTop}{\textsf{Top}}
%\newcommand{\cOpen}{\textsf{Open}}
\newcommand{\Op}{\mathop{\textsf{Open}}\nolimits}
\newcommand{\Ob}{\mathop{\textrm{Ob}}\nolimits}
\newcommand{\id}{\mathop{\mathrm{id}}\nolimits}
\newcommand{\pt}{\mathop{\mathrm{pt}}\nolimits}
\newcommand{\res}{\mathop{\rho}\nolimits}
\newcommand{\A}{\mcal{A}}
\newcommand{\B}{\mcal{B}}
\newcommand{\C}{\mcal{C}}
\newcommand{\D}{\mathscr{D}}
\newcommand{\E}{\mcal{E}}
\newcommand{\G}{\mcal{G}}
%\newcommand{\H}{\mcal{H}}
\newcommand{\I}{\mcal{I}}
\newcommand{\J}{\mcal{J}}
\newcommand{\OO}{\mcal{O}}
\newcommand{\Ring}{\mathop{\textsf{Ring}}\nolimits}
\newcommand{\cAb}{\mathop{\textsf{Ab}}\nolimits}
\newcommand{\Ker}{\mathop{\mathrm{Ker}}\nolimits}
\newcommand{\im}{\mathop{\mathrm{Im}}\nolimits}
\newcommand{\Coker}{\mathop{\mathrm{Coker}}\nolimits}
\newcommand{\Coim}{\mathop{\mathrm{Coim}}\nolimits}
\newcommand{\Ht}{\mathop{\mathrm{Ht}}\nolimits}
\newcommand{\colim}{\mathop{\mathrm{colim}}}
\newcommand{\Tor}{\mathop{\mathrm{Tor}}\nolimits}

\newcommand{\cat}{\mathscr{C}}

\newcommand{\scA}{\mathscr{A}}
\newcommand{\scB}{\mathscr{B}}
\newcommand{\scC}{\mathscr{C}}
\newcommand{\scD}{\mathscr{D}}
\newcommand{\scE}{\mathscr{E}}
\newcommand{\scF}{\mathscr{F}}
\newcommand{\scM}{\mathscr{M}}

\newcommand{\ibA}{\mathop{\text{\textit{\textbf{A}}}}}
\newcommand{\ibB}{\mathop{\text{\textit{\textbf{B}}}}}
\newcommand{\ibC}{\mathop{\text{\textit{\textbf{C}}}}}
\newcommand{\ibD}{\mathop{\text{\textit{\textbf{D}}}}}
\newcommand{\ibE}{\mathop{\text{\textit{\textbf{E}}}}}
\newcommand{\ibF}{\mathop{\text{\textit{\textbf{F}}}}}
\newcommand{\ibG}{\mathop{\text{\textit{\textbf{G}}}}}
\newcommand{\ibH}{\mathop{\text{\textit{\textbf{H}}}}}
\newcommand{\ibI}{\mathop{\text{\textit{\textbf{I}}}}}
\newcommand{\ibJ}{\mathop{\text{\textit{\textbf{J}}}}}
\newcommand{\ibK}{\mathop{\text{\textit{\textbf{K}}}}}
\newcommand{\ibL}{\mathop{\text{\textit{\textbf{L}}}}}
\newcommand{\ibM}{\mathop{\text{\textit{\textbf{M}}}}}
\newcommand{\ibN}{\mathop{\text{\textit{\textbf{N}}}}}
\newcommand{\ibO}{\mathop{\text{\textit{\textbf{O}}}}}
\newcommand{\ibP}{\mathop{\text{\textit{\textbf{P}}}}}
\newcommand{\ibQ}{\mathop{\text{\textit{\textbf{Q}}}}}
\newcommand{\ibR}{\mathop{\text{\textit{\textbf{R}}}}}
\newcommand{\ibS}{\mathop{\text{\textit{\textbf{S}}}}}
\newcommand{\ibT}{\mathop{\text{\textit{\textbf{T}}}}}
\newcommand{\ibU}{\mathop{\text{\textit{\textbf{U}}}}}
\newcommand{\ibV}{\mathop{\text{\textit{\textbf{V}}}}}
\newcommand{\ibW}{\mathop{\text{\textit{\textbf{W}}}}}
\newcommand{\ibX}{\mathop{\text{\textit{\textbf{X}}}}}
\newcommand{\ibY}{\mathop{\text{\textit{\textbf{Y}}}}}
\newcommand{\ibZ}{\mathop{\text{\textit{\textbf{Z}}}}}

\newcommand{\ibx}{\mathop{\text{\textit{\textbf{x}}}}}

\newcommand{\Comp}{\mathop{\mathrm{C}}\nolimits}
\newcommand{\Komp}{\mathop{\mathrm{K}}\nolimits}
\newcommand{\Domp}{{\mathrm{D}}}

\newcommand{\CCat}{\Comp(\cat)}
\newcommand{\KCat}{\Komp(\cat)}
\newcommand{\DCat}{\Domp(\cat)}%圏Cの複体のホモトピー圏
\newcommand{\HOM}{\mathop{\mathscr{H}\hspace{-2pt}om}\nolimits}%内部Hom
\newcommand{\RHOM}{\mathop{\mathrm{R}\hspace{-1.5pt}\HOM}\nolimits}
\newcommand{\TOR}{\mathop{\mathscr{T}\hspace{-2pt}or}\nolimits}%ねじれ


\newcommand{\muS}{\mathop{\mathrm{SS}}\nolimits}
\newcommand{\RG}{\mathop{\mathrm{R}\hspace{-0pt}\Gamma}\nolimits}
\newcommand{\LT}{\mathop{\otimes}\limits^\mathrm{L}}

\newcommand{\simar}{\mathrel{\overset{\sim}{\longrightarrow}}}%内部Hom
\newcommand{\simra}{\mathrel{\overset{\sim}{\longleftarrow}}}%内部Hom

\newcommand{\hocolim}{{\mathrm{hocolim}}}
\newcommand{\indlim}[1][]{\mathop{\varinjlim}\limits_{#1}}
\newcommand{\sindlim}[1][]{\smash{\mathop{\varinjlim}\limits_{#1}}\,}
\newcommand{\Pro}{\mathrm{Pro}}
\newcommand{\Ind}{\mathrm{Ind}}
\newcommand{\prolim}[1][]{\mathop{\varprojlim}\limits_{#1}}
\newcommand{\sprolim}[1][]{\smash{\mathop{\varprojlim}\limits_{#1}}\,}

\newcommand{\supp}{\mathop{\mathrm{supp}}\nolimits}
\newcommand{\car}{\mathop{\mathrm{car}}\nolimits}

% =================================



%================================================
% 自前の定理環境
%   https://mathlandscape.com/latex-amsthm/
% を参考にした
\newtheoremstyle{mystyle}%   % スタイル名
    {5pt}%                   % 上部スペース
    {5pt}%                   % 下部スペース
    {}%              % 本文フォント
    {}%                  % 1行目のインデント量
    {\bfseries}%                      % 見出しフォント
    {.\quad ---}%                     % 見出し後の句読点
    {10pt}%                     % 見出し後のスペース
    {\thmname{#1}\thmnumber{ #2}\thmnote{{\hspace{2pt}\normalfont (#3)}}}% % 見出しの書式

\theoremstyle{mystyle}
\newtheorem{AXM}{公理}
\newtheorem{DFN}{定義}
\newtheorem{THM}{定理}
\newtheorem*{THM*}{定理}
\newtheorem{PRP}{命題}
\newtheorem{LMM}[Axiom]{補題}
\newtheorem{CRL}[PRP]{系}
\newtheorem*{CRL*}{系}
\newtheorem{EG}{例}
\newtheorem{CNV}[Axiom]{規約}
\newtheorem{NTN}[Axiom]{記号}
\newtheorem*{NTN*}{記号}
\newtheorem{CMT}{コメント}
\newtheorem{RMK}{注意}
\newtheorem*{RMK*}{注意}




% 定理環境ここまで
%====================================================

% ---------------------------
% new definition macro
% ---------------------------
% 便利なマクロ集です

% 内積のマクロ
%   例えば \inner<\pphi | \psi> のように用いる
\def\inner<#1>{\langle #1 \rangle}

% ================================
% マクロを追加する場合のスペース 

%=================================





% ----------------------------
% documenet 
% ----------------------------
% 以下, 本文の執筆スペースです. 
% Your main code must be written between 
% begin document and end document.
% ---------------------------

\title{偏微分方程式 --- 層の超局所台:微分加群への応用\footnote{
    M. Kashiwara, P. Schapira, 
    \emph{Micro-support des faisceaux: 
    application aux modules diff\'erentiels}, 
    C. R. Acad. Sc. Paris, 295 (8 novembre 1982) 
    の和訳 (2023/10/28).(\today 更新)
}}
\author{柏原正樹\and ピエール・シャピラ}
\date{1982年9月27日 提出}
\begin{document}
\maketitle
\begin{abstract}
    \(C^1\)級多様体において,層の超局所台を,
    余接束内の包合的な錐的閉集合として導入し,
    その関手的な性質を調べる.
    その後,確定特異点を持つホロノミー加群から誘導される方程式系の
    特性多様体の増大度を導出する.
\end{abstract}
\section{法錐}
\(M\)を\(C^1\)級多様体とし,\(T^\ast M\)をその余法束,
\(\pi\)を\(T^\ast M\)から\(M\)への射影とする.
\(\dot{T}^\ast M\)を\(T^\ast M\)における
零切断\(T_M^\ast M\)の補集合とする.\(a\)を\(T^\ast M\)の
\emph{\ruby{対蹠}{たいせき}写像} (application anti-podale) とし,
\(S^a\)を\(T^\ast M\)の部分集合\(S\)の像とする.
\(M\)の2つの集合\(S\)と\(Z\)に対し,
点\(x\in M\)における\(Z\)に沿った\(S\)の\textbf{法錐} (c\^one 
normal) \(C_x(S,Z)\)とは,
接空間\(T_xM\)の閉錐で,局所座標系を用いて次のように定義されるものである.
\[
    \begin{cases}
        v\in C_x(S,Z)\Leftrightarrow S\times Z\times \rr^+
        \text{内の列} (s_n,z_n,c_n)_n \text{で}\\
        s_n\underset{n}{\rightarrow}x, 
        z_n\underset{n}{\rightarrow}x, 
        c_n(x_n-z_n)\underset{n}{\rightarrow}v \text{をみたすものが存在する.}
    \end{cases}
\]
\(C(S,Z)=\bigcup_xC_x(S,Z)\)とおく.
\(Z\)が\(M\)の部分多様体のとき,\(C_x(S,Z)\)は\(T_xZ\)で安定であり,
\(C_Z(S)\)で\(C(S,Z)\)の法束\(T_ZM\)での像を表す.
\(M\)の閉部分集合\(S\)に対し,
錐\(TM\setminus C(M\setminus S,S)\)を
\emph{強い意味での法錐} (c\^one normal strict) といい
\(N(S)\)とかく.
また\(N(S)\cap T_xM\)の双対閉錐体を\(N_x^\ast S\)とかき,
\(x\)における\(S\)の\textbf{余法錐} (c\^one conrmal) という (cf.\cite{KS79}).

\section{超局所台}
\(\Domp_+(M)\)(\(\Domp_b(M)\))を\(M\)上の\(A\)加群の層の
下に有界な(有界な)複体のなす圏の導来圏とする.\footnote{
    [訳注] \(\Domp_{+}\)と\(\Domp_{b}\)は現在,
    上つきの\(\Domp^{+}\)と
    \(\Domp^{b}\)でそれぞれ表すのが通例である.
}ここで\(A\)は環である.

\begin{PRP}\label{PRP:SS}
    \(F^\bullet\in\Ob(\Domp_+(M))\)とし
    \(\xi\)を\(T^\ast M\)の点とする.
    次の条件は同値である.
    \begin{enumerate}[(i)]
        \item \(\xi\)の錐状近傍\(U\)と\(\pi(\xi)\)の近傍\(V\)で,
        \(V\)の任意の閉部分集合\(Z\)と
        任意の点\(x\in\partial Z\cap V\)で
        \(N^\ast_x(Z)\subset U^a\cup\{0\}\)を満たすものに対し
        \(\left(\RG_Z(F^\bullet)\right)_x=0\)となるもの
        が存在する;
        \item (\(M\)を\(C^r\)級とする.
        ここで\(1\leqq r\leqq\infty\)または\(r=\omega\)とする.)
        境界が\(C^r\)級超曲面であるような任意の\(Z\)に対し,
        条件 (i) が成り立つ;
        \item (\(M\)が\(\rr^n\)の開集合であるとし,
        \(\xi=(x_0,\xi_0)\)とする)
        実数\(\delta>0,\e>0\)で,\(\lvert a-x_0\rvert<\e\)を
        みたす全ての点\(a\in M\)に対し次が成り立つものが存在する.
        \begin{multline*}
            \RG(\{
                    x;-\delta
                    \leqq\langle x-x_0,\xi_0\rangle, 
                    \langle x-a,\xi_0\rangle
                    \leqq-\e\lvert x-a\rvert
            \}, F^\bullet)\\
            \simeq
            \RG(\{
                x;-\delta=\langle x-x_0,\xi_0\rangle, 
                \langle x-a,\xi_0\rangle
                \leqq-\e\lvert x-a\rvert
            \}, F^\bullet).
        \end{multline*}
    \end{enumerate}
\end{PRP}
この命題は\cite[\S4]{KS79}で暗に示されている.

\begin{DFN}
    \(F^\bullet\in\Ob(\Domp_+(M))\)とする.
    \(F^\bullet\)の\emph{超局所台} (micro-support) 
    \(\muS(F^\bullet)\)とは,
    次で定義される\(T^\ast M\)の閉錐である.:
    \begin{enumerate}[(a)]
        \item \(\muS(F^\bullet)\cap T^\ast_MM=\overline{\bigcup_j\supp(\mathscr{H}^j(F^\bullet))}\);
        \item \(\xi\in\dot{T}^\ast M, \xi\in\muS(F^\bullet)\Leftrightarrow\text{命題\ref{PRP:SS}の同値な条件をみたす.}\)
    \end{enumerate}
\end{DFN}

\begin{EG}
    \(M=\rr^n\), \(Z=\{x\in M, x_1\geqq0\}\), \(F=\underline{\cc}_Z\).\footnote{
        [訳注] \(\underline{\cc}_{Z}\)は\(Z\)に台を持つ定数層である.
        現在は\(\cc_Z\)で表すのが通例である.
    }
    
    このとき\(\muS(F)=\{(x,\xi);x_1\geqq0,\xi=0\}\cup\{(x,\xi);\xi_1\geqq0,x_1=\xi_2=\dots=\xi_n=0\}\).
\end{EG}

\begin{EG}
    \(M=\rr^n\), \(\mathscr{U}=\{x\in M, x_1>0\}\), \(F=\underline{\cc}_\mathscr{U}\). 
    
    このとき\(\muS(F)=\{(x,\xi);x_1\geqq0,\xi=0\}\cup\{(x,\xi);\xi_1\leqq0,x_1=\xi_2=\dots=\xi_n=0\}\).
\end{EG}

\begin{EG}
    \(M=\rr^2\), \(Z=\{x\in M, x_1^3\geqq x_2^2\}\), \(F=\underline{\cc}_Z\). 
    
    このとき\(\pi^{-1}(0)\cup\muS(F)=\{\xi;\xi_1\geqq0\}\).
\end{EG}

\section{関手的な性質}

\begin{PRP}
    \(\omega\)を\(M\)上の双対化複体とする.

    このとき,任意の\(F^\bullet\in\Ob(\Domp_b(M))\)に対し,
    \(\muS(\RHOM(F^\bullet,\omega))\subset\muS(F^\bullet)^a\)が
    成り立つ.
\end{PRP}

\(M\)と\(N\)を\(C^1\)級多様体,
\(f\)を\(M\)から\(N\)への\(C^1\)級写像とし,
\(\bar{\omega}\colon T^\ast N\mathop{\times}\limits_N M\to T^\ast N\)と
\(\rho\colon T^\ast N\mathop{\times}\limits_N M\to T^\ast M\)を
\(f\)から定まる写像とする.

\begin{PRP}
    \(F^\bullet\in\Ob(\Domp_+(M))\)を\(f\)の\(
        \overline{\bigcup_j\supp(\mathscr{H}^j(F^\bullet))}
    \)への制限が固有となるものとする.
    このとき\(\muS(\mathrm{R}f_\ast(F^\bullet))\subset\rho(T^\ast N\mathop{\times}\limits_NM)\)が成り立つ.
\end{PRP}

\begin{PRP}
    \(f\)が滑らかであるとする.
    \begin{enumerate}[(i)]
        \item \(G^\bullet\in\Ob(\Domp_+(N))\)とする.
        このとき,\(\muS(f^{-1}(G^\bullet))=\rho\bar{\omega}^{-1}(\muS(G^\bullet))\)である.
        \item \(F^\bullet\in\Ob(\Domp_b(M))\)とする.
        このとき,任意の\(\mathscr{H}^j(F^\bullet)\)が\(f\)のファイバー上で
        局所定数層となるのは,
        \(\muS((F^\bullet))\subset\rho(T^\ast N\mathbin{\mathop{\times}\limits_N} M)\)が成り立つとなるときである.
    \end{enumerate}
\end{PRP}

\begin{PRP}
    \(M\)と\(N\)を\(C^1\)級多様体とし,
    \(p_1\)と\(p_2\)を\(M\times N\)から\(M\)と\(N\)への射影とする.
    \begin{enumerate}[(i)]
        \item \(F^\bullet\in\Ob(\Domp_b(M))\)と\(G^\bullet\in\Ob(\Domp_+(N))\)に対し,\[
            \muS(\RHOM_A(p_1^{-1}F^\bullet,p_2^{-1}G^\bullet))\subset\muS(F^\bullet)^a\times\muS(G^\bullet)
        \]が成り立つ.
        \item \(A\)のコホモロジー次元が有限であるとする.
        このとき,\(F^\bullet\in\Ob(\Domp_b(M))\)と\(G^\bullet\in\Ob(\Domp_+(N))\)に対し,\[
            \muS(p_1^{-1}F^\bullet\LT_A p_2^{-1}G^\bullet)\subset\muS(F^\bullet)\times\muS(G^\bullet)
        \]が成り立つ.  
    \end{enumerate}
\end{PRP}

\section{超局所化}

以降,\(N\)は\(M\)の部分多様体であるとし,
\(M\)と\(N\)は\(C^2\)級であるとする.
\(M\setminus N \cup T_NM\) (\(M\setminus N \cup T^\ast_NM\)) 
には\emph{ブローアップ} (\'eclat\'e) の(\emph{余ブローアップ} 
(co-\'eclat\'e) の)自然な位相を入れる (cf.\cite{SKK}).
\(j\)を\(M\)から\(M\setminus N \cup T_NM\)への包含とし\(\pi\)を
\(M\setminus N \cup T^\ast_NM\)から\(M\)への射影とする.
\(F^\bullet\in\Ob(\Domp_+(M))\)に対し,
\(\lambda(F^\bullet)\)で(\cite{Mal}と同様に)\(\Domp_+(T_NM)\)の対象
\(\mathrm{R}j_\ast(F^\bullet)\rvert_{T_NM}\)を表し,
\(\mu_N(F^\bullet)\)で\(F^\bullet\)に対する\(N\)での佐藤の超局所化,
すなわち,\(\Domp_+(T^\ast_NM)\)の対象\(\RG_{T^\ast_NM}(\pi^{-1}(F^\bullet))^a\)を表す.
\(\Lambda\)で\(M\)の\(N\)に対する余法束\(T_N^\ast M\)を表す.
ハミルトン同相写像\(T^\ast(T^\ast M)\simeq T(T^\ast M)\)によって,
束\(T^\ast\Lambda\)と\(T_\Lambda T^\ast M\)を同一視することができ,
それによって\(T^\ast(T_N M)\)と\(T_\Lambda T^\ast M\)も
同一視することができる.
他方,滑らかな写像\(T_N M\to N\)から
埋め込み\(T^\ast N\mathop{\times}\limits_{N}T_N M
\to T^\ast(T_N M)\simeq T_\Lambda T^\ast M\)が定まる.
したがって,\(T^\ast{N}\)を(\(T_{N}M\)の
零切断を通して)\(T_{\Lambda}T^{\ast}M\)の
部分多様体と同一視することができる.
とくに(\(M\)を\(T^\ast M\)に,
\(N\)を\(M\)に取り替えることで)
\(T^\ast M\)が\(TT^\ast M\)の部分多様体と同一視されることがわかる.

\begin{THM}
    \(F^\bullet\in\Ob(\Domp_+(M))\)とする,次の包含関係が成り立つ.
    \begin{enumerate}[(i)]
        \item \(\muS(\mu_N(F^\ast))\subset C_\Lambda(\muS(F^\bullet))\),
        \item \(\muS(\lambda_N(F^\ast))\subset C_\Lambda(\muS(F^\bullet))\),
        \item \(\muS(\RG_N(F^\ast))\subset T^\ast N\cap C_\Lambda(\muS(F^\bullet))\),
        \item \(\muS(F^\ast\lvert_N)\subset T^\ast N\cap C_\Lambda(\muS(F^\bullet))\).
    \end{enumerate}
\end{THM}
証明は\cite{KS79}の主張の繰り返しである(\cite{BS73}も参照).
(iii) と (iv) は一般に真の包含である.

\begin{CRL}
    \(\muS(F^\ast)\cap T^\ast_N M\subset T^\ast_M M\)とする.
    このとき,
    \[
        \RG_N(F^\bullet)\simeq F^\bullet\otimes\RG_N(\zz_M)
        \quad\text{かつ}\quad
        \muS(F^\ast\lvert_N)\subset\bar{\omega}\rho^{-1}(\muS(F^\ast))
    \]
    が成り立つ.
\end{CRL}
(\cite[1章]{SKK}の 命題 1.2.5 を用いる.)
\begin{CRL}
    \begin{enumerate}[(i)]
        \item \(F^\bullet\in\Ob(\Domp_b(M)), 
        G^\bullet\in\Ob(\Domp_+(M))\)とする.このとき
        \[
            \muS(\RHOM_A(F^\bullet, G^\bullet))
            \subset 
            T^\ast M\cap C(\muS(G^\bullet),\muS(F^\bullet))
        \]
        が成り立つ.
        \item \(A\)のコホモロジー次元が有限であるとする.
        このとき\(F^\bullet,G^\bullet\in\Ob(\Domp_+(M))\)に対し,
        \[
            \muS(F^\bullet\LT_A G^\bullet)\subset T^\ast M\cap C(\muS(F^\bullet),\muS(G^\bullet)^a)
        \]
        が成り立つ.        
    \end{enumerate}
\end{CRL}

\section{包合性}

\begin{THM}
    \(F^\bullet\in\Ob(\Domp_b(M))\)とする.
    このとき\(\muS(F^\bullet)\)は包合的である
    (関数\(f\)が\(\muS(F^\bullet)\)で0になるとき,
    \(\muS(F^\bullet)\)はハミルトンベクトル場\(H_f\)で不変である).
\end{THM}
(\(M\)は\(C^2\)級であるとし,
\(C^1\)級関数\(f\)は\(T^\ast M\)の開集合でしか定義されないと仮定する.)

\section{応用}

以下では\((X,\OO_X)\)を複素解析多様体とし,
\(\mathscr{D}_X\)で有限階の(\(\mathscr{D}^\infty_X\)で無限階の)
正則関数係数の微分作用素のなす\(X\)上の層を表す.
連接\(\D\)加群\(\scM\)に対し,
\(\car(\scM)\)で\(T^\ast X\)における\(\scM\)の特性多様体を表す.
\(Z\)を別の複素解析多様体とするとき,
\(\pi_Z\)で射影\(T^\ast(X\times Z)\to T^\ast X\)を表す.

\begin{PRP}
    \begin{enumerate}[(i)]
        \item \[\car(\scM)=\bigcup_Z\pi_Z(\muS(\RHOM_{\scD}(\scM,\OO_{X\times Z})))\]である(\(Z\)は複素解析多様体である).
        \item \(\scM\)がホロノミー加群であるとき,\[
            \car(\scM)=\muS(\RHOM_{\scD}(\scM,\OO_{X}))
        \]である.
    \end{enumerate}
\end{PRP}

包含の片側は古典的な結果\cite{Kas77}であり,
反対の包含は層\(\scE_X^\rr\) (cf.\cite{SKK,Kas77}) の
コホモロジーを用いた定義と,
\(\scE_X^\rr\)が超局所微分作用素の環\(\scE_X\)の上で
忠実平坦であることから従う.

\begin{CRL}\label{CRL:RH}
    \(Y\)を\(X\)の部分多様体とし,
    \begin{enumerate}[(a)]
        \item \(\scD_X^\infty\otimes\RG_{[Y]}(\scM)=\RG_Y(\scD_X^\infty\otimes\scM)\),
        \item \(\scD_Y\)加群\({\TOR}_j^{\OO_X}(\OO_Y,\scM)\)は連接的である
    \end{enumerate}
    と仮定する.このとき,任意の\(j\)に対し次の包含関係が成り立つ.
    \[
        \car(\TOR_j^{\OO_X}(\OO_Y,\scM))\subset
        T^\ast Y\cap C_{T_Y X}(\car(\scM)).
    \]
\end{CRL}
仮定 (a), (b) は確定特異点型ホロノミー加群ならば
任意の部分多様体\(Y\)に対して成り立つ\cite{KK81}.
系\ref{CRL:RH}は\cite{KK81}の結果を補完し,
例えば,\(\prod_j f_j^{\lambda_j}\)型の分布の
解析的波面集合の上からの評価を得ることができる.

本稿の主要結果は著者の一人が\cite{Sch82}において発表している.
\begin{thebibliography}{20}
    \bibitem[1]{BS73} J. M. Bony, P. Schapira, 
        \emph{Solutions Hyperfonction du Problem de Cauchy}, 
        Springer Lecare Noles in Math., 287, 1973, p. 82--98.
    \bibitem[2]{Kas77}\footnote{[訳注] M. Kashiwara, \emph{Systems of Microdifferential Equations}, Birkh\"auser, 1983. として単行本化されている.} M. Kashiwara, 
        Cours Universit\'e Paris-Nord redige 
        par T. Monteiro-Fernandes, 1977.
    \bibitem[3]{KK81} M. Kashiwara, T. Kawai, 
        \emph{On Holonomic Systems of Microdifferential equations III}, 
        Publ. Rims. Kyoto Univ., 17, 1981, p. 813--979.
    \bibitem[4]{KS79} M. Kashiwara, P. Schapira, \emph{Micro-hyperbolic Systems}, 
        Acta Math., 142, 1979, p. 1-55.
    \bibitem[5]{KS81} M. Kashiwara, P. Schapira, 
        \emph{Micro-support des faisceaux: application aux modules diff\'erentiels}, 
        Journees E.D.P. Saint-Jean-de-Monts, 1981, publ. Ecole Polytechnique, Palaiseau.
    \bibitem[6]{SKK} Sato, T. Kawai, M. Kashiwara, 
        \emph{Microfunctions and Pseudo-differential Equations}, 
        Springer Lecture Notes in Math., 287, 1973, p. 265-529.
    \bibitem[7]{Mal} B. Malgrange, Transformation de Fourier cohomologique, Expose a Nancy, mai 1982.
    \bibitem[8]{Sch82} Schapira, \emph{Micro-support des faisceaux}, 
        Expose a Nancy, mai 1982.
\end{thebibliography}
\end{document}
