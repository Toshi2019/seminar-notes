
%Don't forget to delete
%showkeys
%overfullrule
%\date \ber \er \cmt

%------------------------
\documentclass[11pt, a4paper, dvipdfmx]{jlreq}


%usepackage
%------------------------
\usepackage{amsmath}
\usepackage{amsthm}
%\usepackage[psamsfonts]{amssymb}
\usepackage{color}
\usepackage{ascmac}
\usepackage{amsfonts}
\usepackage{mathrsfs}
\usepackage{mathtools}
\usepackage{amssymb}
\usepackage{graphicx}
\usepackage{fancybox}
%\usepackage{enumerate}
\usepackage{enumitem}
\usepackage{verbatim}
\usepackage{subfigure}
\usepackage{proof}
\usepackage{listings}
\usepackage{otf}
\usepackage{algorithm}
\usepackage{algorithmic}
\usepackage{tikz}
\usetikzlibrary{cd}
\usepackage[all]{xy}
\usepackage{amscd}

\usepackage{pb-diagram}

\usepackage[dvipdfmx]{hyperref}
\usepackage{xcolor}
\definecolor{darkgreen}{rgb}{0,0.45,0} 
\definecolor{darkred}{rgb}{0.75,0,0}
\definecolor{darkblue}{rgb}{0,0,0.6} 
\hypersetup{
    colorlinks=true,
    citecolor=darkgreen,
    linkcolor=darkred,
    urlcolor=darkblue,
}
\usepackage{pxjahyper}

\usepackage{enumitem}

\usepackage{bbm}

% ================================
% パッケージを追加する場合のスペース 
\usepackage{latexsym}
\usepackage{wrapfig}
\usepackage{layout}
\usepackage{url}

\usepackage{okumacro}
%\usepackage{endnotes}
%\usepackage[french]{babel}

%=================================


% --------------------------
% theoremstyle
% --------------------------
\theoremstyle{definition}


% --------------------------
% newtheoem
% --------------------------

% 日本語で定理, 命題, 証明などを番号付きで用いるためのコマンドです. 
% If you want to use theorem environment in Japanece, 
% you can use these code. 
% Attention!
% All theorem enivironment numbers depend on 
% only section numbers.
\newtheorem{Axiom}{公理}[section]
\newtheorem{Definition}[Axiom]{定義}
\newtheorem{Theorem}[Axiom]{定理}
\newtheorem{Proposition}[Axiom]{命題}
\newtheorem{Lemma}[Axiom]{補題}
\newtheorem{Corollary}[Axiom]{系}
\newtheorem{Example}[Axiom]{例}
\newtheorem{Claim}[Axiom]{主張}
\newtheorem{Property}[Axiom]{性質}
\newtheorem{Attention}[Axiom]{注意}
\newtheorem{Question}[Axiom]{問}
\newtheorem{Problem}[Axiom]{問題}
\newtheorem{Consideration}[Axiom]{考察}
\newtheorem{Alert}[Axiom]{警告}
\newtheorem{Fact}[Axiom]{事実}


% 日本語で定理, 命題, 証明などを番号なしで用いるためのコマンドです. 
% If you want to use theorem environment with no number in Japanese, You can use these code.
\newtheorem*{Axiom*}{公理}
\newtheorem*{Definition*}{定義}
\newtheorem*{Theorem*}{定理}
\newtheorem*{Proposition*}{命題}
\newtheorem*{Lemma*}{補題}
\newtheorem*{Example*}{例}
\newtheorem*{Corollary*}{系}
\newtheorem*{Claim*}{主張}
\newtheorem*{Property*}{性質}
\newtheorem*{Attention*}{注意}
\newtheorem*{Question*}{問}
\newtheorem*{Problem*}{問題}
\newtheorem*{Consideration*}{考察}
\newtheorem*{Alert*}{警告}
\newtheorem{Fact*}{事実}


% 英語で定理, 命題, 証明などを番号付きで用いるためのコマンドです. 
% If you want to use theorem environment in English, You can use these code.
%all theorem enivironment number depend on only section number.
\newtheorem{Axiom+}{Axiom}[section]
\newtheorem{Definition+}[Axiom+]{Definition}
\newtheorem{Theorem+}[Axiom+]{Theorem}
\newtheorem{Proposition+}[Axiom+]{Proposition}
\newtheorem{Lemma+}[Axiom+]{Lemma}
\newtheorem{Example+}[Axiom+]{Example}
\newtheorem{Corollary+}[Axiom+]{Corollary}
\newtheorem{Claim+}[Axiom+]{Claim}
\newtheorem{Property+}[Axiom+]{Property}
\newtheorem{Attention+}[Axiom+]{Attention}
\newtheorem{Question+}[Axiom+]{Question}
\newtheorem{Problem+}[Axiom+]{Problem}
\newtheorem{Consideration+}[Axiom+]{Consideration}
\newtheorem{Alert+}{Alert}
\newtheorem{Fact+}[Axiom+]{Fact}
\newtheorem{Remark+}[Axiom+]{Remark}

% ----------------------------
% commmand
% ----------------------------
% 執筆に便利なコマンド集です. 
% コマンドを追加する場合は下のスペースへ. 

% 集合の記号 (黒板文字)
\newcommand{\NN}{\mathbb{N}}
\newcommand{\ZZ}{\mathbb{Z}}
\newcommand{\QQ}{\mathbb{Q}}
\newcommand{\RR}{\mathbb{R}}
\newcommand{\CC}{\mathbb{C}}
\newcommand{\PP}{\mathbb{P}}
\newcommand{\KK}{\mathbb{K}}


% 集合の記号 (太文字)
\newcommand{\nn}{\mathbf{N}}
\newcommand{\zz}{\mathbf{Z}}
\newcommand{\qq}{\mathbf{Q}}
\newcommand{\rr}{\mathbf{R}}
\newcommand{\cc}{\mathbf{C}}
\newcommand{\pp}{\mathbf{P}}
\newcommand{\kk}{\mathbf{K}}

% 特殊な写像の記号
\newcommand{\ev}{\mathop{\mathrm{ev}}\nolimits} % 値写像
\newcommand{\pr}{\mathop{\mathrm{pr}}\nolimits} % 射影

% スクリプト体にするコマンド
%   例えば {\mcal C} のように用いる
\newcommand{\mcal}{\mathcal}

% 花文字にするコマンド 
%   例えば {\h C} のように用いる
\newcommand{\h}{\mathscr}

% ヒルベルト空間などの記号
\newcommand{\F}{\mcal{F}}
\newcommand{\X}{\mcal{X}}
\newcommand{\Y}{\mcal{Y}}
\newcommand{\HH}{\mcal{H}}
\newcommand{\RKHS}{\Hil_{k}}
\newcommand{\Loss}{\mcal{L}_{D}}
\newcommand{\MLsp}{(\X, \Y, D, \Hil, \Loss)}

% 偏微分作用素の記号
\newcommand{\p}{\partial}

% 角カッコの記号 (内積は下にマクロがあります)
\newcommand{\lan}{\langle}
\newcommand{\ran}{\rangle}



% 圏の記号など
\newcommand{\Set}{{\bf Set}}
\newcommand{\Vect}{{\bf Vect}}
\newcommand{\FDVect}{{\bf FDVect}}
\newcommand{\Ring}{{\bf Ring}}
\newcommand{\Ab}{{\bf Ab}}
\newcommand{\Mod}{\mathop{\mathrm{Mod}}\nolimits}
\newcommand{\CGA}{{\bf CGA}}
\newcommand{\GVect}{{\bf GVect}}
\newcommand{\Lie}{{\bf Lie}}
\newcommand{\dLie}{{\bf Liec}}



% 射の集合など
\newcommand{\Map}{\mathop{\mathrm{Map}}\nolimits}
\newcommand{\Hom}{\mathop{\mathrm{Hom}}\nolimits}
\newcommand{\End}{\mathop{\mathrm{End}}\nolimits}
\newcommand{\Aut}{\mathop{\mathrm{Aut}}\nolimits}
\newcommand{\Mor}{\mathop{\mathrm{Mor}}\nolimits}

% その他便利なコマンド
\newcommand{\dip}{\displaystyle} % 本文中で数式モード
\newcommand{\e}{\varepsilon} % イプシロン
\newcommand{\dl}{\delta} % デルタ
\newcommand{\pphi}{\varphi} % ファイ
\newcommand{\ti}{\tilde} % チルダ
\newcommand{\pal}{\parallel} % 平行
\newcommand{\op}{{\rm op}} % 双対を取る記号
\newcommand{\lcm}{\mathop{\mathrm{lcm}}\nolimits} % 最小公倍数の記号
\newcommand{\Probsp}{(\Omega, \F, \P)} 
\newcommand{\argmax}{\mathop{\rm arg~max}\limits}
\newcommand{\argmin}{\mathop{\rm arg~min}\limits}





% ================================
% コマンドを追加する場合のスペース 
\newcommand{\UU}{\mcal{U}}
\newcommand{\OO}{\mcal{O}}
\newcommand{\emp}{\varnothing}
\newcommand{\ceq}{\coloneqq}
\newcommand{\sbs}{\subset}
\newcommand{\mapres}[2]{\left. #1 \right|_{#2}}
\newcommand{\ded}{\hfill $\blacksquare$}
\newcommand{\id}{\mathrm{id}}
\newcommand{\isom}{\overset{\sim}{\longrightarrow}}
\newcommand{\tTop}{\textsf{Top}}
\newcommand{\pfb}{\textbf{証明}}
\newcommand{\Int}{\mathop{\mathrm{Int}}\nolimits} % 内部


% 自前の定理環境
%   https://mathlandscape.com/latex-amsthm/
% を参考にした
\newtheoremstyle{mystyle}%   % スタイル名
    {5pt}%                   % 上部スペース
    {5pt}%                   % 下部スペース
    {}%              % 本文フォント
    {}%                  % 1行目のインデント量
    {\bfseries}%                      % 見出しフォント
    {.}%                     % 見出し後の句読点
    {12pt}%                     % 見出し後のスペース
    {\thmname{#1}\thmnumber{ #2 }\thmnote{{\normalfont (#3)}}}% % 見出しの書式

\theoremstyle{mystyle}
\newtheorem{AXM}{公理}[section]
\newtheorem{DFN}[Axiom]{定義}
\newtheorem{THM}[Axiom]{定理}
\newtheorem*{THM*}{定理}
\newtheorem{PRP}[Axiom]{命題}
\newtheorem{LMM}[Axiom]{補題}
\newtheorem{CRL}[Axiom]{系}
\newtheorem{EG}[Axiom]{例}

%\newtheorem{}{Axiom}[]
\numberwithin{equation}{section} % 式番号を「(3.5)」のように印刷

\newcommand{\MM}{\mcal{M}}
\newcommand{\bk}{\mathbf{k}}

% =================================


% ---------------------------
% new definition macro
% ---------------------------
% 便利なマクロ集です

% 内積のマクロ
%   例えば \inner<\pphi | \psi> のように用いる
\def\inner<#1>{\langle #1 \rangle}

% ================================
% マクロを追加する場合のスペース 

%=================================





% ----------------------------
% documenet 
% ----------------------------
% 以下, 本文の執筆スペースです. 
% Your main code must be written between 
% begin document and end document.
% ---------------------------




\begin{document}

\title{時空でずらした層\footnote{
    LNM
    に掲載された
    Pierre Schapira による記事 ``Shifted sheaves for space-time'' の翻訳.
}}
\author{Pierre Schapira}
\date{}
\maketitle
\begin{abstract}
    歴史的な考察を少ししたのち,
    ある奇妙な現象を\cite{GKS12}から抜粋して述べる.
    その現象によると,ユークリッドベクトル空間にある
    半径が時間$t$に比例して増加する閉球体から定まる層を考えたとき,
    それを$t<0$に対して自然に延長したものは
    空間の次元で(層理論の意味で)\emph{ずらした}開球体になる.
    球面に関しても,極を通るごとにずれが増えるという類似の結果がある.
\end{abstract}

本は科学において根本的な重要性を持っている.
未来や読者についてはもちろんのこと,執筆者にとってもそうである.
自分の知識を拡散するのに本よりも良い手段があるだろうか.
私自身,長年シュプリンガーとともに本を出版してきた.
最初はLNM126でじつに1970年のことである.
だが最も重要な体験は1990年に柏原正樹と書いたGL192の出版である.
出版の過程でカトリーナ・バーンは重要な役割を担って下さった.
諸々の提案や訂正をして下さったり批判を頂戴したのに加え,
何にも増して精神的なご助力を賜った.どうも有難う.

上述の本は層の超局所理論に関するものである.
ここで,3つの名前を挙げねばならない.
ジャン・ルレー.
彼は40年代に戦争捕虜でありながら層の理論を発明した.
アレクサンドル・グロタンディーク.
彼は圏論の枠組みを用いて,
そして「六則演算」を通した関手の視点に重みを置くことで
その理論の強みを最大限に引き出した.
佐藤幹夫.
彼は超局所という視点を導入して,
多様体上で「局所的」に見えるものはある意味で大域的なものである,
すなわち,それは余接束上で起きている現象を多様体上に射影したものである
ということを示した.
上述の柏原との共著はその重要なアイデアを層の言葉で定式化したものである.

では層理論とは何か.
それは局所・大域の隔たりを数学的に取り扱うものである.
ある対象や性質は,局所的に見るのと大域的に見るのとでは
全く異なることがある.
大域的には存在しても局所的には存在しないというものもあれば,
局所的にしか存在しないものもある.
この事実の数学的な実例としてはメビウスの帯が一般的である.
メビウスの帯は局所的には向きづけできるが,
一周してくるとその向きが逆になってしまう.
この対立は日常の現象でも多く見られる.
とりわけ政治の世界にもあって,大きな争いの種になる.

次に,超局所\footnote{細かい概説は\cite{Sch21}を参照}とはどういう意味か.
実多様体$M$の点$x$にいるとする.
局所的というのは,あらゆるものを自分を中心とする
小さい球体の中で眺めるということである.
しかし多様体には接束$\tau_M\colon TM\to M$とその双対,
つまり余接束$\pi_M\colon T^{\ast}M\to M$がある.
端的にいうと,
$x$における接空間$T_xM$とは,
$x$を通るすべての光線のなすベクトル空間であり,
その双対空間$T^{\ast}_xM$は,$x$を通る壁であって,
その光線を遮るもの全てのなす空間である.
接空間の方が直観的であるが,
余接空間の方がシンプレクティック多様体をなすため重要である.
余接束は物理学者の言う「相空間」である.
一言で言うと,超局所的というのは,
$M$を$T^{\ast}M$で取り替えるというのではなく,
$T^{\ast}M$を考えながら$M$上で議論することを意味する.
例えば,$M$上の層$F$の超局所台は層$F$が伝播しない余方向の集合である.
複素の場合\footnote{
    これは\cite{SKK73}で示された重要かつ証明の難しい定理である.
    のちに\cite{Gab81}で純代数的な仕方で再定式化・証明がされた.
    }
における連接D加群の特性多様体と同様に,
超局所台は$T^{\ast}M$のシンプレクティック構造に関して余等方的である.
実際,連接D加群の正則関数解の層の超局所台はD加群の特性多様体に他ならない.

空でない最小の(劣解析的などの弱い正則性を仮定した)余等方的集合を
ラグランジュ部分代数多様体という.
超局所台がラグランジュ部分代数多様体となる層はちょうど構成可能層になる.
層が構成可能であるとは$M$の滑層分割で,
各滑層ごとにその上の局所定数層になるものが存在することをいう.
構成可能層は数学の様々な分野においても,
また物理においても非常に重要である.
一般に,滑らかで射影$\pi_M$の階数が定数であるとき,
ラグランジュ部分多様体$\Lambda$の$\pi_M$による射影は$M$の部分多様体となるが,
$\Lambda$はその射影の$M$における余法束となる.
しかし,射影の階数が定数にならないときには,
射影は特異多様体となり,コースティック (caustics) が現れる.
コースティックの近傍での漸近展開を計算するために,
ヴィクトール・マスロフ\cite{Mas65}は
ラグランジュ部分多様体の中の閉曲線の指数を導入した.
この所謂マスロフ指数については,\cite{Arn67,Ler76}を含む
何人かの著者によって研究・再定式化され,
その後,柏原正樹により非常に単純で流麗な記述が与えられた(\cite[Appendix]{KS90}参照).

「純層」$F$(ここでは説明しない.\cite[\S 7.5]{KS90}を
挙げるにとどめる.)が各点$p$において滑らかなラグランジュ多様体$\Lambda$を
超局所台にもつとき,$F$に対し$p$におけるシフトと呼ばれる
半整数を対応させることができる.
射影の階数が変わるとき,このシフトも変わる.
これは物理学における「相転移」と呼ばれるものと関連があるはずである.

この点について,時空と宇宙の膨張(ビッグバン)を用いて説明しよう.
もちろん,以下で純数学的に述べることは非常に粗く,
何らかの物理的な現実に対応すると主張するものではない.
宇宙を$n$次元の球体(もちろん我々にとっては$n=3$である)表すことにしよう.
球体の半径$R$が時間$t$に比例して増加するとすると,
ミンコフスキー空間内の光錐体と同じ様に,
時空は$t=0$を頂点とする$\rr^4$内の閉錐体で表される.
誰かが問う「$t<0$のときはどうなるか」.
時空を,それを台に持つ定数層で置き換えると,
すなわち,
$t\geqq 0$上定義された(与えられた体$\bk$に対する)層
$\bk_{\left\{\lvert x\rvert\leqq t\right\}}$
を考えるとき,これを$t<0$に自然に拡張する必要がある.
この層の超局所台は内部余法的である.






\begin{thebibliography}{15}
    \bibitem[Arn67]{Arn67} Vladimir I. Arnold, 
    \textit{On a characteristic class entering into conditions of quantization}, 
    Funkcional. Anal. i Prilozen (1967), 1–14. in Russian.
    \bibitem[Gab81]{Gab81}
    Ofer Gabber, \textit{The integrability of the characteristic variety}, 
    Amer. Journ. Math. 103 (1981), 445-–468.
    \bibitem[GKS12]{GKS12} 
    Stéphane Guillermou, Masaki Kashiwara and Pierre Schapira, 
    \textit{Sheaf quantization of Hamiltonian isotopies and applications to nondisplaceability problems}, 
    Duke Math. J. 161(2): 201--245 (2012).
    \bibitem[KS90]{KS90} Masaki Kashiwara and Pierre Schapira, 
    \textit{Sheaves on manifolds}, Grundlehren der Mathematischen Wissenschaften, 
    vol. 292, Springer-Verlag, Berlin, 1990.
    \bibitem[Ler76]{Ler76} 
    Jean Leray, \textit{Analyse Lagrangienne et m\'ecanique quantique}, 
    Coll\`ege de France, 1976.
    \bibitem[Mas65]{Mas65} Viktor P. Maslov, 
    \textit{Theory of perturbations and asymptotic methods}, 
    Moskow Gos. Univ., 1965.
    [邦訳] マスロフ, 摂動論と漸近的方法, 岩波書店, 1976年.
    \bibitem[Sch21]{Sch21} 
    Pierre Schapira, 
    \textit{Microlocal analysis and beyond}, 
    New spaces in Mathematics, edited by Mathieu Anel 
    and Gabriel Catren, 
    Cambridge University Press, 2021, pp. 117--152.
    \bibitem[SKK73]{SKK73}
    Mikio Sato, Takahiro Kawai, and Masaki Kashiwara, 
    \textit{Microfunctions and pseudo-differential equations}, 
    Hyperfunctions and pseudo-differential equations 
    (Proc. Conf., Katata, 1971; dedicated to the memory of 
    Andr\'e Martineau), 
    Springer, Berlin, 1973, pp. 265–-529. 
    Lecture Notes in Math., Vol. 287.

\end{thebibliography}
%\bibliographystyle{junsrt}
%\bibliography{ref}

\end{document}


