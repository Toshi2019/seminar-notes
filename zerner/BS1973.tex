%================================================
%    Bony-Schapira ノート
%================================================

% -----------------------
% preamble
% -----------------------
% ここから本文 (\begin{document}) までの
% ソースコードに変更を加えた場合は
% 編集者まで連絡してください. 
% Don't change preamble code yourself. 
% If you add something
% (usepackage, newtheorem, newcommand, renewcommand),
% please tell it 
% to the editor of institutional paper of RUMS.

% ------------------------
% documentclass
% ------------------------
\documentclass[11pt, a4paper, dvipdfmx, leqno]{jsarticle}

% ------------------------
% usepackage
% ------------------------
\usepackage{algorithm}
\usepackage{algorithmic}
\usepackage{amscd}
\usepackage{amsfonts}
\usepackage{amsmath}
\usepackage[psamsfonts]{amssymb}
\usepackage{amsthm}
\usepackage{ascmac}
\usepackage{color}
\usepackage{enumerate}
\usepackage{fancybox}
\usepackage[stable]{footmisc}
\usepackage{graphicx}
\usepackage{listings}
\usepackage{mathrsfs}
\usepackage{mathtools}
\usepackage{otf}
\usepackage{pifont}
\usepackage{proof}
\usepackage{subfigure}
\usepackage{tikz}
\usepackage{verbatim}
\usepackage[all]{xy}
\usepackage{url}
\usetikzlibrary{cd}



% ================================
% パッケージを追加する場合のスペース 
%\usepackage{calligra}
\usepackage[dvipdfmx]{hyperref}
\usepackage{xcolor}
\definecolor{darkgreen}{rgb}{0,0.45,0} 
\definecolor{darkred}{rgb}{0.75,0,0}
\definecolor{darkblue}{rgb}{0,0,0.6} 
\hypersetup{
    colorlinks=true,
    citecolor=darkgreen,
    linkcolor=darkred,
    urlcolor=darkblue,
}
\usepackage{pxjahyper}

%=================================


% --------------------------
% theoremstyle
% --------------------------
\theoremstyle{definition}

% --------------------------
% newtheoem
% --------------------------

% 日本語で定理, 命題, 証明などを番号付きで用いるためのコマンドです. 
% If you want to use theorem environment in Japanece, 
% you can use these code. 
% Attention!
% All theorem enivironment numbers depend on 
% only section numbers.
\newtheorem{Axiom}{公理}[section]
\newtheorem{Definition}[Axiom]{定義}
\newtheorem{Theorem}[Axiom]{定理}
\newtheorem{Proposition}[Axiom]{命題}
\newtheorem{Lemma}[Axiom]{補題}
\newtheorem{Corollary}[Axiom]{系}
\newtheorem{Example}[Axiom]{例}
\newtheorem{Claim}[Axiom]{主張}
\newtheorem{Property}[Axiom]{性質}
\newtheorem{Attention}[Axiom]{注意}
\newtheorem{Question}[Axiom]{問}
\newtheorem{Problem}[Axiom]{問題}
\newtheorem{Consideration}[Axiom]{考察}
\newtheorem{Alert}[Axiom]{警告}
\newtheorem{Fact}[Axiom]{事実}
\newtheorem{com}[Axiom]{コメント}


% 日本語で定理, 命題, 証明などを番号なしで用いるためのコマンドです. 
% If you want to use theorem environment with no number in Japanese, You can use these code.
\newtheorem*{Axiom*}{公理}
\newtheorem*{Definition*}{定義}
\newtheorem*{Theorem*}{定理}
\newtheorem*{Proposition*}{命題}
\newtheorem*{Lemma*}{補題}
\newtheorem*{Example*}{例}
\newtheorem*{Corollary*}{系}
\newtheorem*{Claim*}{主張}
\newtheorem*{Property*}{性質}
\newtheorem*{Attention*}{注意}
\newtheorem*{Question*}{問}
\newtheorem*{Problem*}{問題}
\newtheorem*{Consideration*}{考察}
\newtheorem*{Alert*}{警告}
\newtheorem*{Fact*}{事実}
\newtheorem*{com*}{コメント}



% 英語で定理, 命題, 証明などを番号付きで用いるためのコマンドです. 
% If you want to use theorem environment in English, You can use these code.
%all theorem enivironment number depend on only section number.
\newtheorem{Axiom+}{Axiom}[section]
\newtheorem{Definition+}[Axiom+]{Definition}
\newtheorem{Theorem+}[Axiom+]{Theorem}
\newtheorem{Proposition+}[Axiom+]{Proposition}
\newtheorem{Lemma+}[Axiom+]{Lemma}
\newtheorem{Example+}[Axiom+]{Example}
\newtheorem{Corollary+}[Axiom+]{Corollary}
\newtheorem{Claim+}[Axiom+]{Claim}
\newtheorem{Property+}[Axiom+]{Property}
\newtheorem{Attention+}[Axiom+]{Attention}
\newtheorem{Question+}[Axiom+]{Question}
\newtheorem{Problem+}[Axiom+]{Problem}
\newtheorem{Consideration+}[Axiom+]{Consideration}
\newtheorem{Alert+}{Alert}
\newtheorem{Fact+}[Axiom+]{Fact}
\newtheorem{Remark+}[Axiom+]{Remark}

% ----------------------------
% commmand
% ----------------------------
% 執筆に便利なコマンド集です. 
% コマンドを追加する場合は下のスペースへ. 

% 集合の記号 (黒板文字)
\newcommand{\NN}{\mathbb{N}}
\newcommand{\ZZ}{\mathbb{Z}}
\newcommand{\QQ}{\mathbb{Q}}
\newcommand{\RR}{\mathbb{R}}
\newcommand{\CC}{\mathbb{C}}
\newcommand{\PP}{\mathbb{P}}
\newcommand{\KK}{\mathbb{K}}


% 集合の記号 (太文字)
\newcommand{\nn}{\mathbf{N}}
\newcommand{\zz}{\mathbf{Z}}
\newcommand{\qq}{\mathbf{Q}}
\newcommand{\rr}{\mathbf{R}}
\newcommand{\cc}{\mathbf{C}}
\newcommand{\pp}{\mathbf{P}}
\newcommand{\kk}{\mathbf{K}}

% 特殊な写像の記号
\newcommand{\ev}{\mathop{\mathrm{ev}}\nolimits} % 値写像
\newcommand{\pr}{\mathop{\mathrm{pr}}\nolimits} % 射影
\newcommand{\grad}{\mathop{\mathrm{grad}}\nolimits} % 射影



% スクリプト体にするコマンド
%   例えば {\mcal C} のように用いる
\newcommand{\mcal}{\mathcal}

% 花文字にするコマンド 
%   例えば {\h C} のように用いる
\newcommand{\h}{\mathscr}

% ヒルベルト空間などの記号
\newcommand{\F}{\mcal{F}}
\newcommand{\X}{\mcal{X}}
\newcommand{\Y}{\mcal{Y}}
\newcommand{\Hil}{\mcal{H}}
\newcommand{\RKHS}{\Hil_{k}}
\newcommand{\Loss}{\mcal{L}_{D}}
\newcommand{\MLsp}{(\X, \Y, D, \Hil, \Loss)}

% 偏微分作用素の記号
\newcommand{\p}{\partial}

% 角カッコの記号 (内積は下にマクロがあります)
\newcommand{\lan}{\langle}
\newcommand{\ran}{\rangle}



% 圏の記号など
\newcommand{\Set}{{\bf Set}}
\newcommand{\Vect}{{\bf Vect}}
\newcommand{\FDVect}{{\bf FDVect}}
\newcommand{\Mod}{\mathop{\mathrm{Mod}}\nolimits}
\newcommand{\CGA}{{\bf CGA}}
\newcommand{\GVect}{{\bf GVect}}
\newcommand{\Lie}{{\bf Lie}}
\newcommand{\dLie}{{\bf Liec}}



% 射の集合など
\newcommand{\Map}{\mathop{\mathrm{Map}}\nolimits}
\newcommand{\Hom}{\mathop{\mathrm{Hom}}\nolimits}
\newcommand{\End}{\mathop{\mathrm{End}}\nolimits}
\newcommand{\Aut}{\mathop{\mathrm{Aut}}\nolimits}
\newcommand{\Mor}{\mathop{\mathrm{Mor}}\nolimits}

% その他便利なコマンド
\newcommand{\dip}{\displaystyle} % 本文中で数式モード
\newcommand{\e}{\varepsilon} % イプシロン
\newcommand{\dl}{\delta} % デルタ
\newcommand{\pphi}{\varphi} % ファイ
\newcommand{\ti}{\tilde} % チルダ
\newcommand{\pal}{\parallel} % 平行
\newcommand{\op}{{\rm op}} % 双対を取る記号
\newcommand{\lcm}{\mathop{\mathrm{lcm}}\nolimits} % 最小公倍数の記号
\newcommand{\Probsp}{(\Omega, \F, \P)} 
\newcommand{\argmax}{\mathop{\rm arg~max}\limits}
\newcommand{\argmin}{\mathop{\rm arg~min}\limits}





% ================================
% コマンドを追加する場合のスペース 
\renewcommand\proofname{\bf 証明} % 証明
%\numberwithin{equation}{subsection}
\newcommand{\cTop}{\textsf{Top}}
%\newcommand{\cOpen}{\textsf{Open}}
\newcommand{\Op}{\mathop{\textsf{Open}}\nolimits}
\newcommand{\Ob}{\mathop{\textrm{Ob}}\nolimits}
\newcommand{\id}{\mathop{\mathrm{id}}\nolimits}
\newcommand{\pt}{\mathop{\mathrm{pt}}\nolimits}
\newcommand{\res}{\mathop{\rho}\nolimits}
\newcommand{\A}{\mcal{A}}
\newcommand{\B}{\mcal{B}}
\newcommand{\C}{\mcal{C}}
\newcommand{\D}{\mcal{D}}
\newcommand{\E}{\mcal{E}}
\newcommand{\G}{\mcal{G}}
%\newcommand{\H}{\mcal{H}}
\newcommand{\I}{\mcal{I}}
\newcommand{\J}{\mcal{J}}
\newcommand{\OO}{\mcal{O}}
\newcommand{\Ring}{\mathop{\textsf{Ring}}\nolimits}
\newcommand{\cAb}{\mathop{\textsf{Ab}}\nolimits}
\newcommand{\Ker}{\mathop{\mathrm{Ker}}\nolimits}
\newcommand{\im}{\mathop{\mathrm{Im}}\nolimits}
\newcommand{\Coker}{\mathop{\mathrm{Coker}}\nolimits}
\newcommand{\Coim}{\mathop{\mathrm{Coim}}\nolimits}
\newcommand{\Ht}{\mathop{\mathrm{Ht}}\nolimits}
\newcommand{\colim}{\mathop{\mathrm{colim}}}
\newcommand{\Tor}{\mathop{\mathrm{Tor}}\nolimits}

\newcommand{\cat}{\mathscr{C}}

\newcommand{\scA}{\mathscr{A}}
\newcommand{\scB}{\mathscr{B}}
\newcommand{\scC}{\mathscr{C}}
\newcommand{\scD}{\mathscr{D}}
\newcommand{\scE}{\mathscr{E}}
\newcommand{\scF}{\mathscr{F}}

\newcommand{\ibA}{\mathop{\text{\textit{\textbf{A}}}}}
\newcommand{\ibB}{\mathop{\text{\textit{\textbf{B}}}}}
\newcommand{\ibC}{\mathop{\text{\textit{\textbf{C}}}}}
\newcommand{\ibD}{\mathop{\text{\textit{\textbf{D}}}}}
\newcommand{\ibE}{\mathop{\text{\textit{\textbf{E}}}}}
\newcommand{\ibF}{\mathop{\text{\textit{\textbf{F}}}}}
\newcommand{\ibG}{\mathop{\text{\textit{\textbf{G}}}}}
\newcommand{\ibH}{\mathop{\text{\textit{\textbf{H}}}}}
\newcommand{\ibI}{\mathop{\text{\textit{\textbf{I}}}}}
\newcommand{\ibJ}{\mathop{\text{\textit{\textbf{J}}}}}
\newcommand{\ibK}{\mathop{\text{\textit{\textbf{K}}}}}
\newcommand{\ibL}{\mathop{\text{\textit{\textbf{L}}}}}
\newcommand{\ibM}{\mathop{\text{\textit{\textbf{M}}}}}
\newcommand{\ibN}{\mathop{\text{\textit{\textbf{N}}}}}
\newcommand{\ibO}{\mathop{\text{\textit{\textbf{O}}}}}
\newcommand{\ibP}{\mathop{\text{\textit{\textbf{P}}}}}
\newcommand{\ibQ}{\mathop{\text{\textit{\textbf{Q}}}}}
\newcommand{\ibR}{\mathop{\text{\textit{\textbf{R}}}}}
\newcommand{\ibS}{\mathop{\text{\textit{\textbf{S}}}}}
\newcommand{\ibT}{\mathop{\text{\textit{\textbf{T}}}}}
\newcommand{\ibU}{\mathop{\text{\textit{\textbf{U}}}}}
\newcommand{\ibV}{\mathop{\text{\textit{\textbf{V}}}}}
\newcommand{\ibW}{\mathop{\text{\textit{\textbf{W}}}}}
\newcommand{\ibX}{\mathop{\text{\textit{\textbf{X}}}}}
\newcommand{\ibY}{\mathop{\text{\textit{\textbf{Y}}}}}
\newcommand{\ibZ}{\mathop{\text{\textit{\textbf{Z}}}}}

\newcommand{\ibx}{\mathop{\text{\textit{\textbf{x}}}}}

\newcommand{\Comp}{\mathop{\mathrm{C}}\nolimits}
\newcommand{\Komp}{\mathop{\mathrm{K}}\nolimits}
\newcommand{\CCat}{\Comp(\cat)}
\newcommand{\KCat}{\Komp(\cat)}
\renewcommand{\Re}{\mathop{\mathrm{Re}}\nolimits}

% =================================



%================================================
% 自前の定理環境
%   https://mathlandscape.com/latex-amsthm/
% を参考にした
\newtheoremstyle{mystyle}%   % スタイル名
    {5pt}%                   % 上部スペース
    {5pt}%                   % 下部スペース
    {}%              % 本文フォント
    {}%                  % 1行目のインデント量
    {\bfseries}%                      % 見出しフォント
    {. }%                     % 見出し後の句読点
    {10pt}%                     % 見出し後のスペース
    {\thmname{#1}\thmnumber{ #2}\thmnote{{\hspace{2pt}\normalfont (#3)}}}% % 見出しの書式

\theoremstyle{mystyle}
\newtheorem{AXM}{公理}[section]
\newtheorem{DFN}{定義}
\newtheorem{THM}[Axiom]{定理}
\newtheorem*{THM*}{定理}
\newtheorem{PRP}{命題}
\newtheorem{LMM}[Axiom]{補題}
\newtheorem{CRL}{系}
\newtheorem*{CRL*}{系}
\newtheorem{EG}[Axiom]{例}
\newtheorem{CNV}[Axiom]{規約}
\newtheorem{NTN}[Axiom]{記号}
\newtheorem*{NTN*}{記号}
\newtheorem{CMT}{コメント}
\newtheorem{RMK}{注意}
\newtheorem*{RMK*}{注意}




% 定理環境ここまで
%====================================================

% ---------------------------
% new definition macro
% ---------------------------
% 便利なマクロ集です

% 内積のマクロ
%   例えば \inner<\pphi | \psi> のように用いる
\def\inner<#1>{\langle #1 \rangle}

% ================================
% マクロを追加する場合のスペース 

%=================================





% ----------------------------
% documenet 
% ----------------------------
% 以下, 本文の執筆スペースです. 
% Your main code must be written between 
% begin document and end document.
% ---------------------------

\title{コーシー問題の超函数解}
\author{ジャン・ミッシェル・ボニー\and ピエール・シャピラ}
\date{}
\begin{document}
\maketitle
\setcounter{section}{-1}
\section{序}
\(P\left(x,\frac{\partial}{\partial x}\right)\)を\(\rr^n\)の
開集合\(U\)で定義された解析的係数の\(m\)階微分作用素で,
主要部がある方向\(N\)に双曲型であるとする(\(P\)の特性多様体に関しては
何も仮定しない).
\((w)\)が超曲面$(x, N) = 0$上の\(m\)組の超函数であり,
\(v\)がこの超曲面の近傍で定義された超函数であり,
\(N\)方向に「解析的」であるとき,
コーシー問題$Pu = v$, $\gamma(u) = (w)$が
超函数の空間で解けることを示す.

証明は,超函数を
正則関数の境界値の和としてあらわし,
複素領域におけるコーシー問題を解き,
そして得られた解が境界値をもつことを示す,という方法で行う.
このとき重要となる道具が2つある.
ひとつは偏微分方程式の正則解の延長定理である.
もうひとつは双曲不等式で,これはボホナーの管定理の局所版にあたる
小松・柏原の定理から導かれる.

我々は同時に,双曲型方程式の解析解を調べ,とくに,
余方向が双曲型ならば,同次方程式の解は
開集合の\(C^1\)級の境界を超えて延長できることを示す.

単純特性的な双曲型作用素の研究は
超函数の範囲については河合\cite{7}が行っている.
定数係数の方程式の解の延長については
キーセルマン\cite{8}によって全く異なる手法で研究されている.

ここで述べる手法は
準備中の論文 (cf. \cite{BS72a} \cite{BS72b} \cite{3} \cite{4}) からの
抜粋である.

\section{記法と復習}

本稿を通して,
\(P=P\left(x,\frac{\p}{\p{x}}\right)\)は\(\rr^n\)の
開集合\(U\)上の解析関数係数微分作用素で,
その係数が\(\cc^n\)の開集合\(\widetilde{U}\)に延長できるものを表す.
\(\widetilde{U}\)上で定義された\(P\)の複素化を\(
    P=P\left(z,\frac{\p}{\p z}\right)
\)で表す.
\(\cc^n\)にエルミート積\(\dip
    \langle z,\zeta\rangle=\sum_{i}^{}z_i\zeta_i
\)を入れたものを,ユークリッド空間\(\rr^{2n}\)にスカラー積\(
    \Re\langle z,\zeta\rangle
\)を入れたものと同一視する.
超平面というときには,特に断らないかぎり,
\(\rr^{2n}\)内の実超平面を意味する.
\(z=x+iy\), \(\zeta=\xi+i\eta\)とかく.
方程式\(\Re\langle z-z_0,\zeta\rangle\)で定義された超曲面が
\(p(z_0,\zeta)=0\)をみたすとき,
\(z_0\)で
\textbf{特性的} (caract\'eristique) であるという.
ここで\(p\)は\(
    \widetilde{U}\times\left(\cc^n-\{0\}\right)
\)で定義された\(P\)の主要表象を表す.
同様に,ベクトル\(\zeta\)のことも\(z_0\)で特性的であるという.

\(\rr^n\)と\(\rr^{2n}\)の単位球面を
それぞれ\(S^{n-1}\), \(S^{2n-1}\)とかく.
\(I\)が\(S^{n-1}\)の部分集合であるとき,
\(I\)が\textbf{凸} (convexe), 
あるいは\textbf{固有} (propre) であるとは,\(I\)の生成する錐が凸,
あるいは直線を一切含まないことをいう.
\(I\)の\textbf{極} (polaire) とは,
原点を通る半空間で方空間の外部が\(I\)に属するものの共通部分である閉錐である.

\(I\)の極の内部を\(\varGamma\)とかくことが多い.
逆に\(\varGamma\)が\(\rr^n\)の凸錐であるとき,
その極\(I\subset S^{n-1}\)は
原点を通る\(\varGamma\)の法線の(閉)集合である.

後で必要となる\cite{3}の定理を復習する.

\begin{THM}
    \(\widetilde{\omega}\)と\(\widetilde{\varOmega}\)を
    \(\cc^n\)の凸集合とし,
    \(\widetilde{\omega}\)は局所コンパクト,
    \(\widetilde{\varOmega}\)は開集合で\(
        \widetilde{\omega}\subset\widetilde{\varOmega}
    \)をみたすとする.
    \(\widetilde{\varOmega}\)の(少なくとも)ひとつの点で
    特性的な方向の極限が法方向であるような超平面を考える.
    そのような超平面で\(\widetilde{\varOmega}\)と交わるものは
    すべて\(\widetilde{\omega}\)とも交わると仮定する.
    このとき,\(f\)が\(\widetilde{\omega}\)の近傍上の正則関数であり,
    \(Pf\)が\(\widetilde{\varOmega}\)上の正則関数に延長されるならば,
    関数\(f\)も\(\widetilde{\varOmega}\)上の正則関数に延長される.
\end{THM}

\begin{RMK*}
    この定理はツェルナー\cite{15}の定理から
    ヘルマンダーによる幾何学的な議論によって
    証明される(\cite{5}の定理5.3.3を参照.).
    第4節で類似の状況が現れる.
    この定理は方程式系\(P_if=g_i\)の解\(f\)に関しても成り立つ
    ことに注意しておく.
    この定理はキーセルマン\cite{8}によって
    定数係数作用素の場合に証明されている.
\end{RMK*}

\section{双曲不等式}

\begin{THM}\label{thm21}
    \(Q(z,\tau)\)を\(\cc^p\)の開集合\(\widetilde{V}\)上の
    正則関数\(\tau\)を係数とする\(m\)次多項式とする.
    \(\tau^m\)の係数は\(1\)で,
    方程式\(Q(z,\tau)=0\)は
    \textcolor{red}{
        \(z\)が実数 (\(
            z\in V=\widetilde{V}\cap \rr^p
        \)) のとき,\(\tau\)の実根を持たない    
    }と仮定する.

    このとき\(\widetilde{V}\cap \rr^p\)の
    任意のコンパクト集合\(K\)に対し\(\varepsilon>0\)と\(C>0\)で\[
        Q(z,\tau)=0,\quad \Re{z}\in K,
        \quad
        \lvert\im{z}\rvert<\varepsilon
        \Longrightarrow
        \lvert\im{\tau}\rvert\leqq C\lvert\im{z}\rvert
    \]
    となるものが存在する.
\end{THM}

柏原の指摘したように,この定理はボホナーの管定理の次の定式化から
ただちに従う(Kashiwara \cite{6}, Komatsu \cite{8bis}).

\begin{THM}\label{thm22}
    \(\cc^{p+1}\)の部分集合\(L\)を次で定める.
    \[
        L=\left\{
            x+iy\mid
            \lvert{x}\rvert<r,\ 
            y_1=\cdots=y_p=0,\ 
            0<y_{p+1}\leqq b
        \right\}.
    \]
    \(f\)が\(L\)の近傍上の正則関数のとき,
    任意の\(a<r\)に対し定数\(C>0\)で\(f\)が開集合
    \[
        \widetilde{L}=\left\{
            x+iy\mid
            \lvert{x}\rvert<a,\ 
            \lvert{y_1}\rvert+\cdots+\lvert{y_p}\rvert<C\lvert{y_{p+1}}\rvert,\ 
            0<y_{p+1}< b
        \right\}.
    \]
    で正則となる.
\end{THM}

以下では\(\rr^n\)のベクトル\((0,\dots,0,1)\)を\(N\)で表し,
\(z=(z',z_n)\), \(\zeta=(\zeta',\zeta_n)\)とかく.

方程式\(p(x_0,\xi',\zeta_n)=0\)が\(\zeta_n\)しか
実根を持たない(\(\xi'\in\rr^{n-1}\))とき,
作用素\(P(x,\frac{\p}{\p x})\)は点\(x_0\)で\(N\)の方向に
双曲的な主部をもつのだった.
とくに,方向\(N\)は非特性的である.

\begin{THM}\label{thm23}
    \(P\)の主部は\(x_0\)の近傍の任意の点において\(N\)の方向に
    双曲的であると仮定する.
    定数\(C>0\)と数\(\varepsilon>0\)で次をみたすものが存在する.
    \[\begin{multlined}
        \lvert{z-x_0}\rvert<\varepsilon,\ 
        \zeta\in\cc^{n},\ 
        p(z,\zeta+\tau N)=0\\
        \Longrightarrow
        \lvert{\im{\tau}}\rvert
        \leqq
        C\left[\lvert{\xi'}\rvert\lvert{y}\rvert+\lvert{\eta}\rvert\right].
    \end{multlined}\]
\end{THM}

\begin{proof}
    定理\ref{thm21}より,
    \(x\in K\), \(\lvert{y}\rvert<\varepsilon\), 
    \(\lvert{\xi'}\rvert=1\), 
    \(\lvert{\eta'}\rvert<\varepsilon\)とすれば,
    方程式\(p(z,\zeta',\tau)=0\)の解は
    \[
        \lvert\im{z}\rvert
        \leqq 
        C_{1}\left[
            \lvert{y}\rvert+\lvert{\eta'}\rvert
        \right]
    \]をみたす.
    同様に\(
        \lvert{\eta'}\rvert
        \leqq\varepsilon\lvert\xi'\rvert
    \)とすることで\[
        \lvert\im{z}\rvert
        \leqq 
        C_{1}\left[
            \lvert{y}\rvert\lvert\xi'\rvert+\lvert{\eta'}\rvert
        \right]
    \]を得る.

    他方で,\(x_0\)の近傍で\(N\)は非特性的なので,
    \(p(z,\zeta+\tau N)\)の解の上からの評価\[
        \lvert{\tau+\zeta_n}\rvert\leqq C_2\lvert{\zeta'}\rvert
    \]とそこから\[
        \lvert{\im{\tau}}\rvert\leqq C_3\left[\lvert{\zeta'}\rvert+\lvert{\zeta_n}\rvert\right]
    \]が得られ,これより\(
        \lvert{\eta'}\rvert\leqq\alpha\lvert{\xi'}\rvert
    \)に対して\[
        \lvert{\im{\tau}}\rvert\leqq C_4\lvert{\eta}\rvert
    \]が成り立つ.
\end{proof}
\section{延長に関する2つの補題}
\(B(0,a)\)で\(0\)を中心とする半径\(a\)の開球を表す.
\(\overline{B(0,a)}\)でその閉包を表し,
\(B'(0,a)\)で超曲面\(\inner<x,N>=0\)との共通部分を表す.

\(K(a,\delta)\)で\(B'(0,a)\)の凸閉包の内部を表し,
(座標\(0,\dots,0,\delta a\)における)その点を\(\delta aN\)で表す.

\begin{LMM}
    \(P\)を球体\(\overline{B(0,r)}\)の各点で\(N\)の方向に
    双曲型な主部を持つ偏微分作用素とする.
    定数\(\delta>0\)で,全ての\(a<r\)と\(
        \overline{B'(0,a)}
    \)の近傍で正則な任意の関数\(f\)で\(Pf\)が\(K(a,\delta)\)に解析接続されるものに対し,
    その開集合上に\(f\)自体も解析接続されるものが存在する.
\end{LMM}

\begin{proof}
    \(\varepsilon>0\)とする.
    \(A_{\varepsilon}\)で,
    \(\lvert x\rvert\leqq a\)と\(\lvert x\rvert\leqq r\)に対する
    点\(z=x+iy\)において\(P(z,\frac{\partial}{\partial z})\)に対し
    特性的な方向\(\zeta\in S^{2n-1}\)のなす(閉)集合を表す.
\end{proof}
\section{解析解}

\section{超函数の復習}

\section{コーシー問題の超函数解}

\section{超函数解の存在と延長}

%===============================================
% 参考文献スペース
%===============================================
\begin{thebibliography}{20} 
    \bibitem[1]{BS72a} Bony, J. -M. et Schapira, P.: 
    Existence et prolongement des solutions analytiques 
    des systèmes hyperboliques non stricts. 
    C. R. Acad. Sci. Paris, 274 (1972), 86--89.

    \bibitem[2]{BS72b} Bony, J. -M. et Schapira, P.: 
    Problème de Cauchy, existence et prolongement 
    pour les hyperfonctions solutions d'équations hyperboliques 
    non strictes. C. R. Acad. Sci. Paris, 274 (1972), 188--191.

    \bibitem[3]{3} Bony, J. -M. et Schapira, P.: 
    Existence et prolongement des solutions holomorphes 
    des \'equations aux d\'eriv\'ees partielles. 
    Article \`a para\^itre aux Inventiones Mathematicae.
   
    \bibitem[4]{4} Bony, J. -M. et Schapira, P.: 
    Solutions analytiques et solutions hyperfonctions 
    des \'equations hyperboliques non strictes. 
    Article \`a para\^itre.

    \bibitem[5]{5} H\"ormander, L.: 
    Linear Partial Differential Operators. Springer, 1963.

    \bibitem[6]{6} Kashiwara, M.: 
    On the structure of hyperfunctions (after M. Sato). 
    Sugaku no Ayumi, 15 (1970), 19--72 (en Japonais).

    \bibitem[7]{7}
    Kawai, T.: 
    Construction of elementary solutions of 
    I-hyperbolic operators and solutions 
    with small singularities. 
    Proc. Japan Acad., 46 (1970), 912--915.

    \bibitem[7bis]{7bis}
    Kawai, T.: 
    On the theory of Fourier transform 
    in the theory of hyperfunctions and its applications, 
    Surikaiseki Kenkyusho Kokyuroku, RIMS, 
    Kyoto Univ., 108 (1969), 84--288 (en Japonais).
    
    \bibitem[8]{8}
    Kiselman, C.-O.: 
    Prolongement des solutions d’une \'equation aux d\'eriv\'ees 
    partielles \'a coefficients constants. 
    Bull. Soc. Math. France, 97 (1969), 329--356.

    \bibitem[8bis]{8bis}
    Komatsu, H.: 
    A local version of Bochner’s tube theorem. 
    J. Fac. Sci. Univ. Tokyo, Sect. IA (\`a para\^itre).

    \bibitem[9]{9}
    Leray, J. et Ohya, Y.: 
    Syst\'emes lin\'eaires hyperboliques non stricts. 
    Colloque sur l’Analyse Fonctionnelle, 
    Li\`ege, 1964, C.B.R.M., pp.105--144.

    \bibitem[10]{10}
    Martineau, A.: 
    Distributions et valeurs au bord des fonctions holomorphes. 
    Proc. of the Intern. Summer Inst. Lisborn, 1964, pp.193--326.

    \bibitem[10bis]{10bis}
    Morimoto, M.: 
    Sur la d\'ecomposition du faisceau des germes 
    de singularit\'es d’hyperfonctions, 
    J. Fac. Sci. Univ. Tokyo, Sect. IA, 17 (1970), 215--239.

    \bibitem[11]{11}
    Sato, M.: 
    Theory of hyperfunctions, I et II. 
    J. Fac. Sci. Univ. Tokyo, Sect. I, 8 (1959–60), 
    139--193 et 398--437.

    \bibitem[12]{12}
    Sato, M.: 
    Regularity of hyperfunction solutions 
    of partial differential equations. 
    Intern. Congress of Math., Nice, 1970.

    \bibitem[13]{13}
    Schapira, P.: 
    Theorie des Hyperfonctions. 
    Lecture Notes in Math. Springer, No.126, 1970.

    \bibitem[14]{14}
    Schapira, P.: 
    Th\'eor\`eme d’unicit\'e de Holmgren et op\'erateurs 
    hyperboliques dans l’espace des hyperfonctions. 
    Anais Acad. Brasil Sc., 43 (1971), 38--44.

    \bibitem[15]{15}
    Zerner, M.: 
    Domaine d’holomorphie des fonctions v\'erifiant 
    une \'equation aux d\'eriv\'ees partielles. 
    C. R. Acad. Sci. Paris, 272 (1971), 1646--1648.
\end{thebibliography}

%===============================================


\end{document}
