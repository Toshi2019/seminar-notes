%================================================
%    sheaves on manifolds ノート
%================================================

% -----------------------
% preamble
% -----------------------
% ここから本文 (\begin{document}) までの
% ソースコードに変更を加えた場合は
% 編集者まで連絡してください. 
% Don't change preamble code yourself. 
% If you add something
% (usepackage, newtheorem, newcommand, renewcommand),
% please tell it 
% to the editor of institutional paper of RUMS.

% ------------------------
% documentclass
% ------------------------
\documentclass[11pt, a4paper, dvipdfmx, draft, leqno]{jsarticle}

% ------------------------
% usepackage
% ------------------------
\usepackage{algorithm}
\usepackage{algorithmic}
\usepackage{amscd}
\usepackage{amsfonts}
\usepackage{amsmath}
\usepackage[psamsfonts]{amssymb}
\usepackage{amsthm}
\usepackage{ascmac}
\usepackage{color}
\usepackage{enumerate}
\usepackage{fancybox}
\usepackage[stable]{footmisc}
\usepackage{graphicx}
\usepackage{listings}
\usepackage{mathrsfs}
\usepackage{mathtools}
\usepackage{otf}
\usepackage{pifont}
\usepackage{proof}
\usepackage{subfigure}
\usepackage{tikz}
\usepackage{verbatim}
\usepackage[all]{xy}
\usepackage{url}
\usetikzlibrary{cd}



% ================================
% パッケージを追加する場合のスペース 
%\usepackage{calligra}
\usepackage[dvipdfmx]{hyperref}
\usepackage{xcolor}
\definecolor{darkgreen}{rgb}{0,0.45,0} 
\definecolor{darkred}{rgb}{0.75,0,0}
\definecolor{darkblue}{rgb}{0,0,0.6} 
\hypersetup{
    colorlinks=true,
    citecolor=darkgreen,
    linkcolor=darkred,
    urlcolor=darkblue,
}
\usepackage{pxjahyper}

%=================================


% --------------------------
% theoremstyle
% --------------------------
\theoremstyle{definition}

% --------------------------
% newtheoem
% --------------------------

% 日本語で定理, 命題, 証明などを番号付きで用いるためのコマンドです. 
% If you want to use theorem environment in Japanece, 
% you can use these code. 
% Attention!
% All theorem enivironment numbers depend on 
% only section numbers.
\newtheorem{Axiom}{公理}[section]
\newtheorem{Definition}[Axiom]{定義}
\newtheorem{Theorem}[Axiom]{定理}
\newtheorem{Proposition}[Axiom]{命題}
\newtheorem{Lemma}[Axiom]{補題}
\newtheorem{Corollary}[Axiom]{系}
\newtheorem{Example}[Axiom]{例}
\newtheorem{Claim}[Axiom]{主張}
\newtheorem{Property}[Axiom]{性質}
\newtheorem{Attention}[Axiom]{注意}
\newtheorem{Question}[Axiom]{問}
\newtheorem{Problem}[Axiom]{問題}
\newtheorem{Consideration}[Axiom]{考察}
\newtheorem{Alert}[Axiom]{警告}
\newtheorem{Fact}[Axiom]{事実}
\newtheorem{com}[Axiom]{コメント}


% 日本語で定理, 命題, 証明などを番号なしで用いるためのコマンドです. 
% If you want to use theorem environment with no number in Japanese, You can use these code.
\newtheorem*{Axiom*}{公理}
\newtheorem*{Definition*}{定義}
\newtheorem*{Theorem*}{定理}
\newtheorem*{Proposition*}{命題}
\newtheorem*{Lemma*}{補題}
\newtheorem*{Example*}{例}
\newtheorem*{Corollary*}{系}
\newtheorem*{Claim*}{主張}
\newtheorem*{Property*}{性質}
\newtheorem*{Attention*}{注意}
\newtheorem*{Question*}{問}
\newtheorem*{Problem*}{問題}
\newtheorem*{Consideration*}{考察}
\newtheorem*{Alert*}{警告}
\newtheorem*{Fact*}{事実}
\newtheorem*{com*}{コメント}



% 英語で定理, 命題, 証明などを番号付きで用いるためのコマンドです. 
% If you want to use theorem environment in English, You can use these code.
%all theorem enivironment number depend on only section number.
\newtheorem{Axiom+}{Axiom}[section]
\newtheorem{Definition+}[Axiom+]{Definition}
\newtheorem{Theorem+}[Axiom+]{Theorem}
\newtheorem{Proposition+}[Axiom+]{Proposition}
\newtheorem{Lemma+}[Axiom+]{Lemma}
\newtheorem{Example+}[Axiom+]{Example}
\newtheorem{Corollary+}[Axiom+]{Corollary}
\newtheorem{Claim+}[Axiom+]{Claim}
\newtheorem{Property+}[Axiom+]{Property}
\newtheorem{Attention+}[Axiom+]{Attention}
\newtheorem{Question+}[Axiom+]{Question}
\newtheorem{Problem+}[Axiom+]{Problem}
\newtheorem{Consideration+}[Axiom+]{Consideration}
\newtheorem{Alert+}{Alert}
\newtheorem{Fact+}[Axiom+]{Fact}
\newtheorem{Remark+}[Axiom+]{Remark}

% ----------------------------
% commmand
% ----------------------------
% 執筆に便利なコマンド集です. 
% コマンドを追加する場合は下のスペースへ. 

% 集合の記号 (黒板文字)
\newcommand{\NN}{\mathbb{N}}
\newcommand{\ZZ}{\mathbb{Z}}
\newcommand{\QQ}{\mathbb{Q}}
\newcommand{\RR}{\mathbb{R}}
\newcommand{\CC}{\mathbb{C}}
\newcommand{\PP}{\mathbb{P}}
\newcommand{\KK}{\mathbb{K}}


% 集合の記号 (太文字)
\newcommand{\nn}{\mathbf{N}}
\newcommand{\zz}{\mathbf{Z}}
\newcommand{\qq}{\mathbf{Q}}
\newcommand{\rr}{\mathbf{R}}
\newcommand{\cc}{\mathbf{C}}
\newcommand{\pp}{\mathbf{P}}
\newcommand{\kk}{\mathbf{K}}

% 特殊な写像の記号
\newcommand{\ev}{\mathop{\mathrm{ev}}\nolimits} % 値写像
\newcommand{\pr}{\mathop{\mathrm{pr}}\nolimits} % 射影
\newcommand{\grad}{\mathop{\mathrm{grad}}\nolimits} % 射影



% スクリプト体にするコマンド
%   例えば {\mcal C} のように用いる
\newcommand{\mcal}{\mathcal}

% 花文字にするコマンド 
%   例えば {\h C} のように用いる
\newcommand{\h}{\mathscr}

% ヒルベルト空間などの記号
\newcommand{\F}{\mcal{F}}
\newcommand{\X}{\mcal{X}}
\newcommand{\Y}{\mcal{Y}}
\newcommand{\Hil}{\mcal{H}}
\newcommand{\RKHS}{\Hil_{k}}
\newcommand{\Loss}{\mcal{L}_{D}}
\newcommand{\MLsp}{(\X, \Y, D, \Hil, \Loss)}

% 偏微分作用素の記号
\newcommand{\p}{\partial}

% 角カッコの記号 (内積は下にマクロがあります)
\newcommand{\lan}{\langle}
\newcommand{\ran}{\rangle}



% 圏の記号など
\newcommand{\Set}{{\bf Set}}
\newcommand{\Vect}{{\bf Vect}}
\newcommand{\FDVect}{{\bf FDVect}}
\newcommand{\Mod}{\mathop{\mathrm{Mod}}\nolimits}
\newcommand{\CGA}{{\bf CGA}}
\newcommand{\GVect}{{\bf GVect}}
\newcommand{\Lie}{{\bf Lie}}
\newcommand{\dLie}{{\bf Liec}}



% 射の集合など
\newcommand{\Map}{\mathop{\mathrm{Map}}\nolimits}
\newcommand{\Hom}{\mathop{\mathrm{Hom}}\nolimits}
\newcommand{\End}{\mathop{\mathrm{End}}\nolimits}
\newcommand{\Aut}{\mathop{\mathrm{Aut}}\nolimits}
\newcommand{\Mor}{\mathop{\mathrm{Mor}}\nolimits}

% その他便利なコマンド
\newcommand{\dip}{\displaystyle} % 本文中で数式モード
\newcommand{\e}{\varepsilon} % イプシロン
\newcommand{\dl}{\delta} % デルタ
\newcommand{\pphi}{\varphi} % ファイ
\newcommand{\ti}{\tilde} % チルダ
\newcommand{\pal}{\parallel} % 平行
\newcommand{\op}{{\rm op}} % 双対を取る記号
\newcommand{\lcm}{\mathop{\mathrm{lcm}}\nolimits} % 最小公倍数の記号
\newcommand{\Probsp}{(\Omega, \F, \P)} 
\newcommand{\argmax}{\mathop{\rm arg~max}\limits}
\newcommand{\argmin}{\mathop{\rm arg~min}\limits}





% ================================
% コマンドを追加する場合のスペース 
\renewcommand\proofname{\bf 証明} % 証明
%\numberwithin{equation}{subsection}
\newcommand{\cTop}{\textsf{Top}}
%\newcommand{\cOpen}{\textsf{Open}}
\newcommand{\Op}{\mathop{\textsf{Open}}\nolimits}
\newcommand{\Ob}{\mathop{\textrm{Ob}}\nolimits}
\newcommand{\id}{\mathop{\mathrm{id}}\nolimits}
\newcommand{\pt}{\mathop{\mathrm{pt}}\nolimits}
\newcommand{\res}{\mathop{\rho}\nolimits}
\newcommand{\A}{\mcal{A}}
\newcommand{\B}{\mcal{B}}
\newcommand{\C}{\mcal{C}}
\newcommand{\D}{\mcal{D}}
\newcommand{\E}{\mcal{E}}
\newcommand{\G}{\mcal{G}}
%\newcommand{\H}{\mcal{H}}
\newcommand{\I}{\mcal{I}}
\newcommand{\J}{\mcal{J}}
\newcommand{\OO}{\mcal{O}}
\newcommand{\Ring}{\mathop{\textsf{Ring}}\nolimits}
\newcommand{\cAb}{\mathop{\textsf{Ab}}\nolimits}
\newcommand{\Ker}{\mathop{\mathrm{Ker}}\nolimits}
\newcommand{\im}{\mathop{\mathrm{Im}}\nolimits}
\newcommand{\Coker}{\mathop{\mathrm{Coker}}\nolimits}
\newcommand{\Coim}{\mathop{\mathrm{Coim}}\nolimits}
\newcommand{\Ht}{\mathop{\mathrm{Ht}}\nolimits}
\newcommand{\colim}{\mathop{\mathrm{colim}}}
\newcommand{\Tor}{\mathop{\mathrm{Tor}}\nolimits}

\newcommand{\cat}{\mathscr{C}}

\newcommand{\scA}{\mathscr{A}}
\newcommand{\scB}{\mathscr{B}}
\newcommand{\scC}{\mathscr{C}}
\newcommand{\scD}{\mathscr{D}}
\newcommand{\scE}{\mathscr{E}}
\newcommand{\scF}{\mathscr{F}}

\newcommand{\ibA}{\mathop{\text{\textit{\textbf{A}}}}}
\newcommand{\ibB}{\mathop{\text{\textit{\textbf{B}}}}}
\newcommand{\ibC}{\mathop{\text{\textit{\textbf{C}}}}}
\newcommand{\ibD}{\mathop{\text{\textit{\textbf{D}}}}}
\newcommand{\ibE}{\mathop{\text{\textit{\textbf{E}}}}}
\newcommand{\ibF}{\mathop{\text{\textit{\textbf{F}}}}}
\newcommand{\ibG}{\mathop{\text{\textit{\textbf{G}}}}}
\newcommand{\ibH}{\mathop{\text{\textit{\textbf{H}}}}}
\newcommand{\ibI}{\mathop{\text{\textit{\textbf{I}}}}}
\newcommand{\ibJ}{\mathop{\text{\textit{\textbf{J}}}}}
\newcommand{\ibK}{\mathop{\text{\textit{\textbf{K}}}}}
\newcommand{\ibL}{\mathop{\text{\textit{\textbf{L}}}}}
\newcommand{\ibM}{\mathop{\text{\textit{\textbf{M}}}}}
\newcommand{\ibN}{\mathop{\text{\textit{\textbf{N}}}}}
\newcommand{\ibO}{\mathop{\text{\textit{\textbf{O}}}}}
\newcommand{\ibP}{\mathop{\text{\textit{\textbf{P}}}}}
\newcommand{\ibQ}{\mathop{\text{\textit{\textbf{Q}}}}}
\newcommand{\ibR}{\mathop{\text{\textit{\textbf{R}}}}}
\newcommand{\ibS}{\mathop{\text{\textit{\textbf{S}}}}}
\newcommand{\ibT}{\mathop{\text{\textit{\textbf{T}}}}}
\newcommand{\ibU}{\mathop{\text{\textit{\textbf{U}}}}}
\newcommand{\ibV}{\mathop{\text{\textit{\textbf{V}}}}}
\newcommand{\ibW}{\mathop{\text{\textit{\textbf{W}}}}}
\newcommand{\ibX}{\mathop{\text{\textit{\textbf{X}}}}}
\newcommand{\ibY}{\mathop{\text{\textit{\textbf{Y}}}}}
\newcommand{\ibZ}{\mathop{\text{\textit{\textbf{Z}}}}}

\newcommand{\ibx}{\mathop{\text{\textit{\textbf{x}}}}}

\newcommand{\Comp}{\mathop{\mathrm{C}}\nolimits}
\newcommand{\Komp}{\mathop{\mathrm{K}}\nolimits}
\newcommand{\CCat}{\Comp(\cat)}
\newcommand{\KCat}{\Komp(\cat)}

% =================================



%================================================
% 自前の定理環境
%   https://mathlandscape.com/latex-amsthm/
% を参考にした
\newtheoremstyle{mystyle}%   % スタイル名
    {5pt}%                   % 上部スペース
    {5pt}%                   % 下部スペース
    {}%              % 本文フォント
    {}%                  % 1行目のインデント量
    {\bfseries}%                      % 見出しフォント
    {.\quad ---}%                     % 見出し後の句読点
    {10pt}%                     % 見出し後のスペース
    {\thmname{#1}\thmnumber{ #2}\thmnote{{\hspace{2pt}\normalfont (#3)}}}% % 見出しの書式

\theoremstyle{mystyle}
\newtheorem{AXM}{公理}
\newtheorem{DFN}{定義}
\newtheorem{THM}{定理}
\newtheorem*{THM*}{定理}
\newtheorem{PRP}{命題}
\newtheorem{LMM}[Axiom]{補題}
\newtheorem{CRL}{系}
\newtheorem*{CRL*}{系}
\newtheorem{EG}[Axiom]{例}
\newtheorem{CNV}[Axiom]{規約}
\newtheorem{NTN}[Axiom]{記号}
\newtheorem*{NTN*}{記号}
\newtheorem{CMT}{コメント}
\newtheorem{RMK}{注意}
\newtheorem*{RMK*}{注意}




% 定理環境ここまで
%====================================================

% ---------------------------
% new definition macro
% ---------------------------
% 便利なマクロ集です

% 内積のマクロ
%   例えば \inner<\pphi | \psi> のように用いる
\def\inner<#1>{\langle #1 \rangle}

% ================================
% マクロを追加する場合のスペース 

%=================================





% ----------------------------
% documenet 
% ----------------------------
% 以下, 本文の執筆スペースです. 
% Your main code must be written between 
% begin document and end document.
% ---------------------------

\title{数学解析 --- 偏微分方程式をみたす関数の正則領域\footnote{
    Zerner, M.: \textit{Domaine d'holomorphie des fonctions v\'erifiant une \'equation aux d\'eriv\'ees partielles}
    C. R. Acad. Sc. Paris, t. \textbf{272}, 1646 (1971). の和訳.
}}
\author{マルティン・ツェルナー}
\date{1971年6月7日}
\begin{document}
\maketitle

%\section{\cite{Zer71}の和訳}

コーシー・コワレフスキーの定理から簡単に従う幾何学的な結果を明示する.
我々の用いる主張はルレーによる
この定理\footnote{
    コーシーデータをもつ多様体の近くでの
    解析的な線形コーシー問題の解の一意性
}の正確な形からすぐに従う結果である.

\begin{NTN*}
    \(a(x,\partial/\partial x)\)は
    開集合\(\Omega\subset\cc^n\)上の解析関数を係数とする
    微分作用素とする.
    \(m\)をその階数とし\(g\)をその主要部とする.
    \(\cc^n\)にエルミートノルムを入れる.次のようにおく:
    \[
        \theta(x,\xi)=g(x,\xi)\lVert\xi\rVert^{-m}.
    \]
\end{NTN*}

\begin{THM*}
    任意のコンパクト集合\(K\subset\Omega\)に対し,
    \(C>0\)で次の性質を満たすものが存在する:
    
    任意の\(x_0\in K,\xi\in\cc^n\)で
    \(\theta(x_0,\xi)\ne0, r>0\)
    をみたすものに対し,
    方程式
    \[
        \langle x-x_0,\xi\rangle=0
    \]で定義される超曲面の上で初期データをみたす
    任意のコーシー問題のただ一つの解であって,
    この超平面内の半径\(r\)の球において正則なものは,
    \(x_0\)を中心とし,
    \[
        C\cdot\theta(x_0,\xi)\min\left(\theta(x_0,\xi),r\right)
    \]
    を半径とする球において正則である.
\end{THM*}

\begin{DFN}
    (ノルム空間の)開集合\(\Omega\)が
    点\(x\in\partial\Omega\)において
    \textbf{良い台} (bon appui) をもつとは,
    \(x\)を通る超平面\(H\)と\(x\)に収束する
    点列\(x_\nu\)で,\(x_\nu\)を通り\(H\)に平行な超平面\(H_\nu\)と
    \(H_\nu\)における\(x_\nu\)を中心とする\(H_\nu\cap\Omega\)に
    含まれる球のうち最大のものの半径\(\rho_\nu\)に対して
    \(\lim\lVert x-x_\nu\rVert/\rho_\nu=0\)が成り立つ
    ことをいう.

    もちろん\(H\)を\textbf{良い超平面台} 
    (hyperplan de bon appui) という.
\end{DFN}
\begin{CMT}
    \(\Omega\)が凸ならば,良い台をもつ点において,
    古典的な意味での台の超平面は一意であり,
    一意な良い超平面台でもある
    (凸性が無いと,良い超平面台は一意ではなくなる.
    例:2つの開集合の合併は同じ点でそれぞれ良い台を持つ).
\end{CMT}
\begin{DFN}
    (距離空間の)開集合\(\Omega\)が
    点\(x\in\partial\Omega\)において
    \textbf{有限の内部曲率} (courbure interne finie) を
    もつ(内部曲率が有限である)とは,
    \(\Omega\)に含まれる球で\(x\)を境界にもつもの
    が存在することをいう.
    この性質を持つ球の半径の上界を
    \(x\)における\textbf{内部曲率半径} 
    (rayon de courbure interne) と呼ぶ.
\end{DFN}
\begin{CMT}
    ユークリッド空間において,開集合は
    内部曲率が有限である全ての点において良い台をもつ.
    この開集合に含まれる球の境界に
    その境界上の点で接する超平面は良い超平面台である.
\end{CMT}
\begin{DFN}
    \(\cc^n\)における実の意味での超平面が
    点\(x\in\Omega\)において\textbf{特性的} 
    (caract\'eristique) であるとは,
    それを含む複素の意味での超平面が一意なことをいう.
\end{DFN}
\begin{PRP}
    \(U\subset\Omega\)を開集合とし
    \(x\)を\(\Omega\)における\(U\)の境界の元で,
    \(x\)において\(U\)が特性的でない良い超平面台をもつものとする.    
    このとき\(x\)の近傍\(V\)で,
    \begin{equation}\label{eq:Zer1}
        a\left(x,\frac{\partial}{\partial x}u(x)\right)=0
    \end{equation}
    をみたす全ての\(U\)上の解析関数\(u\)と\(a\)を\(U\cup V\)の被覆の上の解析関数に延長できる
    ものが存在する.
\end{PRP}
\begin{CMT}
    被覆
\end{CMT}
\begin{CRL}
    開集合\(U\subset\Omega\)が\eqref{eq:Zer1}の解の正則領域であり
    その境界が\(U\)の実\(2n-1\)次元正則部分多様体\(S\)であるとすると,
    \(S\)に接する超平面は特性的である.
\end{CRL}
\begin{CMT}
    \(S\)のレヴィ形式が0ならば,
    \(S\)は特性的な解析的超曲面の合併である.
\end{CMT}
\begin{CRL}
    \(U_1,U_2\)が一般の位置にある実\(2n-1\)次元
    正則部分多様体\(S_1,S_2\)を境界にもち
    \(U_1\cup U_2\)が\eqref{eq:Zer1}をみたす関数の正則領域であるとき,
    \(S_1\cap S_2\)は特性的な解析多様体となる.
\end{CRL}
\begin{PRP}
    \eqref{eq:Zer1}をみたす関数の正則領域である凸開集合に対し,
    境界の各点において,良い超平面台の少なくとも一つは特性的である.
\end{PRP}
\begin{CRL*}
    有界な凸開集合で境界の各点における良い超平面台が一意であるものは
    定数係数偏微分方程式で第2項を持たないもの解の正則領域にはなり得ない.
\end{CRL*}

複素超平面
\[
    \langle x-x_0,\xi\rangle=0
\]が\(x_0\)で\textbf{単純特性的}(caract\'eristique simple) であるとは
\[
    g(x_0,\xi)=0 \quad\text{だが}\quad
    \grad_\xi g(x_0,\xi)\ne 0
\]となることであった.

実超平面が(複素)超平面で\(x_0\)において単純特性的であるものを含むとき,
実超平面は\(x_0\)において単純特性的であるという.

\begin{PRP}
    \(a\)を定数係数の作用素とする.
    \(\Omega\)を凸開集合で\(\partial\Omega\)の
    稠密な集合の各点において,単純特性的な超平面台が存在するものとする.
    このとき,\(\Omega\)は\eqref{eq:Zer1}をみたす関数の正則領域である.
\end{PRP}
\begin{RMK}
    同様の、しかし異なる結果が
    キーセルマン\footnote{
        Bull. Soc. math Fr., 97, 1969, p.329--356.
    }とシャピラ(私信)によって得られている.
\end{RMK}
%\section{\cite{BS72}の和訳}



\end{document}
