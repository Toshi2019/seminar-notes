\chapter*{Introduction}
Let $X$ be a real manifold, 
$F$ a sheaf on $X$, or better an objet of $\Drv^+(X)$, 
the derived category of the category of complexes 
bounded from below of sheaves on $X$. 
Let $TX$ be the cotangent bundle to $X$. 
We associate to $F$ a closed conic subset of $TX$, 
denoted $SS (F)$, the ``micro-support of $F$'', as follows:

\begin{DFN}
    Let $p=(x_0,\xi_0)\in T^\ast X$. 
    Then $p\notin SS(F)$ if and only if 
    there exists an open neighborhood $U$ of $p$ in $TX$ 
    such that for any $(x_1,\xi_1) \in U$, 
    any real $C^1$-function $\phi$ on $X$, 
    with $\phi(x_1)=0$, $d\phi(x_1)=\xi_1$, 
    we have: $(\rr \Gamma_{x;\phi(x)\geqq 0} (F))_{x_1} =0$.
\end{DFN}

In other words 
the micro-support of $F$ describes 
the set of codirections of $X$ where $F$, and its cohomology, 
``do not propagate''. 
This definition is motivated by the following situation. 

Assume $X$ is a complex manifold, 
and let $\mcal{M}$ be a coherent module over 
the Ring $\mcal{D}_X$ of (holomorphic, finire order) 
differential operators. 
Let $\Ch(\mcal{M})$ be the characteristic variety 
of $\mcal{M}$ in $T^\ast X$. 
Then we can interpret a well-known result of Zerner [1], 
Bony-Schapira [1], Kashiwara [5], through the formula:
\begin{equation}
    \muS(\RHOM_{\mcal{D}_X}(\mcal{M},\mcal{O}_X))\subset\Ch(\mcal{M}).
\end{equation}
A natural problem then arising 
in the theory of (micro-)differential equations, 
is to evaluate the set of codirections of propagation for
the sheaf of hyperfunction or microfunction solutions 
of $\mcal{M}$ (or more generally 
of a system of micro-differential equations). 
To be more precise, 
let $M$ be a real analytic manifold of dimension $n$, 
$X$ a complexification of $M$. 
Recall that the sheaf $\mcal{B}_M$ (resp. $\mcal{C}_M$) 
of Sato's hyperfunctions on $M$ (resp. 
Sato's microfunctions on $T^{\ast}_{M}X$, 
the conormal bundle to $M$ in $X$) is defined by:
\begin{align*}
    \mcal{B}_M &= \RG_M(\mcal{O}_X)\otimes\omega_M[n]\\
    (\text{resp. }
    \mcal{C}_M &= \mu_M(\mcal{O}_X)\otimes\omega_M[n])
\end{align*}
where $\omega_M$ is the orientation sheaf on $M$, 
$[n]$ means the $n$-shift in $\Drv^+(X)$, 
and $\mu(\boldsymbol{\cdot})$ is the functor of 
Sato's microlocalization along $M$ (cf. Chapter 2).

Then the problem is: 
evaluate $\muS(\RHOM_{\mcal{D}_X}(\mcal{M},\mcal{B}_M))$ or
\linebreak[4]
$\muS(\RHOM_{\mcal{D}_X}(\mcal{M},\mcal{C}_M))$. 
Taking $F = \RHOM_{\mcal{D}_X}(\mcal{M},\mcal{B}_M)$ 
this is a particular case of the following problem: 
given $F \in \Ob(\Drv^+(X))$, and $M$ a real submanifold of $X$, 
calculate $\muS (\RG_M(F))$ or $\muS(\mu_M(F))$.

As we see, in this new formulation, we may forget that $X$ is a
complexmanifold, and we do not study separately 
the $\mcal{D}_X$-module $\mcal{M}$ from one side 
and the sheaf $\mcal{O}_X$ on the other side. 
On the contrary 
we work with the whole complex of solutions of $\mcal{M}$ in $\mcal{O}_X$. 
The only information that we keep is 
the geometrical data of the characteristic variety, 
which is interpreted in terms of micro-support (in fact
we shall prove in Chapter 10 that the inclusion in (0.1) is an
equality). 

Now let us come back to the subject of this paper.

We study in Chapter 4 and 5 
the functorial properties of the micro-support: 
behavior under direct or inverse images, functors
$\RHOM(\boldsymbol{\cdot},\boldsymbol{\cdot})$, $\boldsymbol{\cdot}\otimes^{L}\boldsymbol{\cdot}$, 
specialization, Fourier-Sato transformation, 
microlocalization (the construction of 
these functors are recalled in Chapter 1 and 2). 
But in order to manipulate micro-supports, 
the definition (0.1) given above is too much 
of a local one and one has to replace it 
by a more global criterium. 
This is achieved in Chapter 3, 
using a Mittag-Leffler procedure for sheaves (Theorem 1.4.3).

The calculations of Chapters 4 and 5 are 
all essentially based on
the computation of the micro-support of the direct image 
of a sheaf by an open immersion. 
In this case the procedure, and the result,
are very similar to those encountered in the theory 
of micro-hyperbolic systems (cf. our work [2] ), 
and the set we obtain is defined as a ``normal cone'', 
(Theorem 4.3.1.). 
The preliminaries concerning such normal cones 
are presented in Chapter 1, \S2.

The notion of micro-support allows us to work with 
sheaves ``micro-locally'', that is, locally in \(T^\ast X\). 
In fact for a subset \(\Omega\) of \(T^\ast X\), 
we introduce the triangulated category \(\Drv^+(X;\Omega)\) 
obtained from \(\Drv^+(X)\) by localization on $\Omega$, 
that is, by regarding as the zero object 
the sheaves whose micro-support do not meet $\Omega$. 
A useful tool in the microlocal study of sheaves, 
is the ``$G$-topology''. 
The idea of the $G$-topology is the following: 
in order to work microlocally, let us say on $X\times U$ 
where $X$ is open in a real vector space $E$ 
and $U$ is an open cone in the dual space $E^\ast$, 
the usual topology on $X$ is too strong, 
and may be weakened by introducing a closed convex proper 
cone $G$ in $E$ whose polar set $G^\circ$ is contained 
in $(-U)\cup\{0\}$, and by considering only those open 
sets $\Omega$ of $X$ such that:
\[
    \Omega = (\Omega+G)\cap X.  \tag*{(0.2)}\label{eq:Gtop}
\]
Let $X_G$ be the space $X$ endowed with 
the $G$-topology (i.e.: the open subsets of $X_G$ satisfy \ref{eq:Gtop}) 
and let be the continuous map $X \to X_G$. 
Let $\Omega_0$ and $\Omega_1$ be two $G$-open subsets 
of $E$ such that \(\Omega_0\subset\Omega_1\), \(\Omega_1\setminus\Omega_0\Subset X\). 
Then one proves (cf.\ Theorem 3.2.2.) 
that for $F \in \Ob(\Drv^{+}(X))$ one has the isomorphism:
\[
    \Phi_{G}^{-1}\rr\Phi_{G^\ast}
    \rr\Gamma_{\Omega_1\setminus\Omega_0}(F)
    \overset{\sim}{\longrightarrow}
    F
    \quad\text{in}\quad
    \Drv^{+}(X:\Int(\Omega_1\setminus\Omega_0)\times \Int(-G^{\circ}))
\]
and moreover:
\[
    \muS(\Phi_{G}^{-1}\rr\Phi_{G^\ast}
    \rr\Gamma_{\Omega_1\setminus\Omega_0}(F))
    \subset
    X\times \Int(-G^{\circ})
\]