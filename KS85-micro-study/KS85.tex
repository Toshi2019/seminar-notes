
%Don't forget to delete
%showkeys
%overfullrule
%\date \ber \er \cmt

%------------------------
\documentclass[11pt, letterpaper, draft, dvipdfmx]{book}


%usepackage
%------------------------
\usepackage{amsmath}
\usepackage{amsthm}
%\usepackage[psamsfonts]{amssymb}
\usepackage{color}
\usepackage{ascmac}
\usepackage{amsfonts}
\usepackage{mathrsfs}
\usepackage{mathtools}
\usepackage{amssymb}
\usepackage{graphicx}
\usepackage{fancybox}
%\usepackage{enumerate}
\usepackage{enumitem}
\usepackage{verbatim}
\usepackage{subfigure}
\usepackage{proof}
\usepackage{listings}
\usepackage{otf}
\usepackage{algorithm}
\usepackage{algorithmic}
\usepackage{tikz}
\usetikzlibrary{cd}
\usepackage[all]{xy}
\usepackage{amscd}

\usepackage{pb-diagram}

\usepackage[dvipdfmx]{hyperref}
\usepackage{xcolor}
\definecolor{darkgreen}{rgb}{0,0.45,0} 
\definecolor{darkred}{rgb}{0.75,0,0}
\definecolor{darkblue}{rgb}{0,0,0.6} 
\hypersetup{
    colorlinks=true,
    citecolor=darkgreen,
    linkcolor=darkred,
    urlcolor=darkblue,
}
\usepackage{pxjahyper}

\usepackage{enumitem}

\usepackage{bbm}

% ================================
% パッケージを追加する場合のスペース 
\usepackage{latexsym}
\usepackage{wrapfig}
\usepackage{layout}
\usepackage{url}

\usepackage{okumacro}
%\usepackage{endnotes}
%\usepackage[french]{babel}

%=================================


% --------------------------
% theoremstyle
% --------------------------
\theoremstyle{definition}


% --------------------------
% newtheoem
% --------------------------

% 日本語で定理, 命題, 証明などを番号付きで用いるためのコマンドです. 
% If you want to use theorem environment in Japanece, 
% you can use these code. 
% Attention!
% All theorem enivironment numbers depend on 
% only section numbers.
\newtheorem{Axiom}{公理}[section]
\newtheorem{Definition}[Axiom]{定義}
\newtheorem{Theorem}[Axiom]{定理}
\newtheorem{Proposition}[Axiom]{命題}
\newtheorem{Lemma}[Axiom]{補題}
\newtheorem{Corollary}[Axiom]{系}
\newtheorem{Example}[Axiom]{例}
\newtheorem{Claim}[Axiom]{主張}
\newtheorem{Property}[Axiom]{性質}
\newtheorem{Attention}[Axiom]{注意}
\newtheorem{Question}[Axiom]{問}
\newtheorem{Problem}[Axiom]{問題}
\newtheorem{Consideration}[Axiom]{考察}
\newtheorem{Alert}[Axiom]{警告}
\newtheorem{Fact}[Axiom]{事実}


% 日本語で定理, 命題, 証明などを番号なしで用いるためのコマンドです. 
% If you want to use theorem environment with no number in Japanese, You can use these code.
\newtheorem*{Axiom*}{公理}
\newtheorem*{Definition*}{定義}
\newtheorem*{Theorem*}{定理}
\newtheorem*{Proposition*}{命題}
\newtheorem*{Lemma*}{補題}
\newtheorem*{Example*}{例}
\newtheorem*{Corollary*}{系}
\newtheorem*{Claim*}{主張}
\newtheorem*{Property*}{性質}
\newtheorem*{Attention*}{注意}
\newtheorem*{Question*}{問}
\newtheorem*{Problem*}{問題}
\newtheorem*{Consideration*}{考察}
\newtheorem*{Alert*}{警告}
\newtheorem{Fact*}{事実}


% 英語で定理, 命題, 証明などを番号付きで用いるためのコマンドです. 
% If you want to use theorem environment in English, You can use these code.
%all theorem enivironment number depend on only section number.
\newtheorem{Axiom+}{Axiom}[section]
\newtheorem{Definition+}[Axiom+]{Definition}
\newtheorem{Theorem+}[Axiom+]{Theorem}
\newtheorem{Proposition+}[Axiom+]{Proposition}
\newtheorem{Lemma+}[Axiom+]{Lemma}
\newtheorem{Example+}[Axiom+]{Example}
\newtheorem{Corollary+}[Axiom+]{Corollary}
\newtheorem{Claim+}[Axiom+]{Claim}
\newtheorem{Property+}[Axiom+]{Property}
\newtheorem{Attention+}[Axiom+]{Attention}
\newtheorem{Question+}[Axiom+]{Question}
\newtheorem{Problem+}[Axiom+]{Problem}
\newtheorem{Consideration+}[Axiom+]{Consideration}
\newtheorem{Alert+}{Alert}
\newtheorem{Fact+}[Axiom+]{Fact}
\newtheorem{Remark+}[Axiom+]{Remark}

% ----------------------------
% commmand
% ----------------------------
% 執筆に便利なコマンド集です. 
% コマンドを追加する場合は下のスペースへ. 

% 集合の記号 (黒板文字)
\newcommand{\NN}{\mathbb{N}}
\newcommand{\ZZ}{\mathbb{Z}}
\newcommand{\QQ}{\mathbb{Q}}
\newcommand{\RR}{\mathbb{R}}
\newcommand{\CC}{\mathbb{C}}
\newcommand{\PP}{\mathbb{P}}
\newcommand{\KK}{\mathbb{K}}


% 集合の記号 (太文字)
\newcommand{\nn}{\mathbf{N}}
\newcommand{\zz}{\mathbf{Z}}
\newcommand{\qq}{\mathbf{Q}}
\newcommand{\rr}{\mathbf{R}}
\newcommand{\cc}{\mathbf{C}}
\newcommand{\pp}{\mathbf{P}}
\newcommand{\kk}{\mathbf{K}}

% 特殊な写像の記号
\newcommand{\ev}{\mathop{\mathrm{ev}}\nolimits} % 値写像
\newcommand{\pr}{\mathop{\mathrm{pr}}\nolimits} % 射影

% スクリプト体にするコマンド
%   例えば {\mcal C} のように用いる
\newcommand{\mcal}{\mathcal}

% 花文字にするコマンド 
%   例えば {\h C} のように用いる
\newcommand{\h}{\mathscr}

% ヒルベルト空間などの記号
\newcommand{\F}{\mcal{F}}
\newcommand{\X}{\mcal{X}}
\newcommand{\Y}{\mcal{Y}}
\newcommand{\HH}{\mcal{H}}
\newcommand{\RKHS}{\Hil_{k}}
\newcommand{\Loss}{\mcal{L}_{D}}
\newcommand{\MLsp}{(\X, \Y, D, \Hil, \Loss)}

% 偏微分作用素の記号
\newcommand{\p}{\partial}

% 角カッコの記号 (内積は下にマクロがあります)
\newcommand{\lan}{\langle}
\newcommand{\ran}{\rangle}



% 圏の記号など
\newcommand{\Set}{{\bf Set}}
\newcommand{\Vect}{{\bf Vect}}
\newcommand{\FDVect}{{\bf FDVect}}
\newcommand{\Ring}{{\bf Ring}}
\newcommand{\Ab}{{\bf Ab}}
\newcommand{\Mod}{\mathop{\mathrm{Mod}}\nolimits}
\newcommand{\CGA}{{\bf CGA}}
\newcommand{\GVect}{{\bf GVect}}
\newcommand{\Lie}{{\bf Lie}}
\newcommand{\dLie}{{\bf Liec}}



% 射の集合など
\newcommand{\Map}{\mathop{\mathrm{Map}}\nolimits}
\newcommand{\Hom}{\mathop{\mathrm{Hom}}\nolimits}
\newcommand{\End}{\mathop{\mathrm{End}}\nolimits}
\newcommand{\Aut}{\mathop{\mathrm{Aut}}\nolimits}
\newcommand{\Mor}{\mathop{\mathrm{Mor}}\nolimits}

% その他便利なコマンド
\newcommand{\dip}{\displaystyle} % 本文中で数式モード
\newcommand{\e}{\varepsilon} % イプシロン
\newcommand{\dl}{\delta} % デルタ
\newcommand{\pphi}{\varphi} % ファイ
\newcommand{\ti}{\tilde} % チルダ
\newcommand{\pal}{\parallel} % 平行
\newcommand{\op}{{\rm op}} % 双対を取る記号
\newcommand{\lcm}{\mathop{\mathrm{lcm}}\nolimits} % 最小公倍数の記号
\newcommand{\Probsp}{(\Omega, \F, \P)} 
\newcommand{\argmax}{\mathop{\rm arg~max}\limits}
\newcommand{\argmin}{\mathop{\rm arg~min}\limits}





% ================================
% コマンドを追加する場合のスペース 
\newcommand{\UU}{\mcal{U}}
\newcommand{\OO}{\mcal{O}}
\newcommand{\emp}{\varnothing}
\newcommand{\ceq}{\coloneqq}
\newcommand{\sbs}{\subset}
\newcommand{\mapres}[2]{\left. #1 \right|_{#2}}
\newcommand{\ded}{\hfill $\blacksquare$}
\newcommand{\id}{\mathrm{id}}
\newcommand{\isom}{\overset{\sim}{\longrightarrow}}
\newcommand{\tTop}{\textsf{Top}}
\newcommand{\pfb}{\textbf{証明}}
\newcommand{\Int}{\mathop{\mathrm{Int}}\nolimits} % 内部
\newcommand{\Ch}{\mathop{\mathrm{char}}\nolimits}
\newcommand{\HOM}{\mathop{\mathscr{H}\hspace{-2pt}om}\nolimits}
\newcommand{\RHOM}{\mathop{\mathrm{R}\hspace{-1.5pt}\HOM}\nolimits}
\newcommand{\muS}{\mathop{\mathrm{SS}}\nolimits}
\newcommand{\RG}{\mathop{\mathrm{R}\hspace{-0pt}\Gamma}\nolimits}
\newcommand{\Ob}{\mathop{\mathrm{Ob}}\nolimits}


% 自前の定理環境
%   https://mathlandscape.com/latex-amsthm/
% を参考にした
\newtheoremstyle{mystyle}%   % スタイル名
    {5pt}%                   % 上部スペース
    {5pt}%                   % 下部スペース
    {}%              % 本文フォント
    {}%                  % 1行目のインデント量
    {\bfseries}%                      % 見出しフォント
    {.}%                     % 見出し後の句読点
    {12pt}%                     % 見出し後のスペース
    {\thmname{#1}\thmnumber{ #2 }\thmnote{{\normalfont (#3)}}}% % 見出しの書式

\theoremstyle{mystyle}
\newtheorem{AXM}{公理}[section]
\newtheorem{DFN}[Axiom]{Definition}
\newtheorem{THM}[Axiom]{定理}
\newtheorem*{THM*}{定理}
\newtheorem{PRP}[Axiom]{命題}
\newtheorem{LMM}[Axiom]{補題}
\newtheorem{CRL}[Axiom]{系}
\newtheorem{EG}[Axiom]{例}

%\newtheorem{}{Axiom}[]
\numberwithin{equation}{section} % 式番号を「(3.5)」のように印刷

\newcommand{\MM}{\mcal{M}}
\newcommand{\bk}{\mathbf{k}}

% =================================


% ---------------------------
% new definition macro
% ---------------------------
% 便利なマクロ集です

% 内積のマクロ
%   例えば \inner<\pphi | \psi> のように用いる
\def\inner<#1>{\langle #1 \rangle}

% ================================
% マクロを追加する場合のスペース 

%=================================





% ----------------------------
% documenet 
% ----------------------------
% 以下, 本文の執筆スペースです. 
% Your main code must be written between 
% begin document and end document.
% ---------------------------




\begin{document}

\title{Microlocal Study of Sheaves}
%\author{Pierre Schapira}
\date{}
\maketitle

\chapter*{Introduction}
Let $X$ be a real manifold, 
$F$ a sheaf on $X$, or better an objet of $D^+(X)$, 
the derived category of the category of complexes 
bounded from below of sheaves on $X$. 
Let $TX$ be the cotangent bundle to $X$. 
We associate to $F$ a closed conic subset of $TX$, 
denoted $SS (F)$, the ``micro-support of $F$'', as follows:

\begin{DFN}
    Let $p=(x_0,\xi_0)\in T^\ast X$. 
    Then $p\notin SS(F)$ if and only if 
    there exists an open neighborhood $U$ of $p$ in $TX$ 
    such that for any $(x_1,\xi_1) \in U$, 
    any real $C^1$-function $\phi$ on $X$, 
    with $\phi(x_1)=0$, $d\phi(x_1)=\xi_1$, 
    we have: $(\rr \Gamma_{x;\phi(x)\geqq 0} (F))_{x_1} =0$.
\end{DFN}

In other words 
the micro-support of $F$ describes 
the set of codirections of $X$ where $F$, and its cohomology, 
``do not propagate''. 
This definition is motivated by the following situation. 

Assume $X$ is a complex manifold, 
and let $\mcal{M}$ be a coherent module over 
the Ring $\mcal{D}_X$ of (holomorphic, finire order) 
differential operators. 
Let $\Ch(\mcal{M})$ be the characteristic variety 
of $\mcal{M}$ in $T^\ast X$. 
Then we can interpret a well-known result of Zerner [1], 
Bony-Schapira [1], Kashiwara [5], through the formula:
\begin{equation}
    \muS(\RHOM_{\mcal{D}_X}(\mcal{M},\mcal{O}_X))\subset\Ch(\mcal{M}).
\end{equation}
A natural problem then arising 
in the theory of (micro-)differential equations, 
is to evaluate the set of codirections of propagation for
the sheaf of hyperfunction or microfunction solutions 
of $\mcal{M}$ (or more generally 
of a system of micro-differential equations). 
To be more precise, 
let $M$ be a real analytic manifold of dimension $n$, 
$X$ a complexification of $M$. 
Recall that the sheaf $\mcal{B}_M$ (resp. $\mcal{C}_M$) 
of Sato's hyperfunctions on $M$ (resp. 
Sato's microfunctions on $T^{\ast}_{M}X$, 
the conormal bundle to $M$ in $X$) is defined by:
\begin{align*}
    \mcal{B}_M &= \RG_M(\mcal{O}_X)\otimes\omega_M[n]\\
    (\text{resp. }
    \mcal{C}_M &= \mu_M(\mcal{O}_X)\otimes\omega_M[n])
\end{align*}
where $\omega_M$ is the orientation sheaf on $M$, 
$[n]$ means the $n$-shift in $D^+(X)$, 
and $\mu(\bullet)$ is the functor of 
Sato's microlocalization along $M$ (cf. Chapter 2).

Then the problem is: 
evaluate $\muS(\RHOM_{\mcal{D}_X}(\mcal{M},\mcal{B}_M))$ or
$\muS(\RHOM_{\mcal{D}_X}(\mcal{M},\mcal{C}_M))$. 
Taking $F = \RHOM_{\mcal{D}_X}(\mcal{M},\mcal{B}_M)$ 
this is a particular case of the following problem: 
given $F \in \Ob(D^+(X))$, and $M$ a real submanifold of $X$, 
calculate $\muS (\RG_M(F))$ or $\muS(\mu_M(F))$.

As we see, in this new formulation, we may forget that $X$ is a
complexmanifold, and we do not study separately 
the $\mcal{D}_X$-module $\mcal{M}$ from one side 
and the sheaf $\mcal{O}_X$ on the other side. 
On the contrary 
we work with the whole complex of solutions of $\mcal{M}$ in $\mcal{O}_X$. 
The only information that we keep is 
the geometrical data of the characteristic variety, 
which is interpreted in terms of micro-support (in fact
we shall prove in Chapter 10 that the inclusion in (0.1) is an
equality). 

Now let us come back to the subject of this paper.

We study in Chapter 4 and 5 
the functorial properties of the micro-support: 
behavior under direct or inverse images, functors
$\RHOM(\bullet,\bullet)$, $\bullet\otimes^{L}\bullet$, 
specialization, Fourier-Sato transformation, 
microlocalization (the construction of 
these functors are recalled in Chapter 1 and 2). 
But in order to manipulate micro-supports, 
the definition (0.1) given above is too much 
of a local one and one has to replace it 
by a more global criterium. 
This is achieved in Chapter 3, 
using a Mittag-Leffler procedure for sheaves (Theorem 1.4.3).
























\begin{thebibliography}{15}
    \bibitem[Arn67]{Arn67} Vladimir I. Arnold, 
    \textit{On a characteristic class entering into conditions of quantization}, 
    Funkcional. Anal. i Prilozen (1967), 1–14. in Russian.
    \bibitem[Gab81]{Gab81}
    Ofer Gabber, \textit{The integrability of the characteristic variety}, 
    Amer. Journ. Math. 103 (1981), 445-–468.
    \bibitem[GKS12]{GKS12} 
    Stéphane Guillermou, Masaki Kashiwara and Pierre Schapira, 
    \textit{Sheaf quantization of Hamiltonian isotopies and applications to nondisplaceability problems}, 
    Duke Math. J. 161(2): 201--245 (2012).
    \bibitem[KS90]{KS90} Masaki Kashiwara and Pierre Schapira, 
    \textit{Sheaves on manifolds}, Grundlehren der Mathematischen Wissenschaften, 
    vol. 292, Springer-Verlag, Berlin, 1990.
    \bibitem[Ler76]{Ler76} 
    Jean Leray, \textit{Analyse Lagrangienne et m\'ecanique quantique}, 
    Coll\`ege de France, 1976.
    \bibitem[Mas65]{Mas65} Viktor P. Maslov, 
    \textit{Theory of perturbations and asymptotic methods}, 
    Moskow Gos. Univ., 1965.
    [邦訳] マスロフ, 摂動論と漸近的方法, 岩波書店, 1976年.
    \bibitem[Sch21]{Sch21} 
    Pierre Schapira, 
    \textit{Microlocal analysis and beyond}, 
    New spaces in Mathematics, edited by Mathieu Anel 
    and Gabriel Catren, 
    Cambridge University Press, 2021, pp. 117--152.
    \bibitem[SKK73]{SKK73}
    Mikio Sato, Takahiro Kawai, and Masaki Kashiwara, 
    \textit{Microfunctions and pseudo-differential equations}, 
    Hyperfunctions and pseudo-differential equations 
    (Proc. Conf., Katata, 1971; dedicated to the memory of 
    Andr\'e Martineau), 
    Springer, Berlin, 1973, pp. 265–-529. 
    Lecture Notes in Math., Vol. 287.

\end{thebibliography}
%\bibliographystyle{junsrt}
%\bibliography{ref}

\end{document}


