%================================================
%    層理論まとめノート
%================================================

% -----------------------
% preamble
% -----------------------
% ここから本文 (\begin{document}) までの
% ソースコードに変更を加えた場合は
% 編集者まで連絡してください. 
% Don't change preamble code yourself. 
% If you add something
% (usepackage, newtheorem, newcommand, renewcommand),
% please tell it 
% to the editor of institutional paper of RUMS.

% ------------------------
% documentclass
% ------------------------
\documentclass[11pt, a4paper, dvipdfmx, leqno, draft]{jsbook}

% ------------------------
% usepackage
% ------------------------
\usepackage{algorithm}
\usepackage{algorithmic}
\usepackage{amscd}
\usepackage{amsfonts}
\usepackage{amsmath}
\usepackage[psamsfonts]{amssymb}
\usepackage{amsthm}
\usepackage{ascmac}
\usepackage{bm}
\usepackage{color}
\usepackage{enumerate}
\usepackage{fancybox}
\usepackage[stable]{footmisc}
\usepackage{graphicx}
\usepackage{listings}
\usepackage{mathrsfs}
\usepackage{mathtools}
\usepackage{otf}
\usepackage{pifont}
\usepackage{proof}
\usepackage{subfigure}
\usepackage{tikz}
\usepackage{verbatim}
\usepackage[all]{xy}
\usepackage{url}
\usetikzlibrary{cd}



% ================================
% パッケージを追加する場合のスペース 
%\usepackage{calligra}
\usepackage[dvipdfmx]{hyperref}
\usepackage{xcolor}
\definecolor{darkgreen}{rgb}{0,0.45,0} 
\definecolor{darkred}{rgb}{0.75,0,0}
\definecolor{darkblue}{rgb}{0,0,0.6} 
\hypersetup{
    colorlinks=true,
    citecolor=darkgreen,
    linkcolor=darkred,
    urlcolor=darkblue,
}
\usepackage{pxjahyper}
\usepackage{layout}
%=================================


% --------------------------
% theoremstyle
% --------------------------
\theoremstyle{definition}

% --------------------------
% newtheoem
% --------------------------

% 日本語で定理, 命題, 証明などを番号付きで用いるためのコマンドです. 
% If you want to use theorem environment in Japanece, 
% you can use these code. 
% Attention!
% All theorem enivironment numbers depend on 
% only section numbers.
\newtheorem{Axiom}{公理}[section]
\newtheorem{Definition}[Axiom]{定義}
\newtheorem{Theorem}[Axiom]{定理}
\newtheorem{Proposition}[Axiom]{命題}
\newtheorem{Lemma}[Axiom]{補題}
\newtheorem{Corollary}[Axiom]{系}
\newtheorem{Example}[Axiom]{例}
\newtheorem{Claim}[Axiom]{主張}
\newtheorem{Property}[Axiom]{性質}
\newtheorem{Attention}[Axiom]{注意}
\newtheorem{Question}[Axiom]{問}
\newtheorem{Problem}[Axiom]{問題}
\newtheorem{Consideration}[Axiom]{考察}
\newtheorem{Alert}[Axiom]{警告}
\newtheorem{Fact}[Axiom]{事実}
\newtheorem{com}[Axiom]{コメント}


% 日本語で定理, 命題, 証明などを番号なしで用いるためのコマンドです. 
% If you want to use theorem environment with no number in Japanese, You can use these code.
\newtheorem*{Axiom*}{公理}
\newtheorem*{Definition*}{定義}
\newtheorem*{Theorem*}{定理}
\newtheorem*{Proposition*}{命題}
\newtheorem*{Lemma*}{補題}
\newtheorem*{Example*}{例}
\newtheorem*{Corollary*}{系}
\newtheorem*{Claim*}{主張}
\newtheorem*{Property*}{性質}
\newtheorem*{Attention*}{注意}
\newtheorem*{Question*}{問}
\newtheorem*{Problem*}{問題}
\newtheorem*{Consideration*}{考察}
\newtheorem*{Alert*}{警告}
\newtheorem*{Fact*}{事実}
\newtheorem*{com*}{コメント}



% 英語で定理, 命題, 証明などを番号付きで用いるためのコマンドです. 
% If you want to use theorem environment in English, You can use these code.
%all theorem enivironment number depend on only section number.
\newtheorem{Axiom+}{Axiom}[section]
\newtheorem{Definition+}[Axiom+]{Definition}
\newtheorem{Theorem+}[Axiom+]{Theorem}
\newtheorem{Proposition+}[Axiom+]{Proposition}
\newtheorem{Lemma+}[Axiom+]{Lemma}
\newtheorem{Example+}[Axiom+]{Example}
\newtheorem{Corollary+}[Axiom+]{Corollary}
\newtheorem{Claim+}[Axiom+]{Claim}
\newtheorem{Property+}[Axiom+]{Property}
\newtheorem{Attention+}[Axiom+]{Attention}
\newtheorem{Question+}[Axiom+]{Question}
\newtheorem{Problem+}[Axiom+]{Problem}
\newtheorem{Consideration+}[Axiom+]{Consideration}
\newtheorem{Alert+}{Alert}
\newtheorem{Fact+}[Axiom+]{Fact}
\newtheorem{Remark+}[Axiom+]{Remark}

% ----------------------------
% commmand
% ----------------------------
% 執筆に便利なコマンド集です. 
% コマンドを追加する場合は下のスペースへ. 

% 集合の記号 (黒板文字)
\newcommand{\NN}{\mathbb{N}}
\newcommand{\ZZ}{\mathbb{Z}}
\newcommand{\QQ}{\mathbb{Q}}
\newcommand{\RR}{\mathbb{R}}
\newcommand{\CC}{\mathbb{C}}
\newcommand{\PP}{\mathbb{P}}
\newcommand{\KK}{\mathbb{K}}


% 集合の記号 (太文字)
\newcommand{\nn}{\mathbf{N}}
\newcommand{\zz}{\mathbf{Z}}
\newcommand{\qq}{\mathbf{Q}}
\newcommand{\rr}{\mathbf{R}}
\newcommand{\cc}{\mathbf{C}}
\newcommand{\pp}{\mathbf{P}}
\newcommand{\kk}{\mathbf{K}}

% 特殊な写像の記号
\newcommand{\ev}{\mathop{\mathrm{ev}}\nolimits} % 値写像
\newcommand{\pr}{\mathop{\mathrm{pr}}\nolimits} % 射影

% スクリプト体にするコマンド
%   例えば {\mcal C} のように用いる
\newcommand{\mcal}{\mathcal}

% 花文字にするコマンド 
%   例えば {\h C} のように用いる
\newcommand{\h}{\mathscr}

% ヒルベルト空間などの記号
\newcommand{\F}{\mcal{F}}
\newcommand{\X}{\mcal{X}}
\newcommand{\Y}{\mcal{Y}}
\newcommand{\Hil}{\mcal{H}}
\newcommand{\RKHS}{\Hil_{k}}
\newcommand{\Loss}{\mcal{L}_{D}}
\newcommand{\MLsp}{(\X, \Y, D, \Hil, \Loss)}

% 偏微分作用素の記号
\newcommand{\p}{\partial}

% 角カッコの記号 (内積は下にマクロがあります)
\newcommand{\lan}{\langle}
\newcommand{\ran}{\rangle}



% 圏の記号など
\newcommand{\Set}{{\bf Set}}
\newcommand{\Vect}{{\bf Vect}}
\newcommand{\FDVect}{{\bf FDVect}}
\newcommand{\Mod}{\mathop{\mathrm{Mod}}\nolimits}
\newcommand{\CGA}{{\bf CGA}}
\newcommand{\GVect}{{\bf GVect}}
\newcommand{\Lie}{{\bf Lie}}
\newcommand{\dLie}{{\bf Liec}}



% 射の集合など
\newcommand{\Map}{\mathop{\mathrm{Map}}\nolimits}
\newcommand{\Hom}{\mathop{\mathrm{Hom}}\nolimits}
\newcommand{\End}{\mathop{\mathrm{End}}\nolimits}
\newcommand{\Aut}{\mathop{\mathrm{Aut}}\nolimits}
\newcommand{\Mor}{\mathop{\mathrm{Mor}}\nolimits}

% その他便利なコマンド
\newcommand{\dip}{\displaystyle} % 本文中で数式モード
\newcommand{\e}{\varepsilon} % イプシロン
\newcommand{\dl}{\delta} % デルタ
\newcommand{\pphi}{\varphi} % ファイ
\newcommand{\ti}{\tilde} % チルダ
\newcommand{\pal}{\parallel} % 平行
\newcommand{\op}{{\rm op}} % 双対を取る記号
\newcommand{\lcm}{\mathop{\mathrm{lcm}}\nolimits} % 最小公倍数の記号
\newcommand{\Probsp}{(\Omega, \F, \P)} 
\newcommand{\argmax}{\mathop{\rm arg~max}\limits}
\newcommand{\argmin}{\mathop{\rm arg~min}\limits}





% ================================
% コマンドを追加する場合のスペース 
\renewcommand\proofname{\bf 証明} % 証明
\numberwithin{equation}{subsection}
\newcommand{\cTop}{\textsf{Top}}
%\newcommand{\cOpen}{\textsf{Open}}
\newcommand{\Op}{\mathop{\textsf{Op}}\nolimits}
\newcommand{\Ob}{\mathop{\textrm{Ob}}\nolimits}
\newcommand{\id}{\mathop{\mathrm{id}}\nolimits}
\newcommand{\pt}{\mathop{\mathrm{pt}}\nolimits}
\newcommand{\res}{\mathop{\rho}\nolimits}
\newcommand{\Ring}{\mathop{\textsf{Ring}}\nolimits}
\newcommand{\cAb}{\mathop{\textsf{Ab}}\nolimits}
\newcommand{\Sh}{\mathop{\textsf{Sh}}\nolimits}
\newcommand{\PSh}{\mathop{\textsf{PSh}}\nolimits}
\newcommand{\Ker}{\mathop{\mathrm{Ker}}\nolimits}
\newcommand{\im}{\mathop{\mathrm{Im}}\nolimits}
\newcommand{\Coker}{\mathop{\mathrm{Coker}}\nolimits}
\newcommand{\Coim}{\mathop{\mathrm{Coim}}\nolimits}
\newcommand{\Ht}{\mathop{\mathrm{Ht}}\nolimits}
\newcommand{\colim}{\mathop{\mathrm{colim}}}
\newcommand{\ori}{\mathop{\textsf{or}}\nolimits}
\newcommand{\supp}{\mathop{\mathrm{supp}}\nolimits}


\newcommand{\cat}{\mathcal{C}}%一般の圏の記号
\newcommand{\Comp}{\mathop{\mathrm{C}}\nolimits}%複体の圏
\newcommand{\Komp}{\mathop{\mathrm{K}}\nolimits}%複体のホモトピー圏
\newcommand{\Domp}{\mathop{\mathrm{D}}\nolimits}%複体のホモトピー圏
\newcommand{\CCat}{\Comp(\cat)}%圏Cの複体の圏
\newcommand{\KCat}{\Komp(\cat)}%圏Cの複体のホモトピー圏
\newcommand{\DCat}{\Domp(\cat)}%圏Cの複体のホモトピー圏
\newcommand{\HOM}{\mathop{\mathscr{H}\hspace{-2pt}om}\nolimits}%内部Hom
\newcommand{\RHOM}{\mathop{\mathrm{R}\hspace{-1.5pt}\HOM}\nolimits}
\newcommand{\muS}{\mathop{\mathrm{SS}}\nolimits}
\newcommand{\RG}{\mathop{\mathrm{R}\hspace{-0pt}\Gamma}\nolimits}

\newcommand{\limf}{\mathop{\text{``}\hspace{-0.7pt}\varinjlim\hspace{-1.5pt}\text{''}}}
\newcommand{\sumf}{\mathop{\text{``}\hspace{-0.7pt}\bigoplus\hspace{-1.5pt}\text{''}}}

\newcommand{\hh}{\mathop{\mathrm{h}}\nolimits}
\newcommand{\Ind}{\mathop{\mathrm{Ind}}}


%筆記体
\newcommand{\cA}{\mcal{A}}
\newcommand{\cB}{\mcal{B}}
\newcommand{\cC}{\mcal{C}}
\newcommand{\cD}{\mcal{D}}
\newcommand{\cE}{\mcal{E}}
\newcommand{\cF}{\mcal{F}}
\newcommand{\cG}{\mcal{G}}
\newcommand{\cH}{\mcal{H}}
\newcommand{\cI}{\mcal{I}}
\newcommand{\cJ}{\mcal{J}}
\newcommand{\cK}{\mcal{K}}
\newcommand{\cL}{\mcal{L}}
\newcommand{\cM}{\mcal{M}}
\newcommand{\cN}{\mcal{N}}
\newcommand{\cO}{\mcal{O}}
\newcommand{\cP}{\mcal{P}}
\newcommand{\cQ}{\mcal{Q}}
\newcommand{\cR}{\mcal{R}}
\newcommand{\cS}{\mcal{S}}
\newcommand{\cT}{\mcal{T}}
\newcommand{\cU}{\mcal{U}}
\newcommand{\cV}{\mcal{V}}
\newcommand{\cW}{\mcal{W}}
\newcommand{\cX}{\mcal{X}}
\newcommand{\cY}{\mcal{Y}}
\newcommand{\cZ}{\mcal{Z}}

%スクリプト体
\newcommand{\scA}{\mathscr{A}}
\newcommand{\scB}{\mathscr{B}}
\newcommand{\scC}{\mathscr{C}}
\newcommand{\scD}{\mathscr{D}}
\newcommand{\scE}{\mathscr{E}}
\newcommand{\scF}{\mathscr{F}}
\newcommand{\scG}{\mathscr{G}}
\newcommand{\scH}{\mathscr{H}}
\newcommand{\scI}{\mathscr{I}}
\newcommand{\scJ}{\mathscr{J}}
\newcommand{\scK}{\mathscr{K}}
\newcommand{\scL}{\mathscr{L}}
\newcommand{\scM}{\mathscr{M}}
\newcommand{\scN}{\mathscr{N}}
\newcommand{\scO}{\mathscr{O}}
\newcommand{\scP}{\mathscr{P}}
\newcommand{\scQ}{\mathscr{Q}}
\newcommand{\scR}{\mathscr{R}}
\newcommand{\scS}{\mathscr{S}}
\newcommand{\scT}{\mathscr{T}}
\newcommand{\scU}{\mathscr{U}}
\newcommand{\scV}{\mathscr{V}}
\newcommand{\scW}{\mathscr{W}}
\newcommand{\scX}{\mathscr{X}}
\newcommand{\scY}{\mathscr{Y}}
\newcommand{\scZ}{\mathscr{Z}}


\newcommand{\ibA}{\mathop{\text{\textit{\textbf{A}}}}}
\newcommand{\ibB}{\mathop{\text{\textit{\textbf{B}}}}}
\newcommand{\ibC}{\mathop{\text{\textit{\textbf{C}}}}}
\newcommand{\ibD}{\mathop{\text{\textit{\textbf{D}}}}}
\newcommand{\ibE}{\mathop{\text{\textit{\textbf{E}}}}}
\newcommand{\ibF}{\mathop{\text{\textit{\textbf{F}}}}}
\newcommand{\ibG}{\mathop{\text{\textit{\textbf{G}}}}}
\newcommand{\ibH}{\mathop{\text{\textit{\textbf{H}}}}}
\newcommand{\ibI}{\mathop{\text{\textit{\textbf{I}}}}}
\newcommand{\ibJ}{\mathop{\text{\textit{\textbf{J}}}}}
\newcommand{\ibK}{\mathop{\text{\textit{\textbf{K}}}}}
\newcommand{\ibL}{\mathop{\text{\textit{\textbf{L}}}}}
\newcommand{\ibM}{\mathop{\text{\textit{\textbf{M}}}}}
\newcommand{\ibN}{\mathop{\text{\textit{\textbf{N}}}}}
\newcommand{\ibO}{\mathop{\text{\textit{\textbf{O}}}}}
\newcommand{\ibP}{\mathop{\text{\textit{\textbf{P}}}}}
\newcommand{\ibQ}{\mathop{\text{\textit{\textbf{Q}}}}}
\newcommand{\ibR}{\mathop{\text{\textit{\textbf{R}}}}}
\newcommand{\ibS}{\mathop{\text{\textit{\textbf{S}}}}}
\newcommand{\ibT}{\mathop{\text{\textit{\textbf{T}}}}}
\newcommand{\ibU}{\mathop{\text{\textit{\textbf{U}}}}}
\newcommand{\ibV}{\mathop{\text{\textit{\textbf{V}}}}}
\newcommand{\ibW}{\mathop{\text{\textit{\textbf{W}}}}}
\newcommand{\ibX}{\mathop{\text{\textit{\textbf{X}}}}}
\newcommand{\ibY}{\mathop{\text{\textit{\textbf{Y}}}}}
\newcommand{\ibZ}{\mathop{\text{\textit{\textbf{Z}}}}}

\newcommand{\ibx}{\mathop{\text{\textit{\textbf{x}}}}}

\newcommand{\mres}[2][]{{{#1}\rvert}_{#2}}

% =================================



\usepackage{framed}
\definecolor{lightgray}{rgb}{0.75,0.75,0.75}
\renewenvironment{leftbar}{%
  \def\FrameCommand{\textcolor{lightgray}{\vrule width 4pt} \hspace{10pt}}% 
  \MakeFramed {\advance\hsize-\width \FrameRestore}}%
{\endMakeFramed}
\newenvironment{redleftbar}{%
  \def\FrameCommand{\textcolor{lightgray}{\vrule width 1pt} \hspace{10pt}}% 
  \MakeFramed {\advance\hsize-\width \FrameRestore}}%
 {\endMakeFramed}


%================================================
% 自前の定理環境
%   https://mathlandscape.com/latex-amsthm/
% を参考にした
\newtheoremstyle{mystyle}%   % スタイル名
    {5pt}%                   % 上部スペース
    {5pt}%                   % 下部スペース
    {}%              % 本文フォント
    {}%                  % 1行目のインデント量
    {\bfseries}%                      % 見出しフォント
    {.}%                     % 見出し後の句読点
    {12pt}%                     % 見出し後のスペース
    {\thmname{#1}\thmnumber{ #2}\thmnote{{\hspace{2pt}\normalfont (#3)}}}% % 見出しの書式

\theoremstyle{mystyle}
\newtheorem{AXM}{公理}[section]
\newtheorem{DFN}[Axiom]{定義}
\newtheorem{THM}[Axiom]{定理}
\newtheorem*{THM*}{定理}
\newtheorem{PRP}[Axiom]{命題}
\newtheorem{LMM}[Axiom]{補題}
\newtheorem{CRL}[Axiom]{系}
\newtheorem{EG}[Axiom]{例}
\newtheorem{CNV}[Axiom]{約束}


% 定理環境ここまで
%====================================================

%レイアウト
%\setlength{\oddsidemargin}{-10pt}
%\setlength{\evensidemargin}{10pt}

% ---------------------------
% new definition macro
% ---------------------------
% 便利なマクロ集です

% 内積のマクロ
%   例えば \inner<\pphi | \psi> のように用いる
\def\inner<#1>{\langle #1 \rangle}

% ================================
% マクロを追加する場合のスペース 
\def\ind<#1>{\mathop{\text{``}\hspace{-0.7pt}#1\limits\hspace{-1.5pt}\text{''}}}

%=================================





% ----------------------------
% documenet 
% ----------------------------
% 以下, 本文の執筆スペースです. 
% Your main code must be written between 
% begin document and end document.
% ---------------------------

\title{層理論まとめノート}
\author{Toshi2019}
\date{\today 更新}
\begin{document}
\maketitle
\frontmatter
\tableofcontents
\layout
\mainmatter
\chapter{層}



\begin{CNV}次のことは断りなく用いる.
    \begin{itemize}
        \item 環といえば,結合則をみたす積をもち単位元をもつ環とする.
        \item 位相空間$X$に対し,
        $X$上の環や加群の層をたんに$X$上の環とか$X$上の加群という.
        \item 層の記号は$\cF,\scF$のようにはせず,
        $F,G,H,\ldots$のようにローマン体とする
    \end{itemize}
\end{CNV}

\section{アーベル圏に値をとる層}

\subsection{前層と層の定義}

\begin{leftbar}
\begin{DFN}[前層]
    \(X\)を位相空間とし,\(\cat\)をアーベル圏とする.
    反変関手\(F\colon\Op(X)^{\op}\to\cat\)のことを
    \(\cat\)に値をとる\(X\)上の\textbf{前層} (presheaf) とよぶ.
\end{DFN}
\end{leftbar}

\begin{leftbar}
\begin{DFN}[層]
    \(X\)を位相空間とし,\(\cat\)をアーベル圏とする.
    \(F\)を\(\cat\)に値をとる\(X\)上の前層とする.
    \(F\)が
    任意の開集合\(U\)と\(U\)の開被覆\(
            \cU=\left(U_i\right)_{i\in I}
    \)に対し,次の列が完全になるとき,
    \(F\)は\textbf{層} (sheaf) であるという.
    \[
        0\to F(U)\overset{d^0}{\to} 
        \prod_{i\in I}F(U_i)\overset{d^1}{\to} 
        \prod_{i,j\in I}F(U_{i}\cap U_j).
    \]ただし,\(d^0\), \(d^1\)は\[
        d^0\colon s\mapsto\left(\mres[s]{U_i}\right)_i,
        \quad
        d^1\colon \left(s_i\right)_i 
        \mapsto\left(
            \mres[s_i]{U_{i}\cap U_j}-\mres[s_j]{U_{i}\cap U_j}
        \right)_{i,j}
    \]で定める.    
\end{DFN}
\end{leftbar}
















\subsection{アーベル層}
層の圏における完全列等の概念を明確化しておく.
アーベル圏$\cat$に値を取る位相空間$X$上の層を
\textbf{アーベル層}(abelian sheaf) ということがある.
アーベル層の圏$\Sh(X,\cat)$はアーベル圏になる.
すなわち,アーベル層の圏における核と余核が定まる.実際
アーベル層$F,G\in\Sh(X,\cat)$の間の射$\varphi\colon F\to G$に対し,
\begin{alignat*}{2}
    \Ker\varphi(U)&\coloneqq\Ker(\varphi_U), 
    &\quad 
    \Coker\varphi(U)&\coloneqq a_X\left(\Coker(\varphi_U)\right)
\end{alignat*}
として定めると,これらは$\Sh(X,\cat)$における核と余核になる.

これらを用いて,$\Sh(X,\cat)$における短完全列を次のように定める.
\begin{DFN}
    \begin{equation*}
        0\to F\overset{\varphi}{\to} 
        G\overset{\psi}{\to}H\to 0
        \quad\text{in $\Sh(X,\cat)$}
    \end{equation*}
    が完全であるとは,次の条件(i)--(iii)が成り立つことをいう.
    \begin{enumerate}
        \item [(i)]     $\Ker\phi\cong0$.
        \item [(ii)]    $\Ker\psi\cong\im\varphi$.
        \item [(iii)]   $\im\psi\cong H$.
    \end{enumerate}
\end{DFN}

$\cat=\cAb$のとき,$\Sh(X)=\Sh(X,\cAb)$とかく.

\begin{PRP}[層の同形は茎ごとの同形]\label{prp:stalk-iso}
    $\Sh(X)$の射$\phi\colon F\to G$が同形となるのは,
    各点$x\in X$に対し$\phi_x$が同形となるときである.
\end{PRP}
















\section{完全性}
諸々の操作の完全性についてまとめる.
$X$を位相空間とし,$R$を$X$上の環とする.
\subsection{茎}

\begin{PRP}[茎は完全]
    各点$x\in X$に対し,
    \begin{equation*}
        \begin{array}{rccc}
            {\boldsymbol{\cdot}}_x\colon&\Mod(R)&\longrightarrow&\Mod(R_x)\\
            &\rotatebox{90}{$\in$}&&\rotatebox{90}{$\in$}\\
            &F&\longmapsto&F_x        
        \end{array}
    \end{equation*}
    は完全関手である.    
\end{PRP}
\begin{proof}
    \begin{equation*}
        0\to F\overset{\varphi}{\to} 
        G\overset{\psi}{\to}H\to 0
        \quad\text{in $\Sh(X,\cat)$}
    \end{equation*}
    を完全列とする.
    命題\ref{prp:stalk-iso}から,各点$x\in X$に対し次の条件(i)--(iii)が成り立つ.
    \begin{enumerate}
        \item [(i)]     $(\Ker\phi)_x\cong0$.
        \item [(ii)]    $(\Ker\psi)_x\cong(\im\varphi)_x$.
        \item [(iii)]   $(\im\psi)_x\cong H_x$.
    \end{enumerate}
    これは次の条件($\mathrm{i}'$)--($\mathrm{iii}'$)と同値である.
    \begin{enumerate}
        \item [($\mathrm{i}'$)]     $\Ker\phi_x\cong0$.
        \item [($\mathrm{ii}'$)]    $\Ker\psi_x\cong\im\varphi_x$.
        \item [($\mathrm{iii}'$)]   $\im\psi_x\cong H_x$.
    \end{enumerate}
    これは
    \begin{equation*}
        0\to F_x\overset{\varphi_x}{\longrightarrow} 
        G_x\overset{\psi_x}{\longrightarrow}H_x\to 0
        \quad\text{in $\cat$}
    \end{equation*}
    が完全であることと同値である.    
\end{proof}
とくに,茎ごとの完全性から層の完全性も出てくるので,
層の完全列の概念が各点$x\in X$における完全列
\begin{equation*}
    0\to F_x\overset{\varphi_x}{\longrightarrow} 
    G_x\overset{\psi_x}{\longrightarrow}H_x\to 0
    \quad\text{exact in $\cat$}
\end{equation*}
にすり替わる.

\subsection{切断}
\begin{PRP}[切断は左完全]
    開集合$U\in\Op(X)$に対し,
    \begin{equation*}
        \begin{array}{rccc}
            \Gamma(U;{\boldsymbol{\cdot}})\colon&\Mod(R)&\longrightarrow&\Mod(R(U))\\
            &\rotatebox{90}{$\in$}&&\rotatebox{90}{$\in$}\\
            &F&\longmapsto&\Gamma(U;F)\coloneqq F(U)
        \end{array}
    \end{equation*}
    は左完全関手である.
    よってとくに$\Gamma(X;{\boldsymbol{\cdot}})$も左完全である.
\end{PRP}

\begin{EG}[右完全にならない例1]
    \(X=\cc\)とする.
    \(X\)上の層の射\(\partial_z\colon\cO_X\to\cO_X\)を,
    正則関数\(u(z)\)に対し導関数\(\dip\frac{du}{dz}(z)\)を
    対応させることで定める.このとき次は完全である.
    \[
        0\to  
        \cc_X\to
        \cO_X\overset{\partial_z}{\longrightarrow}
        \cO_X\to 0
        \quad
        \text{exact in \(\Mod(\cc_X)\)}
    \]
    しかし,開集合\(U=\cc-\{0\}\)上の切断を取った
    \[
        0\to  
        \cc_X(U)\to
        \cO_X(U)\overset{\partial_z(U)}{\longrightarrow}
        \cO_X(U)\to 0
        \quad
        \text{in \(\Mod(\cc)\)}
    \]
    は右完全ではない.
    実際,
    \(
        \dip\frac{1}{z}\in\cO_X(U)
    \)
    に対し,原始関数\(\log(z)\)は\(U\)上では正則ではない.
    よって,\(\partial(U)\)は全射ではない.
\end{EG}
\begin{EG}[右完全にならない例2]
    $E$を1次元複素トーラスとする.
    $E$上の層の完全列
    \begin{equation*}
        0\to \cO_E\overset{\iota}{\to} 
        \cO_E(P)\overset{\rho_P}{\to}\cc_P\to 0
        \quad\text{exact in $\Mod(\cc_E)$}
    \end{equation*}
    を考える.
    ただし,$\cO_E(P)$は一点$P\in E$における1位の因子から定まる層
    である.つまり,
    \begin{align*}
        \cO_E(P)(U)=\left\{\text{$P$でのみ高々1位の極を持つ$U$上の有理形関数}\right\}
    \end{align*}
    によって定まる層である.
    また$\cc_P$は一点$P$にのみ台をもつ摩天楼層である.
    このとき
    \begin{equation*}
        0\to \cO_E(E)\overset{\iota_E}{\longrightarrow} 
        \cO_E(P)(E)\overset{(r_P)_E}{\longrightarrow}\cc\to 0
        \quad\text{in $\Mod(\cc)$}
    \end{equation*}
    は完全ではない.実際この系列は
    \begin{equation*}
        0\to \cc\overset{\id_\cc}{\longrightarrow} 
        \cc\overset{0}{\longrightarrow}\cc\to 0
        \quad\text{in $\Mod(\cc)$}
    \end{equation*}
    となり,右側の0は全射にならない.
\end{EG}




\subsection{台を持つ切断}

\begin{PRP}[台を持つ切断は左完全]
    \(Z\)を\(X\)の局所閉集合とする.
    開集合$U\in\Op(X)$に対し,
    \[
        \begin{array}{rccc}
            \Gamma_{Z\cap U}(U;{\boldsymbol{\cdot}})\colon&\Mod(R)&\longrightarrow&\Mod(R(U))\\
            &\rotatebox{90}{$\in$}&&\rotatebox{90}{$\in$}\\
            &F&\longmapsto&\Gamma_{Z\cap U}(U;F)\coloneqq\left\{s\in F(U); \supp{s}\subset Z\cap U\right\}
        \end{array}
    \]
    を対応させる関手は左完全関手である.
    %よってとくに$\Gamma(X;{\boldsymbol{\cdot}})$も左完全である.
\end{PRP}



























\subsection{内部Hom}
\begin{PRP}[Homは左完全]
    $F,G$を$R$加群とする.
    \begin{equation*}
        \begin{array}{rccc}
            \Hom_R({\boldsymbol{\cdot}},{\boldsymbol{\cdot}})\colon
            &\Mod(R)^{\op}\times\Mod(R)
            &\longrightarrow&\Mod(\zz)\\
            &\rotatebox{90}{$\in$}&&\rotatebox{90}{$\in$}\\
            & (F,G)&\longmapsto&\Hom_R(F,G)
        \end{array}
    \end{equation*}
    は左完全な両側関手である.
\end{PRP}


\subsection{内部Hom}
\begin{PRP}[Homは左完全]
    $F,G$を$R$加群とする.
    \begin{equation*}
        \begin{array}{rccc}
            \HOM_R({\boldsymbol{\cdot}},{\boldsymbol{\cdot}})\colon
            &\Mod(R)^{\op}\times\Mod(R)
            &\longrightarrow&\Mod(\zz_X)\\
            &\rotatebox{90}{$\in$}&&\rotatebox{90}{$\in$}\\
            & (F,G)&\longmapsto&\HOM_R(F,G)
        \end{array}
    \end{equation*}
    は左完全な両側関手である.
\end{PRP}

\subsection{内部テンソル積}
層化が要る.

\begin{PRP}
    $F$を右$R$加群,$G$を左$R$加群とする.
    \begin{equation*}
        \begin{array}{rccc}
            {\boldsymbol{\cdot}}\otimes_R{\boldsymbol{\cdot}}\colon
            &\Mod(R^{\op})\times\Mod(R)
            &\longrightarrow&\Mod(\zz_X)\\
            &\rotatebox{90}{$\in$}&&\rotatebox{90}{$\in$}\\
            & (F,G)&\longmapsto&F\otimes_R G
        \end{array}
    \end{equation*}
    は右完全な両側関手である.
\end{PRP}

\subsection{帰納極限}
層化が要る.

帰納系$\alpha\colon I\to \Mod(R)$を考える.
極限を取る操作$\varinjlim\colon\Mod(R)^I\to\Mod(R)$を
\begin{equation*}
    \begin{array}{rccc}
        \varinjlim\colon&\Mod(R)^I&\longrightarrow&\Mod(R)\\
            &\rotatebox{90}{$\in$}&&\rotatebox{90}{$\in$}\\
            &\alpha=(F_i)_i &\longmapsto&\varinjlim\limits_{i\in I}F_i
    \end{array}
\end{equation*}
のようにかく.

\begin{PRP}
    \begin{equation*}
        \begin{array}{rccc}
            \varinjlim\colon&\Mod(R)^I&\longrightarrow&\Mod(R)\\
                &\rotatebox{90}{$\in$}&&\rotatebox{90}{$\in$}\\
                &\alpha=(F_i)_i &\longmapsto&\varinjlim\limits_{i\in I}F_i
        \end{array}
    \end{equation*}
    は右完全関手である.        
\end{PRP}

\subsection{射影極限}
射影系$\beta\colon I^\op\to \Mod(R)$を考える.
極限を取る操作$\varprojlim\colon\Mod(R)^{I^\op}\to\Mod(R)$を
\begin{equation*}
    \begin{array}{rccc}
        \varprojlim\colon&\Mod(R)^{I^\op}&\longrightarrow&\Mod(R)\\
            &\rotatebox{90}{$\in$}&&\rotatebox{90}{$\in$}\\
            &\beta=(F_i)_i &\longmapsto&\varprojlim\limits_{i\in I}F_i
    \end{array}
\end{equation*}
のようにかく.

\begin{PRP}
    \begin{equation*}
        \begin{array}{rccc}
            \varprojlim\colon&\Mod(R)^{I^\op}&\longrightarrow&\Mod(R)\\
                &\rotatebox{90}{$\in$}&&\rotatebox{90}{$\in$}\\
                &\beta=(F_i)_i &\longmapsto&\varprojlim\limits_{i\in I}F_i
        \end{array}
    \end{equation*}
    は左完全関手である.        
\end{PRP}

\subsection{順像}
\(f\colon X\to Y\)を連続写像とする.


\begin{PRP}
    \[
        \begin{array}{rccc}
            f_\ast\colon&   \Sh(X)& \longrightarrow& \Sh(Y) \\
            &   \rotatebox{90}{$\in$}&&\rotatebox{90}{$\in$}\\
            &   F&  \longmapsto&    f_\ast F   
        \end{array}
    \]
    は左完全関手である.
\end{PRP}


\(R\)を\(X\)上の環とする.
%\(G\in\Mod(S)\)に対し,
\begin{PRP}
    \[
        \begin{array}{rccc}
            f_\ast\colon&   \Mod(R)& \longrightarrow& \Mod(f_\ast R)\\
            &   \rotatebox{90}{$\in$}&&\rotatebox{90}{$\in$}\\
            &   F&  \longmapsto&    f_\ast F   
        \end{array}
    \]
    は左完全関手である.
\end{PRP}


\subsection{逆像}
層化が要る.

\(f\colon X\to Y\)を連続写像とする.

\begin{PRP}
    \[
        \begin{array}{rccc}
            f^{-1}\colon&   \Sh(Y) & \longrightarrow& \Sh(X) \\
            &   \rotatebox{90}{$\in$}&&\rotatebox{90}{$\in$}\\
            &   G&  \longmapsto&    f^{-1} G   
        \end{array}
    \]
    は完全関手である.
\end{PRP}


\(S\)を\(Y\)上の環とする.

\begin{PRP}
    \[
        \begin{array}{rccc}
            f^{-1}\colon&   \Mod(S)& \longrightarrow& \Mod(f^{-1}S)\\
            &   \rotatebox{90}{$\in$}&&\rotatebox{90}{$\in$}\\
            &   G&  \longmapsto&    f^{-1} G   
        \end{array}
    \]
    は完全関手である.
\end{PRP}



\subsection{部分集合から定まる層}

\(Z\subset X\)を部分集合とし,
包含写像\(j\colon Z\hookrightarrow X\)の引き戻しで\(Z\)に\(X\)の
誘導位相を入れる.

\subsubsection{制限と切断の一般化}
逆像関手を用いると\(F\in\Sh(X)\)の\(Z\)への制限が
開集合以外にも一般化できる.
\[
    \begin{array}{rccc}
        j^{-1}\colon&   \Sh(X)& \longrightarrow& \Sh(Z)\\
        &   \rotatebox{90}{$\in$}&&\rotatebox{90}{$\in$}\\
        &   F&  \longmapsto&    F|_Z=j^{-1} F   
    \end{array}
\]
また,切断関手の一般化も
\[
    \begin{array}{rccc}
        \Gamma(Z;{\boldsymbol{\cdot}})\colon&\Mod(R)&\longrightarrow&\Mod(R(Z))\\
        &\rotatebox{90}{$\in$}&&\rotatebox{90}{$\in$}\\
        &F&\longmapsto&\Gamma(Z;F)\coloneqq \Gamma(Z;F|_Z)
    \end{array}
\]
によって行える.


\paragraph{2023/09/13分の行間埋め}

\begin{proof}[\textbf{自然な射\(\Gamma(X;F)\to\Gamma(Z;F)\)の存在}]
    \(j^{-1}\)と\(\Gamma(Z;\boldsymbol{\cdot})\)を合成すればよい.
    \[
        F\mapsto F|_Z\mapsto \Gamma(Z;F).
    \]
\end{proof}

\subsubsection{制限の引き戻し}

\(X\)上の層\(F\)に対して\(Z\)から定まる\(X\)上の新しい層\(F_Z\)を
\[
    \begin{array}{rccc}
        {\boldsymbol{\cdot}}_Z\colon&   \Sh(X)& \longrightarrow& \Sh(X)\\
        &   \rotatebox{90}{$\in$}&&\rotatebox{90}{$\in$}\\
        &   F&  \longmapsto&    F_Z=j_\ast j^{-1} F   
    \end{array}
\]
で定める.


















\section{各操作の関係}

諸々の操作の間の関係についてまとめる.

\subsection{茎と極限}
茎と帰納極限は可換

茎と有限射影極限は可換
\subsection{切断と極限}
切断と帰納極限は可換とは限らない.
有限は?

切断と射影極限は可換




























%===============================================
% 参考文献スペース
%===============================================
\begin{thebibliography}{20} 
    \bibitem[KS90]{KS90} Masaki Kashiwara, Pierre Schapira, 
        \textit{Sheaves on Manifolds}, 
        Grundlehren der Mathematischen Wissenschaften, 292, Springer, 1990.
        \bibitem[KS06]{KS06} Masaki Kashiwara, Pierre Schapira, 
        \textit{Categories and Sheaves}, 
        Grundlehren der Mathematischen Wissenschaften, 332, Springer, 2006.
        \bibitem[Sh16]{Sh16} 志甫淳, 層とホモロジー代数, 共立出版, 2016.
    %\bibitem[Og02]{Og02} 小木曽啓示, 代数曲線論, 朝倉書店, 2022.
\end{thebibliography}

%===============================================


\end{document}
