%================================================
%    池ノートの誤植表
%================================================

% -----------------------
% preamble
% -----------------------
% ここから本文 (\begin{document}) までの
% ソースコードに変更を加えた場合は
% 編集者まで連絡してください. 
% Don't change preamble code yourself. 
% If you add something
% (usepackage, newtheorem, newcommand, renewcommand),
% please tell it 
% to the editor of institutional paper of RUMS.

% ------------------------
% documentclass
% ------------------------
\documentclass[9pt, a4paper, dvipdfmx, leqno]{jsarticle}

% ------------------------
% usepackage
% ------------------------
\usepackage{algorithm}
\usepackage{algorithmic}
\usepackage{amscd}
\usepackage{amsfonts}
\usepackage{amsmath}
\usepackage[psamsfonts]{amssymb}
\usepackage{amsthm}
\usepackage{ascmac}
\usepackage{bm}
\usepackage{color}
\usepackage{enumerate}
\usepackage{fancybox}
\usepackage[stable]{footmisc}
\usepackage{graphicx}
\usepackage{listings}
\usepackage{mathrsfs}
\usepackage{mathtools}
\usepackage{otf}
\usepackage{pifont}
\usepackage{proof}
\usepackage{subfigure}
\usepackage{tikz}
\usepackage{verbatim}
\usepackage[all]{xy}
\usepackage{url}
\usetikzlibrary{cd}



% ================================
% パッケージを追加する場合のスペース 
%\usepackage{calligra}
\usepackage[dvipdfmx]{hyperref}
\usepackage{xcolor}
\definecolor{darkgreen}{rgb}{0,0.45,0} 
\definecolor{darkred}{rgb}{0.75,0,0}
\definecolor{darkblue}{rgb}{0,0,0.6} 
\hypersetup{
    colorlinks=true,
    citecolor=darkgreen,
    linkcolor=darkred,
    urlcolor=darkblue,
}
\usepackage{pxjahyper}
%\usepackage[mathscr]{euscript}
\usepackage{layout}
\usepackage{framed}
\definecolor{lightgray}{rgb}{0.75,0.75,0.75}
\renewenvironment{leftbar}{%
  \def\FrameCommand{\textcolor{lightgray}{\vrule width 0.7zw} \hspace{10pt}}% 
  \MakeFramed {\advance\hsize-\width \FrameRestore}}%
{\endMakeFramed}
\newenvironment{redleftbar}{%
  \def\FrameCommand{\textcolor{red}{\vrule width 1pt} \hspace{10pt}}% 
  \MakeFramed {\advance\hsize-\width \FrameRestore}}%
 {\endMakeFramed}
 \usepackage{colortbl}
%=================================


% --------------------------
% theoremstyle
% --------------------------
\theoremstyle{definition}

% --------------------------
% newtheoem
% --------------------------

% 日本語で定理, 命題, 証明などを番号付きで用いるためのコマンドです. 
% If you want to use theorem environment in Japanece, 
% you can use these code. 
% Attention!
% All theorem enivironment numbers depend on 
% only section numbers.
\newtheorem{Axiom}{公理}[section]
\newtheorem{Definition}[Axiom]{定義}
\newtheorem{Theorem}[Axiom]{定理}
\newtheorem{Proposition}[Axiom]{命題}
\newtheorem{Lemma}[Axiom]{補題}
\newtheorem{Corollary}[Axiom]{系}
\newtheorem{Example}[Axiom]{例}
\newtheorem{Claim}[Axiom]{主張}
\newtheorem{Property}[Axiom]{性質}
\newtheorem{Attention}[Axiom]{注意}
\newtheorem{Question}[Axiom]{問}
\newtheorem{Problem}[Axiom]{問題}
\newtheorem{Consideration}[Axiom]{考察}
\newtheorem{Alert}[Axiom]{警告}
\newtheorem{Fact}[Axiom]{事実}
\newtheorem{com}[Axiom]{コメント}


% 日本語で定理, 命題, 証明などを番号なしで用いるためのコマンドです. 
% If you want to use theorem environment with no number in Japanese, You can use these code.
\newtheorem*{Axiom*}{公理}
\newtheorem*{Definition*}{定義}
\newtheorem*{Theorem*}{定理}
\newtheorem*{Proposition*}{命題}
\newtheorem*{Lemma*}{補題}
\newtheorem*{Example*}{例}
\newtheorem*{Corollary*}{系}
\newtheorem*{Claim*}{主張}
\newtheorem*{Property*}{性質}
\newtheorem*{Attention*}{注意}
\newtheorem*{Question*}{問}
\newtheorem*{Problem*}{問題}
\newtheorem*{Consideration*}{考察}
\newtheorem*{Alert*}{警告}
\newtheorem*{Fact*}{事実}
\newtheorem*{com*}{コメント}



% 英語で定理, 命題, 証明などを番号付きで用いるためのコマンドです. 
% If you want to use theorem environment in English, You can use these code.
%all theorem enivironment number depend on only section number.
\newtheorem{Axiom+}{Axiom}[section]
\newtheorem{Definition+}[Axiom+]{Definition}
\newtheorem{Theorem+}[Axiom+]{Theorem}
\newtheorem{Proposition+}[Axiom+]{Proposition}
\newtheorem{Lemma+}[Axiom+]{Lemma}
\newtheorem{Example+}[Axiom+]{Example}
\newtheorem{Corollary+}[Axiom+]{Corollary}
\newtheorem{Claim+}[Axiom+]{Claim}
\newtheorem{Property+}[Axiom+]{Property}
\newtheorem{Attention+}[Axiom+]{Attention}
\newtheorem{Question+}[Axiom+]{Question}
\newtheorem{Problem+}[Axiom+]{Problem}
\newtheorem{Consideration+}[Axiom+]{Consideration}
\newtheorem{Alert+}{Alert}
\newtheorem{Fact+}[Axiom+]{Fact}
\newtheorem{Remark+}[Axiom+]{Remark}

% ----------------------------
% commmand
% ----------------------------
% 執筆に便利なコマンド集です. 
% コマンドを追加する場合は下のスペースへ. 

% 集合の記号 (黒板文字)
\newcommand{\NN}{\mathbb{N}}
\newcommand{\ZZ}{\mathbb{Z}}
\newcommand{\QQ}{\mathbb{Q}}
\newcommand{\RR}{\mathbb{R}}
\newcommand{\CC}{\mathbb{C}}
\newcommand{\PP}{\mathbb{P}}
\newcommand{\KK}{\mathbb{K}}


% 集合の記号 (太文字)
\newcommand{\nn}{\mathbf{N}}
\newcommand{\zz}{\mathbf{Z}}
\newcommand{\qq}{\mathbf{Q}}
\newcommand{\rr}{\mathbf{R}}
\newcommand{\cc}{\mathbf{C}}
\newcommand{\pp}{\mathbf{P}}
\newcommand{\kk}{\mathbf{K}}

% 特殊な写像の記号
\newcommand{\ev}{\mathop{\mathrm{ev}}\nolimits} % 値写像
\newcommand{\pr}{\mathop{\mathrm{pr}}\nolimits} % 射影

% スクリプト体にするコマンド
%   例えば {\mcal C} のように用いる
\newcommand{\mcal}{\mathcal}

% 花文字にするコマンド 
%   例えば {\h C} のように用いる
\newcommand{\h}{\mathscr}

% ヒルベルト空間などの記号
\newcommand{\F}{\mcal{F}}
\newcommand{\X}{\mcal{X}}
\newcommand{\Y}{\mcal{Y}}
\newcommand{\Hil}{\mcal{H}}
\newcommand{\RKHS}{\Hil_{k}}
\newcommand{\Loss}{\mcal{L}_{D}}
\newcommand{\MLsp}{(\X, \Y, D, \Hil, \Loss)}

% 偏微分作用素の記号
\newcommand{\p}{\partial}

% 角カッコの記号 (内積は下にマクロがあります)
\newcommand{\lan}{\langle}
\newcommand{\ran}{\rangle}



% 圏の記号など
\newcommand{\Set}{{\bf Set}}
\newcommand{\Vect}{{\bf Vect}}
\newcommand{\FDVect}{{\bf FDVect}}
\newcommand{\Mod}{\mathop{\mathrm{Mod}}\nolimits}
\newcommand{\CGA}{{\bf CGA}}
\newcommand{\GVect}{{\bf GVect}}
\newcommand{\Lie}{{\bf Lie}}
\newcommand{\dLie}{{\bf Liec}}



% 射の集合など
\newcommand{\Map}{\mathop{\mathrm{Map}}\nolimits}
\newcommand{\Hom}{\mathop{\mathrm{Hom}}\nolimits}
\newcommand{\End}{\mathop{\mathrm{End}}\nolimits}
\newcommand{\Aut}{\mathop{\mathrm{Aut}}\nolimits}
\newcommand{\Mor}{\mathop{\mathrm{Mor}}\nolimits}

% その他便利なコマンド
\newcommand{\dip}{\displaystyle} % 本文中で数式モード
\newcommand{\e}{\varepsilon} % イプシロン
\newcommand{\dl}{\delta} % デルタ
\newcommand{\pphi}{\varphi} % ファイ
\newcommand{\ti}{\tilde} % チルダ
\newcommand{\pal}{\parallel} % 平行
\newcommand{\op}{{\rm op}} % 双対を取る記号
\newcommand{\lcm}{\mathop{\mathrm{lcm}}\nolimits} % 最小公倍数の記号
\newcommand{\Probsp}{(\Omega, \F, \P)} 
\newcommand{\argmax}{\mathop{\rm arg~max}\limits}
\newcommand{\argmin}{\mathop{\rm arg~min}\limits}





% ================================
% コマンドを追加する場合のスペース 
\renewcommand\proofname{\bf 証明} % 証明
\numberwithin{equation}{section}
\newcommand{\cTop}{\textsf{Top}}
%\newcommand{\cOpen}{\textsf{Open}}
\newcommand{\Op}{\mathop{\textsf{Open}}\nolimits}
\newcommand{\Ob}{\mathop{\textrm{Ob}}\nolimits}
\newcommand{\id}{\mathop{\mathrm{id}}\nolimits}
\newcommand{\pt}{\mathop{\mathrm{pt}}\nolimits}
\newcommand{\res}{\mathop{\rho}\nolimits}
\newcommand{\A}{\mcal{A}}
\newcommand{\B}{\mcal{B}}
\newcommand{\C}{\mcal{C}}
\newcommand{\D}{\mcal{D}}
\newcommand{\E}{\mcal{E}}
\newcommand{\G}{\mcal{G}}
%\newcommand{\H}{\mcal{H}}
\newcommand{\I}{\mcal{I}}
\newcommand{\J}{\mcal{J}}
\newcommand{\OO}{\mcal{O}}
\newcommand{\Ring}{\mathop{\textsf{Ring}}\nolimits}
\newcommand{\cAb}{\mathop{\textsf{Ab}}\nolimits}
\newcommand{\Ker}{\mathop{\mathrm{Ker}}\nolimits}
\newcommand{\im}{\mathop{\mathrm{Im}}\nolimits}
\newcommand{\Coker}{\mathop{\mathrm{Coker}}\nolimits}
\newcommand{\Coim}{\mathop{\mathrm{Coim}}\nolimits}
\newcommand{\Ht}{\mathop{\mathrm{Ht}}\nolimits}
\newcommand{\supp}{\mathop{\mathrm{supp}}\nolimits}
\newcommand{\colim}{\mathop{\mathrm{colim}}}
\newcommand{\Tor}{\mathop{\mathrm{Tor}}\nolimits}

\newcommand{\cat}{\mathscr{C}}

\newcommand{\scA}{\mathscr{A}}
\newcommand{\scB}{\mathscr{B}}
\newcommand{\scC}{\mathscr{C}}
\newcommand{\scD}{\mathscr{D}}
\newcommand{\scE}{\mathscr{E}}
\newcommand{\scF}{\mathscr{F}}

\newcommand{\ibA}{\mathop{\text{\textit{\textbf{A}}}}}
\newcommand{\ibB}{\mathop{\text{\textit{\textbf{B}}}}}
\newcommand{\ibC}{\mathop{\text{\textit{\textbf{C}}}}}
\newcommand{\ibD}{\mathop{\text{\textit{\textbf{D}}}}}
\newcommand{\ibE}{\mathop{\text{\textit{\textbf{E}}}}}
\newcommand{\ibF}{\mathop{\text{\textit{\textbf{F}}}}}
\newcommand{\ibG}{\mathop{\text{\textit{\textbf{G}}}}}
\newcommand{\ibH}{\mathop{\text{\textit{\textbf{H}}}}}
\newcommand{\ibI}{\mathop{\text{\textit{\textbf{I}}}}}
\newcommand{\ibJ}{\mathop{\text{\textit{\textbf{J}}}}}
\newcommand{\ibK}{\mathop{\text{\textit{\textbf{K}}}}}
\newcommand{\ibL}{\mathop{\text{\textit{\textbf{L}}}}}
\newcommand{\ibM}{\mathop{\text{\textit{\textbf{M}}}}}
\newcommand{\ibN}{\mathop{\text{\textit{\textbf{N}}}}}
\newcommand{\ibO}{\mathop{\text{\textit{\textbf{O}}}}}
\newcommand{\ibP}{\mathop{\text{\textit{\textbf{P}}}}}
\newcommand{\ibQ}{\mathop{\text{\textit{\textbf{Q}}}}}
\newcommand{\ibR}{\mathop{\text{\textit{\textbf{R}}}}}
\newcommand{\ibS}{\mathop{\text{\textit{\textbf{S}}}}}
\newcommand{\ibT}{\mathop{\text{\textit{\textbf{T}}}}}
\newcommand{\ibU}{\mathop{\text{\textit{\textbf{U}}}}}
\newcommand{\ibV}{\mathop{\text{\textit{\textbf{V}}}}}
\newcommand{\ibW}{\mathop{\text{\textit{\textbf{W}}}}}
\newcommand{\ibX}{\mathop{\text{\textit{\textbf{X}}}}}
\newcommand{\ibY}{\mathop{\text{\textit{\textbf{Y}}}}}
\newcommand{\ibZ}{\mathop{\text{\textit{\textbf{Z}}}}}

\newcommand{\ibx}{\mathop{\text{\textit{\textbf{x}}}}}

\newcommand{\Comp}{\mathop{\mathrm{C}}\nolimits}
\newcommand{\Komp}{\mathop{\mathrm{K}}\nolimits}
\newcommand{\Domp}{\mathop{\mathrm{D}}\nolimits}%複体のホモトピー圏

\newcommand{\CCat}{\Comp(\cat)}
\newcommand{\KCat}{\Komp(\cat)}
\newcommand{\DCat}{\Domp(\cat)}%圏Cの複体のホモトピー圏
\newcommand{\HOM}{\mathop{\mathscr{H}\hspace{-2pt}om}\nolimits}%内部Hom
\newcommand{\RHOM}{\mathop{\mathrm{R}\hspace{-1.5pt}\HOM}\nolimits}

\newcommand{\muS}{\mathop{\mathrm{SS}}\nolimits}
\newcommand{\RG}{\mathop{\mathrm{R}\hspace{-0pt}\Gamma}\nolimits}

\newcommand{\simar}{\mathrel{\overset{\sim}{\longrightarrow}}}%内部Hom
\newcommand{\simra}{\mathrel{\overset{\sim}{\longleftarrow}}}%内部Hom

\newcommand{\hocolim}{{\mathrm{hocolim}}}
\newcommand{\indlim}[1][]{\mathop{\varinjlim}\limits_{#1}}
\newcommand{\sindlim}[1][]{\smash{\mathop{\varinjlim}\limits_{#1}}\,}
\newcommand{\Pro}{\mathrm{Pro}}
\newcommand{\Ind}{\mathrm{Ind}}
\newcommand{\prolim}[1][]{\mathop{\varprojlim}\limits_{#1}}
\newcommand{\sprolim}[1][]{\smash{\mathop{\varprojlim}\limits_{#1}}\,}

\newcommand{\Sh}{\mathrm{Sh}}
\newcommand{\PSh}{\mathrm{PSh}}


% =================================



%================================================
% 自前の定理環境
%   https://mathlandscape.com/latex-amsthm/
% を参考にした
\newtheoremstyle{mystyle}%   % スタイル名
    {5pt}%                   % 上部スペース
    {5pt}%                   % 下部スペース
    {}%              % 本文フォント
    {}%                  % 1行目のインデント量
    {\bfseries}%                      % 見出しフォント
    {.}%                     % 見出し後の句読点
    {12pt}%                     % 見出し後のスペース
    {\thmname{#1}\thmnumber{ #2}\thmnote{{\hspace{2pt}\normalfont (#3)}}}% % 見出しの書式

\theoremstyle{mystyle}
\newtheorem{AXM}{公理}[section]
\newtheorem{DFN}[Axiom]{定義}
\newtheorem{THM}[Axiom]{定理}
\newtheorem*{THM*}{定理}
\newtheorem{PRP}[Axiom]{命題}
\newtheorem{LMM}[Axiom]{補題}
\newtheorem{CRL}[Axiom]{系}
\newtheorem{EG}[Axiom]{例}
\newtheorem{CNV}[Axiom]{規約}
\newtheorem{CMT}[Axiom]{コメント}


% 定理環境ここまで
%====================================================

% ---------------------------
% new definition macro
% ---------------------------
% 便利なマクロ集です

% 内積のマクロ
%   例えば \inner<\pphi | \psi> のように用いる
\def\inner<#1>{\langle #1 \rangle}

% ================================
% マクロを追加する場合のスペース 

%=================================





% ----------------------------
% documenet 
% ----------------------------
% 以下, 本文
% ---------------------------

\title{『超局所層理論概説』(2021年9月1日版)の誤植表}
\author{大柴寿浩}
%\date{2023年6月26日--30日}
\begin{document}
\maketitle
\paragraph{凡例}
\begin{itemize}
    \item l.\(-4\)は下から4行目の意味.
    \item ページ数の横に?がついているものは間違いかどうか曖昧なもの.(意図を汲むとこう書きたかったのかも?というものも含む)
\end{itemize}
\begin{table}[h]
%    \caption{色付きの表}
    \centering
    \begin{tabular}{|c|c|c|c|}
    \hline
    \rowcolor{lightgray}p&位置 & 誤 & 正 \\
    \hline
        14&例1.1.28 (i) の完全列&\(0\to F\to G\to G/\textcolor{red}{H}\to0\)&\(0\to F\to G\to G/\textcolor{red}{F}\to0\)\\    
    \hline
        23&l.\(-8\)&\(H^n(X\textcolor{red}{:}\ZZ_X)\)&\(H^n(X\textcolor{red}{;}\ZZ_X)\)\\
    \hline
        24&例1.2.17(i) 4行目&
        \(0\to \Coker\varepsilon\to C^{\textcolor{red}{0}}(F)\to\Coker{d^0}\to0\)
        &\(0\to \Coker\e\to C^{\textcolor{red}{1}}(F)\to\Coker{d^0}\to0\)\\
    \hline
        24&例1.2.17(i) 11行目&
        \(C^{\textcolor{red}{0}}(F)(X)\)&\(C^{\textcolor{red}{1}}(F)(X)\)\\
    \hline
        27&定義1.2.21 (2) 1行目&\(\mathcal{B}\)における&\(\mathcal{A}\)における\\
    \hline
        29&l.\(-7\)&分解\textcolor{red}{する}して&分解して\\
    \hline
        30&(iii)&入射分解たち間&入射分解たち\textcolor{red}{の}間\\
        \hline
\end{tabular}
\end{table}

\subparagraph{p. 31 命題1.2.31. (iii) の2つ目の図式:}
    \[\begin{CD}
        &&0&&0&&0\\
        &&@VVV@VVV@VVV\\
    0@>>>A_1@>{f_1}>>A_2@>{f_2}>>A_3@>>>0\\
    &&@VVV@VVV@VVV\\
    0@>>>I^\bullet_1@>{f^\bullet_1}>>I^\bullet_{\textcolor{red}{1}}@>{f^\bullet_2}>>I^\bullet_{\textcolor{red}{1}}@>>>0
    \end{CD}
    \]
は
  \[
    \begin{CD}
        &&0&&0&&0\\
        &&@VVV@VVV@VVV\\
    0@>>>A_1@>{f_1}>>A_2@>{f_2}>>A_3@>>>0\\
    &&@VVV@VVV@VVV\\
    0@>>>I^\bullet_1@>{f^\bullet_1}>>I^\bullet_{\textcolor{red}{2}}@>{f^\bullet_2}>>I^\bullet_{\textcolor{red}{3}}@>>>0
    \end{CD}
    \]
が正しい.

\subparagraph{p. 31 命題1.2.31. (iii) の3つ目の図式}:
    \[\begin{CD}
        &&0&&0&&0\\
        &&@VVV@VVV@VVV\\
    0@>>>B_1@>{g_1}>>B_2@>{g_2}>>B_3@>>>0\\
    &&@VVV@VVV@VVV\\
    0@>>>J^\bullet_1@>{g^\bullet_1}>>J^\bullet_{\textcolor{red}{1}}@>{g^\bullet_2}>>I^\bullet_{\textcolor{red}{1}}@>>>0
    \end{CD}\]
は  \[
    \begin{CD}
        &&0&&0&&0\\
        &&@VVV@VVV@VVV\\
    0@>>>B_1@>{g_1}>>B_2@>{g_2}>>B_3@>>>0\\
    &&@VVV@VVV@VVV\\
    0@>>>J^\bullet_1@>{g^\bullet_1}>>J^\bullet_{\textcolor{red}{2}}@>{g^\bullet_2}>>I^\bullet_{\textcolor{red}{3}}@>>>0
    \end{CD}
    \]
が正しい.


\begin{table}[h]\centering
\begin{tabular}{|c|c|c|c|}
    \hline
    \rowcolor{lightgray}p&位置 & 誤 & 正 \\
    \hline
    32&l.5
    &\(n\in\ZZ_{\textcolor{red}{\ge}}\)
    &\(n\in\ZZ_{\textcolor{red}{\ge0}}\)\\
    \hline
    32&定義1.2.32の2行上
    &\(H^n(T(I_1^\bullet))\simeq H^n(T(I_{\textcolor{red}{1}}^\bullet))\)
    &\(H^n(T(I_1^\bullet))\simeq H^n(T(I_{\textcolor{red}{2}}^\bullet))\)\\    
    \hline
    33&補題1.2.37の証明l.\(-5\)
    &\(\varphi\coloneqq\psi\circ\textcolor{red}{\varphi}\)
    &\(\varphi\coloneqq\psi\circ\textcolor{red}{\varepsilon}\)
    \\\hline
    35&l.1
    &右導来函手同じ
    &右導来函手\textcolor{red}{と}同じ
    \\\hline
    37&l.2
    &(inner Hom functor) \textcolor{red}{または}と呼ぶ.
    &(inner Hom functor)と呼ぶ.
    \\\hline
    ?38&1.3.2のl.2
    &\(X\)上の層を\textcolor{red}{押して}
    &\(X\)上の層を\textcolor{red}{押し出して}\footnotemark[1]
    \\\hline
    ?39&定義1.3.10のl.3
    &\textcolor{red}{押し出し}
    &\textcolor{red}{\textbf{押し出し} (pushforward)}\footnotemark[2]
    \\\hline
    45&1.3.3のl.2
    &押し出す方法\textcolor{red}{を定義}です.
    &押し出す方法です.
    \\\hline
\end{tabular}
\end{table}
\footnotetext[1]{
    すぐ後に「\(Y\)上の層を引き戻して」とあるので,
    押し出し・引き戻しの対応的にこう書くつもりだったのかもしれない
    と忖度したので.
}
\footnotetext[2]{ボールド体+英訳にする.
これも引き戻しの方にはついているので忖度.
ただ,押し出しの方の英訳はpushforward以外にもpushout等が見つかったので,
あえてつけなかったかも?その場合はごめんなさい.}    

\subparagraph{p.60 定義2.1.4. (iii) の図式}:
    \[\begin{CD}
    L@>{f}>>M@>{g}>>N@>{h}>>L[1]\\
    @V{{\text{\rotatebox{-90}{\(\sim\)}}}}V{u}V@V{{\text{\rotatebox{-90}{\(\sim\)}}}}V{v}V@V{{\text{\rotatebox{-90}{\(\sim\)}}}}V{w}V@V{{\text{\rotatebox{-90}{\(\sim\)}}}}V{u[1]}V\\
    L@>>{f'}>M@>>{\alpha(f')}>\mathrm{Mc}(f)@>>{\beta(f')}>\textcolor{red}{L[1]}
    \end{CD}\]
は 
\[\begin{CD}
    L@>{f}>>M@>{g}>>N@>{h}>>L[1]\\
    @V{{\text{\rotatebox{-90}{\(\sim\)}}}}V{u}V@V{{\text{\rotatebox{-90}{\(\sim\)}}}}V{v}V@V{{\text{\rotatebox{-90}{\(\sim\)}}}}V{w}V@V{{\text{\rotatebox{-90}{\(\sim\)}}}}V{u[1]}V\\
    L@>>{f'}>M@>>{\alpha(f')}>\mathrm{Mc}(f)@>>{\beta(f')}>\textcolor{red}{L'[1]}
    \end{CD}\]
が正しい.
\clearpage
\subparagraph{p.62 命題2.1.8. 証明 (ii) の1つ目の図式}:
\[\begin{tikzcd}
    L
    \arrow[r,"{\id_L}"]
    \arrow[d,"{\id_L}"]
    &
    L
    \arrow[d,"f"]
    \arrow[r] 
    & 
    0
    \arrow[r] 
    \arrow[d,dashed]
    &
    \textcolor{red}{K}[1]
    \arrow[d,"{\id_L[1]}"]
    \\ 
    L\arrow[r,"f"']
    &
    M\arrow[r,"g"'] 
    & 
    N\arrow[r] 
    &
    {L}[1]. 
\end{tikzcd}\]
は 
\[\begin{tikzcd}
    L
    \arrow[r,"{\id_L}"]
    \arrow[d,"{\id_L}"]
    &
    L
    \arrow[d,"f"]
    \arrow[r] 
    & 
    0
    \arrow[r] 
    \arrow[d,dashed]
    &
    \textcolor{red}{L}[1]
    \arrow[d,"{\id_L[1]}"]
    \\ 
    L\arrow[r,"f"']
    &
    M\arrow[r,"g"'] 
    & 
    N\arrow[r] 
    &
    {L}[1]. 
\end{tikzcd}\]
が正しい.

\begin{table}[h]\centering
\begin{tabular}{|c|c|c|c|}
    \hline
    \rowcolor{lightgray}p&位置 & 誤 & 正 \\
    \hline
    62&1つ目の図式の次の行
    &\(\phi\Hom(K,M)\)
    &\(\phi\textcolor{red}{\in}\Hom(K,M)\)
    \\\hline
    63&定義2.1.11(M4) l.1
    &\(f,g\colon\textcolor{red}{X}\to\textcolor{red}{Y}\)
    &\(f,g\colon\textcolor{red}{A}\to\textcolor{red}{B}\)
    \\\hline
    68&定義2.1.24(普遍性)
    &\(U\colon\textcolor{red}{\mathsf{K}}^+(\mathcal{A})\to\textcolor{red}{\mathsf{K}}^+(\mathcal{B})\)
    &\(U\colon\textcolor{red}{\mathsf{D}}^+(\mathcal{A})\to\textcolor{red}{\mathsf{D}}^+(\mathcal{B})\)
    \\\hline
    70&注意2.1.29 l.2
    &\(\textcolor{red}{L}\in\mathsf{K}^{\textcolor{red}{+}}(\mathcal{J})\)
    &\(\textcolor{red}{J}\in\mathsf{K}^{\textcolor{red}{\mathrm{b}}}(\mathcal{J})\)\footnotemark[1]
    \\\hline
    70&例2.1.30 l.2
    &\(\RG(X,\ast)\colon\mathsf{D}^+(\Sh(X))\to \textcolor{red}{Ab}\)
    &\(\RG(X,\ast)\colon\mathsf{D}^+(\Sh(X))\to \textcolor{red}{\mathsf{D}^+(Ab)}\)
    \\\hline
    71&例2.1.32 (i) l.3
    &\(\mathrm{R}(\Gamma(Y;)\circ{f_\ast})\)
    &\(\mathrm{R}(\Gamma(Y;\textcolor{red}{\ast})\circ{f_\ast})\)
    \\\hline
    75&l.\(-8\)
    &\(F\otimes^\bullet G\mathsf{C}(\mathbf{k}_X)\)
    &\(F\otimes^\bullet G\textcolor{red}{\in}\mathsf{C}(\mathbf{k}_X)\)
    \\\hline
    77&命題2.2.6 l.\(1\)
    &\(\textcolor{red}{G}\in\mathsf{D}^+(\mathbf{k}_X)\)
    &\(\textcolor{red}{H}\in\mathsf{D}^+(\mathbf{k}_X)\)
    \\\hline
    81&命題2.2.19 証明 l.\(2\)
    &\(\Gamma_c(X;\textcolor{red}{F}\otimes M_X)\)
    &\(\Gamma_c(X;\textcolor{red}{S}\otimes M_X)\)
    \\\hline
    82&命題2.2.20から3行下
    &\(\textcolor{red}{RG}_Z(F)\)
    &\(\textcolor{red}{\RG}_Z(F)\)
    \\\hline
    87&l.1
    &\(\omega_X\coloneqq a_X^!\)
    &\(\omega_X\coloneqq a_X^!\textcolor{red}{\textbf{k}}\)
    \\\hline    
    115&l.7
    &\(\dip \indlim[x_0\in{B}]H^n(\{\varphi<\textcolor{red}{0}\}\cap{B};F)\)
    &\(\dip \indlim[x_0\in{B}]H^n(\{\varphi<\textcolor{red}{\varphi(x_0)}\}\cap{B};F)\)
    \\\hline
\end{tabular}
\end{table}
\footnotetext[1]{
    元の複体\(L\)は上下に有界であるが,
    取り替える複体\(J\)も上下に有界とは限らないという文脈だと思うので.
}    



\end{document}
