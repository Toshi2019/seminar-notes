%================================================
%    帰納圏について
%================================================

% -----------------------
% preamble
% -----------------------
% ここから本文 (\begin{document}) までの
% ソースコードに変更を加えた場合は
% 編集者まで連絡してください. 
% Don't change preamble code yourself. 
% If you add something
% (usepackage, newtheorem, newcommand, renewcommand),
% please tell it 
% to the editor of institutional paper of RUMS.

% ------------------------
% documentclass
% ------------------------
\documentclass[11pt, a4paper, dvipdfmx, leqno]{jsarticle}

% ------------------------
% usepackage
% ------------------------
\usepackage{algorithm}
\usepackage{algorithmic}
\usepackage{amscd}
\usepackage{amsfonts}
\usepackage{amsmath}
\usepackage[psamsfonts]{amssymb}
\usepackage{amsthm}
\usepackage{ascmac}
\usepackage{color}
\usepackage{enumerate}
\usepackage{fancybox}
\usepackage[stable]{footmisc}
\usepackage{graphicx}
\usepackage{listings}
\usepackage{mathrsfs}
\usepackage{mathtools}
\usepackage{otf}
\usepackage{pifont}
\usepackage{proof}
\usepackage{subfigure}
\usepackage{tikz}
\usepackage{verbatim}
\usepackage[all]{xy}
\usepackage{url}
\usetikzlibrary{cd}



% ================================
% パッケージを追加する場合のスペース 
%\usepackage{calligra}
\usepackage[dvipdfmx]{hyperref}
\usepackage{xcolor}
\definecolor{darkgreen}{rgb}{0,0.45,0} 
\definecolor{darkred}{rgb}{0.75,0,0}
\definecolor{darkblue}{rgb}{0,0,0.6} 
\hypersetup{
    colorlinks=true,
    citecolor=darkgreen,
    linkcolor=darkred,
    urlcolor=darkblue,
}
\usepackage{pxjahyper}

%=================================


% --------------------------
% theoremstyle
% --------------------------
\theoremstyle{definition}

% --------------------------
% newtheoem
% --------------------------

% 日本語で定理, 命題, 証明などを番号付きで用いるためのコマンドです. 
% If you want to use theorem environment in Japanece, 
% you can use these code. 
% Attention!
% All theorem enivironment numbers depend on 
% only section numbers.
\newtheorem{Axiom}{公理}[section]
\newtheorem{Definition}[Axiom]{定義}
\newtheorem{Theorem}[Axiom]{定理}
\newtheorem{Proposition}[Axiom]{命題}
\newtheorem{Lemma}[Axiom]{補題}
\newtheorem{Corollary}[Axiom]{系}
\newtheorem{Example}[Axiom]{例}
\newtheorem{Claim}[Axiom]{主張}
\newtheorem{Property}[Axiom]{性質}
\newtheorem{Attention}[Axiom]{注意}
\newtheorem{Question}[Axiom]{問}
\newtheorem{Problem}[Axiom]{問題}
\newtheorem{Consideration}[Axiom]{考察}
\newtheorem{Alert}[Axiom]{警告}
\newtheorem{Fact}[Axiom]{事実}
\newtheorem{com}[Axiom]{コメント}
\newtheorem{Notation}[Axiom]{記号}


% 日本語で定理, 命題, 証明などを番号なしで用いるためのコマンドです. 
% If you want to use theorem environment with no number in Japanese, You can use these code.
\newtheorem*{Axiom*}{公理}
\newtheorem*{Definition*}{定義}
\newtheorem*{Theorem*}{定理}
\newtheorem*{Proposition*}{命題}
\newtheorem*{Lemma*}{補題}
\newtheorem*{Example*}{例}
\newtheorem*{Corollary*}{系}
\newtheorem*{Claim*}{主張}
\newtheorem*{Property*}{性質}
\newtheorem*{Attention*}{注意}
\newtheorem*{Question*}{問}
\newtheorem*{Problem*}{問題}
\newtheorem*{Consideration*}{考察}
\newtheorem*{Alert*}{警告}
\newtheorem*{Fact*}{事実}
\newtheorem*{com*}{コメント}



% 英語で定理, 命題, 証明などを番号付きで用いるためのコマンドです. 
% If you want to use theorem environment in English, You can use these code.
%all theorem enivironment number depend on only section number.
\newtheorem{Axiom+}{Axiom}[section]
\newtheorem{Definition+}[Axiom+]{Definition}
\newtheorem{Theorem+}[Axiom+]{Theorem}
\newtheorem{Proposition+}[Axiom+]{Proposition}
\newtheorem{Lemma+}[Axiom+]{Lemma}
\newtheorem{Example+}[Axiom+]{Example}
\newtheorem{Corollary+}[Axiom+]{Corollary}
\newtheorem{Claim+}[Axiom+]{Claim}
\newtheorem{Property+}[Axiom+]{Property}
\newtheorem{Attention+}[Axiom+]{Attention}
\newtheorem{Question+}[Axiom+]{Question}
\newtheorem{Problem+}[Axiom+]{Problem}
\newtheorem{Consideration+}[Axiom+]{Consideration}
\newtheorem{Alert+}{Alert}
\newtheorem{Fact+}[Axiom+]{Fact}
\newtheorem{Remark+}[Axiom+]{Remark}

% ----------------------------
% commmand
% ----------------------------
% 執筆に便利なコマンド集です. 
% コマンドを追加する場合は下のスペースへ. 

% 集合の記号 (黒板文字)
\newcommand{\NN}{\mathbb{N}}
\newcommand{\ZZ}{\mathbb{Z}}
\newcommand{\QQ}{\mathbb{Q}}
\newcommand{\RR}{\mathbb{R}}
\newcommand{\CC}{\mathbb{C}}
\newcommand{\PP}{\mathbb{P}}
\newcommand{\KK}{\mathbb{K}}


% 集合の記号 (太文字)
\newcommand{\nn}{\mathbf{N}}
\newcommand{\zz}{\mathbf{Z}}
\newcommand{\qq}{\mathbf{Q}}
\newcommand{\rr}{\mathbf{R}}
\newcommand{\cc}{\mathbf{C}}
\newcommand{\pp}{\mathbf{P}}
\newcommand{\kk}{\mathbf{K}}

% 特殊な写像の記号
\newcommand{\ev}{\mathop{\mathrm{ev}}\nolimits} % 値写像
\newcommand{\pr}{\mathop{\mathrm{pr}}\nolimits} % 射影

% スクリプト体にするコマンド
%   例えば {\mcal C} のように用いる
\newcommand{\mcal}{\mathcal}

% 花文字にするコマンド 
%   例えば {\h C} のように用いる
\newcommand{\h}{\mathscr}

% ヒルベルト空間などの記号
\newcommand{\F}{\mcal{F}}
\newcommand{\X}{\mcal{X}}
\newcommand{\Y}{\mcal{Y}}
\newcommand{\Hil}{\mcal{H}}
\newcommand{\RKHS}{\Hil_{k}}
\newcommand{\Loss}{\mcal{L}_{D}}
\newcommand{\MLsp}{(\X, \Y, D, \Hil, \Loss)}

% 偏微分作用素の記号
\newcommand{\p}{\partial}

% 角カッコの記号 (内積は下にマクロがあります)
\newcommand{\lan}{\langle}
\newcommand{\ran}{\rangle}



% 圏の記号など
\newcommand{\Set}{\mathop{\textsf{Set}}\nolimits}
\newcommand{\Vect}{{\bf Vect}}
\newcommand{\FDVect}{{\bf FDVect}}
\newcommand{\Mod}{\mathop{\textsf{Mod}}\nolimits}
\newcommand{\CGA}{{\bf CGA}}
\newcommand{\GVect}{{\bf GVect}}
\newcommand{\Lie}{{\bf Lie}}
\newcommand{\dLie}{{\bf Liec}}



% 射の集合など
\newcommand{\Map}{\mathop{\mathrm{Map}}\nolimits}
\newcommand{\Hom}{\mathop{\mathrm{Hom}}\nolimits}
\newcommand{\End}{\mathop{\mathrm{End}}\nolimits}
\newcommand{\Aut}{\mathop{\mathrm{Aut}}\nolimits}
\newcommand{\Mor}{\mathop{\mathrm{Mor}}\nolimits}
\newcommand{\Fct}{\mathop{\mathrm{Fct}}\nolimits}
\newcommand{\f}{\mathop{\mathit{for}}\nolimits}


% その他便利なコマンド
\newcommand{\dip}{\displaystyle} % 本文中で数式モード
\newcommand{\e}{\varepsilon} % イプシロン
\newcommand{\dl}{\delta} % デルタ
\newcommand{\pphi}{\varphi} % ファイ
\newcommand{\ti}{\tilde} % チルダ
\newcommand{\pal}{\parallel} % 平行
\newcommand{\op}{{\rm op}} % 双対を取る記号
\newcommand{\lcm}{\mathop{\mathrm{lcm}}\nolimits} % 最小公倍数の記号
\newcommand{\Probsp}{(\Omega, \F, \P)} 
\newcommand{\argmax}{\mathop{\rm arg~max}\limits}
\newcommand{\argmin}{\mathop{\rm arg~min}\limits}





% ================================
% コマンドを追加する場合のスペース 
\renewcommand\proofname{\bf 証明} % 証明
\numberwithin{equation}{section}
\newcommand{\cTop}{\textsf{Top}}
%\newcommand{\cOpen}{\textsf{Open}}
\newcommand{\Op}{\mathop{\textsf{Open}}\nolimits}
\newcommand{\Ob}{\mathop{\textrm{Ob}}\nolimits}
\newcommand{\id}{\mathop{\mathrm{id}}\nolimits}
\newcommand{\pt}{\mathop{\mathrm{pt}}\nolimits}
\newcommand{\res}{\mathop{\rho}\nolimits}
\newcommand{\A}{\mcal{A}}
\newcommand{\B}{\mcal{B}}
\newcommand{\C}{\mcal{C}}
\newcommand{\D}{\mcal{D}}
\newcommand{\E}{\mcal{E}}
\newcommand{\G}{\mcal{G}}
\newcommand{\U}{\mcal{U}}
%\newcommand{\H}{\mcal{H}}
\newcommand{\I}{\mcal{I}}
\newcommand{\J}{\mcal{J}}
\newcommand{\calS}{\mcal{S}}
\newcommand{\OO}{\mcal{O}}
\newcommand{\Ring}{\mathop{\textsf{Ring}}\nolimits}
\newcommand{\cAb}{\mathop{\textsf{Ab}}\nolimits}
\newcommand{\Ker}{\mathop{\mathrm{Ker}}\nolimits}
\newcommand{\im}{\mathop{\mathrm{Im}}\nolimits}
\newcommand{\Coker}{\mathop{\mathrm{Coker}}\nolimits}
\newcommand{\Coim}{\mathop{\mathrm{Coim}}\nolimits}
\newcommand{\Ht}{\mathop{\mathrm{Ht}}\nolimits}
\newcommand{\colim}{\mathop{\mathrm{colim}}}

\newcommand{\limf}{\mathop{\text{``}\hspace{-0.7pt}\varinjlim\hspace{-1.5pt}\text{''}}}
\newcommand{\sumf}{\mathop{\text{``}\hspace{-0.7pt}\bigoplus\hspace{-1.5pt}\text{''}}}

\newcommand{\hh}{\mathop{\mathrm{h}}\nolimits}
\newcommand{\Ind}{\mathop{\mathrm{Ind}}}




\newcommand{\cat}{\mathcal{C}}

\newcommand{\scA}{\mathscr{A}}
\newcommand{\scB}{\mathscr{B}}
\newcommand{\scC}{\mathscr{C}}
\newcommand{\scD}{\mathscr{D}}
\newcommand{\scE}{\mathscr{E}}
\newcommand{\scF}{\mathscr{F}}

\newcommand{\ibA}{\mathop{\text{\textit{\textbf{A}}}}}
\newcommand{\ibB}{\mathop{\text{\textit{\textbf{B}}}}}
\newcommand{\ibC}{\mathop{\text{\textit{\textbf{C}}}}}
\newcommand{\ibD}{\mathop{\text{\textit{\textbf{D}}}}}
\newcommand{\ibE}{\mathop{\text{\textit{\textbf{E}}}}}
\newcommand{\ibF}{\mathop{\text{\textit{\textbf{F}}}}}
\newcommand{\ibG}{\mathop{\text{\textit{\textbf{G}}}}}
\newcommand{\ibH}{\mathop{\text{\textit{\textbf{H}}}}}
\newcommand{\ibI}{\mathop{\text{\textit{\textbf{I}}}}}
\newcommand{\ibJ}{\mathop{\text{\textit{\textbf{J}}}}}
\newcommand{\ibK}{\mathop{\text{\textit{\textbf{K}}}}}
\newcommand{\ibL}{\mathop{\text{\textit{\textbf{L}}}}}
\newcommand{\ibM}{\mathop{\text{\textit{\textbf{M}}}}}
\newcommand{\ibN}{\mathop{\text{\textit{\textbf{N}}}}}
\newcommand{\ibO}{\mathop{\text{\textit{\textbf{O}}}}}
\newcommand{\ibP}{\mathop{\text{\textit{\textbf{P}}}}}
\newcommand{\ibQ}{\mathop{\text{\textit{\textbf{Q}}}}}
\newcommand{\ibR}{\mathop{\text{\textit{\textbf{R}}}}}
\newcommand{\ibS}{\mathop{\text{\textit{\textbf{S}}}}}
\newcommand{\ibT}{\mathop{\text{\textit{\textbf{T}}}}}
\newcommand{\ibU}{\mathop{\text{\textit{\textbf{U}}}}}
\newcommand{\ibV}{\mathop{\text{\textit{\textbf{V}}}}}
\newcommand{\ibW}{\mathop{\text{\textit{\textbf{W}}}}}
\newcommand{\ibX}{\mathop{\text{\textit{\textbf{X}}}}}
\newcommand{\ibY}{\mathop{\text{\textit{\textbf{Y}}}}}
\newcommand{\ibZ}{\mathop{\text{\textit{\textbf{Z}}}}}

\newcommand{\ibx}{\mathop{\text{\textit{\textbf{x}}}}}

\newcommand{\Comp}{\mathop{\mathsf{C}}\nolimits}
\newcommand{\Komp}{\mathop{\mathsf{K}}\nolimits}
\newcommand{\Domp}{\mathop{\mathsf{D}}\nolimits}%複体のホモトピー圏
\newcommand{\CCat}{\Comp(\cat)}
\newcommand{\KCat}{\Komp(\cat)}

% =================================





% ---------------------------
% new definition macro
% ---------------------------
% 便利なマクロ集です

% 内積のマクロ
%   例えば \inner<\pphi | \psi> のように用いる
\def\inner<#1>{\langle #1 \rangle}
\def\ind<#1>{\mathop{\text{``}\hspace{-0.7pt}#1\limits\hspace{-1.5pt}\text{''}}}


% ================================
% マクロを追加する場合のスペース 

%=================================





% ----------------------------
% documenet 
% ----------------------------
% 以下, 本文の執筆スペースです. 
% Your main code must be written between 
% begin document and end document.
% ---------------------------

\title{続・帰納圏について}
\author{うるち米(@RisE\_HU23)}
\date{}
\begin{document}
\maketitle
%\thispagestyle{empty}

\section*{はじめに}
2023/12/03に行う月一圏論ゼミでの発表資料.
走り書きなので,不備だらけ.
ほとんどの出典は\cite{KS01,KS06}.
モチベーションは\textbf{帰納層}(ind-sheaf)にあって,
圏論の細かいところはダルいので,
あまり厳密にやるつもりは無い.

前回(8月)に引き続き圏の帰納化について発表する.
前回は帰納化の定義とアーベル圏の帰納化について見た.
今月は局所化と時間があれば導来圏と帰納化の関係について言及する.


\section{帰納対象}

$\U$を宇宙とする.
集合$x$が$\U$に関して小さいとは,
$\U$内の集合$u\in\U$で$x\cong u$と
なるものが存在することをいう.
$\U$圏$\cat$とは,
任意の対象$X,Y\in\cat$に対し,$\Hom_\cat(X,Y)$が$\U$に関して
小さい集合となることをいう.
$\U$圏$\cat$であって,
$\Ob(\cat)$が$\U$に関して小さい集合であるものを
$\U$に関して小さい圏とよぶ.
圏が$\U$に関して本質的に小さいとは,
小さい圏と同値であることをいう.

宇宙$\U$を固定して考える.
圏は$\U$圏のことを意味するとする.
小さくない圏を大きい圏という.
$\Set$は$\U$集合のなす圏を意味する.

\begin{Definition}
    $\cat$を圏とする.
    \begin{equation*}
        \cat^\wedge\coloneqq\Fct(\cat^\op,\Set)
    \end{equation*}
    とおく.
\end{Definition}

米田埋め込みを$\hh_\cat\colon\cat\to\cat^\wedge$で表す.
このとき,$G\in\cat^\wedge$と$X\in\cat$に対し,
\begin{equation}
    \Hom_{\cat^\wedge}(\hh_\cat(X),G)\cong G(X)
\end{equation}
が成り立つ.
特に,
\begin{equation*}
    \Hom_{\cat^\wedge}(\hh_\cat(X),\hh_\cat(Y))
    \cong 
    \Hom_\cat(X,Y)
\end{equation*}
であり,$\hh_\cat$は充満忠実である.
$\hh_\cat$によって$\cat$を$\cat^\wedge$の充満部分圏とみなす.

$\cat^\wedge$は小さい帰納極限を持つが,
$\hh_\cat$と$\varinjlim$は一般に交換しない.
混同を防ぐために$\cat^\wedge$における帰納極限を$\limf$とかく.
$I$が小さいとし,$\alpha\colon I\to\cat$を関手とする.
$\limf\alpha=\limf({\hh}_\cat\circ\alpha)$とおく.
すなわち,$\limf\alpha$は
\begin{equation*}
    \limf\alpha\colon\cat\ni X\mapsto\varinjlim_i\Hom_\cat(X,\alpha(i))
\end{equation*}
で定まる$\cat^\wedge$の対象である.
この約束のもとで,
\begin{align*}
    \varinjlim_i\Hom_\cat(X,\alpha(i))
    &=\limf\alpha(X)\\
    &\cong\Hom_{\cat^\wedge}(\hh_\cat(X),\limf\alpha)
\end{align*}
が成り立つ.
\(X\mapsto\hh_\cat(X)\)の略記\(X\mapsto{X}\)を用いると
\begin{equation*}
    \varinjlim_i\Hom_\cat(X,\alpha(i))
    \cong\Hom_{\cat^\wedge}(X,\limf\alpha)
\end{equation*}
という書き方になる.
\begin{Definition}
    \(I\)が\textbf{フィルターづけられている} (filtrant, filtered) とは,
    次の条件\eqref{enum:fil1}--\eqref{enum:fil3}をみたすことをいう.
    \begin{enumerate}[(i)]
        \item \(I\ne\varnothing\)\label{enum:fil1}
        \item 各\(i,j\in{I}\)に対し,\(k\in{I}\)で,\(i\to{k},\j\to{k}\)と射が存在する.\label{enum:fil2}
        \item 各\(f,g\colon i\to{j}\)に対し,\(h\colon j\to k\)で,\(h\circ f= h\circ g\)となるものが存在する.\label{enum:fil3}    
    \end{enumerate}

    また,\(I\)が\textbf{共終的に小さい} (cofinally small) とは,
    小さい集合\(S\subset\Ob(I)\)で,任意の\(i\in I\)が
    射\(i\to j\) (\(j \in S\)) を持つものが存在することをいう.
\end{Definition}

\begin{Definition}
    (i) 
    $\cat$を$\U$圏とする.
    $A\in\cat^\wedge$が$\cat$の\textbf{帰納対象} (ind-object) であるとは,
    $\U$に関して小さい\footnote{
        $\U$圏$\cat$であって,
        $\Ob(\cat)$が$\U$に関して小さい集合であるものを
        $\U$に関して小さい圏とよぶ.
    }フィルター圏$I$と関手$\alpha\colon I\to\cat$で,
    $A\cong\limf\alpha$となるものが存在することをいう.

    (ii) 
    $\cat$の帰納対象からなる$\cat^\wedge$の大きい部分圏
    を$\Ind^\U(\cat)$(混同の恐れがないときはたんに$\Ind(\cat)$)と
    かき,$\cat$の\textbf{帰納化} (indization) と
    よぶ.($\hh_\cat$のひきおこす)自然な
    関手を$\iota_\cat\colon\cat\to\Ind(\cat)$で表す.
\end{Definition}

$\alpha\colon I\to\cat$, $\beta\colon J\to\cat$を
小さい圏で定義された関手とすると,次の同形を得る.
\begin{equation}
    \begin{aligned}\label{eq:ind-in-U}
        \Hom_{\cat^\wedge}(
            \ind<\varinjlim_{i\in I}>\alpha(i),\ind<\varinjlim_{j\in J}>\beta(j)
            )
        &\cong\varprojlim_i\Hom_{\cat^\wedge}(\alpha(i),\ind<\varinjlim_{j\in J}>\beta(j))\\
        &\cong\varprojlim_i\varinjlim_j\Hom_{\cat}(\alpha(i),\beta(j)).
    \end{aligned}
\end{equation}


\begin{Lemma}
    $\Ind(\cat)$は$\U$圏である.
\end{Lemma}
\begin{proof}
    $A,B\in\Ind(\cat)$とする.
    $I$, $J$を小さいフィルター圏, 
    $\alpha\colon I\to\cat$, $\beta\colon J\to\cat$を関手で
    $A\cong\ind<\varinjlim_{i\in I}>\alpha(i)$, 
    $B\cong\ind<\varinjlim_{j\in J}>\beta(j)$をみたすものとする.
    このとき\eqref{eq:ind-in-U}より$\Hom_\cat(A,B)\cong\varprojlim_i\varinjlim_j\Hom_{\cat}(\alpha(i),\beta(j))\in\U$となる.
\end{proof}

$A\in\cat^\wedge$に対し,圏$\cat_A$と関手$\alpha_A\colon\cat_A\to\cat$を
\begin{align*}
    \Ob(\cat_A)&\coloneqq
    \{(X,a);X\in\cat,a\in A(X)\},\\
    \Hom_{\cat_A}((X,a),(Y,b))&\coloneqq
    \{f\colon X\to Y; a=b\circ f\},\\
    \alpha_A&\colon (X,a)\mapsto X
\end{align*}
で定める.


\begin{Proposition}
    $A\in\cat^\wedge$とする.
    $A\in\Ind(\cat)$となるのは,
    $\cat_A$がフィルターづけられており,共終的に小さいときである.
    このとき,$A\cong\limf\alpha_A$となる.
\end{Proposition}
\begin{proof}
    \(\dip A=\limf_i\alpha\), 
    \((\alpha\colon I\to \cat)\in\Ind(\cat)\) とする.
    \(\cat_A\)がフィルター圏の条件を満たすことを示す.

    \eqref{enum:fil1} 
    \((A,\id_A)\in\cat_A\)なので,\(\cat_A\ne\varnothing\). 
    
    \eqref{enum:fil2} 
    \((X,a),(Y,b)\in\cat_A\)とすると,\(\cat_A\)の定義より,
    \((A,\id_A)\)が条件を満たす.

    \eqref{enum:fil3} 
    \(f,g\colon (X,a)\to(Y,b)\)とすると,
    \((A,\id_A)\)と\(b\)が条件を満たす.

    以上より,\(\cat_A\)はフィルター圏である.

    \(\Ob(\cat_A)\)の部分集合\(\left\{(A,\id_A)\right\}\)について,
    任意の\((X,a)\in\cat_A\)に対し,\(a\)が
    \[
        (X,a)\to(A,\id_A)
    \]
    を満たす.したがって\(\cat_A\)は共終的に小さい.
\end{proof}

関手$F\colon\cat\to\cat'$を$IF\colon\Ind(\cat)\to\Ind(\cat')$
に拡張することができる.
$A\in\Ind(\cat)$に対し,$IF(A)\in\Ind(\cat')$を
\begin{equation*}
    IF(A)\coloneqq\ind<\varinjlim_{X\in\cat_A}>F(X)
\end{equation*}
で定める.
$B\in\Ind(\cat)$と$\Ind(\cat)$における射$f\colon A\to B$に対し,
関手$\cat_A\to\cat_B$が$A(X)\ni a\mapsto f\circ a\in B(X)$で定まる.
したがって,射$IF(f)$が
\begin{equation*}
    IF(f)\colon
    \ind<\varinjlim_{X\in\cat_A}>F(X)
    \to
    \ind<\varinjlim_{Y\in\cat_B}>F(Y)
\end{equation*}
で得られる.

$A\cong\ind<\varinjlim_{i\in I}>\alpha(i)$, 
$B\cong\ind<\varinjlim_{j\in J}>\beta(j)$のとき,
\begin{equation*}
    \Hom_{\Ind(\cat)}(A,B)
    \cong
    \varprojlim_i\varinjlim_j\Hom_{\cat}(\alpha(i),\beta(j))
\end{equation*}
が成り立ち,写像$IF\colon\Hom(A,B)\to\Hom(IF(A),IF(B))$は
\begin{equation*}
    \varprojlim_i\varinjlim_j\Hom_{\cat}(\alpha(i),\beta(j))
    \to
    \varprojlim_i\varinjlim_j\Hom_{\cat'}(F(\alpha(i)),F(\beta(j)))
\end{equation*}
で与えられる.


\begin{Proposition}
    $F\colon\cat\to\cat'$とする.

    (i) 
    図式
    \begin{equation}
        \vcenter{\xymatrix@C=40pt@R=32pt{
        \cat
        \ar[r]^-{F}
        \ar[rd]^-{\hh_{\cat'}\circ F}
        \ar[d]_-{\hh_{\cat}}
        &\cat'
        \ar[d]^-{\hh_{\cat'}}
        \\
        \Ind(\cat)
        \ar@{.>}[r]_-{IF}
        &
        \Ind(\cat')
        }}
    \end{equation}
    は可換.

    (ii) 
    $IF\colon\Ind(\cat)\to\Ind(\cat')$はフィルター帰納極限と交換する.

    (iii) 
    $F$が(充満)忠実なら$IF$も(充満)忠実になる.
\end{Proposition}
(i)はつまり,
\begin{equation*}
    IF=\hh_\cat^\dag(\hh_{\cat'}\circ F).
\end{equation*}


\section{帰納化と局所化}

\subsection{圏の局所化}

\begin{Definition}\label{DFN:mul-sys}
    \(\cat\)の射の族\(\mathcal{S}\subset\Mor(\cat)\)が
    \textbf{右乗法系} (right multiplicative system) であるとは,
    次の公理S\ref{enum:mul-sys1}--S\ref{enum:mul-sys4}を
    みたすことをいう.
    \begin{enumerate}[S1]
        \item \(\mathcal{S}\)は\(\cat\)の全ての同型射を含む.
        \label{enum:mul-sys1}
        \item \(f\colon X\to Y\), \(g\colon Y\to Z\)が
        \(\calS\)に属すならば,\(g\circ{f}\)も\(\calS\)に属する.
        \label{enum:mul-sys2}
        \item \(f\colon{X}\to{Y}\)と
        \(s\colon{X}\to{X'}\in\calS\)に対し,
        \(t\colon{Y}\to{Y'}\in\calS\)と
        \(g\colon{X'}\to{Y'}\)で
        \(g\circ{s}=t\circ{f}\)をみたすものが存在する.
        \[\vcenter{\xymatrix@C=32pt@R=32pt{
        X
        \ar[r]^-{f}
        \ar[d]_-{s}
        &Y
        \ar@{.>}[d]^-{t}
        \\
        X'
        \ar@{.>}[r]_-{g}
        &
        Y'.
        }}\]
        \label{enum:mul-sys3}
        \item \(f,g\colon{X}\to{Y}\)を平行射とする.
        \(s\colon{W}\to{X}\in\calS\)で\(f\circ{s}=g\circ{s}\)
        をみたすものが存在するとき,
        \(t\colon{Y}\to{Z}\in\calS\)で
        \(t\circ{f}=t\circ{g}\)をみたすものが存在する.
        \[\vcenter{\xymatrix@C=32pt@R=32pt{
        W
        \ar[r]^-{s}
        &
        X
        \ar@<0.5ex>[r]^-{f}
        \ar@<-0.5ex>[r]_-{g}
        &Y
        \ar@{.>}[r]^-{t}
        &Z.
        }}\]\label{enum:mul-sys4}
    \end{enumerate}
\end{Definition}
\begin{Attention}
    定義\ref{DFN:mul-sys}の
    公理S\ref{enum:mul-sys3}とS\ref{enum:mul-sys4}を
    次の条件S\('\)\ref{enum:mul-sys33}と
    S\('\)\ref{enum:mul-sys44}に置き換えることで,
    \textbf{左乗法系} (left multiplicative system) を
    定義する.
    \begin{enumerate}[S\('\)1]
        \setcounter{enumi}{2}
        \item \(f\colon{X}\to{Y}\)と
        \(t\colon{Y'}\to{Y}\in\calS\)に対し,
        \(s\colon{X'}\to{X}\in\calS\)と
        \(g\colon{X'}\to{Y'}\)で
        \(g\circ{s}=t\circ{f}\)をみたすものが存在する.
        \[\vcenter{\xymatrix@C=32pt@R=32pt{
        X'
        \ar@{.>}[r]^-{g}
        \ar@{.>}[d]_-{s}
        &Y'
        \ar[d]^-{t}
        \\
        X
        \ar[r]_-{f}
        &
        Y.
        }}\]
        \label{enum:mul-sys33}
        \item \(f,g\colon{X}\to{Y}\)を平行射とする.
        \(t\colon{Y}\to{Z}\in\calS\)で
        \(t\circ{f}=t\circ{g}\)をみたすものが存在するとき,
        \(s\colon{W}\to{X}\in\calS\)で
        \(f\circ{s}=g\circ{s}\)
        をみたすものが存在する.
        \[\vcenter{\xymatrix@C=32pt@R=32pt{
        W
        \ar@{.>}[r]^-{s}
        &
        X
        \ar@<0.5ex>[r]^-{f}
        \ar@<-0.5ex>[r]_-{g}
        &Y
        \ar[r]^-{t}
        &Z.
        }}\]\label{enum:mul-sys44}
    \end{enumerate}
\end{Attention}

右乗法系でも左乗法系でもあるとき,たんに乗法系という.

\begin{Definition}
    \(\cat\)の\(\calS\)による局所化とは,
    圏\(\cat_\calS\)と関手\(Q\colon\cat\to\cat_\calS\)の組で
    次の\eqref{enum:loc1}--\eqref{enum:loc3}をみたすものをいう.
    \begin{enumerate}[(i)]
        \item 全ての\(s\in\calS\)に対し,\(Q(s)\)は同型射である.\label{enum:loc1}
        \item 任意の関手\(F\colon\cat\to\cat'\)で\(F(s)\)が
        どの\(s\in\calS\)についても同型であるものに対し,
        関手\(F_\calS\colon\cat_\calS\to\cat'\)と
        自然同型\(F\cong{F_\calS\circ{Q}}\)が存在する.\label{enum:loc2}
        \item 任意の\(G_1,G_2\in\Fct(\cat_\calS,\cat')\)に対し,
        自然な射\(\Hom(G_1,G_2\to\Hom(G_1\circ{Q},G_2\circ{Q})\)は
        一対一対応である.\label{enum:loc3}
    \end{enumerate}
\end{Definition}

乗法系が存在すれば,局所化ができる.

\subsection{帰納化}
任意の対象\(X\in\cat\)に対し,圏\(\calS^X\)を次で定める.
\begin{align*}
    \Ob(\calS^X)
    &\coloneqq
    \left\{s\colon{X}\to{X'};s\in\calS\right\}\\
    \Hom_{\calS^X}((s\colon{X}\to{X'}),s'\colon{X}\to{X''})
    &\coloneqq
    \left\{h\colon{X'}\to{X''};h\circ{s}=s'\right\}\\
    \alpha^X\colon\calS^X\to\cat&;\alpha^X(X\underset{s}{\to}X')=X'.
\end{align*}
\begin{Fact}
    \(\calS\)が積閉系であることと,\(\calS^X\)がフィルター圏で\(\id_X\)を含むことは同値.
\end{Fact}
これを用いて,\(X,Y\in\cat\)に対し,
帰納極限\(\varinjlim\Hom(Y,\alpha^X)\)を考える.
これは次のようにかける.
\[
    \varinjlim_{X\underset{s}{\to}X',s\in\calS}\Hom(Y,X').
\]
所謂「屋根」の図式たちの集合.

こうすると,次が成り立つ.
\begin{align*}
    \Hom_{\cat_\calS}(X,Y)
    &\cong\varinjlim\Hom_\cat(X,\alpha^Y)\\
    &=\varinjlim_{(Y\underset{t}{\to}Y')\in\calS}\Hom(X,Y').
\end{align*}

関手
\[
    \alpha\colon\cat\to\Ind(\cat)\subset\cat^\wedge
\]
を
\[
    \alpha(X)\coloneqq\limf\alpha^X
    =\limf_{(X\to X')\in\calS^X}X'
\]
で定める.
\begin{Proposition}
    \begin{enumerate}[(i)]
        \item 関手\(\alpha\)は\(\cat_\calS\)で分解する.
        したがって,関手\(\alpha_\calS\colon\cat_\calS\to\Ind(\cat)\)が定まる.
        \item \(\alpha_\calS\)は充満忠実関手である.
    \end{enumerate}
\end{Proposition}
\begin{proof}
    自然同型
    \[
        \limf_{(Y\to{Y'})\in\calS^Y}\Hom_\cat(X,Y)
        \overset{\sim}{\longrightarrow}\varprojlim_{(X\to{X'})\in\calS^X}\varinjlim_{(Y\to{Y'})\in\calS^Y}\Hom_\cat(X',Y')
    \]は同型
    \[
        \Hom_{\cat_\calS}(X,Y)
        \to
        \Hom_{\Ind(\cat)}(\limf\alpha^X,\limf\alpha^Y)
    \]をひきおこす.これは合成と可換.
\end{proof}
図式
\[\vcenter{\xymatrix@C=32pt@R=32pt{
    \cat
    \ar[r]^-{Q}
    \ar[rd]_-{\iota_\cat}
    &
    \cat_\calS
    \ar[d]^-{\alpha_\calS}
    \\
    &\Ind(\cat)
}}\]
は一般には可換ではない.
(\(\alpha_\calS\)は局所化の普遍性を用いて構成したわけではない.)
ただ,次の自然な射はある.
\begin{equation}
    \iota_\cat\to\alpha=\alpha_\calS\circ{Q},
    \quad 
    \iota_\cat(X)\to\limf\alpha^X\cong (\alpha_\calS\circ Q)(X).
\end{equation}
\subsection{関手の局所化}
\(\cat\)を圏,\(\calS\)を乗法系,
\(F\colon\cat\to\cat'\)を関手とする.
\begin{Definition}
    \begin{enumerate}[(i)]
        \item \(F\)の右局所化とは,
        関手\(F_\calS\colon\cat_\calS\to\cat'\)で,アレをみたすもの.
        \item \(F\)が普遍的に右局所化可能であるとは,アレをみたすこと.
    \end{enumerate}
\end{Definition}


%\setcounter{subsection}{2}
\section[アーベル圏の帰納化]{アーベル圏の帰納化}

$\cat$をアーベル$\U$圏とする.
このとき,$\cat^\op$から$\Mod(\zz)$への加法関手のなす
大きい圏$\cat^{\wedge,add}$はアーベル圏になる.
$\cat^{\wedge,add,l}$で左完全関手
のなす$\cat^{\wedge,add}$の大きい充満部分圏を表す.
%$\hh_\cat\colon\cat\to\cat^\wedge$により,
%$\cat$を$\cat^{\wedge,add}$の充満部分アーベル圏ととみなせて,
%$\hh_\cat$は左完全となる.

\begin{Notation}
    %$\cat$が圏であるとき,
    %$\limf$によって$\cat^\wedge$における帰納極限を表す.
    $(X_i)_{i\in I}$が$I$で添字づけられた加法圏$\cat$の対象の
    小さい族であるとき,$\ind<\bigoplus_{i\in I}>X_i$で
    $\ind<\varinjlim_J>(\bigoplus\limits_{i\in J}X_i)$を表す.
    ここで,$J$は$I$の有限部分集合を走る.
    したがって,$Z\in\cat$に対し
    \begin{equation*}
        \Hom_{\cat^\wedge}(Z,\ind<\bigoplus_{i\in I}>X_i)
        \cong
        \bigoplus_{i\in I}\Hom_{\cat^\wedge}(Z,X_i)
    \end{equation*}
    が成り立つ.
\end{Notation}

関手
\begin{equation*}
    \hh_\cat\colon\cat\to\cat^{\wedge,add},\quad X\mapsto\Hom_{\cat}(\boldsymbol{\cdot},X)
\end{equation*}
によって,$\cat$を$\cat^{\wedge,add}$の充満部分圏とみなせる.
しかもこの関手は左完全である.ただし,一般に完全ではない.

\begin{Proposition}
    $A\in\cat^{\wedge,add}$とする.
    次の条件(i)と(ii)は同値である.
    \begin{enumerate}
        \item [(i)]関手$A$は$\Ind(\cat)$に属する.
        \item [(ii)]関手$A$は左完全であり$\cat_A$は共終で小さい圏である.
    \end{enumerate}
\end{Proposition}

\begin{Corollary}
    $\cat$を小さいアーベル圏とする.
    このとき,$\Ind(\cat)$は左完全関手のなす$\cat^{\wedge,add}$の
    充満加法部分圏$\cat^{\wedge,add,l}$と圏同値である.
\end{Corollary}

\begin{Lemma}\label{lem:ind-property}
    (i) 
    圏$\Ind(\cat)$は加法的であり,核と余核を持つ.

    (ii) 
    $I$を小さいフィルター圏とし,
    $\alpha, \beta\colon I\to\cat$を関手,$\varphi\colon\alpha\to\beta$を関手の射とする.
    $f\coloneqq\limf\varphi$とすると,
    $\Ker f\cong\limf(\Ker\varphi)$,
    $\Coker f\cong\limf(\Coker\varphi)$が成り立つ.

    (iii) 
    $\varphi\colon A\to B$を$\Ind(\cat)$の射とする.
    このとき,$\cat^{\wedge,add}$における$\varphi$の核は$\Ind(\cat)$における核である.
\end{Lemma}

%これは命題\ref{prop:ind-lim}と命題\ref{prop:ind-fin-colim}の特殊な場合である.

\begin{Theorem}
    (i) 
    $\cat$はアーベル圏である.

    (ii) 
    自然な関手$\cat\to\Ind(\cat)$は充満忠実かつ完全であり,
    自然な関手$\Ind(\cat)\to\cat^{\wedge,add}$は
    充満忠実かつ左完全である.

    (iii) 
    $\Ind(\cat)$は小さい帰納極限を持つ.
    さらに,小さいフィルター圏上の帰納極限は完全である.

    (iv) 
    $\cat$が小さい射影極限を持つならば,
    $\Ind(\cat)$も小さい射影極限を持つ.

    (v) 
    $\ind<\bigoplus>$は$\Ind(\cat)$における余積である.
%    (vi)
%    $\cat$が本質的に小さいとする.
\end{Theorem}



\begin{Proposition}\label{prop:ind-exact}
    $0\to A'\overset{f}{\to}A\overset{g}{\to}A''\to 0$
    を$\Ind(\cat)$の完全列とし,$\mcal{J}$を$\cat$の充満部分加法圏とする.
    このとき,小さいフィルター圏$I$と$I$から$\cat$への関手の
    完全列$0\to \alpha'
    \overset{\varphi}{\to}\alpha
    \overset{\psi}{\to}\alpha''\to 0$で$f\cong\limf\varphi$と$g\cong\limf\psi$をみたすものが存在する.
\end{Proposition}

\begin{comment}
\begin{Lemma}
    $I$を小さいフィルター圏,
    $\alpha\colon T\to \cat$を関手,
    $A=\limf\alpha$とし,
    $f\colon A\hookrightarrow B$を$\Ind(\cat)$における単射とする.
    このとき,小さいフィルター圏$K$と共終関手$p$
\end{Lemma}
\end{comment}

\begin{Corollary}
    $F\colon \cat\to\cat'$をアーベル圏の間の加法関手とし,
    $IF\colon\Ind(\cat)\to\Ind(\cat')$を対応する関手とする.
    $F$が左完全(右完全)なら$IF$も左完全(右完全)である.
\end{Corollary}

\begin{proof}
    命題\ref{prop:ind-exact}から従う.
\end{proof}

\begin{Proposition}
    \(\Ind(\cat)\)における射の列
    \(A\underset{f}{\to}B\underset{g}{\to}C\)で
    \(g\circ{f}=0\)をみたすものが完全になるのは,
    \(Y\in\cat\)であるような\(\Ind(\cat)\)における
    任意の実線の可換図式
    \[\vcenter{\xymatrix@C=32pt@R=32pt{
        X
        \ar@{.>}[r]^-{h}
        \ar@{.>}[d]_-{}
        &Y
        \ar[d]^-{}
        \ar[rd]^-{0}
        \\
        A
        \ar[r]_-{f}
        &
        B
        \ar[r]_-{g}
        &
        C
    }}\]に対し,\(X\in\cat\)かつ\(h\)が全射となるような
    点線の射があることをいう.
\end{Proposition}

\begin{Proposition}
    \(\cat\)をアーベル圏とする.
    \begin{enumerate}[(i)]
        \item \(\cat\)は\(\Ind(\cat)\)の中で拡大に関して閉じている.\footnote{
            \(\J\subset\cat\)が\(\cat\)の中で拡大に関して閉じているとは,
            \(\cat\)における任意の完全列
            \(0\to{X'}\to{X}\to{X''}\to0\)
            で\(X',X''\in\J\)であるものに対し,\(X\in\J\)となることをいう.
        }
        \item \(\cat_0\subset\cat\)を\(\cat\)の中で
        拡大に関して閉じている部分アーベル圏とする.
        このとき,\(\Ind(\cat_0)\)は\(\Ind(\cat)\)の中で
        拡大に関して閉じている.
    \end{enumerate}
\end{Proposition}

\(\cat\)をアーベル圏とし,\(\J\)を充満部分加法圏とする.

\begin{Definition}
    \(\J\)が\(\cat\)において\textbf{余生成的} (cogenerating) とは,
    任意の対象\(X\in\cat\)に対し,
    \(\J\)の対象\(J\)への単射\(0\to{X}\to{J}\)が存在することをいう.
\end{Definition}


















\section{導来圏}

\(\A\)をアーベル圏とする.
\(\A\)の複体の圏\(\Comp(\A)\)のホモトピー圏\(\Komp(\A)\)の
零系\(\mathcal{N}\)による局所化\(\Komp(\A)/\mathcal{N}\)を
導来圏といい,\(\Domp(\A)\)とかくのだった.




\section{帰納化と導来圏}
\(\Domp(\Ind(\cat))\)を考える.

\(\cat\)が充分単射的対象を持っていても,
\(\Ind(\cat)\)が充分単射的対象を持つとは限らない.
そこで,「準単射的対象」と呼ばれる概念を導入する.

\(\cat\)をアーベル圏とし,
\(\cat'\subset\cat\)を充満部分アーベル圏とする.
\(\Domp^b_{\cat'}(\cat)\subset\Domp^b(\cat)\)を
\[
    \Domp^b_{\cat'}(\cat)\coloneqq\left\{
        X\in\Domp^b(\cat);H(X)\in\cat'
    \right\}
\]
すなわち,
コホモロジーが\(\cat'\)に含まれる対象のなす充満部分圏として定める.

\begin{Proposition}
    \(\cat\)をアーベル圏とする.
    自然な関手\(\Domp^b(\cat)\to\Domp^b_{\cat'}(\Ind(\cat))\)は
    三角圏の同値である.
\end{Proposition}
\begin{Definition}
    \(A\in\Ind(\cat)\)とする.
    \(A\)が\textbf{準単射的} (quasi-injective) であるとは,
    関手\[
        \begin{array}{cccl}
            \cat^\op& \longrightarrow& \Mod(\zz)&\\
               \rotatebox{90}{$\in$}&&\rotatebox{90}{$\in$}&\\
               X&  \longmapsto&    A(X)&\left(=\Hom_{\Ind(\cat)}(X,A)\right)   
        \end{array}
    \]
    が完全であることをいう.
\end{Definition}

\begin{Definition}
    \(\cat\)をアーベル圏とする.
    \(\cat\)における
    \textbf{狭義の\(\U\)生成系} (a system of 
    strict \(\U\)-generators) とは,
    \(\cat\)の対象の族\(\left\{G_a;a\in{A}\right\}\)で,
    \(A\)が\(\U\)に関して小さく,
    次の条件を満たすものをいう.
    \begin{enumerate}[(i)]
        \item 全ての\(X\in\cat\)と\(a\in{A}\)に対し,対象\(G_a^{\oplus\Hom(G_a,X)}\)が存在する.
        \item 全ての\(X\in\cat\)に対し,\(a\in{A}\)で射\(G_a^{\oplus\Hom(G_a,X)}\to{X}\)が全射となるものが存在する.
    \end{enumerate}
\end{Definition}
\begin{Fact}
    \(\cat\)が充分単射的対象をもち,狭義の生成系をもつとする.
    このとき,\(\Ind(\cat)\)は充分準単射的対象をもつ.
\end{Fact}




































%===============================================
% 参考文献スペース
%===============================================
\begin{thebibliography}{20} 
    \bibitem[KS90]{KS90} Masaki Kashiwara, Pierre Schapira, 
    \textit{Sheaves on Manifolds}, 
    Grundlehren der Mathematischen Wissenschaften, 292, Springer, 1990.
    \bibitem[KS99]{KS99} Masaki Kashiwara, Pierre Schapira, 
    \textit{Ind-Sheaves,distributions, and microlocalization}, 
    Sem Ec. Polytechnique, May 18, 1999.
    
    \bibitem[KS01]{KS01} Masaki Kashiwara, Pierre Schapira, 
    \textit{Ind-sheaves}, 
    Ast\`erisque, 271, Soci\`et\`e Math. de France, 2001.
    \bibitem[KS06]{KS06} Masaki Kashiwara, Pierre Schapira, 
    \textit{Categories and Sheaves}, 
    Grundlehren der Mathematischen Wissenschaften, 332, Springer, 2006.
    %\bibitem[Og02]{Og02} 小木曽啓示, 代数曲線論, 朝倉書店, 2022.
\end{thebibliography}

%===============================================


\end{document}
