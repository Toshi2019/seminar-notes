\chapter{Theory of Microfunctions}

\section{Construction of the sheaf of microfunctions}

\subsection{Hyperfunctions}


Let $M$ be an $n$-dimentional real analytic manifold 
and $X$ be a complex neighborhood of $M$. 
$X$ is uniquely determined by $M$ if we pay attention 
only to a neighborhood of $M$. 
We denote 
by $\shhol{X}$ the sheaf of holomorphic functions on $X$ 
and 
by $\shan{M}$ the sheaf of real analytic functions on $M$, 
that is, 
$\shan{M} = \iota^{-1}\shhol{X}$ by definition, 
where $\iota \colon M\hookrightarrow X$ is 
the canonical injection.
We denote by $\shor{M}$ the sheaf of orientation of $M$. 
$\shor{M}$ is isomorphic to $\HH_M^n(\ZZ_M)$. 
$\shor{M}$ is locally isomorphic to $\ZZ_M$, and giving
an isomorphism 
$\mapres{\shor{M}}{U}\simeq \mapres{\ZZ_M}{U}$ 
on an open subset $U$ of $M$ is equivalent to 
giving an orientation of $U$.

As in \cite{Sato1}, we define the sheaf of hyperfunctions on $M$: 

\begin{dfn}
    The sheaf $\shhyp{M}$ is by definition 
    \begin{equation}
        \shhyp{M} = \HH_M^n(\shhol{X})\otimes_{\ZZ_{M}}\shor{M}.
    \end{equation}

    A section of $\shhyp{M}$ is called a \emph{hyperfunction}.
\end{dfn}

As stated in \cite{Sato1}, $H_M^i(\shhol{X}) = 0$ for $i \neq n$ 
and $\shhyp{M}$ constitutes a flabby sheaf on $M$.

We first recall the fallowing general lemma:

\begin{lem}
    Let $Y$ be a $d$-codimensional submanifold of 
    a topological manifold $X$ of dimension $n$. 
    Then, for any sheaf (or complex of sheaves) $\mathscr{F}$
    on $X$, we can define the following homomorphism
    \begin{equation}
        \mapres{\mathscr{F}}{Y} \longrightarrow \RR\Gamma_Y(\mathscr{F})[d]\otimes\shor{Y/X},
    \end{equation}
    where $\shor{Y/X} = \HH_Y^d(\ZZ_X)$ is the orientation sheaf 
    of $Y\subset X$ and $\RR$ and $\Gamma_Y$ denote respectively 
    the derived functor in the derived category and 
    the functor of taking the sub sheaf with support 
    in $Y$ of \cite{Hartshorne1}.
\end{lem}

\begin{proof}
    Since $\RR\Gamma_Y(\ZZ_X) = \shor{Y/X}[-d]$, we obtain the desired homomorphism
    as the composite of the following:
    \begin{align*}
        \mapres{\F}{Y} 
        \simeq \F&\otimes_{\ZZ_X} \ZZ_Y 
        \simeq \F\otimes_{\ZZ_X} \RR\Gamma_Y(\ZZ_X)\otimes\shor{Y/X}[d]\\
        &\longrightarrow \RR\Gamma_Y(\mathscr{F})[d]\otimes\shor{Y/X}[d].
    \end{align*}
\end{proof}

We apply this lemma to our case where $\F,X,Y$ correspond to
$\shhol{X},X$ and $M$ respectively. Then we obtain the sheaf homomorphism
\begin{equation}
    \shan{M}\longrightarrow \shhyp{M},
\end{equation}
which will be proved to be injective later. 
This injection allows us to consider hyperfunctions 
as a generalization of functions.
The purpose of this section is 
to analyse the structure of the quotient sheaf 
$\shhyp{M}/\shan{M}$ from a very new point of view.


\subsection{Real monoidal transformation and real comonoidal transformation}


Now consider the following situation, although we apply it 
to a special case in this section. 

Let $N$ and $M$ be real analytic manifolds 
and $f\colon M \to N$ be a real analytic map. 
We denote by $TN$ (resp.\ $TM$) the tangent vector bundle 
of $N$ (resp.\ $M$) and by $T^\ast N$ (resp.\ $T^\ast M$) the
cotangent vector bundle over $N$ (resp.\ $M$). 
We can define the following canonical homomorphisms: 
\begin{equation}
    \begin{aligned}
        0\to TM &\to TN\times_{N} M \to T_{M} N \to 0 \quad\text{(when $f$ is an embedding)}\\
        T^\ast M &\leftarrow T^\ast N\times_{N} M \leftarrow T_{M}^\ast N \leftarrow 0
    \end{aligned}    
\end{equation}%in the original paper, this eq is numbered (1.2.1)
where $T_{M} N$ (resp.\ $T_{M}^\ast N$) is 
the normal (resp.\ conormal) fiber space.
We denote by $SM$ (resp.\ $S^\ast M$, $SN$, $S^\ast N$, 
$S_{M} N$, $S_{M}^\ast N$) the spherical 
bundle $(TM - M)/ \RR^+$ (resp.\ $(T^\ast M - M)/ \RR^+$,\ldots), where $\RR^+$ is the 
multiplicative group of strictly positive real numbers. 
$S^\ast_{M}N$ is not necessarily a fiber bundle. 

Then,
\begin{align*}
    S^\ast_{M}N \hookrightarrow S^\ast N \times_{N} M
\end{align*}
and we have a projection
\begin{equation}
    \rho \colon S^\ast N \times_{N} M - S^\ast_{M}N \longrightarrow S^\ast M.
\end{equation}%in the original paper, this eq is numbered (1.2.2)
Suppose moreover that $\iota\colon M \to N$ is an embedding. 
Then we can provide the disjoint 
union $\widetilde{M_N} = (N - M)\sqcup S_{M}N$ with 
a structure of real analytic manifold with boundary $S_{M}N$.
Since this is constructed in the same way as monoidal transforms 
of complex manifolds, we call $\widetilde{M_N}$ the \emph{real monoidal 
transform} of $N$ with center $M$.
Let $\{U_j\}$ be a set of coordinate patches of $N$ with 
a local coordinate $x_j = (x_j^1, \ldots ,x_j^n)$ such that
\begin{align*}
    M\cap U_j = \{x_j\in U_j ;\ x_j^1 = \cdots =x_j^m=0\}.
\end{align*}
Let 
\begin{align}
    x_j^\nu &= f_{jk}^\nu(x_k) \quad\nu = m+1,\ldots,n,  \label{eq:123}\\
    x_j^\nu &= \sum_{\mu=1}^{m}x_k^\mu g_{jk,\mu}^\nu(x_k) \quad\nu = 1,\ldots,m \label{eq:124}
\end{align}%in the original paper, this eq is numbered (1.2.3), (1.2.4)
be a coordinate transformation. We put
\begin{align*}
    U_j' = \Big{\{} 
        (x_j,\xi_j);\quad  
        &x_j = (x_j^1, \ldots ,x_j^n)\in U_j,\ 
        \xi_j = (\xi_j^1, \ldots ,\xi_j^m) \in\RR^m - \{0\}\\
        \text{such that}&\quad 
        x_j^\nu\xi_j^\mu = x_j^\mu \xi_j^\nu 
        \quad\text{for}\quad 
        \nu,\mu = 1,\ldots,m, 
        \quad x_j^\nu \xi_j^\nu \geqq 0
    \Big{\}}.
\end{align*}    
The multiplicative group $\RR^+$ of positive numbers 
operates on $U_j'$ by \[
    \left(\left(x_j, \xi_j\right), t\right)
    \mapsto \left(x_j,t\xi_j\right).
\] 
We denote by $\widetilde{U_j}$ the quotient $U'_j/\RR^+$. 
We glue together $\widetilde{U_j}$ in the following 
manner: $(x_j,\xi_j)\in U_j$ and $(x_k,\xi_k)\in U_k$ are 
identified if $x_j$ and $x_k$ satisfy \eqref{eq:123} and \eqref{eq:124} and
\begin{align*}
    {\xi}_j^{\nu} = \sum_{\mu=1}^{m}{\xi}_k^{\mu}g_{jk,\mu}^{\nu}(x_k)\quad {\nu} = 1,\ldots,m.
\end{align*}

We denote by $\widetilde{M_N}$ the real analytic manifold 
with boundary obtained by gluing $\widetilde{U_j}$.
Then, $\tau \colon \widetilde{M_N} \to N$ is the projection defined 
by $\widetilde{U_j}\ni (x_j,\xi_j)\overset{\tau}{\mapsto}x_j\in U_j$. 
Then, $\tau^{-1}(M)$ is isomorphic to the normal spherical 
bundle $S_{M}N$, and seen to be the boundary 
of $\widetilde{M_N}$. Moreover, $\tau$ gives 
an isomorphism $\widetilde{M_N}-S_{M}N \to N - M$. 
For a tangent vector $\xi \in T_{M}N_X - \{0\}$, 
we denote by $x+\xi0$ the corresponding point 
of $S_{M}N\subset \widetilde{M_N}$. 

$D_{M}N$ is a subset of the fiber 
product $S_{M}N\times_{M}S_{M}^{\ast}N$ defined 
by $\{(\xi,\eta)\in S_{M}N\times_{M}S_{M}^{\ast}N; 
\inner<\xi,\eta>\geqq 0\}$. 
We define the toplogy on the 
set $\widetilde{M_{N^+}} = (N-M) \sqcup D_{M}N$ as 
follows: $N^M \subset \widetilde{M_{N^+}}$ is an open set 
and the toplogy of $N-M$ induced from $\widetilde{M_{N^+}}$ is 
usual one, and for 
a point $x\in D_{M}N \subset \widetilde{M_{N^+}}$, a 
neighborhood of $x$ is a subset $U$ such 
that $U\cap D_{M}N$ is a neighborhood of $x$ with 
respect to the usual toplogy of $D_{M}N$ ant that 
the image of $U$ under the 
projection $\pi\colon \widetilde{M_{N^+}}\to 
\widetilde{M_{N}}$ is a neighborhood of $\pi(x)$. We note that 
the topology of $\widetilde{M_{N^+}}$ is not Hausdorff.

Let $\widetilde{M_N^{\ast}}$ be disjoint union 
of $(N-M)$ and $S_{M}^{\ast}N$, $\tau\colon 
\widetilde{M_{N^{\ast}}}$, $\pi\colon 
\widetilde{M_{N^{\ast}}}\to N$ be the canonical projections. 
$\widetilde{M_{N^{\ast}}}$ will be equipped 
with the quotient toplogy 
of $\widetilde{M_{N^{\ast}}}$ under $\tau$. 

In this way we obtain a diagram of maps of topological spaces:
\begin{equation}
    \vcenter{\xymatrix
    @C=32pt@R=32pt
    {
        &\widetilde{M_{N^+}}
        \ar[dl]_(.6){\pi}
        \ar[dr]_(.3){\tau}
        &
        D_{M}{N}
        \ar@{_{(}->}[l]^-{}
        \ar[dl]_(.3){\pi}
        \ar[dr]^(.5){\tau}
        &\\
        \widetilde{M_N}
        \ar[dr]_(.5){\tau}
        &S_M{N}
        \ar@{_{(}->}[l]^-{}
        \ar[dr]_(.3){\tau}
        &\widetilde{M_{N^\ast}}
        \ar[dl]^(.3){\pi}
        &S_M^\ast{N}
        \ar@{_{(}->}[l]^-{}
        \ar[dl]_(.6){\pi}
        \\
        &N
        &M
        \ar@{_{(}->}[l]^-{}
    }}.
\end{equation}
Note that
\begin{enumerate}[1)]
    \item all horizontal inclusionas are closed embeddings;
    \item \(\widetilde{M_{N^+}}\) can be considered as a closed subspace of \(\widetilde{M_N}\mathop{\times}\limits_{N}\widetilde{M_{N^\ast}}\);
    \item \(\widetilde{M_N}\to N\) and \(\widetilde{M_{N^+}}\to \widetilde{M_{N^\ast}}\) are proper and separated.
\end{enumerate}

\begin{rem}
    The map \(f\colon X\to Y\) of topological spaces is 
    said to be \emph{separated} if \(X\) is closed 
    in \(X\times_Y X\). 
    \(f\) is said to be \emph{proper} 
    if every fibre of \(f\) is compact 
    and \(f\) is closed (that is, the image of a closed set 
    in \(X\) by \(f\) is closed in \(Y\)). 
    The following lemma is used frequetly in this note.
\end{rem}

\begin{lem}
    Let \(f\colon X\to Y\) be separated and proper, 
    \(\scF\) be a sheaf on \(X\). 
    Then, for every point \(y\) of \(Y\), the homomorphism
    \[
        \Rder^k{f}_\ast(\scF)_{y}\to
        H^k\left(f^{-1}(y);\scF\rvert{f^{-1}(y)}\right)
    \]
    is isomorphic for every integer \(k\).
\end{lem}
For the proof, we refer to Bredon.

In the sequel, 
the notion of derived category will be of constant use. 
We refer to Hartshorne as to derived category. 
We will not distinguishe the sheaf, 
the complex of sheaves and the corresponding object 
of the derived category. 

\begin{prp}
    Let \(\scF\) be a complex of sheves on \(N\) (or more precisely an object of the derived category of sheaves on \(N\)). 
    Then we have an isomorphism
    \[
        \Rder{\tau_\ast}\pi^{-1}\RG_{S_{M}N}\left(
            \tau^{-1}\scF
        \right)
        \simarr
        \RG_{S_{M}^\ast{N}}\left(\pi^{-1}\scF\right).
    \]
\end{prp}
\begin{proof}
    At first note that
    \[
        \pi^{-1}\RG_{S_{M}N}\left(
            \tau^{-1}\scF
        \right)
        \simarr
        \RG_{D_{M}{N}}\left(\pi^{-1}\tau^{-1}\scF\right)
    \]
    is an isomorphism. 
    This follows from the fact that 
    for every point \(x\) in \(D_M{N}\), 
    the family \(\{U-S_M{N}\}\) where \(U\) runs through 
    the neighborhoods of \(\pi(x)\) is equivalent to 
    the family \(V - D_M{N}\) where \(V\) runs through 
    the neighborhoods of \(x\). 

    Now we have a triangle:
    \[    \vcenter{\xymatrix
    @C=2pt@R=42pt
    {
        &\RG_{S^{\ast}_{M}N}\Rder{\Dist}_{\tau}\left(
            \pi^{-1}\scF
        \right)
        \ar[dr]
        &\\
        \Rder{\tau_\ast}\RG_{D_{M}N}\left(
            \tau^{-1}\pi^{-1}\scF
        \right)
        \ar[ur]^-{+1}
        &&
        \RG_{S^{\ast}_{M}N}\left(
            \pi^{-1}\scF
        \right).
        \ar[ll]
    }}
\]
(See Hartshorne for the notion of triangle.) 
Since \(\tau\colon \widetilde{M_{N^+}}\to\widetilde{M_{N^\ast}}\) is proper and separated with contractible fiber, 
\[
    \Rder\Dist_{\tau}\left(\pi^{-1}\scF\right)=0.
\]
This proves the isomorphism
\[
    \Rder{\tau_\ast}\RG_{D_{M}N}\left(
            \tau^{-1}\pi^{-1}\scF
    \right)
    \simarr    
    \RG_{S^{\ast}_{M}N}\left(
            \pi^{-1}\scF
    \right).\]
\end{proof}

\begin{rem}
    Let \(f\colon X\to Y\) be a continuous map, 
    and \(\scF\) be a sheaf on \(Y\). 
    Let \(\scF\to \scL^\bullet\), 
    \(f^{-1}\scF\to\scM^\bullet\) be flabby resolutions 
    of \(\scF\) and \(f^{-1}\scF\) with \(
        \scL^\bullet\to f_\ast\scM^\bullet
    \).
\end{rem}
\subsection{Definition of microfunctions}


Now we will come back to the original situation. 




\subsection{Sheaves on sphere bundle and on cosphere bundle}


We consider the following situation.



\subsection{Fundamental diagram on $\mathscr{C}$}


We will apply the arguements in the preceding section 
to a special case.
