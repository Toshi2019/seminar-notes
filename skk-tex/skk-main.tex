%==========================================
%   SKK TeX project
%   2021.Sep.11 - 
%   written by Toshihiro Oshiba
%==========================================

\documentclass[leqno]{book}


%usepackage
%------------------------
\usepackage{amsmath}
\usepackage{amsthm}
\usepackage[psamsfonts]{amssymb}
\usepackage{color}
\usepackage{ascmac}
\usepackage{amsfonts}
\usepackage{calligra}
\usepackage{mathrsfs}
%\usepackage{rsfso}
\usepackage{mathtools}
\usepackage{amssymb}
\usepackage{graphicx}
\usepackage{fancybox}
\usepackage{enumerate}
\usepackage{verbatim}
\usepackage{subfigure}
\usepackage{proof}
\usepackage{listings}
\usepackage{otf}
\usepackage{algorithm}
\usepackage{algorithmic}
\usepackage{tikz}
%\usetikzlibrary{arrows.meta}
\usepackage[all]{xy}
\usepackage{amscd}

\usepackage[pdfborder=0,dvipdfmx]{hyperref}
\usepackage{xcolor}
\definecolor{darkgreen}{rgb}{0,0.45,0} 
\definecolor{darkred}{rgb}{0.75,0,0}
\definecolor{darkblue}{rgb}{0,0,0.6} 
\hypersetup{
    colorlinks=true,
    citecolor=darkgreen,
    linkcolor=darkblue,
    urlcolor=darkblue,
}
\usepackage{pxjahyper}
\usepackage{comment}
%\usepackage{enumitem}
\usepackage{layout}

\usetikzlibrary{cd}

%\theoremstyle{definition}
\theoremstyle{plain}
\newtheorem{thm}{Theorem}[section]
\newtheorem{prp}[thm]{Proposition}

\theoremstyle{definition}
\newtheorem{dfn}[thm]{Definition}
\newtheorem{lem}[thm]{Lemma}

\theoremstyle{remark}
\newtheorem{rem}[thm]{Remark}

\renewcommand{\qedsymbol}{q.e.d.}

\newcommand{\ZZ}{\mathbb{Z}}
\newcommand{\RR}{\mathbb{R}}
\newcommand{\CC}{\mathbb{C}}

% category theory
\newcommand{\fin}[1]{{#1}^{\mathrm{f}}} % finiteness


% sheaf theory
\newcommand{\F}{\mathscr{F}}
\newcommand{\HH}{\mathscr{H}}
\newcommand{\shhol}[1]{\mathscr{O}_{#1}}
\newcommand{\shan}[1]{\mathscr{A}_{#1}}
\newcommand{\shor}[1]{\omega_{#1}}
\newcommand{\shhyp}[1]{\mathscr{B}_{#1}}

% D-modules
\newcommand{\DD}{\mathscr{D}}
\newcommand{\PP}{\mathscr{P}}

\newcommand{\rmD}{\mathrm{D}}

\newcommand{\scA}{\mathscr{A}}
\newcommand{\scB}{\mathscr{B}}
\newcommand{\scC}{\mathscr{C}}
\newcommand{\scD}{\mathscr{D}}
\newcommand{\scE}{\mathscr{E}}
\newcommand{\scF}{\mathscr{F}}
\newcommand{\scL}{\mathscr{L}}
\newcommand{\scM}{\mathscr{M}}
\newcommand{\scN}{\mathscr{N}}
\newcommand{\scO}{\mathscr{O}}
\newcommand{\scV}{\mathscr{V}}

\newcommand{\Rder}{\mathrm{R}}
\newcommand{\RG}{\mathop{\mathrm{R}\hspace{-0pt}\Gamma}\nolimits}
\newcommand{\RHom}{\mathop{\mathrm{R}\hspace{-1.5pt}\Hom}\nolimits}

\newcommand{\simar}{\mathrel{\overset{\sim}{\rightarrow}}}%同型右矢印
\newcommand{\simarr}{\mathrel{\overset{\sim}{\longrightarrow}}}%同型右矢印
\newcommand{\simra}{\mathrel{\overset{\sim}{\leftarrow}}}%同型左矢印
\newcommand{\simrra}{\mathrel{\overset{\sim}{\longleftarrow}}}%同型左矢印

\newcommand{\Dist}{\mathscr{D}ist}


%new definition macro
%-------------------------
\def\inner<#1>{\langle #1 \rangle}

\newcommand{\mapres}[2]{\left. #1 \right|_{#2}}

\numberwithin{equation}{subsection}


%==============================================================
% page layout 
%--------------------------------------------------------------

%\setlength{\oddsidemargin}{-1cm}
\setlength{\evensidemargin}{\oddsidemargin}
%\pagestyle{myheadings}






%==============================================================






\title{Hyperfunctions and Pseudo-differential Equations}
\author{}

\begin{document}
\maketitle

\frontmatter

\tableofcontents
\markboth{Contents}{Contents}

\chapter*{Preface for Part II}

This is the last of two parts of the Proceedings of
the conference on Hyperfunctions and Pseudo-Differential 
Equations held at Katata on October 12--14, 1971. 

This part consists of a paper by 
M. Sato, T. Kawai and M. Kashiwara which is an enlarged 
version of four lectures by them delivered at the conference.

We received the final manuscript in June, 1971 but have
postponed the publication because the authors had the intention
of adding an introduction to the paper. 
Since we do not think it appropriate to wait for it forever, 
we have decided to publish this part in the present form. 

In place of the introduction, we advise the reader to read 
the lectures by the authors at different occasions, 
the Nice Congress, 1970, the A. M. S. Symposium on 
Partial Differential Equations at Berkeley, 1971, 
and the Colloque C. N. R. S. Equations aux 
D\'eriv\'ees Partielles Lin\'eaires at Orsay, 1972. 

We thank Miss C. Sagawa for typing and Mr.\ T. Miwa and 
Mr.\ T.\ Oshima for proof-reading. 


\begin{flushright}
    December 28, 1972
    \linebreak

    Hikosaburo Komatsu
\end{flushright}

\layout
\mainmatter

%\chapter{Theory of Microfunctions}

\section{Construction of the sheaf of microfunctions}

\subsection{Hyperfunctions}

Let $M$ be an $n$-dimentional real analytic manifold 
and $X$ be a complex neifgborhood of $M$. 




\subsection{Real monoidal transformation and real comonoidal transformation}

Now consider the following situation, although we apply it 
to a special case in this section.




\subsection{Definition of microfunctions}

Now we will come back to the original situation. 




\subsection{Sheaves on sphere bundle and on cosphere bundle}

We consider the following situation.



\subsection{Fundamental diagram on $\mathscr{C}$}

We will apply the arguements in the preceding section 
to a special case.

\section{Several oprations on hyperfunctions and microfunctions}

\chapter{Theory of Microfunctions}

\section{Construction of the sheaf of microfunctions}

\subsection{Hyperfunctions}


Let $M$ be an $n$-dimentional real analytic manifold 
and $X$ be a complex neighborhood of $M$. 
$X$ is uniquely determined by $M$ if we pay attention 
only to a neighborhood of $M$. 
We denote 
by $\shhol{X}$ the sheaf of holomorphic functions on $X$ 
and 
by $\shan{M}$ the sheaf of real analytic functions on $M$, 
that is, 
$\shan{M} = \iota^{-1}\shhol{X}$ by definition, 
where $\iota \colon M\hookrightarrow X$ is 
the canonical injection.
We denote by $\shor{M}$ the sheaf of orientation of $M$. 
$\shor{M}$ is isomorphic to $\HH_M^n(\ZZ_M)$. 
$\shor{M}$ is locally isomorphic to $\ZZ_M$, and giving
an isomorphism 
$\mapres{\shor{M}}{U}\simeq \mapres{\ZZ_M}{U}$ 
on an open subset $U$ of $M$ is equivalent to 
giving an orientation of $U$.

As in \cite{Sato1}, we define the sheaf of hyperfunctions on $M$: 

\begin{dfn}
    The sheaf $\shhyp{M}$ is by definition 
    \begin{equation}
        \shhyp{M} = \HH_M^n(\shhol{X})\otimes_{\ZZ_{M}}\shor{M}.
    \end{equation}

    A section of $\shhyp{M}$ is called a \emph{hyperfunction}.
\end{dfn}

As stated in \cite{Sato1}, $H_M^i(\shhol{X}) = 0$ for $i \neq n$ 
and $\shhyp{M}$ constitutes a flabby sheaf on $M$.

We first recall the fallowing general lemma:

\begin{lem}
    Let $Y$ be a $d$-codimensional submanifold of 
    a topological manifold $X$ of dimension $n$. 
    Then, for any sheaf (or complex of sheaves) $\mathscr{F}$
    on $X$, we can define the following homomorphism
    \begin{equation}
        \mapres{\mathscr{F}}{Y} \longrightarrow \RR\Gamma_Y(\mathscr{F})[d]\otimes\shor{Y/X},
    \end{equation}
    where $\shor{Y/X} = \HH_Y^d(\ZZ_X)$ is the orientation sheaf 
    of $Y\subset X$ and $\RR$ and $\Gamma_Y$ denote respectively 
    the derived functor in the derived category and 
    the functor of taking the sub sheaf with support 
    in $Y$ of \cite{Hartshorne1}.
\end{lem}

\begin{proof}
    Since $\RR\Gamma_Y(\ZZ_X) = \shor{Y/X}[-d]$, we obtain the desired homomorphism
    as the composite of the following:
    \begin{align*}
        \mapres{\F}{Y} 
        \simeq \F&\otimes_{\ZZ_X} \ZZ_Y 
        \simeq \F\otimes_{\ZZ_X} \RR\Gamma_Y(\ZZ_X)\otimes\shor{Y/X}[d]\\
        &\longrightarrow \RR\Gamma_Y(\mathscr{F})[d]\otimes\shor{Y/X}[d].
    \end{align*}
\end{proof}

We apply this lemma to our case where $\F,X,Y$ correspond to
$\shhol{X},X$ and $M$ respectively. Then we obtain the sheaf homomorphism
\begin{equation}
    \shan{M}\longrightarrow \shhyp{M},
\end{equation}
which will be proved to be injective later. 
This injection allows us to consider hyperfunctions 
as a generalization of functions.
The purpose of this section is 
to analyse the structure of the quotient sheaf 
$\shhyp{M}/\shan{M}$ from a very new point of view.


\subsection{Real monoidal transformation and real comonoidal transformation}


Now consider the following situation, although we apply it 
to a special case in this section. 

Let $N$ and $M$ be real analytic manifolds 
and $f\colon M \to N$ be a real analytic map. 
We denote by $TN$ (resp.\ $TM$) the tangent vector bundle 
of $N$ (resp.\ $M$) and by $T^\ast N$ (resp.\ $T^\ast M$) the
cotangent vector bundle over $N$ (resp.\ $M$). 
We can define the following canonical homomorphisms: 
\begin{equation}
    \begin{aligned}
        0\to TM &\to TN\times_{N} M \to T_{M} N \to 0 \quad\text{(when $f$ is an embedding)}\\
        T^\ast M &\leftarrow T^\ast N\times_{N} M \leftarrow T_{M}^\ast N \leftarrow 0
    \end{aligned}    
\end{equation}%in the original paper, this eq is numbered (1.2.1)
where $T_{M} N$ (resp.\ $T_{M}^\ast N$) is 
the normal (resp.\ conormal) fiber space.
We denote by $SM$ (resp.\ $S^\ast M$, $SN$, $S^\ast N$, 
$S_{M} N$, $S_{M}^\ast N$) the spherical 
bundle $(TM - M)/ \RR^+$ (resp.\ $(T^\ast M - M)/ \RR^+$,\ldots), where $\RR^+$ is the 
multiplicative group of strictly positive real numbers. 
$S^\ast_{M}N$ is not necessarily a fiber bundle. 

Then,
\begin{align*}
    S^\ast_{M}N \hookrightarrow S^\ast N \times_{N} M
\end{align*}
and we have a projection
\begin{equation}
    \rho \colon S^\ast N \times_{N} M - S^\ast_{M}N \longrightarrow S^\ast M.
\end{equation}%in the original paper, this eq is numbered (1.2.2)
Suppose moreover that $\iota\colon M \to N$ is an embedding. 
Then we can provide the disjoint 
union $\widetilde{M_N} = (N - M)\sqcup S_{M}N$ with 
a structure of real analytic manifold with boundary $S_{M}N$.
Since this is constructed in the same way as monoidal transforms 
of complex manifolds, we call $\widetilde{M_N}$ the \emph{real monoidal 
transform} of $N$ with center $M$.
Let $\{U_j\}$ be a set of coordinate patches of $N$ with 
a local coordinate $x_j = (x_j^1, \ldots ,x_j^n)$ such that
\begin{align*}
    M\cap U_j = \{x_j\in U_j ;\ x_j^1 = \cdots =x_j^m=0\}.
\end{align*}
Let 
\begin{align}
    x_j^\nu &= f_{jk}^\nu(x_k) \quad\nu = m+1,\ldots,n,  \label{eq:123}\\
    x_j^\nu &= \sum_{\mu=1}^{m}x_k^\mu g_{jk,\mu}^\nu(x_k) \quad\nu = 1,\ldots,m \label{eq:124}
\end{align}%in the original paper, this eq is numbered (1.2.3), (1.2.4)
be a coordinate transformation. We put
\begin{align*}
    U_j' = \Big{\{} 
        (x_j,\xi_j);\quad  
        &x_j = (x_j^1, \ldots ,x_j^n)\in U_j,\ 
        \xi_j = (\xi_j^1, \ldots ,\xi_j^m) \in\RR^m - \{0\}\\
        \text{such that}&\quad 
        x_j^\nu\xi_j^\mu = x_j^\mu \xi_j^\nu 
        \quad\text{for}\quad 
        \nu,\mu = 1,\ldots,m, 
        \quad x_j^\nu \xi_j^\nu \geqq 0
    \Big{\}}.
\end{align*}    
The multiplicative group $\RR^+$ of positive numbers 
operates on $U_j'$ by \[
    \left(\left(x_j, \xi_j\right), t\right)
    \mapsto \left(x_j,t\xi_j\right).
\] 
We denote by $\widetilde{U_j}$ the quotient $U'_j/\RR^+$. 
We glue together $\widetilde{U_j}$ in the following 
manner: $(x_j,\xi_j)\in U_j$ and $(x_k,\xi_k)\in U_k$ are 
identified if $x_j$ and $x_k$ satisfy \eqref{eq:123} and \eqref{eq:124} and
\begin{align*}
    {\xi}_j^{\nu} = \sum_{\mu=1}^{m}{\xi}_k^{\mu}g_{jk,\mu}^{\nu}(x_k)\quad {\nu} = 1,\ldots,m.
\end{align*}

We denote by $\widetilde{M_N}$ the real analytic manifold 
with boundary obtained by gluing $\widetilde{U_j}$.
Then, $\tau \colon \widetilde{M_N} \to N$ is the projection defined 
by $\widetilde{U_j}\ni (x_j,\xi_j)\overset{\tau}{\mapsto}x_j\in U_j$. 
Then, $\tau^{-1}(M)$ is isomorphic to the normal spherical 
bundle $S_{M}N$, and seen to be the boundary 
of $\widetilde{M_N}$. Moreover, $\tau$ gives 
an isomorphism $\widetilde{M_N}-S_{M}N \to N - M$. 
For a tangent vector $\xi \in T_{M}N_X - \{0\}$, 
we denote by $x+\xi0$ the corresponding point 
of $S_{M}N\subset \widetilde{M_N}$. 

$D_{M}N$ is a subset of the fiber 
product $S_{M}N\times_{M}S_{M}^{\ast}N$ defined 
by $\{(\xi,\eta)\in S_{M}N\times_{M}S_{M}^{\ast}N; 
\inner<\xi,\eta>\geqq 0\}$. 
We define the toplogy on the 
set $\widetilde{M_{N^+}} = (N-M) \sqcup D_{M}N$ as 
follows: $N^M \subset \widetilde{M_{N^+}}$ is an open set 
and the toplogy of $N-M$ induced from $\widetilde{M_{N^+}}$ is 
usual one, and for 
a point $x\in D_{M}N \subset \widetilde{M_{N^+}}$, a 
neighborhood of $x$ is a subset $U$ such 
that $U\cap D_{M}N$ is a neighborhood of $x$ with 
respect to the usual toplogy of $D_{M}N$ ant that 
the image of $U$ under the 
projection $\pi\colon \widetilde{M_{N^+}}\to 
\widetilde{M_{N}}$ is a neighborhood of $\pi(x)$. We note that 
the topology of $\widetilde{M_{N^+}}$ is not Hausdorff.

Let $\widetilde{M_N^{\ast}}$ be disjoint union 
of $(N-M)$ and $S_{M}^{\ast}N$, $\tau\colon 
\widetilde{M_{N^{\ast}}}$, $\pi\colon 
\widetilde{M_{N^{\ast}}}\to N$ be the canonical projections. 
$\widetilde{M_{N^{\ast}}}$ will be equipped 
with the quotient toplogy 
of $\widetilde{M_{N^{\ast}}}$ under $\tau$. 

In this way we obtain a diagram of maps of topological spaces:
\begin{equation}
    \vcenter{\xymatrix
    @C=32pt@R=32pt
    {
        &\widetilde{M_{N^+}}
        \ar[dl]_(.6){\pi}
        \ar[dr]_(.3){\tau}
        &
        D_{M}{N}
        \ar@{_{(}->}[l]^-{}
        \ar[dl]_(.3){\pi}
        \ar[dr]^(.5){\tau}
        &\\
        \widetilde{M_N}
        \ar[dr]_(.5){\tau}
        &S_M{N}
        \ar@{_{(}->}[l]^-{}
        \ar[dr]_(.3){\tau}
        &\widetilde{M_{N^\ast}}
        \ar[dl]^(.3){\pi}
        &S_M^\ast{N}
        \ar@{_{(}->}[l]^-{}
        \ar[dl]_(.6){\pi}
        \\
        &N
        &M
        \ar@{_{(}->}[l]^-{}
    }}.
\end{equation}
Note that
\begin{enumerate}[1)]
    \item all horizontal inclusionas are closed embeddings;
    \item \(\widetilde{M_{N^+}}\) can be considered as a closed subspace of \(\widetilde{M_N}\mathop{\times}\limits_{N}\widetilde{M_{N^\ast}}\);
    \item \(\widetilde{M_N}\to N\) and \(\widetilde{M_{N^+}}\to \widetilde{M_{N^\ast}}\) are proper and separated.
\end{enumerate}

\begin{rem}
    The map \(f\colon X\to Y\) of topological spaces is 
    said to be \emph{separated} if \(X\) is closed 
    in \(X\times_Y X\). 
    \(f\) is said to be \emph{proper} 
    if every fibre of \(f\) is compact 
    and \(f\) is closed (that is, the image of a closed set 
    in \(X\) by \(f\) is closed in \(Y\)). 
    The following lemma is used frequetly in this note.
\end{rem}

\begin{lem}
    Let \(f\colon X\to Y\) be separated and proper, 
    \(\scF\) be a sheaf on \(X\). 
    Then, for every point \(y\) of \(Y\), the homomorphism
    \[
        \Rder^k{f}_\ast(\scF)_{y}\to
        H^k\left(f^{-1}(y);\scF\rvert{f^{-1}(y)}\right)
    \]
    is isomorphic for every integer \(k\).
\end{lem}
For the proof, we refer to Bredon.

In the sequel, 
the notion of derived category will be of constant use. 
We refer to Hartshorne as to derived category. 
We will not distinguishe the sheaf, 
the complex of sheaves and the corresponding object 
of the derived category. 

\begin{prp}
    Let \(\scF\) be a complex of sheves on \(N\) (or more precisely an object of the derived category of sheaves on \(N\)). 
    Then we have an isomorphism
    \[
        \Rder{\tau_\ast}\pi^{-1}\RG_{S_{M}N}\left(
            \tau^{-1}\scF
        \right)
        \simarr
        \RG_{S_{M}^\ast{N}}\left(\pi^{-1}\scF\right).
    \]
\end{prp}
\begin{proof}
    At first note that
    \[
        \pi^{-1}\RG_{S_{M}N}\left(
            \tau^{-1}\scF
        \right)
        \simarr
        \RG_{D_{M}{N}}\left(\pi^{-1}\tau^{-1}\scF\right)
    \]
    is an isomorphism. 
    This follows from the fact that 
    for every point \(x\) in \(D_M{N}\), 
    the family \(\{U-S_M{N}\}\) where \(U\) runs through 
    the neighborhoods of \(\pi(x)\) is equivalent to 
    the family \(V - D_M{N}\) where \(V\) runs through 
    the neighborhoods of \(x\). 

    Now we have a triangle:
    \[    \vcenter{\xymatrix
    @C=2pt@R=42pt
    {
        &\RG_{S^{\ast}_{M}N}\Rder{\Dist}_{\tau}\left(
            \pi^{-1}\scF
        \right)
        \ar[dr]
        &\\
        \Rder{\tau_\ast}\RG_{D_{M}N}\left(
            \tau^{-1}\pi^{-1}\scF
        \right)
        \ar[ur]^-{+1}
        &&
        \RG_{S^{\ast}_{M}N}\left(
            \pi^{-1}\scF
        \right).
        \ar[ll]
    }}
\]
(See Hartshorne for the notion of triangle.) 
Since \(\tau\colon \widetilde{M_{N^+}}\to\widetilde{M_{N^\ast}}\) is proper and separated with contractible fiber, 
\[
    \Rder\Dist_{\tau}\left(\pi^{-1}\scF\right)=0.
\]
This proves the isomorphism
\[
    \Rder{\tau_\ast}\RG_{D_{M}N}\left(
            \tau^{-1}\pi^{-1}\scF
    \right)
    \simarr    
    \RG_{S^{\ast}_{M}N}\left(
            \pi^{-1}\scF
    \right).\]
\end{proof}

\begin{rem}
    Let \(f\colon X\to Y\) be a continuous map, 
    and \(\scF\) be a sheaf on \(Y\). 
    Let \(\scF\to \scL^\bullet\), 
    \(f^{-1}\scF\to\scM^\bullet\) be flabby resolutions 
    of \(\scF\) and \(f^{-1}\scF\) with \(
        \scL^\bullet\to f_\ast\scM^\bullet
    \).
\end{rem}
\subsection{Definition of microfunctions}


Now we will come back to the original situation. 




\subsection{Sheaves on sphere bundle and on cosphere bundle}


We consider the following situation.



\subsection{Fundamental diagram on $\mathscr{C}$}


We will apply the arguements in the preceding section 
to a special case.

\section{Several oprations on hyperfunctions and microfunctions}

\subsection{Linear differential operators}

\subsection{Substitution}

\subsection{Integration along fibers}

\subsection{Products}

\subsection{Micro-local operators}

\subsection{Complex conjugation}
\section{Techniques for construction of hyperfunctions and microfunctions}

\subsection{Real analytic functions of positive type}

\subsection{Boundary values of hyperfunctions with holomorphic parameters and examples}
\chapter{Foundation of the Theory of Pseudo-differential Equations}

\section{Definition of pseudo-differential operators}

Is a


\section{Fundamental properties of pseudo-differential operators}

\subsection{Theorems on ellipticity and the equivalence of pseudo-differential operators}

\subsection{Theorems on division of pseudo-differential operators}
\section{Algebraic properties of the sheaf of pseudo-differential operators}


\subsection{Pseudo-differential operators with holomorphic parameters}


\subsection{Properties of the ring of formal pseudo-differential operators}


\subsection{Contact structure and quantized contact transforms}


\subsection{Faithful flatness}


\begin{rem}
    Let $X$ be a complex manifold. We denote by $\DD_X$ (resp. $\fin{\DD}_X$) 
    the sheaf of differential operators on $X$ (resp. differential 
    operators of finite order on $X$). 
    $\PP_X$ (resp. $\fin{\PP}_X$) is a $\pi^{-1}\DD_X$-Algebra 
    (resp. $\pi^{-1}\fin{\DD}_X$-Algebra). 
    By using the method 
\end{rem}

\subsection{Operations on systems of pseudo-differential equations}


\section{Maximally overdetermined systems}

\subsection{Definition of maximally overdetermined systems}

\subsection{Invariants of maximally overdetermined systems}

\subsection{Quantized contact transform --- general case ---}
\section{Structure theorem for systems of pseudo-differential equations in the complex domain}
\markright{\scriptsize{2.5 Structure theorem for systems of pseudo-differential equations in the complex domain}}
In this section we establish the fundamental theorem 
concerning the structure of a system of pseudo-differential equations 
of finite order in complex domain at generic points, 
i.e., we will firstly prove in theorem \ref{thm512} 
as the simplest case that any system $\mathscr{M}$ of 
pseudo-differential equations of finite order with one unknown
function and simple characteristics can be transformed micro-locally 
into the partial de Rham systems
\begin{align*}
    \mathscr{N} \colon \frac{\partial}{\partial x'_i}u = 0, \quad i = 1,\ldots,d
\end{align*}
by a suitable ``quantized'' contact transformation.
Here ``micro-locally'' means ``locally on $P^\ast X$, not on $X$''. 
In the sequal we use the word ``micro-locally'' 
in this sense (and sometimes in the sense that ``locally on 
$S^{\ast}M$, not on $M$'' when we consider the problems in the real domain).
Later we extend Theorem\ref{thm512} to more general systems 
by the aid of pseudo-differential operators of infinite order.

\subsection{Structure theorem for systems of pseudo-differential equations with simple characteristics}

\begin{thm}\label{thm512}
    a
    % theorem5.1.2
\end{thm}

\subsection{Equivalence of pseudo-differential operators with constant multiple characteristics}

\begin{rem}
    This example shows that the structure of the hyperfunction
    solution sheaf, not merely the microfunction solution sheaf, of 
    the equation of $P_1(D)u = 0$ and that of $P_2(D)u = 0$ are the same, 
    because the operators $A_j(x,D)$ are differential operators, not merely 
    pseudo-differential operators. Note that, more generally, 
    if $P(x,D)$ is a linear differential operator of order $m$ 
    defined in a neighborfood of the origin of $\CC^n$ whose 
    principal symbol is $\iota^m_1$, then the differential equation 
    $P(x,D)u=0$ and $D_1^mu=0$ are equivalent as left $\DD$-modules.
\end{rem}
\subsection{Structure theorem for regular systems of pseudo-differential equations}

\chapter{Structure of Systems of Pseudo-differential Equations}

\section{Realification of holomorphic microfunctions}

\subsection{Realification of holomorphic hyperfunctions}

\subsection{Realification of holomorphic microfunctions}

\subsection{Real ``quantized'' contact transforms}

\section{Structure theorems for systems of pseudo-differential equations in the real domain}

\subsection{Structure theorem I --- partial de Rham type ---}

\subsection{Structure theorem II --- partial Cauchy Riemann type ---}

\subsection{Structure theorem III --- Lewy-Mizohata type ---}

\subsection{Structure theorem IV --- general case ---}


\backmatter

%\begin{thebibliography}{99}
    \par
      \bibitem[Liu]{Liu} Qing Liu, 
      \textit{Algebraic Geometry and Arithmetic Curves}, 
      Oxford Graduate Text in Mathematics, 
      \textbf{6}, 2010.
\end{thebibliography}
\begin{thebibliography}{20}
    \par
    \bibitem[Hatshorne1]{Hartshorne1} R. Hartshorne, 
    \textit{Residues and duality}, Lecture Notes in Mathematics, 
    Vol.\ 20, Springer-Verlag, Berlin, 1966.

    \bibitem[Sato1]{Sato1} Mikio Sato, \textit{Theory of hyperfunctions II}, J. Fac.\ Sci.\ Univ.\ Tokyo, {\bf{8}} (1960), 387--437.

    \end{thebibliography}
\end{document}