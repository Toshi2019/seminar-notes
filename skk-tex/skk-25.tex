\section{Structure theorem for systems of pseudo-differential equations in the complex domain}
\markright{\scriptsize{2.5 Structure theorem for systems of pseudo-differential equations in the complex domain}}
In this section we establish the fundamental theorem 
concerning the structure of a system of pseudo-differential equations 
of finite order in complex domain at generic points, 
i.e., we will firstly prove in theorem \ref{thm512} 
as the simplest case that any system $\mathscr{M}$ of 
pseudo-differential equations of finite order with one unknown
function and simple characteristics can be transformed micro-locally 
into the partial de Rham systems
\begin{align*}
    \mathscr{N} \colon \frac{\partial}{\partial x'_i}u = 0, \quad i = 1,\ldots,d
\end{align*}
by a suitable ``quantized'' contact transformation.
Here ``micro-locally'' means ``locally on $P^\ast X$, not on $X$''. 
In the sequal we use the word ``micro-locally'' 
in this sense (and sometimes in the sense that ``locally on 
$S^{\ast}M$, not on $M$'' when we consider the problems in the real domain).
Later we extend Theorem\ref{thm512} to more general systems 
by the aid of pseudo-differential operators of infinite order.

\subsection{Structure theorem for systems of pseudo-differential equations with simple characteristics}

\begin{thm}\label{thm512}
    a
    % theorem5.1.2
\end{thm}

\subsection{Equivalence of pseudo-differential operators with constant multiple characteristics}

\begin{rem}
    This example shows that the structure of the hyperfunction
    solution sheaf, not merely the microfunction solution sheaf, of 
    the equation of $P_1(D)u = 0$ and that of $P_2(D)u = 0$ are the same, 
    because the operators $A_j(x,D)$ are differential operators, not merely 
    pseudo-differential operators. Note that, more generally, 
    if $P(x,D)$ is a linear differential operator of order $m$ 
    defined in a neighborfood of the origin of $\CC^n$ whose 
    principal symbol is $\iota^m_1$, then the differential equation 
    $P(x,D)u=0$ and $D_1^mu=0$ are equivalent as left $\DD$-modules.
\end{rem}
\subsection{Structure theorem for regular systems of pseudo-differential equations}
