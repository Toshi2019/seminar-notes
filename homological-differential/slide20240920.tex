%=====================================
%   homological-subdifferential-slide.tex
%   Homological Subdifferential について
%   toshi2019
%=====================================
\documentclass[dvipdfmx,12pt,aspectratio=169,leqno]{beamer}% dvipdfmxしたい
\title{Homological Subdifferential
\subtitle{本永さんゼミ}}
\date{2024年9月27日}
\author{大柴 寿浩}





    

\usetheme{Singapore}
\usecolortheme{rose}
\usefonttheme{professionalfonts}
\setbeamertemplate{navigaton symbols}{\false}
\setbeamertemplate{footline}[frame number]
%\usepackage{graphicx,xcolor}





\usepackage{bxdpx-beamer}% dvipdfmxなので必要
\usepackage{pxjahyper}% 日本語で'しおり'したい
\usepackage{minijs}% min10ヤダ
\renewcommand{\kanjifamilydefault}{\gtdefault}
\renewcommand{\emph}[1]{{\upshape\bfseries #1}}

\usepackage{amsmath}
\usepackage{amsthm}
\usepackage{tikz}
\usepackage{color}
\usepackage{ascmac}
\usepackage{amsfonts}
\usepackage{mathrsfs}
\usepackage[mathscr]{eucal}
\usepackage{mathtools}
\usepackage{amssymb}
\usepackage{graphicx}
\usepackage{fancybox}
\usepackage{enumerate}
\usepackage{verbatim}
\usepackage{subfigure}
\usepackage{proof}
\usepackage{listings}
\usepackage{otf}
\usepackage[all]{xy}
\usepackage{amscd}
%\usepackage[dvipdfmx]{hyperref}

\usepackage{xcolor}
\definecolor{darkgreen}{rgb}{0,0.45,0} 
\definecolor{darkred}{rgb}{0.75,0,0}
\definecolor{darkblue}{rgb}{0,0,0.6} 
\hypersetup{
    colorlinks=true,
    citecolor=darkgreen,
    linkcolor=darkblue,
    urlcolor=darkblue,
}

\usetikzlibrary{positioning}


\usepackage{latexsym}
\usepackage{wrapfig}
\usepackage{layout}
\usepackage{url}

\usepackage{okumacro}

\usepackage{comment}
%\usepackage{pxjahyper}
% ----------------------------
% commmand
% ----------------------------
% 執筆に便利なコマンド集です. 
% コマンドを追加する場合は下のスペースへ. 

% 集合の記号 (黒板文字)
\newcommand{\NN}{\mathbb{N}}
\newcommand{\ZZ}{\mathbb{Z}}
\newcommand{\QQ}{\mathbb{Q}}
\newcommand{\RR}{\mathbb{R}}
\newcommand{\CC}{\mathbb{C}}
\newcommand{\PP}{\mathbb{P}}
\newcommand{\KK}{\mathbb{K}}


% 集合の記号 (太文字)
\newcommand{\nn}{\mathbf{N}}
\newcommand{\zz}{\mathbf{Z}}
\newcommand{\qq}{\mathbf{Q}}
\newcommand{\rr}{\mathbf{R}}
\newcommand{\cc}{\mathbf{C}}
\newcommand{\pp}{\mathbf{P}}
\newcommand{\kk}{\mathbf{k}}

% 特殊な写像の記号
\newcommand{\ev}{\mathop{\mathrm{ev}}\nolimits} % 値写像
\newcommand{\pr}{\mathop{\mathrm{pr}}\nolimits} % 射影

% スクリプト体にするコマンド
%   例えば {\mcal C} のように用いる
\newcommand{\mcal}{\mathcal}

% 花文字にするコマンド 
%   例えば {\h C} のように用いる
\newcommand{\h}{\mathscr}

% ヒルベルト空間などの記号
\newcommand{\F}{\mcal{F}}
\newcommand{\X}{\mcal{X}}
\newcommand{\Y}{\mcal{Y}}
\newcommand{\Hil}{\mcal{H}}
\newcommand{\RKHS}{\Hil_{k}}
\newcommand{\Loss}{\mcal{L}_{D}}
\newcommand{\MLsp}{(\X, \Y, D, \Hil, \Loss)}

% 偏微分作用素の記号
\newcommand{\p}{\partial}

% 角カッコの記号 (内積は下にマクロがあります)
\newcommand{\lan}{\langle}
\newcommand{\ran}{\rangle}



% 圏の記号など
\newcommand{\Set}{{\bf Set}}
\newcommand{\Vect}{{\bf Vect}}
\newcommand{\FDVect}{{\bf FDVect}}
%\newcommand{\Ring}{{\bf Ring}}
\newcommand{\Ab}{{\bf Ab}}
\newcommand{\Mod}{\mathop{\mathrm{Mod}}\nolimits}
\newcommand{\Modf}{\mathop{\mathrm{Mod}^\mathrm{f}}\nolimits}
\newcommand{\CGA}{{\bf CGA}}
\newcommand{\GVect}{{\bf GVect}}
\newcommand{\Lie}{{\bf Lie}}
\newcommand{\dLie}{{\bf Liec}}



% 射の集合など
\newcommand{\Map}{\mathop{\mathrm{Map}}\nolimits} % 写像の集合
\newcommand{\Hom}{\mathop{\mathrm{Hom}}\nolimits} % 射集合
\newcommand{\End}{\mathop{\mathrm{End}}\nolimits} % 自己準同型の集合
\newcommand{\Aut}{\mathop{\mathrm{Aut}}\nolimits} % 自己同型の集合
\newcommand{\Mor}{\mathop{\mathrm{Mor}}\nolimits} % 射集合
\newcommand{\Ker}{\mathop{\mathrm{Ker}}\nolimits} % 核
\newcommand{\Img}{\mathop{\mathrm{Im}}\nolimits} % 像
\newcommand{\Cok}{\mathop{\mathrm{Coker}}\nolimits} % 余核
\newcommand{\Cim}{\mathop{\mathrm{Coim}}\nolimits} % 余像

% その他便利なコマンド
\newcommand{\dip}{\displaystyle} % 本文中で数式モード
\newcommand{\e}{\varepsilon} % イプシロン
\newcommand{\dl}{\delta} % デルタ
\newcommand{\pphi}{\varphi} % ファイ
\newcommand{\ti}{\tilde} % チルダ
\newcommand{\pal}{\parallel} % 平行
\newcommand{\op}{{\rm op}} % 双対を取る記号
\newcommand{\lcm}{\mathop{\mathrm{lcm}}\nolimits} % 最小公倍数の記号
\newcommand{\Probsp}{(\Omega, \F, \P)} 
\newcommand{\argmax}{\mathop{\rm arg~max}\limits}
\newcommand{\argmin}{\mathop{\rm arg~min}\limits}





% ================================
% コマンドを追加する場合のスペース 
%\newcommand{\OO}{\mcal{O}}



\makeatletter
\renewenvironment{proof}[1][\proofname]{\par
  \pushQED{\qed}%
  \normalfont \topsep6\p@\@plus6\p@\relax
  \trivlist
  \item[\hskip\labelsep
%        \itshape
         \bfseries
%    #1\@addpunct{.}]\ignorespaces
    {#1}]\ignorespaces
}{%
  \popQED\endtrivlist\@endpefalse
}
\makeatother

\renewcommand{\proofname}{\textrm{Proof.}}



%\renewcommand\proofname{\bf 証明} % 証明
\numberwithin{equation}{subsection}
\newcommand{\cTop}{\textsf{Top}}
%\newcommand{\cOpen}{\textsf{Open}}
\newcommand{\Op}{\mathop{\textsf{Op}}\nolimits}
\newcommand{\Ob}{\mathop{\textrm{Ob}}\nolimits}
\newcommand{\id}{\mathop{\mathrm{id}}\nolimits}
\newcommand{\pt}{\mathop{\mathrm{pt}}\nolimits}
\newcommand{\res}{\mathop{\rho}\nolimits}
\newcommand{\A}{\mcal{A}}
\newcommand{\B}{\mcal{B}}
\newcommand{\C}{\mcal{C}}
\newcommand{\D}{\mcal{D}}
\newcommand{\E}{\mcal{E}}
\newcommand{\G}{\mcal{G}}
%\newcommand{\H}{\mcal{H}}
\newcommand{\I}{\mcal{I}}
\newcommand{\J}{\mcal{J}}
\newcommand{\OO}{\mcal{O}}
\newcommand{\Ring}{\mathop{\textsf{Ring}}\nolimits}
\newcommand{\cAb}{\mathop{\textsf{Ab}}\nolimits}
%\newcommand{\Ker}{\mathop{\mathrm{Ker}}\nolimits}
\newcommand{\im}{\mathop{\mathrm{Im}}\nolimits}
\newcommand{\Coker}{\mathop{\mathrm{Coker}}\nolimits}
\newcommand{\Coim}{\mathop{\mathrm{Coim}}\nolimits}
\newcommand{\rank}{\mathop{\mathrm{rank}}\nolimits}
\newcommand{\Ht}{\mathop{\mathrm{Ht}}\nolimits}
\newcommand{\supp}{\mathop{\mathrm{supp}}\nolimits}
\newcommand{\colim}{\mathop{\mathrm{colim}}}
\newcommand{\Tor}{\mathop{\mathrm{Tor}}\nolimits}

\newcommand{\cat}{\mathscr{C}}

%筆記体
\newcommand{\cA}{\mcal{A}}
\newcommand{\cB}{\mcal{B}}
\newcommand{\cC}{\mcal{C}}
\newcommand{\cD}{\mcal{D}}
\newcommand{\cE}{\mcal{E}}
\newcommand{\cF}{\mcal{F}}
\newcommand{\cG}{\mcal{G}}
\newcommand{\cH}{\mcal{H}}
\newcommand{\cI}{\mcal{I}}
\newcommand{\cJ}{\mcal{J}}
\newcommand{\cK}{\mcal{K}}
\newcommand{\cL}{\mcal{L}}
\newcommand{\cM}{\mcal{M}}
\newcommand{\cN}{\mcal{N}}
\newcommand{\cO}{\mcal{O}}
\newcommand{\cP}{\mcal{P}}
\newcommand{\cQ}{\mcal{Q}}
\newcommand{\cR}{\mcal{R}}
\newcommand{\cS}{\mcal{S}}
\newcommand{\cT}{\mcal{T}}
\newcommand{\cU}{\mcal{U}}
\newcommand{\cV}{\mcal{V}}
\newcommand{\cW}{\mcal{W}}
\newcommand{\cX}{\mcal{X}}
\newcommand{\cY}{\mcal{Y}}
\newcommand{\cZ}{\mcal{Z}}


\newcommand{\scA}{\mathscr{A}}
\newcommand{\scB}{\mathscr{B}}
\newcommand{\scC}{\mathscr{C}}
\newcommand{\scD}{\mathscr{D}}
\newcommand{\scE}{\mathscr{E}}
\newcommand{\scF}{\mathscr{F}}
\newcommand{\scN}{\mathscr{N}}
\newcommand{\scO}{\mathscr{O}}
\newcommand{\scV}{\mathscr{V}}
\newcommand{\scU}{\mathscr{U}}


\newcommand{\ibA}{\mathop{\text{\textit{\textbf{A}}}}}
\newcommand{\ibB}{\mathop{\text{\textit{\textbf{B}}}}}
\newcommand{\ibC}{\mathop{\text{\textit{\textbf{C}}}}}
\newcommand{\ibD}{\mathop{\text{\textit{\textbf{D}}}}}
\newcommand{\ibE}{\mathop{\text{\textit{\textbf{E}}}}}
\newcommand{\ibF}{\mathop{\text{\textit{\textbf{F}}}}}
\newcommand{\ibG}{\mathop{\text{\textit{\textbf{G}}}}}
\newcommand{\ibH}{\mathop{\text{\textit{\textbf{H}}}}}
\newcommand{\ibI}{\mathop{\text{\textit{\textbf{I}}}}}
\newcommand{\ibJ}{\mathop{\text{\textit{\textbf{J}}}}}
\newcommand{\ibK}{\mathop{\text{\textit{\textbf{K}}}}}
\newcommand{\ibL}{\mathop{\text{\textit{\textbf{L}}}}}
\newcommand{\ibM}{\mathop{\text{\textit{\textbf{M}}}}}
\newcommand{\ibN}{\mathop{\text{\textit{\textbf{N}}}}}
\newcommand{\ibO}{\mathop{\text{\textit{\textbf{O}}}}}
\newcommand{\ibP}{\mathop{\text{\textit{\textbf{P}}}}}
\newcommand{\ibQ}{\mathop{\text{\textit{\textbf{Q}}}}}
\newcommand{\ibR}{\mathop{\text{\textit{\textbf{R}}}}}
\newcommand{\ibS}{\mathop{\text{\textit{\textbf{S}}}}}
\newcommand{\ibT}{\mathop{\text{\textit{\textbf{T}}}}}
\newcommand{\ibU}{\mathop{\text{\textit{\textbf{U}}}}}
\newcommand{\ibV}{\mathop{\text{\textit{\textbf{V}}}}}
\newcommand{\ibW}{\mathop{\text{\textit{\textbf{W}}}}}
\newcommand{\ibX}{\mathop{\text{\textit{\textbf{X}}}}}
\newcommand{\ibY}{\mathop{\text{\textit{\textbf{Y}}}}}
\newcommand{\ibZ}{\mathop{\text{\textit{\textbf{Z}}}}}

\newcommand{\ibx}{\mathop{\text{\textit{\textbf{x}}}}}

%\newcommand{\Comp}{\mathop{\mathrm{C}}\nolimits}
%\newcommand{\Komp}{\mathop{\mathrm{K}}\nolimits}
%\newcommand{\Domp}{\mathop{\mathsf{D}}\nolimits}%複体のホモトピー圏
%\newcommand{\Comp}{\mathrm{C}}
%\newcommand{\Komp}{\mathrm{K}}
%\newcommand{\Domp}{\mathsf{D}}%複体のホモトピー圏

\newcommand{\Comp}{\mathsf{C}}
\newcommand{\Komp}{\mathsf{K}}
\newcommand{\Domp}{\mathsf{D}}
\newcommand{\Kompl}{\mathop{\mathsf{K}^\mathrm{+}}\nolimits}
\newcommand{\Kompu}{\mathop{\mathsf{K}^\mathrm{-}}\nolimits}
\newcommand{\Kompb}{\mathop{\mathsf{K}^\mathrm{b}}\nolimits}
\newcommand{\Dompl}{\mathop{\mathsf{D}^\mathrm{+}}\nolimits}
\newcommand{\Dompu}{\mathop{\mathsf{D}^\mathrm{-}}\nolimits}
\newcommand{\Dompb}{\mathop{\mathsf{D}^\mathrm{b}}\nolimits}
\newcommand{\Dompbf}{\mathop{\mathsf{D}_\mathrm{f}^\mathrm{b}}\nolimits}




\newcommand{\CCat}{\Comp(\cat)}
\newcommand{\KCat}{\Komp(\cat)}
\newcommand{\DCat}{\Domp(\cat)}%圏Cの複体のホモトピー圏
\newcommand{\HOM}{\mathop{\mathscr{H}\hspace{-2pt}om}\nolimits}%内部Hom
\newcommand{\RHOM}{\mathop{\mathrm{R}\hspace{-1.5pt}\HOM}\nolimits}

\newcommand{\muS}{\mathop{\mathrm{SS}}\nolimits}
\newcommand{\RG}{\mathop{\mathrm{R}\hspace{-0pt}\Gamma}\nolimits}
\newcommand{\RHom}{\mathop{\mathrm{R}\hspace{-1.5pt}\Hom}\nolimits}
\newcommand{\Rder}{\mathrm{R}}

\newcommand{\simar}{\mathrel{\overset{\sim}{\rightarrow}}}%同型右矢印
\newcommand{\simarr}{\mathrel{\overset{\sim}{\longrightarrow}}}%同型右矢印
\newcommand{\simra}{\mathrel{\overset{\sim}{\leftarrow}}}%同型左矢印
\newcommand{\simrra}{\mathrel{\overset{\sim}{\longleftarrow}}}%同型左矢印

\newcommand{\hocolim}{{\mathrm{hocolim}}}
\newcommand{\indlim}[1][]{\mathop{\varinjlim}\limits_{#1}}
\newcommand{\sindlim}[1][]{\smash{\mathop{\varinjlim}\limits_{#1}}\,}
\newcommand{\Pro}{\mathrm{Pro}}
\newcommand{\Ind}{\mathrm{Ind}}
\newcommand{\prolim}[1][]{\mathop{\varprojlim}\limits_{#1}}
\newcommand{\sprolim}[1][]{\smash{\mathop{\varprojlim}\limits_{#1}}\,}

\newcommand{\Sh}{\mathrm{Sh}}
\newcommand{\PSh}{\mathrm{PSh}}

\newcommand{\rmD}{\mathrm{D}}

\newcommand{\Lloc}[1][]{\mathord{\mathcal{L}^1_{\mathrm{loc},{#1}}}}
\newcommand{\ori}{\mathord{\mathrm{or}}}
\newcommand{\Db}{\mathord{\mathscr{D}b}}

\newcommand{\codim}{\mathop{\mathrm{codim}}\nolimits}



\newcommand{\gld}{\mathop{\mathrm{gld}}\nolimits}
\newcommand{\wgld}{\mathop{\mathrm{wgld}}\nolimits}


\newcommand{\tens}[1][]{\mathbin{\otimes_{\raise1.5ex\hbox to-.1em{}{#1}}}}
\newcommand{\etens}{\mathbin{\boxtimes}}
\newcommand{\ltens}[1][]{\mathbin{\overset{\mathrm{L}}\tens}_{#1}}
\newcommand{\mtens}[1][]{\mathbin{\overset{\mathrm{\mu}}\tens}_{#1}}
\newcommand{\lltens}[1][]{{\mathop{\tens}\limits^{\mathrm{L}}_{#1}}}
\newcommand{\letens}{\overset{\mathrm{L}}{\etens}}
\newcommand{\detens}{\underline{\etens}}
\newcommand{\ldetens}{\overset{\mathrm{L}}{\underline{\etens}}}
\newcommand{\dtens}[1][]{{\overset{\mathrm{L}}{\underline{\otimes}}}_{#1}}

\newcommand{\blk}{\mathord{\ \cdot\ }}
\newcommand{\mres}[2][]{{\left.{#1}\right\rvert}_{#2}}


%\newcommand{\hocolim}{{\mathrm{hocolim}}}
%\newcommand{\indlim}[1][]{\mathop{\varinjlim}\limits_{#1}}
%\newcommand{\sindlim}[1][]{\smash{\mathop{\varinjlim}\limits_{#1}}\,}
%\newcommand{\Pro}{\mathrm{Pro}}
%\newcommand{\Ind}{\mathrm{Ind}}
%\newcommand{\prolim}[1][]{\mathop{\varprojlim}\limits_{#1}}
%\newcommand{\sprolim}[1][]{\smash{\mathop{\varprojlim}\limits_{#1}}\,}
\newcommand{\proolim}[1][]{\mathop{\text{\rm``{$\varprojlim$}''}}\limits_{#1}}
\newcommand{\sproolim}[1][]{\smash{\mathop{\rm``{\varprojlim}''}\limits_{#1}}}
\newcommand{\inddlim}[1][]{\mathop{\text{\rm``{$\varinjlim$}''}}\limits_{#1}}
\newcommand{\sinddlim}[1][]{\smash{\mathop{\text{\rm``{$\varinjlim$}''}}\limits_{#1}}\,}
\newcommand{\ooplus}{\mathop{\text{\rm``{$\oplus$}''}}\limits}
\newcommand{\bbigsqcup}{\mathop{``\bigsqcup''}\limits}
\newcommand{\bsqcup}{\mathop{``\sqcup''}\limits}
\newcommand{\dsum}[1][]{\mathbin{\oplus_{#1}}}

\newcommand{\Fct}{\mathop{\mathsf{Fct}}\nolimits}
\newcommand{\Red}{\mathop{\mathsf{Red}}\nolimits}
\newcommand{\Cone}{\mathop{\mathsf{Cone}}\nolimits}
\newcommand{\epi}{\mathop{\mathsf{epi}}\nolimits}





%================================================
% 自前の定理環境
%   https://mathlandscape.com/latex-amsthm/
% を参考にした
\newtheoremstyle{mystyle}%   % スタイル名
    {5pt}%                   % 上部スペース
    {5pt}%                   % 下部スペース
    {}%              % 本文フォント
    {}%                  % 1行目のインデント量
    {\bfseries}%                      % 見出しフォント
    {.}%                     % 見出し後の句読点
    {12pt}%                     % 見出し後のスペース
    {\thmname{#1}\thmnumber{ #2}\thmnote{{\hspace{2pt}\normalfont (#3)}}}% % 見出しの書式

\theoremstyle{mystyle}
\newtheorem{AXM}{公理}%[section]
\newtheorem{DFN}[AXM]{定義}
\newtheorem{THM}[AXM]{定理}
\newtheorem*{THM*}{定理}
\newtheorem{PRP}[AXM]{命題}
\newtheorem{LMM}[AXM]{補題}
\newtheorem{CRL}[AXM]{系}
\newtheorem{EG}[AXM]{例}
\newtheorem*{EG*}{例}
\newtheorem{RMK}[AXM]{注意}
\newtheorem{CNV}[AXM]{約束}
\newtheorem{CMT}[AXM]{コメント}
\newtheorem*{CMT*}{コメント}
\newtheorem{NTN}[AXM]{記号}
\newtheorem{CLM}[AXM]{Claim}
\newtheorem{Prop}[AXM]{Proposition}

% 定理環境ここまで
%====================================================

\usepackage{framed}
\definecolor{lightgray}{rgb}{0.75,0.75,0.75}
\renewenvironment{leftbar}{%
  \def\FrameCommand{\textcolor{lightgray}{\vrule width 4pt} \hspace{10pt}}% 
  \MakeFramed {\advance\hsize-\width \FrameRestore}}%
{\endMakeFramed}
\newenvironment{redleftbar}{%
  \def\FrameCommand{\textcolor{lightgray}{\vrule width 1pt} \hspace{10pt}}% 
  \MakeFramed {\advance\hsize-\width \FrameRestore}}%
 {\endMakeFramed}



\renewcommand{\Re}{\mathop{\mathrm{Re}}\nolimits}


% =================================





% ---------------------------
% new definition macro
% ---------------------------
% 便利なマクロ集です

% 内積のマクロ
%   例えば \inner<\pphi | \psi> のように用いる
\def\inner<#1>{\langle #1 \rangle}

% ================================
% マクロを追加する場合のスペース 

%=================================







\def\fracinline#1/#2{\mbox{\raise0.5ex\hbox{\footnotesize$#1$}{\hskip-.1em$/$\hskip-.1em}\raise-0.5ex\hbox{\footnotesize$#2$}}}

\begin{document}




\begin{frame}
    \titlepage
\end{frame}
\begin{comment}
    \begin{frame}\frametitle{書くこと}
    \begin{itemize}
        \item 特に興味を持った定理や理論の背景
        \item それを記述するための記号や概念の導入
        \item 正確な主張の紹介
        \item 証明の基本的なアイデアやアウトラインの紹介
        \item または定理の過程を満たす具体例や定理の応用例の紹介
    \end{itemize}
    \end{frame}
\end{comment}
%\setcounter{section}{3}
\section*{はじめに}
\begin{frame}
    \frametitle{目次}
    \tableofcontents
\end{frame}

\begin{frame}
    \frametitle{はじめに}
    \cite{Ike24}は導来圏を使わずに\(\SS(F)\)を定義している.

    \bigskip
    劣微分の定義をダイジェスト的に述べると
    \begin{itemize}
        \item \(f\colon X\to \rr\):下半連続関数
        \item \(\epi(f)\subset X\times \rr\):\(f\)のエピグラフ
        \item \(F_{f}=\kk_{\epi(f)}\):エピグラフに台を持つ定数層
        \item \(\SS(F_{f})\subset T^{\ast}X\times T^{\ast}\rr\):超局所台をとる
        \item \(\p f = -\Red(\SS(F_{f}))\subset T^{\ast}X\):「射影」(簡約)をとってひねる
    \end{itemize}
\end{frame}
\section[層]{層理論}
\subsection{定義}

\begin{frame}
    \frametitle{設定と記号}
    \begin{itemize}
        \item \(X\):位相空間
        \item \(\Op(X)\):開集合のなす順序集合
        \item \(I_P\):\(P\in X\)の開近傍のなす有向順序集合
        \item \(U\in \Op(X)\):開集合
        \item \(C^0(U)\):\(U\)上の連続関数環
        \item \((U_i)_{i\in I}\):\(U\)の被覆.
    \end{itemize}
\end{frame}

\begin{frame}
    \frametitle{簡単な例}

    連続関数に対する2つの操作:
    \begin{description}
        \item[制限]\(f\in C^0(U)\)の\(V\subset U\)上の\(f\rvert_V\in C^0(V)\)への制限.
        \[
            \text{\(U\supset V\supset W\)ならば,}
            \left.({f\rvert_V})\right\rvert_W=f\rvert_W.
        \]
        \item[はりあわせ]
        \((f_i)_{i}\in\prod_{i\in I}^{}C^{0}(U_i)\)で\(f_i=f_j\) on \(U_i\cap U_j\)のとき,
        \(f\rvert_{U_i}=f_i\)として\(f\in C^{0}(U)\)に貼り合わせ.
    \end{description}

    \bigskip
    \begin{itemize}
        \item 制限\(\to\)前層
        \item 貼り合わせ\(\to\)層
    \end{itemize}
    として定式化
\end{frame}



\begin{frame}
    \frametitle{前層の定義}

    \begin{definition}[前層]
        \(\kk\)加群の前層\(\mathscr{F}\)とは\(\kk\)加群の族と\(\kk\)加群の射の族の組\[\left(
        \left(\mathscr{F}(U)\right)_{U\in\Op(X)},
        \left(\rho_{VU}\colon\mathscr{F}(U)\to\mathscr{F}(V)\right)_{(V\hookrightarrow U)}
    \right)\]で次をみたすもののこと
    \begin{enumerate}
        %\renewcommand{\labelenumi}{({\arabic{enumi}})}
        \item $\rho_{UU} = \id_{\mathscr{F}(U)}$ 
        \item $W \subset V \subset U$ ならば, 
        $\rho_{WU} = \rho_{WV} \circ \rho_{VU}$.\label{enum:res}
    \end{enumerate}
    \end{definition}
\end{frame}
\begin{frame}
    \frametitle{前層の定義}

    \begin{definition}[前層の射]
        \(\kk\)加群の前層の射\(\mathscr{F}\to\mathscr{G}\)とは
        \(\kk\)加群の射の族\[
            (\phi_{U}\colon \mathscr{F}(U)
            \to \mathscr{G}(U))_{U\in\Op(X)}
        \]
        で次を可換にするもののこと
        \[\vcenter{\xymatrix@C=26pt@R=26pt{
            \mathscr{F}(U)
            \ar[r]^-{\phi_{U}}
            \ar[d]^-{\rho^{\mathscr{F}}_{VU}}
            &
            \mathscr{G}(U)
            \ar[d]^-{\rho^{\mathscr{G}}_{VU}} 
            \\
            \mathscr{F}(V)
            \ar[r]^-{\phi_{V}}
            &
            \mathscr{G}(V)
            }}
        \]
    \end{definition}
\end{frame}
\begin{frame}
    \frametitle{前層の例}
\begin{EG}
    \(X\)を位相空間とする.
    \begin{enumerate}
        \item \(C_X^r\colon U\mapsto C_X^r(U)\coloneqq\left\{\text{\(U\)上の\(C^r\)級関数}\right\}\)は\(X\)上の
        前層である.
        \item \(
            \mathscr{O}_X\colon U\mapsto 
            \mathscr{O}_X(U)\coloneqq\left\{\text{\(U\)上の正則関数}\right\}
        \)は\(X\)上の前層である.
        \item \(
            \mathscr{M}_X\colon U\mapsto 
            \mathscr{M}_X(U)\coloneqq\left\{\text{\(U\)上の有理型関数}\right\}
        \)は\(X\)上の前層である.
        \item \(M\)をアーベル群とする.
        \(U\mapsto M\)は\(X\)上の前層である.
    \end{enumerate}
\end{EG}
\end{frame}

\begin{frame}
    \frametitle{前層の例}
    \begin{EG}
        \(E\to X\)を\(C^{\infty}\)ベクトル束とする.
        \[
            U\mapsto \Gamma(U;E)\coloneqq
            \left\{s\colon U\to E;s\colon C^{\infty},\pi\circ s=\id_U\right\}
        \]は前層である.
    \end{EG}
\end{frame}

\begin{frame}
    \frametitle{層の定義}
    \begin{definition}[層]\label{def:sheaf}
        \(\mathscr{F}\):\(X\)上の\(\kk\)加群の前層.
        
        \(\mathscr{F}\)が\textbf{層} (sheaf) であるとは,
        \(\forall U\in\Op(X)\)と
        その開被覆\(\left(U_i\right)_{i\in I}\)に対して
        \begin{enumerate}[(S1)]
            \setcounter{enumi}{-1}
            \item \(\mathscr{F}(\varnothing)=0\).
            \item \(s\in\mathscr{F}(U)\)が各\(i\in I\)に対して\(U_i\)上\(s\rvert_{U_{i}}=0\)ならば,
            \(s=0\).\label{sheaf-cond1}
            \item \(\left(s_i\right)_i\in\prod_{i}\mathscr{F}(U_i)\)が
            各\(i,j\in I\)で\(U_i\cap U_j\ne\varnothing\)となるものに対して\(U_i\cap U_j\)上\(s_i\rvert_{U_i\cap U_j}-s_j\rvert_{U_i\cap U_j}=0\)ならば,
            \(s\in\mathscr{F}(U)\)で各\(U_i\)上\(s\rvert_i=s_i\)となるものが存在する.\label{sheaf-cond2}
        \end{enumerate}
    \end{definition}    
\end{frame}

\begin{frame}
    \frametitle{層の定義}
    層の射は前層としての射

    \bigskip
    記号を定める.
    \begin{itemize}
        \item \(\PSh(X)\coloneqq\left\{\text{\(X\)上の前層}\right\}\)
        \item \(\Sh(X)\coloneqq\left\{\text{\(X\)上の層}\right\}\)
        \item \(\Mod(\kk_X)\coloneqq\left\{\text{\(X\)上の\(\kk\)加群の層}\right\}\)
        \item \(\Ab\coloneqq\left\{\text{アーベル群}\right\}\)
    \end{itemize}
\end{frame}

\begin{frame}
    \frametitle{切断関手}

    \begin{DFN}[切断関手]
        %\(F\in\Sh(X)\), 
        \(U\in\Op(X)\):fixed
        \[
            \Gamma(U;\blk)\colon \Sh(X)\to \Ab;\quad F\mapsto
            \Gamma(U;F)=F(U)
        \]
    \end{DFN}    
\end{frame}



\begin{frame}
    \frametitle{層の例}
    \begin{EG}\label{eg:sheaf}
        \(X\)を位相空間とする.
        \begin{enumerate}
            \item 前層\(C_X^r\)は\(X\)上の層である.
            \item 前層\(\mathscr{O}_X\)は\(X\)上の層である.
            \item 前層\(\mathscr{M}_X\)は\(X\)上の層である.
            \item \(M\)をアーベル群とする.前層\(M\colon U\mapsto M\)は\(X\)上の層ではない.
            実際,\(X\)を2点からなる離散空間\(2=\left\{0,1\right\}\)とし,その上の
            前層として\(M=\cc\)を考えると,
            \(1\in\cc(\{0\})\), \(i\in\cc(\{1\})\)に対して,
            \(\cc(\{0\}\cap\{1\})=\cc(\varnothing)=0\)
            より,\(\rho_{\varnothing,{0}}(1)=0
            =\rho_{\varnothing,{1}}(i)\)となるが,
            \(2\)上の切断\(z\in\cc(2)\)で\(z\rvert_{\{0\}}=1,
            z\rvert_{\{1\}}=i\)をみたすものは存在しない.
            したがって,条件 (S\ref{sheaf-cond2}) が成り立たない.\label{eg:non-sheaf}
        \end{enumerate}
    \end{EG}    
\end{frame}

\begin{frame}
    \frametitle{前層の例}
    \begin{EG}
        \(E\to X\)を\(C^{\infty}\)ベクトル束とする.
        \[
            C^{\infty}(E)\colon U\mapsto \Gamma(U;E)\coloneqq
            \left\{s\colon U\to E;s\colon C^{\infty}, \pi\circ s=\id_U\right\}
        \]は層である.
    \end{EG}
\end{frame}


\subsection{層化と茎}

\begin{frame}
    \frametitle{層化}
    層ではない前層に対して,
    「最も近い」層を自然に対応させる.
    \begin{THM}\label{thm:sheafification}
        \(X\)上の任意の前層\(\mathscr{F}\)に対して,
        \(X\)上の層\(\mathscr{F}^\dagger\)と
        前層の射\(\theta\colon \mathscr{F}\to\mathscr{F}^\dagger\)で
        次の普遍性をみたすものが
        ただ一つ存在する.
        \begin{quote}
            任意の層 \(\mathcal{G}\in\Sh(X)\)と
            前層の射\(\varphi\colon \mathscr{F}\to\mathscr{G}\)に
            対し,層の射\(\varphi^\dagger\colon\mathscr{F}^\dagger
            \to\mathscr{G}\)で,
            \[\varphi^\dagger\circ \theta=\varphi\]をみたすものが
            ただ一つ存在する.
        \end{quote}
    \end{THM}    
\end{frame}

\begin{frame}
    \frametitle{茎の例から}
    \begin{itemize}
        \item \(X\):例えばリーマン面,\(U\subset X\):開集合
        \item 正則関数\(f\in\OO_{X}(U)\)は
        各点\(P\)のまわりで冪級数展開可能
        \item 関数\(f\in \mathscr{O}_X(U)\)と\(g\in \mathscr{O}_X(V)\)が\(P\)の近くで等しいとは
        
        \begin{quote}
            \(U\cap V\)に含まれる\(P\)の近傍\(W\)とって,
            \(P\)のまわりの座標\(z\colon W\to z(W)\)
            を用いて\(f,g\)を展開したとき,
            収束冪級数として等しいということ
        \end{quote}
    \end{itemize}

定義域の異なる関数(環)に対して,
各点まわりに注目するという操作を茎として定式化.

\end{frame}

\begin{frame}
    \frametitle{茎の定義}
    \begin{DFN}[茎]
        \begin{itemize}
            \item \(\mathscr{F}\in\Sh(X)\)
            \item \(P\in X\)
        \end{itemize}
        \(\mathscr{F}\)の\(P\)での\textbf{茎} (stalk) \(
            \mathscr{F}_P
        \)を次で定める.
        \[
            \mathscr{F}_P\coloneqq
            \varinjlim_{U\in I_P}\mathscr{F}(U).
        \]
    
        \(\mathscr{F}(U)\to\mathscr{F}_P\)から定まる
        \(s\in\mathscr{F}(U)\)の行き先を\(s_P\)とかき,
        \(s\)の\textbf{芽} (germ) という.
    \end{DFN}
\end{frame}

\begin{frame}
    \frametitle{茎の定義}
    \[
        \mathscr{F}_{P}\cong
        \left.\left(
            \bigsqcup_{U\in I_P}\mathscr{F}(U)
        \right)\right/\mathop{\sim}
    \]
ここで,\((f,U)\), \((g,V)\in \bigsqcup_{U\in I_P}\mathscr{F}(U)\)に
対し,\[
    (f,U)\sim(g,V)
    \colon\Longleftrightarrow
    \begin{cases}
        \text{\(P\)の開近傍\(W\in I_P\)で\(W\subset U\cap V\)と}\\
        \text{\(f\rvert_W=g\rvert_W\)をみたすものが存在する.}
    \end{cases}
\]
つまり,\(P\)のまわりでの挙動が同じ切断を
同一視した類が芽.
\end{frame}


\begin{frame}
    \frametitle{層化の構成}

    開集合\(U\)に対し,
    \[
        \mathscr{F}^\dagger(U)\coloneqq
        \left\{
            (s^P)_{P\in U}\in\prod_{P\in U}^{}\mathscr{F}_P;
            \begin{aligned}
                \text{任意の\(P\in U\)に対し,近傍\(W\in I_P\cap U\)と}\\
                \text{切断\(t\in\mathscr{F}(W)\)で,任意の\(Q\in W\)に対し,}\\            
                \text{\(s^Q=t_Q\)となるものが存在する.}            
            \end{aligned}
        \right\}
    \]
    とおき,
    制限射\(\mathscr{F}^\dagger(U)\to\mathscr{F}^\dagger(V)\)を
    \((s^P)_{P\in U}\mapsto(s^P)_{P\in V}\)で定めると,
    前層\(\mathscr{F}^\dagger\)が定まり,
    \(\mathscr{F}^\dagger\)が層の条件をみたすことも確かめられる.

    \begin{CMT}層化は茎を保つ.
        \(\mathscr{F}_P=\mathscr{F}^\dagger_P\)
    \end{CMT}
    
\end{frame}

\begin{frame}
    \frametitle{層化の例}

    \begin{EG}
        \(M\colon U\mapsto M\)をアーベル群\(M\)から定まる前層とする.
        \(M\)の層化\(M^\dagger\)は\[
            M^\dagger(U)=\left\{
            \text{\(U\)上の\(M\)に値をとる局所定数関数}\right\}
        \]である.
        \(M^\dagger\)を\(M_X\)とかき,\(X\)上の定数層と呼ぶ.
    \end{EG}

    \begin{PRP}
        \[M_X(U)\cong M^{\# \pi_0(U)}\]
        ここで,\(\pi_0(U)\)は\(U\)の連結成分の集合.
    \end{PRP}

\end{frame}





\section[層係数コホモロジー]{層係数コホモロジー}
\begin{frame}
    \frametitle{層の全射・単射}
    \begin{DFN}
        \(X\)上の層の射\(\varphi\colon\mathscr{F}\to\mathscr{G}\)が
        \begin{enumerate}
            \item 単射とは各点\(x\)に対し,\(\varphi_x\colon\mathscr{F}_x\to\mathscr{G}_x\)が単射.
            \item 全射とは各点\(x\)に対し,\(\varphi_x\colon\mathscr{F}_x\to\mathscr{G}_x\)が全射.
        \end{enumerate}
    \end{DFN}
\end{frame}

\begin{frame}
    \frametitle{層の完全列}
    \begin{DFN}
        \[
            \mathscr{F}\to
            \mathscr{G}\to
            \mathscr{H}
            \quad
            \text{in \(\Sh(X)\)}
        \]が完全列であるとは各点\(x\)に対し,
        \[
            \mathscr{F}_x\to
            \mathscr{G}_x\to
            \mathscr{H}_x
            \quad
            \text{in \(\Mod(\kk)\)}
        \]
        が完全列であることである.
    \end{DFN}    
\end{frame}

\begin{frame}
    \frametitle{切断は左完全}
    \begin{PRP}
        \[
            0\to F\to G\to H\to 0
            \quad
            \text{in \(\Sh(X)\)}
        \]が完全列のとき
        \[
            0\to \Gamma(U;F)\to \Gamma(U;F)\to \Gamma(U;H)
            \quad
            \text{in \(\Ab\)}
        \]は完全列.
    \end{PRP}
\end{frame}

\begin{frame}
    \frametitle{導来関手}
    右完全でない度合いを定量化する量としてコホモロジーを導入
\end{frame}








\section[導来圏]{導来圏}

\begin{frame}\frametitle{アーベル圏から導来圏まで}
    \[\vcenter{\xymatrix@C=32pt@R=32pt{
        \text{アーベル圏\(\cat\)}
        \ar@{=>}[d]^-{\text{複体をとる}}
        \\
        \text{複体の圏\(\CCat\)}
        \ar@{=>}[d]^-{\text{射の集合をホモトピー同値で割る}}
        \\
        \text{ホモトピー圏\(\KCat\)}
        \ar@{=>}[d]^-{\text{局所化}}
        \\
        \text{導来圏\(\DCat\)}
    }}\]
\end{frame}
\begin{frame}
    \frametitle{導来圏の動機}
    完全列
    \[
        0\to X\to Y\to Z\to 0
    \]
    を考える.
\end{frame}


\begin{frame}
    \frametitle{層係数コホモロジー}
    空間\(X\)の上の層\(\mathscr{F}\)について,
    コホモロジー
    \[
        H^i(X;\mathscr{F})
    \]は重要な量.
\end{frame}

\begin{frame}
    \frametitle{層係数コホモロジー}
    計算するときには,``良い層''\(\mathscr{I}^i\)たちで分解して
    \[\vcenter{\xymatrix@C=22pt@R=26pt{
        \text{もとの層}\mathscr{F}
        \ar@{~>}[d]^-{\text{分解}}
        \\ 
        \text{層の複体\((\mathscr{I}^i)\)}
        \ar@{~>}[d]^-{H^i}
        \\        
        \text{コホモロジー}H^i(X;\mathscr{I})
    }}\]
    のようにする
\end{frame}
\begin{frame}
    \frametitle{層係数コホモロジー}
    もっと一般に,アーベル圏\(\mathscr{A}, \mathscr{B}\)に対して,
    左完全関手\(F\colon \mathscr{A}\to \mathscr{B}\)があるとき,
    \[
        \rr^iF\colon\mathscr{A}\to \mathscr{B}
    \]
    を定めた
\end{frame}

\begin{frame}
    \frametitle{問題点}
    \begin{itemize}
        \item 導来関手どうしの合成の計算が面倒
        \item ``双対性''の定式化がしづらい
    \end{itemize}
\end{frame}

\begin{frame}
    \frametitle{複体の圏}
    \begin{Definition}[アーベル圏の複体の圏\(\mathsf{C}(\cat)\)]
        \begin{description}\setlength{\leftskip}{-25pt}
            \item[対象]:$\Ob(\Comp(\cat))=\left\{
                X=\left(
                    (X^n)_{n\in\zz},(d_X^{n})_{n\in\zz}
                \right); 
                d_X^{n+1}\circ d_X^{n}=0\quad  (n\in\zz)
                \right\}$
            \item[射]:$\Hom_{\Comp(\cat)}(X,Y)
            =\{\text{$\cat$の複体の射}\}$
        \end{description}
    \end{Definition}

複体の射\(f\colon X\to Y\)は,
射の族$(f^n\colon X^n\to Y^n)_{n\in\zz}$で,次を可換にするもの.
\begin{equation*}
    \vcenter{\xymatrix@C=26pt@R=26pt{
    \cdots \ar[r]
    & 
    X^{n}
    \ar[r]^-{d_X^n}
    \ar[d]^-{f^n}
    &
    X^{n+1}
    \ar[r]
    \ar[d]^-{f^{n+1}} 
    &\cdots
    \\
    \cdots \ar[r]
    & 
    Y^{n}
    \ar[r]^-{d_Y^n}
    &
    Y^{n+1}
    \ar[r]
    &\cdots
    }}
\end{equation*}
\end{frame}

\begin{frame}
    \frametitle{ホモトピー圏}
\begin{Definition}[$f,g\colon X\to Y$ in $\CCat$が0にホモトピック]
    $\cat$の射の族$(s^n\colon X^n\to Y^{n-1})$で,
    \begin{equation*}
        f^n-g^n=d_Y^{n-1}\circ s^n+s^{n+1}\circ d_X^{n}\quad(n\in\zz)
    \end{equation*}
    となるものが存在すること.
\end{Definition}    
\begin{equation*}
    \vcenter{\xymatrix@C=26pt@R=26pt{
    \cdots \ar[r]
    & 
    X^{n}
    \ar[r]%^-{d_X^n}
    \ar[d]
    &
    X^{n}
    \ar[r]^-{d_X^n}
    \ar[d]^-{f^n-g^n}
    \ar[dl]_-{s^n}
    &
    X^{n+1}
    \ar[r]
    \ar[d]
    \ar[dl]^-{s^{+1}}
    &\cdots
    \\
    \cdots \ar[r]
    & 
    Y^{n-1}
    \ar[r]_-{d_Y^{n-1}}
    &
    Y^{n}
    \ar[r]%_-{d_Y^n}
    &
    Y^{n+1}
    \ar[r]
    &\cdots
    }}
\end{equation*}
%    $f-g\colon X\to Y$が0にホモトピックであるとき,
%    $f$と$g$はホモトピック ($f\simeq g$)
\end{frame}
\begin{frame}
    \frametitle{ホモトピー圏}
\begin{Definition}[$f,g\colon X\to Y$ in $\CCat$が0にホモトピック]
    $\cat$の射の族$(s^n\colon X^n\to Y^{n-1})$で,
    \begin{equation*}
        f^n-g^n=d_Y^{n-1}\circ s^n+s^{n+1}\circ d_X^{n}\quad(n\in\zz)
    \end{equation*}
    となるものが存在すること.
\end{Definition}    
これは同値関係.
\[\Ht(X,Y)\coloneqq\{f\colon X\to Y\text{のホモトピー類}\}\]
とおく.
\end{frame}
\begin{frame}\frametitle{ホモトピー圏}
    \begin{Definition}[圏$\cat$のホモトピー圏$\KCat$]
        \begin{description}\setlength{\leftskip}{-25pt}
            \item[対象]:$\Ob(\KCat)=\Ob(\CCat)$
            \item[射]:$\Hom_{\KCat}(X,Y)=\Hom_{\CCat}(X,Y)/\Ht(X,Y)$
        \end{description}        
    \end{Definition}
    \begin{alertblock}{注意} 
        $\KCat$はアーベル圏ではない!
    \end{alertblock}
    完全列の代わりに完全三角というものを用いる.
\end{frame}
\begin{frame}\frametitle{写像錐}
    \begin{Definition}[$f$の写像錐$M(f)$]
        \begin{align*}
            \begin{cases}
                M(f)^n=X^{n+1}\oplus Y^{n},\\
                d^{n}_{M(f)}=\begin{bmatrix*}
                    d_{X[1]}^n & 0\\
                    f^{n+1} & d_Y^n\\
                \end{bmatrix*}
            \end{cases}
        \end{align*}
    \end{Definition}
\end{frame}
\begin{frame}\frametitle{写像錐}
    射$\alpha(f)\colon Y\to M(f)$と$\beta(f)\colon M(f)\to X[1]$を
    次で定める.
    \begin{align}
        \alpha(f)^{n}=\begin{bmatrix*}
            0\\\id_{Y^n}
        \end{bmatrix*},\\
        \beta(f)^{n}=\begin{bmatrix*}
            \id_{X^{n+1}}&0
        \end{bmatrix*}.
    \end{align}    
\end{frame}
\begin{frame}\frametitle{ホモトピー圏$\KCat$は三角圏}
    \begin{definition}[三角]
        射の列
        \[
            X\to Y\to Z\to X[1] \quad\text{in }\KCat    
        \]
        を三角という
    \end{definition}
    \[
        X\to Y\to M(f)\to X[1]
    \]
    と同形な$\triangle$を完全三角という
\end{frame}

\section{層の導来圏}
\begin{frame}\frametitle{層の複体の圏}
    \begin{Definition}[層の導来圏\(\Domp^b(\kk_X)\)]
        \[
            \Domp^b(\kk_X)\coloneqq \Domp^b(\Mod(\kk_X))
        \]
    \end{Definition}
\end{frame}

\section{層の超局所台}
\begin{frame}\frametitle{層の超局所台}
    \(\Dompb(\kk_X)\ni F\mapsto \SS(F)\subset T^{\ast}X\)
    \begin{Definition}[層の超局所台]
        \(p\notin \SS(F)\)となるのは,
        \(\exists U\in I_{p}\)で\(
            \forall x_0\in X,
            \varphi\colon X\to\rr
        \)で\(d\varphi(x_0)\in U\)となるものに対し,\[
            \indlim[x_0\in B]H^{n}(B;F)\simarr
            \indlim[x_0\in B]H^{n}(B\cap \{\varphi<\varphi(x_0)\};F)
        \]
        となるものが存在するとき.
    \end{Definition}
\end{frame}

\begin{frame}\frametitle{超局所台の性質}
    \(F, F_i~ (i=1,2,3) \in\Dompb(\kk_M)\)
    \begin{itemize}
        \item 超局所台は\(T^{\ast}M\)の錐状閉集合
        \item \(\SS(F)\cap{T^{\ast}_{M}M}=\supp(F)\)
        \item \(F_1\to F_2\to F_3\to+1\): d.t.のとき,\(\SS(F_i)\subset\SS(F_j)\cup\SS(F_k)\) (\(j\ne k\))
    \end{itemize}
\end{frame}

\begin{frame}\frametitle{超局所台の性質}
    \(F\in\Dompb(\kk_M)\)
    \begin{THM}
        \(SS(F)\)は包合的.
    \end{THM}
\end{frame}

\begin{frame}\frametitle{超局所台の例}
    \begin{enumerate}[a)]
        \item \(F\)を連結多様体\(M\)上の零でない局所系とすると\(SS(F)=T^{\ast}_{M}M\).
        \item \(Z\subset M\):滑らかな閉部分多様体,\(F=\kk_Z\)とすると\(\SS(F)=T^{\ast}_{Z}M\).
        \item \(\phi\)を\(\phi(x)=0\)となる\(x\)で\(d\phi(x)\ne0\)となる\(C^1\)級関数とする.\(Z\coloneqq\left\{x\in M;\phi(x)\geqq0\right\}\)とすると\[
            \SS(\kk_Z)
            =Z\times_{M}T^{\ast}_{M}M
            \cup
            \left\{
                (x;\lambda d\phi(x));\phi(x)=0,\lambda\geqq0
            \right\}.
        \]
    \end{enumerate}
\end{frame}


\section[劣微分]{ホモロジカル劣微分}

\begin{frame}
    \frametitle{補助的な定義}

    \begin{itemize}
        \item \(X\):\(C^\infty\) 多様体
        \item \(J^{1}(X)\coloneqq T^{\ast}X\times \rr\):1ジェット空間,\((x,t;\xi)\)
        \item \(\tilde{\lambda}=\lambda+dt\):\(J^{1}(X)\)の接触形式
    \end{itemize}
    
    \[
        \vcenter{\xymatrix@C=26pt@R=26pt{
        \left\{\tau>0\right\}\cap T^{\ast}(X\times\rr)
        \ar[rd]_-{\tilde{\rho}_{t}}
        \ar[rr]^-{\rho_{t}}
        &&
        T^{\ast}X
        \\
        &
        J^{1}(X)
        \ar[ru]_-{r}
      }}
    \]
    \[
        \rho_{t}\left(x,t;\xi,\tau\right)
        \coloneqq
        \left(x,\frac{\xi}{\tau}\right),
        \quad
        \tilde{\rho}_{t}\left(x,t;\xi,\tau\right)
        \coloneqq
        \left(x,\frac{\xi}{\tau},t\right),
        \quad
        r\colon\text{projection}.
    \]

\end{frame}

\begin{frame}
    \frametitle{錐化と簡約}
    \begin{definition}[錐化集合 (conification)]
        \begin{itemize}
            \item \(L\subset T^{\ast}X\):滑らかなラグランジュ部分多様体
        \end{itemize}
        \[
            \Cone(L)\coloneqq
            \rho_{t}^{-1}(L)\subset 
            T^{\ast}X\times T^{\ast}\rr
            \colon\text{\((n+2)\)次元部分多様体}
        \]
    \end{definition}
    
    \begin{definition}[簡約集合 (reduction)]
        \begin{itemize}
            \item \(A\subset T^{\ast}X\times T^{\ast}\rr\):錐状部分集合
        \end{itemize}
        \[\Red(A)\coloneqq\rho_{t}(A\cap\left\{\tau>0\right\})\]
    \end{definition}

\end{frame}
\begin{frame}
    \frametitle{錐化と簡約}

    \begin{CLM}
        \(\Red(\Cone(L))=L\subset T^{\ast}X\)
    \end{CLM}
\end{frame}


\begin{frame}
    \frametitle{層の representative}

    \begin{itemize}
        \item \(X\): \(C^\infty\) 多様体
        \item \(\mathbf{k}\): 単位元を持つ大域次元が有限な可換環
    \end{itemize}
    
    \begin{definition}[層の representative]
        \begin{itemize}
            \item \(F\in \Dompb(\kk_{X\times \rr})\)に対し,
            次の集合\(\mathrm{R}(F)\)を\(F\)の representative という.
        \end{itemize}
        \[\mathrm{R}(F)\coloneqq\Red(\SS(F))\subset T^{\ast}X\]
    \end{definition}

\end{frame}

\begin{frame}
    \frametitle{エピグラフ}
    
    \begin{definition}[エピグラフ (epigraph)]
        \begin{itemize}
            \item \(f\colon X\to \rr\)に対し,
            次の集合\(\epi(f)\)を\(f\)のエピグラフという.
        \end{itemize}
        \[
            \epi(f)\coloneqq
            \left\{(x,t)\in X\times \rr;f(x)\leqq t\right\}.
        \]
    \end{definition}
    \(\epi(f)\)を\(Z_f\)ともかく.
    エピグラフに台を持つ層を\(
            F_{f}\coloneqq\kk_{\epi(f)}\)で表す.
    \begin{definition}[ホモロジカル劣微分]
        下半連続関数\(f\colon X\to \rr\)に対し,
        \(f\)の劣微分\(\p f\)とは
        \[\p f\coloneqq \mathrm{R}(F_{f})^{a}\]
    \end{definition}

\end{frame}

\begin{frame}
    \frametitle{\(C^1\)級関数の劣微分は1点}
    \begin{NTN}
        \begin{itemize}
            \item \(\mres[\partial f]{x}=\partial f\cap T^{\ast}_{x}X\)
            \item \(\partial f(x)=\mres[\partial f]{x}\)
        \end{itemize}
    \end{NTN}
    \begin{EG}
        \(f\in C^{1}(X)\)とすると\[
            \mres[\partial f]{x}=\left\{df(x)\right\}.
        \]
    \end{EG}
\end{frame}

\begin{frame}
    \frametitle{劣微分の特徴づけ}
    劣微分は\(\epi(f)\)上の定数層の超局所台の計算に帰着.

    \(\leadsto\)等位集合コホモロジーの局所的な振る舞いを見れば良い.
    \begin{DFN}
        \(f\colon X\to \rr\):連続関数 (!?)
        \begin{itemize}
            \item \(x\in X\)が\(f\)の特異点とは次が同型でないことをいう.\[
                \indlim[U\ni x,\epsilon\to0]
                H^{\ast}(U\cap f^{<\epsilon+a})
                \to
                \indlim[U\ni x,\epsilon\to0]
                H^{\ast}(U\cap f^{<a})
                \quad\text{ただし\(a=f(x)\)}
            \]
            \item \(x\)が臨界点とは,
            列\((\phi_n,x_n)\in C^{1}(X)\times X\)で
            \(f-\phi_n\)が\(x_n\)で特異かつ\(x_n\to x\)かつ
            \(d\phi_n(x_n)\to0\)となるものが存在することをいう.
        \end{itemize}
    \end{DFN}
\end{frame}

\begin{frame}
    \frametitle{劣微分の特徴づけ}
    \begin{PRP}
        \(f\colon X\to \rr\).
        \(\xi\in\mres[\partial f]{x}\)となるのは,
        \(x\)が\(x\mapsto f(x)-\inner<\xi,x>\)の臨界点となるときである.
    \end{PRP}
\end{frame}


\section{References}

\begin{frame}[allowframebreaks]{References}
    \begin{thebibliography}{90}\beamertemplatetextbibitems
        \bibitem[Ike24]{Ike24} 池 祐一, 
        層理論と層のモース理論, \url{https://drive.google.com/file/d/1x1ibUAqXxNQHSLJrIYg72Rbu6O-043JZ/view?usp=drive_link}.
        \bibitem[KS90]{KS90} Kashiwara, Schapira, 
        \textit{Sheaves on Manifolds}, Springer, 1990.
        \bibitem[Vic13]{Vic13} Vichery, 
        \textit{Homological Differential Calculus}, \url{https://doi.org/10.48550/arXiv.1310.4845}.
    \end{thebibliography}
\end{frame}

\begin{comment}
\begin{frame}[noframenumbering]
aaa
\end{frame}
\end{comment}

\end{document}