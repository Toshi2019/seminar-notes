%=====================================
%   current.tex
%   カレントまとめノート
%   2024/01/20 -
%=====================================

% -----------------------
% preamble
% -----------------------
% ここから本文 (\begin{document}) までの
% ソースコードに変更を加えた場合は
% 編集者まで連絡してください. 
% Don't change preamble code yourself. 
% If you add something
% (usepackage, newtheorem, newcommand, renewcommand),
% please tell it 
% to the editor of institutional paper of RUMS.

% ------------------------
% documentclass
% ------------------------
\documentclass[11pt, a4paper, dvipdfmx]{jsarticle}

% ------------------------
% usepackage
% ------------------------
\usepackage{algorithm}
\usepackage{algorithmic}
\usepackage{amscd}
\usepackage{amsfonts}
\usepackage{amsmath}
\usepackage[psamsfonts]{amssymb}
\usepackage{amsthm}
\usepackage{ascmac}
\usepackage{color}
\usepackage{enumerate}
\usepackage{fancybox}
\usepackage[stable]{footmisc}
\usepackage{graphicx}
\usepackage{listings}
\usepackage{mathrsfs}
\usepackage{mathtools}
\usepackage{otf}
\usepackage{pifont}
\usepackage{proof}
\usepackage{subfigure}
\usepackage{tikz}
\usepackage{verbatim}
\usepackage[all]{xy}

\usetikzlibrary{cd}



% ================================
% パッケージを追加する場合のスペース 
\usepackage[dvipdfmx]{hyperref}
\usepackage{xcolor}
\definecolor{darkgreen}{rgb}{0,0.45,0} 
\definecolor{darkred}{rgb}{0.75,0,0}
\definecolor{darkblue}{rgb}{0,0,0.6} 
\hypersetup{
    colorlinks=true,
    citecolor=darkgreen,
    linkcolor=darkred,
    urlcolor=darkblue,
}
\usepackage{pxjahyper}

%=================================


% --------------------------
% theoremstyle
% --------------------------
\theoremstyle{definition}

% --------------------------
% newtheoem
% --------------------------

% 日本語で定理, 命題, 証明などを番号付きで用いるためのコマンドです. 
% If you want to use theorem environment in Japanece, 
% you can use these code. 
% Attention!
% All theorem enivironment numbers depend on 
% only section numbers.
\newtheorem{Axiom}{公理}[section]
\newtheorem{Definition}[Axiom]{定義}
\newtheorem{Theorem}[Axiom]{定理}
\newtheorem{Proposition}[Axiom]{命題}
\newtheorem{Lemma}[Axiom]{補題}
\newtheorem{Corollary}[Axiom]{系}
\newtheorem{Example}[Axiom]{例}
\newtheorem{Claim}[Axiom]{主張}
\newtheorem{Property}[Axiom]{性質}
\newtheorem{Attention}[Axiom]{注意}
\newtheorem{Question}[Axiom]{問}
\newtheorem{Problem}[Axiom]{問題}
\newtheorem{Consideration}[Axiom]{考察}
\newtheorem{Alert}[Axiom]{警告}
\newtheorem{Fact}[Axiom]{事実}


% 日本語で定理, 命題, 証明などを番号なしで用いるためのコマンドです. 
% If you want to use theorem environment with no number in Japanese, You can use these code.
\newtheorem*{Axiom*}{公理}
\newtheorem*{Definition*}{定義}
\newtheorem*{Theorem*}{定理}
\newtheorem*{Proposition*}{命題}
\newtheorem*{Lemma*}{補題}
\newtheorem*{Example*}{例}
\newtheorem*{Corollary*}{系}
\newtheorem*{Claim*}{主張}
\newtheorem*{Property*}{性質}
\newtheorem*{Attention*}{注意}
\newtheorem*{Question*}{問}
\newtheorem*{Problem*}{問題}
\newtheorem*{Consideration*}{考察}
\newtheorem*{Alert*}{警告}
\newtheorem{Fact*}{事実}


% 英語で定理, 命題, 証明などを番号付きで用いるためのコマンドです. 
% If you want to use theorem environment in English, You can use these code.
%all theorem enivironment number depend on only section number.
\newtheorem{Axiom+}{Axiom}[section]
\newtheorem{Definition+}[Axiom+]{Definition}
\newtheorem{Theorem+}[Axiom+]{Theorem}
\newtheorem{Proposition+}[Axiom+]{Proposition}
\newtheorem{Lemma+}[Axiom+]{Lemma}
\newtheorem{Example+}[Axiom+]{Example}
\newtheorem{Corollary+}[Axiom+]{Corollary}
\newtheorem{Claim+}[Axiom+]{Claim}
\newtheorem{Property+}[Axiom+]{Property}
\newtheorem{Attention+}[Axiom+]{Attention}
\newtheorem{Question+}[Axiom+]{Question}
\newtheorem{Problem+}[Axiom+]{Problem}
\newtheorem{Consideration+}[Axiom+]{Consideration}
\newtheorem{Alert+}{Alert}
\newtheorem{Fact+}[Axiom+]{Fact}
\newtheorem{Remark+}[Axiom+]{Remark}

% ----------------------------
% commmand
% ----------------------------
% 執筆に便利なコマンド集です. 
% コマンドを追加する場合は下のスペースへ. 

% 集合の記号 (黒板文字)
\newcommand{\NN}{\mathbb{N}}
\newcommand{\ZZ}{\mathbb{Z}}
\newcommand{\QQ}{\mathbb{Q}}
\newcommand{\RR}{\mathbb{R}}
\newcommand{\CC}{\mathbb{C}}
\newcommand{\PP}{\mathbb{P}}
\newcommand{\KK}{\mathbb{K}}


% 集合の記号 (太文字)
\newcommand{\nn}{\mathbf{N}}
\newcommand{\zz}{\mathbf{Z}}
\newcommand{\qq}{\mathbf{Q}}
\newcommand{\rr}{\mathbf{R}}
\newcommand{\cc}{\mathbf{C}}
\newcommand{\pp}{\mathbf{P}}
\newcommand{\kk}{\mathbf{K}}

% 特殊な写像の記号
\newcommand{\ev}{\mathop{\mathrm{ev}}\nolimits} % 値写像
\newcommand{\pr}{\mathop{\mathrm{pr}}\nolimits} % 射影

% スクリプト体にするコマンド
%   例えば {\mcal C} のように用いる
\newcommand{\mcal}{\mathcal}

% 花文字にするコマンド 
%   例えば {\h C} のように用いる
\newcommand{\h}{\mathscr}

% ヒルベルト空間などの記号
\newcommand{\F}{\mcal{F}}
\newcommand{\X}{\mcal{X}}
\newcommand{\Y}{\mcal{Y}}
\newcommand{\Hil}{\mcal{H}}
\newcommand{\RKHS}{\Hil_{k}}
\newcommand{\Loss}{\mcal{L}_{D}}
\newcommand{\MLsp}{(\X, \Y, D, \Hil, \Loss)}

% 偏微分作用素の記号
\newcommand{\p}{\partial}

% 角カッコの記号 (内積は下にマクロがあります)
\newcommand{\lan}{\langle}
\newcommand{\ran}{\rangle}



% 圏の記号など
\newcommand{\Set}{{\bf Set}}
\newcommand{\Vect}{{\bf Vect}}
\newcommand{\FDVect}{{\bf FDVect}}
%\newcommand{\Ring}{{\bf Ring}}
\newcommand{\Ab}{{\bf Ab}}
\newcommand{\Mod}{\mathop{\mathrm{Mod}}\nolimits}
\newcommand{\CGA}{{\bf CGA}}
\newcommand{\GVect}{{\bf GVect}}
\newcommand{\Lie}{{\bf Lie}}
\newcommand{\dLie}{{\bf Liec}}



% 射の集合など
\newcommand{\Map}{\mathop{\mathrm{Map}}\nolimits} % 写像の集合
\newcommand{\Hom}{\mathop{\mathrm{Hom}}\nolimits} % 射集合
\newcommand{\End}{\mathop{\mathrm{End}}\nolimits} % 自己準同型の集合
\newcommand{\Aut}{\mathop{\mathrm{Aut}}\nolimits} % 自己同型の集合
\newcommand{\Mor}{\mathop{\mathrm{Mor}}\nolimits} % 射集合
\newcommand{\Ker}{\mathop{\mathrm{Ker}}\nolimits} % 核
\newcommand{\Img}{\mathop{\mathrm{Im}}\nolimits} % 像
\newcommand{\Cok}{\mathop{\mathrm{Coker}}\nolimits} % 余核
\newcommand{\Cim}{\mathop{\mathrm{Coim}}\nolimits} % 余像

% その他便利なコマンド
\newcommand{\dip}{\displaystyle} % 本文中で数式モード
\newcommand{\e}{\varepsilon} % イプシロン
\newcommand{\dl}{\delta} % デルタ
\newcommand{\pphi}{\varphi} % ファイ
\newcommand{\ti}{\tilde} % チルダ
\newcommand{\pal}{\parallel} % 平行
\newcommand{\op}{{\rm op}} % 双対を取る記号
\newcommand{\lcm}{\mathop{\mathrm{lcm}}\nolimits} % 最小公倍数の記号
\newcommand{\Probsp}{(\Omega, \F, \P)} 
\newcommand{\argmax}{\mathop{\rm arg~max}\limits}
\newcommand{\argmin}{\mathop{\rm arg~min}\limits}





% ================================
% コマンドを追加する場合のスペース 
%\newcommand{\OO}{\mcal{O}}



\renewcommand\proofname{\bf 証明} % 証明
\numberwithin{equation}{section}
\newcommand{\cTop}{\textsf{Top}}
%\newcommand{\cOpen}{\textsf{Open}}
\newcommand{\Op}{\mathop{\textsf{Open}}\nolimits}
\newcommand{\Ob}{\mathop{\textrm{Ob}}\nolimits}
\newcommand{\id}{\mathop{\mathrm{id}}\nolimits}
\newcommand{\pt}{\mathop{\mathrm{pt}}\nolimits}
\newcommand{\res}{\mathop{\rho}\nolimits}
\newcommand{\A}{\mcal{A}}
\newcommand{\B}{\mcal{B}}
\newcommand{\C}{\mcal{C}}
\newcommand{\D}{\mcal{D}}
\newcommand{\E}{\mcal{E}}
\newcommand{\G}{\mcal{G}}
%\newcommand{\H}{\mcal{H}}
\newcommand{\I}{\mcal{I}}
\newcommand{\J}{\mcal{J}}
\newcommand{\OO}{\mcal{O}}
\newcommand{\Ring}{\mathop{\textsf{Ring}}\nolimits}
\newcommand{\cAb}{\mathop{\textsf{Ab}}\nolimits}
%\newcommand{\Ker}{\mathop{\mathrm{Ker}}\nolimits}
\newcommand{\im}{\mathop{\mathrm{Im}}\nolimits}
\newcommand{\Coker}{\mathop{\mathrm{Coker}}\nolimits}
\newcommand{\Coim}{\mathop{\mathrm{Coim}}\nolimits}
\newcommand{\rank}{\mathop{\mathrm{rank}}\nolimits}
\newcommand{\Ht}{\mathop{\mathrm{Ht}}\nolimits}
\newcommand{\supp}{\mathop{\mathrm{supp}}\nolimits}
\newcommand{\colim}{\mathop{\mathrm{colim}}}
\newcommand{\Tor}{\mathop{\mathrm{Tor}}\nolimits}

\newcommand{\cat}{\mathscr{C}}

\newcommand{\scA}{\mathscr{A}}
\newcommand{\scB}{\mathscr{B}}
\newcommand{\scC}{\mathscr{C}}
\newcommand{\scD}{\mathscr{D}}
\newcommand{\scE}{\mathscr{E}}
\newcommand{\scF}{\mathscr{F}}

\newcommand{\ibA}{\mathop{\text{\textit{\textbf{A}}}}}
\newcommand{\ibB}{\mathop{\text{\textit{\textbf{B}}}}}
\newcommand{\ibC}{\mathop{\text{\textit{\textbf{C}}}}}
\newcommand{\ibD}{\mathop{\text{\textit{\textbf{D}}}}}
\newcommand{\ibE}{\mathop{\text{\textit{\textbf{E}}}}}
\newcommand{\ibF}{\mathop{\text{\textit{\textbf{F}}}}}
\newcommand{\ibG}{\mathop{\text{\textit{\textbf{G}}}}}
\newcommand{\ibH}{\mathop{\text{\textit{\textbf{H}}}}}
\newcommand{\ibI}{\mathop{\text{\textit{\textbf{I}}}}}
\newcommand{\ibJ}{\mathop{\text{\textit{\textbf{J}}}}}
\newcommand{\ibK}{\mathop{\text{\textit{\textbf{K}}}}}
\newcommand{\ibL}{\mathop{\text{\textit{\textbf{L}}}}}
\newcommand{\ibM}{\mathop{\text{\textit{\textbf{M}}}}}
\newcommand{\ibN}{\mathop{\text{\textit{\textbf{N}}}}}
\newcommand{\ibO}{\mathop{\text{\textit{\textbf{O}}}}}
\newcommand{\ibP}{\mathop{\text{\textit{\textbf{P}}}}}
\newcommand{\ibQ}{\mathop{\text{\textit{\textbf{Q}}}}}
\newcommand{\ibR}{\mathop{\text{\textit{\textbf{R}}}}}
\newcommand{\ibS}{\mathop{\text{\textit{\textbf{S}}}}}
\newcommand{\ibT}{\mathop{\text{\textit{\textbf{T}}}}}
\newcommand{\ibU}{\mathop{\text{\textit{\textbf{U}}}}}
\newcommand{\ibV}{\mathop{\text{\textit{\textbf{V}}}}}
\newcommand{\ibW}{\mathop{\text{\textit{\textbf{W}}}}}
\newcommand{\ibX}{\mathop{\text{\textit{\textbf{X}}}}}
\newcommand{\ibY}{\mathop{\text{\textit{\textbf{Y}}}}}
\newcommand{\ibZ}{\mathop{\text{\textit{\textbf{Z}}}}}

\newcommand{\ibx}{\mathop{\text{\textit{\textbf{x}}}}}

%\newcommand{\Comp}{\mathop{\mathrm{C}}\nolimits}
%\newcommand{\Komp}{\mathop{\mathrm{K}}\nolimits}
%\newcommand{\Domp}{\mathop{\mathsf{D}}\nolimits}%複体のホモトピー圏
%\newcommand{\Comp}{\mathrm{C}}
%\newcommand{\Komp}{\mathrm{K}}
%\newcommand{\Domp}{\mathsf{D}}%複体のホモトピー圏

\newcommand{\Comp}{\mathop{\mathrm{C}}\nolimits}
\newcommand{\Komp}{\mathop{\mathsf{K}}\nolimits}
\newcommand{\Domp}{\mathop{\mathsf{D}}\nolimits}
\newcommand{\Kompl}{\mathop{\mathsf{K}^\mathrm{+}}\nolimits}
\newcommand{\Kompu}{\mathop{\mathsf{K}^\mathrm{-}}\nolimits}
\newcommand{\Kompb}{\mathop{\mathsf{K}^\mathrm{b}}\nolimits}
\newcommand{\Dompl}{\mathop{\mathsf{D}^\mathrm{+}}\nolimits}
\newcommand{\Dompu}{\mathop{\mathsf{D}^\mathrm{-}}\nolimits}
\newcommand{\Dompb}{\mathop{\mathsf{D}^\mathrm{b}}\nolimits}




\newcommand{\CCat}{\Comp(\cat)}
\newcommand{\KCat}{\Komp(\cat)}
\newcommand{\DCat}{\Domp(\cat)}%圏Cの複体のホモトピー圏
\newcommand{\HOM}{\mathop{\mathscr{H}\hspace{-2pt}om}\nolimits}%内部Hom
\newcommand{\RHOM}{\mathop{\mathrm{R}\hspace{-1.5pt}\HOM}\nolimits}

\newcommand{\muS}{\mathop{\mathrm{SS}}\nolimits}
\newcommand{\RG}{\mathop{\mathrm{R}\hspace{-0pt}\Gamma}\nolimits}
\newcommand{\RHom}{\mathop{\mathrm{R}\hspace{-1.5pt}\Hom}\nolimits}
\newcommand{\Rder}{\mathrm{R}}

\newcommand{\simar}{\mathrel{\overset{\sim}{\rightarrow}}}%同型右矢印
\newcommand{\simarr}{\mathrel{\overset{\sim}{\longrightarrow}}}%同型右矢印
\newcommand{\simra}{\mathrel{\overset{\sim}{\leftarrow}}}%同型左矢印
\newcommand{\simrra}{\mathrel{\overset{\sim}{\longleftarrow}}}%同型左矢印

\newcommand{\hocolim}{{\mathrm{hocolim}}}
\newcommand{\indlim}[1][]{\mathop{\varinjlim}\limits_{#1}}
\newcommand{\sindlim}[1][]{\smash{\mathop{\varinjlim}\limits_{#1}}\,}
\newcommand{\Pro}{\mathrm{Pro}}
\newcommand{\Ind}{\mathrm{Ind}}
\newcommand{\prolim}[1][]{\mathop{\varprojlim}\limits_{#1}}
\newcommand{\sprolim}[1][]{\smash{\mathop{\varprojlim}\limits_{#1}}\,}

\newcommand{\Sh}{\mathrm{Sh}}
\newcommand{\PSh}{\mathrm{PSh}}

\newcommand{\rmD}{\mathrm{D}}

\newcommand{\ori}{\mathord{\mathrm{or}}}












%================================================
% 自前の定理環境
%   https://mathlandscape.com/latex-amsthm/
% を参考にした
\newtheoremstyle{mystyle}%   % スタイル名
    {5pt}%                   % 上部スペース
    {5pt}%                   % 下部スペース
    {}%              % 本文フォント
    {}%                  % 1行目のインデント量
    {\bfseries}%                      % 見出しフォント
    {.}%                     % 見出し後の句読点
    {12pt}%                     % 見出し後のスペース
    {\thmname{#1}\thmnumber{ #2}\thmnote{{\hspace{2pt}\normalfont (#3)}}}% % 見出しの書式

\theoremstyle{mystyle}
\newtheorem{AXM}{公理}[section]
\newtheorem{DFN}[AXM]{定義}
\newtheorem{THM}[AXM]{定理}
\newtheorem*{THM*}{定理}
\newtheorem{PRP}[AXM]{命題}
\newtheorem{LMM}[AXM]{補題}
\newtheorem{CRL}[AXM]{系}
\newtheorem{EG}[AXM]{例}
\newtheorem*{EG*}{例}
\newtheorem{CNV}[AXM]{規約}
\newtheorem{CMT}[AXM]{コメント}

% 定理環境ここまで
%====================================================

\usepackage{framed}
\definecolor{lightgray}{rgb}{0.75,0.75,0.75}
\renewenvironment{leftbar}{%
  \def\FrameCommand{\textcolor{lightgray}{\vrule width 0.7zw} \hspace{10pt}}% 
  \MakeFramed {\advance\hsize-\width \FrameRestore}}%
{\endMakeFramed}
\newenvironment{redleftbar}{%
  \def\FrameCommand{\textcolor{lightgray}{\vrule width 1pt} \hspace{10pt}}% 
  \MakeFramed {\advance\hsize-\width \FrameRestore}}%
 {\endMakeFramed}

\interfootnotelinepenalty=10000

% =================================





% ---------------------------
% new definition macro
% ---------------------------
% 便利なマクロ集です

% 内積のマクロ
%   例えば \inner<\pphi | \psi> のように用いる
\def\inner<#1>{\langle #1 \rangle}

% ================================
% マクロを追加する場合のスペース 

%=================================





% ----------------------------
% documenet 
% ----------------------------
% 以下, 本文の執筆スペースです. 
% Your main code must be written between 
% begin document and end document.
% ---------------------------

\title{カレントまとめノート\footnote{2024/01/20執筆開始}}
\author{Toshi2019}
%\date{}
\begin{document}
\maketitle

\section*{用語について}

\begin{itemize}
    \item distribution は分布と訳す.
\end{itemize}
\section{いろんな本の記述}
\cite[Chap.6. Notes]{Hor89}の記述
\begin{quotation}
    As indicated at the end of Section 6.3 
    one can define \(\scD'(X)\) when \(X\) is a manifold 
    as the dual of the space of \(C^\infty\) densities 
    of compact support. 
    This is a special case of the theory of currents 
    of de Rham [1], 
    which also contains a study of 
    distribution valued differential forms of arbitrary degree. 
    The results in Section 6.1 are thus essentially 
    contained in de Rham's theory, 
    for composition with a map \(f\) having 
    surjective differential can locally be split into 
    a tensor product with the function 1 
    in some new variables and a change of variables 
    in the domain of \(f\). 
    The formula (6.1.5) is from John [5] though. 
\end{quotation}

和訳は
\begin{quotation}
    6.3節で述べたように,\(X\)が多様体であるときの\(\scD'(X)\)を,
    コンパクト台を持つ\(C^\infty\)密度の空間の双対として
    定義することができる.
    これは de Rham [1] のカレントの理論の特別な場合である.
    この本には,分布に値を取る任意の次数の微分形式の議論も含まれている.
    従って,6.1節の結果は本質的にはde Rhamの理論に含まれる.
    微分が全射である写像\(f\)との合成は局所的に,
    新しい変数に関する関数\(1\)とのテンソル積
    と\(f\)の定義域での変数変換に分解できる.
    ただし,公式 (6.1.5) は John [5] からの引用である.
\end{quotation}

\paragraph{\cite{Hor63,Hor89}の場合}
\cite{Hor63}を見ると,次のRemarkが見つかった.
\begin{quotation}
    \(\mathscr{D}'(\varOmega)\) may also be defined 
    as the space of all continuous linear forms 
    on the space of infinitely differentiable densities 
    with compact support. 
    Here a density means a linear form \(L\) 
    on the space \(C_0^\infty(\varOmega)\) of functions 
    in \(C^\infty(\varOmega)\) with compact support, 
    such that to every coordinate system \(\kappa\) 
    there exists a function \(L^\kappa\in C^\infty(\tilde{\varOmega}_\kappa)\) for which
    \[
        L(\varphi)= \int L^\kappa(\varphi\circ\kappa^{-1})dx
        \quad \text{if} \quad \varphi\in C_0^\infty(\varOmega).
    \]
    For details we refer to DE RHAM [1]\footnote{de Rham, ``Vari\'et\'es diff\'erentiables'' のこと.} 
    where a density is called odd form of degree \(n\) 
    and a distribution is an even current of degree 0.
    \footnote{\cite[p.28]{Hor63}}
\end{quotation}

\begin{quotation}
    de Rham[1]では密度を\(n\)次の奇形式,分布を0次の偶カレントと呼んでいる.
\end{quotation}

シュワルツ
\begin{quotation}
    カレントの理論は\dots 多様体上の超函数微分形式の理論によって
    単純かつ完全にされる.カレントの完全な理論 (超函数微分形式の意味で) は
    de Rham の最近の書物に述べられている.
\end{quotation}

%\clearpage
\section{シュワルツの本}

\subsection*{梗概}
\S1では(境界のある)可微分多様体とそれの上の常形式または捩形式の
意味を想起する.
人々はしばしば,捩形式を用いることを避ける;それを避けるには,
多様体が単に向きづけ可能なだけではなく,
向きづけされていると仮定しなければならないが,
それはかなり窮屈なことである;
しかも捩形式のとり扱いは非常に容易である.

さて\S2では多様体上で通常の,または捩れたカレントを定義する.
いくつかの例,特に物理学からの例(電流)を述べる.
向きづけされた多様体の場合に,両種のカレントは差異がなくなる.
この節の終りには,有限次元のベクトル空間をファイバーとする
ファイバー空間の超函数断面を定義する.

\S3ではカレントの上での演算:外積,ベクトル馬との内積,
双対境界または外微分と種々の例,無限小変換による微分についてしらべる.
この節の最後では
カレントのコホモロジーを述べる(一般化された de Rham の定理).

\S4では写像によるカレントの順像を研究する.
定理2でその主要な性質をまとめ,次に例を与える.
定理2の2は実用上有用な同型写像を与える.

\S5ではカレントの逆像,すなわちカレントにおける変数変換を研究する.
この逆像は順次に導入され,その性質は定理3においてまとめられる.
この定理がいかなる場合に適用可能かを見なければならない.
まず局所微分同相写像の場合をとり扱い,例を挙げる.
次にファイバー多様体上で,微分形式のファイバー上での偏微分の概念を研究する;
これからさらに一般的な多様体\(U^m\)から\(n\)次元の多様体\(V^n\)の
中への階数\(n\)の写像による変数変換の場合に導かれる(定理4).
例が与えられる.

\S6では不変な形のFourier変換,すなわち有限次元ベクトル空間上の
偶および奇の緩いカレントのFourier変換を研究する.
これは前にScarfiello[1]により研究された.

\subsection{無限回可微分多様体上の偶形式と奇形式}
\paragraph{偶または常形式}
可微分多様体の主な性質は既知とする.
そして冗長さと複雑さを避けるために,
しばしば証明は概略だけを述べる\footnote{
    多様体の研究のためには,たとえば 
    de Rham[3], Helgason[1]を参照することができる.
}.
特に断らないかぎり,\textbf{次のものを多様体とよぶことにする:
無限遠点で可算な,分離的な,実数体上の無限回可微分多様体.}

\(n\)次元多様体(境界をもつものでもよい\footnote{
    境界をもった無限回可微分多様体という概念は暗によく知られてはいるが,
    書物に必ずしもはっきり書かれてはいない.
    境界点の近傍の局所座標系はその近傍を
    位相空間\(\rr^n_-\)(\(\rr^n\)の\(x_1\leqq0\)なる
    部分空間)
    の開集合の上に表現するものである.
    そのとき\(V^n\)上の可微分函数という概念は
    \(\rr^n_-\)上の可微分函数という概念に帰着する;
    \(\rr^n_-\)で定義された函数に対しては,偏微分係数は
    ただちに定義できる(超平面\(x_1=0\)の点での,
    \(x_1\)に関する偏微分の計算には,
    単に関数の定義域\(x_1\leqq0\)での函数値のみが関与する).
    \(\rr^n_-\)上の\(m\)回連続的微分可能な任意の函数\(\varphi\)は,\nopagebreak
    \(\rr^n\)上の\(m\)回連続的微分可能なある函数\(\varPhi\)を\nopagebreak
    \(\rr^n_-\)に制限したものになっている,\nopagebreak
    という知識がしばしば必要になる.
    そのような拡張\(\varPhi\)は,もし\(m\)が有限ならば,
    全く初歩的な方法で作ることができる:それには,\(x_1>0\)に対して
    次のようにおく:
    \[
        \varPhi(x_1,x_2,\ldots,x_n)
        =\sum_{\nu=0}^{m}
        C_\nu\varphi(-\nu{x_1},x_2,\ldots,x_{n-1},x_n),
    \]ここに\(C_\nu\)はVandermondeの連立方程式:
    \(\displaystyle
        \sum_{\nu=0}^{m}C_\nu(-\nu)^k=1
    \), 
    \(k=0,1,2,\dots,m;\)
    が成立するように選ぶ,
    したがって\(x_1>0\)での\(m\)回までの微係数は,
    \(x_1\to0\)のとき\(\varphi\)の微係数とうまく接続する.
    もし\(\varphi\)がコンパクトな台をもてば\(\varPhi\)もそうである.
    \(m\)が無限であると事情はもっと難しい.
    たとえば,Whitney[4]の65頁,定理1すなわち拡張定理または,
    Selly[1]の定理を適用することができる.
    Whitney[1]は\(x_1>0\)で解析的な拡張\(\varPhi\)を与えたが,
    ここではその必要はない;もし\(\varphi\)がコンパクトな台をもち,
    \(\alpha\)が\(\rr^n\)の関数で\(\scD\)に属し,
    \(\varphi\)の台の近傍で1ならば
    \(\alpha\varphi\)もまた\(\varphi\)の拡張であり,
    その台はコンパクトである.\nopagebreak
})の上では,
\(m\)回連続的微分可能(すなわち\(C^m\)級)函数とか,
無限回微分可能(すなわち\(C^\infty\)級)函数という概念が
定まっているのであった.
さらにまた,1つの多様体から他の多様体への無限回微分可能な写像
という概念も定まっている.

とくに,逆写像もまた無限回微分可能であるような
無限回微分可能写像を微分同相写像という.

さらに\(V\)の局所座標系とは,
\(V\)の開集合---局所座標系\footnote{
    [訳注] 微分同相写像は diff\'eomorpisme,局所座標系は carte,
    座標系は atlas,部分座標系は atlas partiel.
}の定義域---および,
その定義域から\(\rr^n\)または
\(\rr^n_-\)(\(\rr^n\)の部分空間\(x_1\leqq0\))の
開集合の上への微分同相写像\(H\)の与えられた組である.













%===============================================
% 参考文献スペース
%===============================================
\begin{thebibliography}{20} 
    \bibitem[GP74]{GP74} Victor Guillemin, Alan Pollack, 
    \textit{Differential Topology}, 
    Prentice-Hall, 1974.
    \bibitem[Hor63]{Hor63} Lars H\"ormander, 
    \textit{Linear Pertial Differential Operators}, Fourth Printing, 
    Grundlehren der Mathematischen Wissenschaften, 116, Springer, 1963.
    \bibitem[Hor89]{Hor89} Lars H\"ormander, 
    \textit{The Analysis of Linear Pertial Differential Operators I}, Second Edition,
    Grundlehren der Mathematischen Wissenschaften, 256, Springer, 1989.
    \bibitem[KS90]{KS90} Masaki Kashiwara, Pierre Schapira, 
    \textit{Sheaves on Manifolds}, 
    Grundlehren der Mathematischen Wissenschaften, 292, Springer, 1990.
    \bibitem[Le13]{Le13} John M. Lee, 
    \textit{Introduction to Smooth Manifolds}, Second Edition,
    Graduate Texts in Mathematics, \textbf{218}, Springer, 2013.
    \bibitem[Sp65]{Sp65} Michael Spivak, 
    \textit{Calculus on Manifolds}, 
    Benjamin, 1965.
\end{thebibliography}

%===============================================


\end{document}
