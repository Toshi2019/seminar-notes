%================================================
%    多様体上の密度について
%================================================

% -----------------------
% preamble
% -----------------------
% ここから本文 (\begin{document}) までの
% ソースコードに変更を加えた場合は
% 編集者まで連絡してください. 
% Don't change preamble code yourself. 
% If you add something
% (usepackage, newtheorem, newcommand, renewcommand),
% please tell it 
% to the editor of institutional paper of RUMS.

% ------------------------
% documentclass
% ------------------------
\documentclass[11pt, a4paper, dvipdfmx, leqno]{jsarticle}

% ------------------------
% usepackage
% ------------------------
\usepackage{algorithm}
\usepackage{algorithmic}
\usepackage{amscd}
\usepackage{amsfonts}
\usepackage{amsmath}
\usepackage[psamsfonts]{amssymb}
\usepackage{amsthm}
\usepackage{ascmac}
\usepackage{color}
\usepackage{enumerate}
\usepackage{fancybox}
\usepackage[stable]{footmisc}
\usepackage{graphicx}
\usepackage{listings}
\usepackage{mathrsfs}
\usepackage{mathtools}
\usepackage{otf}
\usepackage{pifont}
\usepackage{proof}
\usepackage{subfigure}
\usepackage{tikz}
\usepackage{verbatim}
\usepackage[all]{xy}
\usepackage{url}
\usetikzlibrary{cd}



% ================================
% パッケージを追加する場合のスペース 
%\usepackage{calligra}
\usepackage[dvipdfmx]{hyperref}
\usepackage{xcolor}
\definecolor{darkgreen}{rgb}{0,0.45,0} 
\definecolor{darkred}{rgb}{0.75,0,0}
\definecolor{darkblue}{rgb}{0,0,0.6} 
\hypersetup{
    colorlinks=true,
    citecolor=darkgreen,
    linkcolor=darkred,
    urlcolor=darkblue,
}
\usepackage{pxjahyper}

%=================================


% --------------------------
% theoremstyle
% --------------------------
\theoremstyle{definition}

% --------------------------
% newtheoem
% --------------------------

% 日本語で定理, 命題, 証明などを番号付きで用いるためのコマンドです. 
% If you want to use theorem environment in Japanece, 
% you can use these code. 
% Attention!
% All theorem enivironment numbers depend on 
% only section numbers.
\newtheorem{Axiom}{公理}[section]
\newtheorem{Definition}[Axiom]{定義}
\newtheorem{Theorem}[Axiom]{定理}
\newtheorem{Proposition}[Axiom]{命題}
\newtheorem{Lemma}[Axiom]{補題}
\newtheorem{Corollary}[Axiom]{系}
\newtheorem{Example}[Axiom]{例}
\newtheorem{Claim}[Axiom]{主張}
\newtheorem{Property}[Axiom]{性質}
\newtheorem{Attention}[Axiom]{注意}
\newtheorem{Question}[Axiom]{問}
\newtheorem{Problem}[Axiom]{問題}
\newtheorem{Consideration}[Axiom]{考察}
\newtheorem{Alert}[Axiom]{警告}
\newtheorem{Fact}[Axiom]{事実}
\newtheorem{com}[Axiom]{コメント}
\newtheorem{Notation}[Axiom]{記号}


% 日本語で定理, 命題, 証明などを番号なしで用いるためのコマンドです. 
% If you want to use theorem environment with no number in Japanese, You can use these code.
\newtheorem*{Axiom*}{公理}
\newtheorem*{Definition*}{定義}
\newtheorem*{Theorem*}{定理}
\newtheorem*{Proposition*}{命題}
\newtheorem*{Lemma*}{補題}
\newtheorem*{Example*}{例}
\newtheorem*{Corollary*}{系}
\newtheorem*{Claim*}{主張}
\newtheorem*{Property*}{性質}
\newtheorem*{Attention*}{注意}
\newtheorem*{Question*}{問}
\newtheorem*{Problem*}{問題}
\newtheorem*{Consideration*}{考察}
\newtheorem*{Alert*}{警告}
\newtheorem*{Fact*}{事実}
\newtheorem*{com*}{コメント}



% 英語で定理, 命題, 証明などを番号付きで用いるためのコマンドです. 
% If you want to use theorem environment in English, You can use these code.
%all theorem enivironment number depend on only section number.
\newtheorem{Axiom+}{Axiom}[section]
\newtheorem{Definition+}[Axiom+]{Definition}
\newtheorem{Theorem+}[Axiom+]{Theorem}
\newtheorem{Proposition+}[Axiom+]{Proposition}
\newtheorem{Lemma+}[Axiom+]{Lemma}
\newtheorem{Example+}[Axiom+]{Example}
\newtheorem{Corollary+}[Axiom+]{Corollary}
\newtheorem{Claim+}[Axiom+]{Claim}
\newtheorem{Property+}[Axiom+]{Property}
\newtheorem{Attention+}[Axiom+]{Attention}
\newtheorem{Question+}[Axiom+]{Question}
\newtheorem{Problem+}[Axiom+]{Problem}
\newtheorem{Consideration+}[Axiom+]{Consideration}
\newtheorem{Alert+}{Alert}
\newtheorem{Fact+}[Axiom+]{Fact}
\newtheorem{Remark+}[Axiom+]{Remark}

% ----------------------------
% commmand
% ----------------------------
% 執筆に便利なコマンド集です. 
% コマンドを追加する場合は下のスペースへ. 

% 集合の記号 (黒板文字)
\newcommand{\NN}{\mathbb{N}}
\newcommand{\ZZ}{\mathbb{Z}}
\newcommand{\QQ}{\mathbb{Q}}
\newcommand{\RR}{\mathbb{R}}
\newcommand{\CC}{\mathbb{C}}
\newcommand{\PP}{\mathbb{P}}
\newcommand{\KK}{\mathbb{K}}


% 集合の記号 (太文字)
\newcommand{\nn}{\mathbf{N}}
\newcommand{\zz}{\mathbf{Z}}
\newcommand{\qq}{\mathbf{Q}}
\newcommand{\rr}{\mathbf{R}}
\newcommand{\cc}{\mathbf{C}}
\newcommand{\pp}{\mathbf{P}}
\newcommand{\kk}{\mathbf{K}}

% 特殊な写像の記号
\newcommand{\ev}{\mathop{\mathrm{ev}}\nolimits} % 値写像
\newcommand{\pr}{\mathop{\mathrm{pr}}\nolimits} % 射影

% スクリプト体にするコマンド
%   例えば {\mcal C} のように用いる
\newcommand{\mcal}{\mathcal}

% 花文字にするコマンド 
%   例えば {\h C} のように用いる
\newcommand{\h}{\mathscr}

% ヒルベルト空間などの記号
\newcommand{\F}{\mcal{F}}
\newcommand{\X}{\mcal{X}}
\newcommand{\Y}{\mcal{Y}}
\newcommand{\Hil}{\mcal{H}}
\newcommand{\RKHS}{\Hil_{k}}
\newcommand{\Loss}{\mcal{L}_{D}}
\newcommand{\MLsp}{(\X, \Y, D, \Hil, \Loss)}

% 偏微分作用素の記号
\newcommand{\p}{\partial}

% 角カッコの記号 (内積は下にマクロがあります)
\newcommand{\lan}{\langle}
\newcommand{\ran}{\rangle}



% 圏の記号など
\newcommand{\Set}{\mathop{\textsf{Set}}\nolimits}
\newcommand{\Vect}{{\bf Vect}}
\newcommand{\FDVect}{{\bf FDVect}}
\newcommand{\Mod}{\mathop{\textsf{Mod}}\nolimits}
\newcommand{\CGA}{{\bf CGA}}
\newcommand{\GVect}{{\bf GVect}}
\newcommand{\Lie}{{\bf Lie}}
\newcommand{\dLie}{{\bf Liec}}



% 射の集合など
\newcommand{\Map}{\mathop{\mathrm{Map}}\nolimits}
\newcommand{\Hom}{\mathop{\mathrm{Hom}}\nolimits}
\newcommand{\End}{\mathop{\mathrm{End}}\nolimits}
\newcommand{\Aut}{\mathop{\mathrm{Aut}}\nolimits}
\newcommand{\Mor}{\mathop{\mathrm{Mor}}\nolimits}
\newcommand{\Fct}{\mathop{\mathrm{Fct}}\nolimits}
\newcommand{\f}{\mathop{\mathit{for}}\nolimits}


% その他便利なコマンド
\newcommand{\dip}{\displaystyle} % 本文中で数式モード
\newcommand{\e}{\varepsilon} % イプシロン
\newcommand{\dl}{\delta} % デルタ
\newcommand{\pphi}{\varphi} % ファイ
\newcommand{\ti}{\tilde} % チルダ
\newcommand{\pal}{\parallel} % 平行
\newcommand{\op}{{\rm op}} % 双対を取る記号
\newcommand{\lcm}{\mathop{\mathrm{lcm}}\nolimits} % 最小公倍数の記号
\newcommand{\Probsp}{(\Omega, \F, \P)} 
\newcommand{\argmax}{\mathop{\rm arg~max}\limits}
\newcommand{\argmin}{\mathop{\rm arg~min}\limits}





% ================================
% コマンドを追加する場合のスペース 
\renewcommand\proofname{\bf 証明} % 証明
\numberwithin{equation}{section}
\newcommand{\cTop}{\textsf{Top}}
%\newcommand{\cOpen}{\textsf{Open}}
\newcommand{\Op}{\mathop{\textsf{Open}}\nolimits}
\newcommand{\Ob}{\mathop{\textrm{Ob}}\nolimits}
\newcommand{\id}{\mathop{\mathrm{id}}\nolimits}
\newcommand{\pt}{\mathop{\mathrm{pt}}\nolimits}
\newcommand{\res}{\mathop{\rho}\nolimits}
\newcommand{\A}{\mcal{A}}
\newcommand{\B}{\mcal{B}}
\newcommand{\C}{\mcal{C}}
\newcommand{\D}{\mcal{D}}
\newcommand{\E}{\mcal{E}}
\newcommand{\G}{\mcal{G}}
\newcommand{\U}{\mcal{U}}
%\newcommand{\H}{\mcal{H}}
\newcommand{\I}{\mcal{I}}
\newcommand{\J}{\mcal{J}}
\newcommand{\calS}{\mcal{S}}
\newcommand{\OO}{\mcal{O}}
\newcommand{\Ring}{\mathop{\textsf{Ring}}\nolimits}
\newcommand{\cAb}{\mathop{\textsf{Ab}}\nolimits}
\newcommand{\Ker}{\mathop{\mathrm{Ker}}\nolimits}
\newcommand{\im}{\mathop{\mathrm{Im}}\nolimits}
\newcommand{\Coker}{\mathop{\mathrm{Coker}}\nolimits}
\newcommand{\Coim}{\mathop{\mathrm{Coim}}\nolimits}
\newcommand{\Ht}{\mathop{\mathrm{Ht}}\nolimits}
\newcommand{\colim}{\mathop{\mathrm{colim}}}

\newcommand{\limf}{\mathop{\text{``}\hspace{-0.7pt}\varinjlim\hspace{-1.5pt}\text{''}}}
\newcommand{\sumf}{\mathop{\text{``}\hspace{-0.7pt}\bigoplus\hspace{-1.5pt}\text{''}}}

\newcommand{\hh}{\mathop{\mathrm{h}}\nolimits}
\newcommand{\Ind}{\mathop{\mathrm{Ind}}}




\newcommand{\cat}{\mathcal{C}}

\newcommand{\scA}{\mathscr{A}}
\newcommand{\scB}{\mathscr{B}}
\newcommand{\scC}{\mathscr{C}}
\newcommand{\scD}{\mathscr{D}}
\newcommand{\scE}{\mathscr{E}}
\newcommand{\scF}{\mathscr{F}}

\newcommand{\ibA}{\mathop{\text{\textit{\textbf{A}}}}}
\newcommand{\ibB}{\mathop{\text{\textit{\textbf{B}}}}}
\newcommand{\ibC}{\mathop{\text{\textit{\textbf{C}}}}}
\newcommand{\ibD}{\mathop{\text{\textit{\textbf{D}}}}}
\newcommand{\ibE}{\mathop{\text{\textit{\textbf{E}}}}}
\newcommand{\ibF}{\mathop{\text{\textit{\textbf{F}}}}}
\newcommand{\ibG}{\mathop{\text{\textit{\textbf{G}}}}}
\newcommand{\ibH}{\mathop{\text{\textit{\textbf{H}}}}}
\newcommand{\ibI}{\mathop{\text{\textit{\textbf{I}}}}}
\newcommand{\ibJ}{\mathop{\text{\textit{\textbf{J}}}}}
\newcommand{\ibK}{\mathop{\text{\textit{\textbf{K}}}}}
\newcommand{\ibL}{\mathop{\text{\textit{\textbf{L}}}}}
\newcommand{\ibM}{\mathop{\text{\textit{\textbf{M}}}}}
\newcommand{\ibN}{\mathop{\text{\textit{\textbf{N}}}}}
\newcommand{\ibO}{\mathop{\text{\textit{\textbf{O}}}}}
\newcommand{\ibP}{\mathop{\text{\textit{\textbf{P}}}}}
\newcommand{\ibQ}{\mathop{\text{\textit{\textbf{Q}}}}}
\newcommand{\ibR}{\mathop{\text{\textit{\textbf{R}}}}}
\newcommand{\ibS}{\mathop{\text{\textit{\textbf{S}}}}}
\newcommand{\ibT}{\mathop{\text{\textit{\textbf{T}}}}}
\newcommand{\ibU}{\mathop{\text{\textit{\textbf{U}}}}}
\newcommand{\ibV}{\mathop{\text{\textit{\textbf{V}}}}}
\newcommand{\ibW}{\mathop{\text{\textit{\textbf{W}}}}}
\newcommand{\ibX}{\mathop{\text{\textit{\textbf{X}}}}}
\newcommand{\ibY}{\mathop{\text{\textit{\textbf{Y}}}}}
\newcommand{\ibZ}{\mathop{\text{\textit{\textbf{Z}}}}}

\newcommand{\ibx}{\mathop{\text{\textit{\textbf{x}}}}}

\newcommand{\Comp}{\mathop{\mathsf{C}}\nolimits}
\newcommand{\Komp}{\mathop{\mathsf{K}}\nolimits}
\newcommand{\Domp}{\mathop{\mathsf{D}}\nolimits}%複体のホモトピー圏
\newcommand{\CCat}{\Comp(\cat)}
\newcommand{\KCat}{\Komp(\cat)}

% =================================





% ---------------------------
% new definition macro
% ---------------------------
% 便利なマクロ集です

% 内積のマクロ
%   例えば \inner<\pphi | \psi> のように用いる
\def\inner<#1>{\langle #1 \rangle}
\def\ind<#1>{\mathop{\text{``}\hspace{-0.7pt}#1\limits\hspace{-1.5pt}\text{''}}}


% ================================
% マクロを追加する場合のスペース 

%=================================





% ----------------------------
% documenet 
% ----------------------------
% 以下, 本文の執筆スペースです. 
% Your main code must be written between 
% begin document and end document.
% ---------------------------

\title{density について}
\author{大柴 寿浩\thanks{北海道大学大学院理学院数学専攻修士1年
%\\\url{oshiba.toshihiro.l9@elms.hokudai.ac.jp}
}}
\date{2024/01/04}
\begin{document}
\maketitle
%\thispagestyle{empty}

\section*{はじめに}
密度 (density) について調べたことをまとめたノート.

\paragraph{\cite{F77}の場合}
最初に見つかった和書は藤原\cite{F77}である.
第7章 (p.179--196) のタイトルが「超関数と密度の平方根」となっており,
とくに
\begin{quotation}
    \(X\)は\(\sigma\)コンパクト\(C^\infty\)多様体で,
    必ずしも向きづけられていないとする.
    \(X\)上の関数空間に自然なHermite内積を定義するのは困難であった.
    H\"ormanderは,密度の平方根の概念を導入して,
    この困難を取り除いた.\footnote{\cite[p.183]{F77}}
\end{quotation}
とある.

\paragraph{\cite{Hor63,Hor89}の場合}

そこで,\cite{Hor63}を見ると,次のRemarkが見つかった.
\begin{quotation}
    \(\mathscr{D}'(\varOmega)\) may also be defined 
    as the space of all continuous linear forms 
    on the space of infinitely differentiable densities 
    with compact support. 
    Here a density means a linear form \(L\) 
    on the space \(C_0^\infty(\varOmega)\) of functions 
    in \(C^\infty(\varOmega)\) with compact support, 
    such that to every coordinate system \(\kappa\) 
    there exists a function \(L^\kappa\in C^\infty(\tilde{\varOmega}_\kappa)\) for which
    \[
        L(\varphi)= \int L^\kappa(\varphi\circ\kappa^{-1})dx
        \quad \text{if} \quad \varphi\in C_0^\infty(\varOmega).
    \]
    For details we refer to DE RHAM [1]\footnote{de Rham, ``Vari\'et\'es diff\'erentiables'' のこと.} 
    where a density is called odd form of degree \(n\) 
    and a distribution is an even current of degree 0.
    \footnote{\cite[p.28]{Hor63}}
\end{quotation}
\cite[p.145]{Hor89}でも密度の定義が登場する.

\paragraph{\cite{Hor85}の場合}
この2つには密度の平方根は現れていない.
4巻本の第3巻\cite[18章]{Hor85}に説明が出てくる.
\begin{quotation}
    In Section 6.4 we defined the density bundle \(\varOmega\) 
    on \(X\): a section of \(\varOmega\) expressed 
    in local coordinates \(x_1,\dots,x_n\) is a function \(u\) 
    such that the measure \(u|dx|\) is independent of 
    how they are chosen, \(|dx|\) denoting the Lebesgue measure 
    in the local coordinates. 
    For the representation \(u'\) in the local coordinates \(x'\) 
    we therefore have
    \[u' |dx'|=u|dx|.\]
    We can define the powers \(\varOmega^a\) 
    of \(\varOmega\) for any \(a\in\cc\) by just changing the
    transformation law to
    \[u' |dx'|^a=u|dx|^a.\]
    or, more formally, we take the transition functions
    \[
        g_{\kappa\kappa'}=\lvert
            \det(\kappa\circ\kappa'^{-1})'
        \rvert^a\circ\kappa'
        \quad
        \text{in}
        X_{\kappa}\cap X_{\kappa'}
    \]
    if \(\kappa\) and \(\kappa'\) are 
    arbitrary local coordinates 
    with coordinate patches \(X_{\kappa}\) and \(X_{\kappa}\). 
    We shall now work out the transformation law 
    for the second term in the symbol 
    of a polyhomogeneous operator acting on half densities, 
    that is, sections of \(\varOmega^{\frac{1}{2}}\). \footnote{\cite[p.92]{Hor85}}
\end{quotation}

\paragraph{初出\cite{Hor71}}
では,密度の平方根の初出は何か.
これはおそらくフーリエ積分作用素の論文\cite{Hor71}と思われる.
この論文の117--118ページの部分に次の記述がある.
\begin{quotation}
    Densities of order \(\alpha\) can of course be regarded 
    as sections of a line bundle \(\varOmega\) on \(Y\), 
    defined by the transition functions (中略). 
    The notions of real or positive densities are 
    therefore well defined, 
    and every positive density has 
    a unique positive square root in \(\varOmega_{\frac{1}{2}}\). 
    (中略)
    Concerning the terminology 
    we note that Atiyah and Bott [3]\footnote{
        ATIYAH, M. F., BOTT, R., 
        ``A Lefschetz fixed point formula 
        for elliptic complexes I.'' Ann. of Math., 86 (1967), 374--407. のこと.
    } have called \(\varOmega_1\) 
    the volume bundle of \(Y\).
    \footnote{\cite[p.117-118]{Hor71}}
\end{quotation}



\section*{その他の文献}
積分に関連して,体積要素について,
金子超関数\cite[参考4.2 (p.161), 参考4.3 (p.165)]{K96}に記述がある.


密度束のことが載っている文献で見つかったのは
和書では吉田朋好\cite[2.3.3項 (p.46--48)]{Y98},
洋書では\cite[p.29]{BGV92}, \cite[p.427--434]{L13}が見つかった.

カレントについては,ド・ラームの多様体の本とか,
シュワルツの超関数の本とかがいいんでしょう,きっと.
秋月調和積分にもカレントは載っている.

他に,幾何学的量子化と関連して,\cite{BW97}が面白そう.

\section{Duistermaat から}

\paragraph{ベクトル空間上の密度}
\(E\)を\(\rr\)上の\(n\)次元ベクトル空間とし,
\(\Lambda^n E\)を\(E\)の\(n\)ベクトルの空間とする.
これは例えば\(n\)交代線形形式\(E^n\to \rr\)の空間の
双対として定義される.
\(\Lambda^n{E}\)は\(\rr\)上1次元である.
任意の\(\alpha\in\rr\)に対し,
写像\(\rho\colon\Lambda^n{E}-\{0\}\to\cc\)で
各\(v\in\Lambda^n{E}-\{0\}\), \(\lambda\in\rr-\{0\}\)に対し\(
    \rho(\lambda v)
    =\lvert{\lambda}\rvert^{\alpha}\cdot\rho(v)
\)をみたすものを
それぞれ階数\(\alpha\)の\textbf{複素数値密度} (complex valued 
density)(または\textbf{無向体積} (nonoriented volume))とよぶ.
階数\(\alpha\)の密度全体は\(\cc\)上1次元のベクトル空間であり,
\(\Omega_\alpha(E)\)で表す.

\paragraph{多様体上の密度}
今度は\(X\)を\(n\)次元\(C^{\infty}\)多様体とする.
\(X\)の\(x\in X\)における接空間を\(T_{x}(X)\)で表す.
\(\Omega_{\alpha}(T_{x}(X))\), \(x\in X\)は自然に
複素\(C^{\infty}\)ベクトル束のファイバーとなる.
このとき,階数\(\alpha\)の\(C^{\infty}\)\textbf{密度}を
\(C^{\infty}\)切断\(\rho\colon X\to \Omega_{\alpha}(X)\)として
定義する.
\(X\)上の階数\(\alpha\)の\(C^{\infty}\)密度の
空間を\(C^{\infty}(X,\Omega_{\alpha})\)で表す.
至るところ0にならない階数\(\alpha\)の標準的な密度を選べば,
空間\(C^{\infty}(X,\Omega_{\alpha})\)は\(C^{\infty}(X)\)と
同一視できる.
\(X\)がパラコンパクト(ここでは\(C^{\infty}\)多様体に対してはいつも
パラコンパクトであることを仮定する)であるときには,
単位の分割を用いることで,\(X\)上の\(C^{\infty}\)密度として,
常に正の値をとるものをいつでも構成できる.

%中略==========================================

\paragraph{分布密度}
\(\rho\in C^{\infty}(X,\Omega_{\alpha})\), \(
    \sigma\in C^{\infty}(X,\Omega_{\beta})
\)のとき,各点での積で積\(
    \rho\cdot\sigma\in C^{\infty}(X,\Omega_{\alpha+\beta})
\)が定まる.特に\[
    (\rho,\sigma)\to\int\rho\cdot\sigma dx
\]とすることで\(
    C^{\infty}(X,\Omega_{\alpha})
    \times 
    C^{\infty}_{0}(X,\Omega_{1-\alpha})
\)上の双線形形式が定まり,
これにより\(\sigma\to\int(\rho\cdot\sigma)dx\)は\(
    (C^{\infty}_{0}(X,\Omega_{1-\alpha}))'
\)の元になる.これも\(\rho\)で表す.
連続なうめこみ\(
    C^{\infty}(X,\Omega_{\alpha})
    \to
    (C^{\infty}_{0}(X,\Omega_{1-\alpha}))'
\)が定まることが分かるので,\(
    (C^{\infty}_{0}(X,\Omega_{1-\alpha}))'
\)を階数\(\alpha\)の\textbf{分布密度} (distribution density) の空間とよび,
\(\D'(X,\Omega_{\alpha})\)で表す.(分布密度
は de Rham [20]\footnote{
    de Rham, ``Vari\'et\'es diff\'erentiables'' のこと.
} によって初めて導入された.)






%===============================================
% 参考文献スペース
%===============================================
\begin{thebibliography}{20} 

    \bibitem[BGV92]{BGV92} Nicole Berline, Ezra Getzler, Mich\'ele Vergne, 
    \textit{Heat Kernels and Dirac Operators}, 
    Grundlehren der Mathematischen Wissenschaften, 298, Springer, 1992.
    \bibitem[BW97]{BW97} Sean Bates, Alan Weinstein, 
    \textit{Lectures on the Geometry of Quantization}, 
    Berkeley Mathematics Lecture Notes, \textbf{8}, 1997.
    \bibitem[F77]{F77} 藤原大輔, 
    線型偏微分方程式論における漸近的方法 I, II, 
    岩波講座基礎数学 解析学 (II) \textbf{viii}, 岩波書店, 1977.
    \bibitem[Hor63]{Hor63} Lars H\"ormander, 
    \textit{Linear Pertial Differential Operators}, Fourth Printing, 
    Grundlehren der Mathematischen Wissenschaften, 116, Springer, 1963.
    \bibitem[Hor71]{Hor71} Lars H\"ormander, 
    \textit{Fourier Integral Operators I}, 
    Acta Math. 127, 79--183, (1971).
    \bibitem[Hor89]{Hor89} Lars H\"ormander, 
    \textit{The Analysis of Linear Pertial Differential Operators I}, Second Edition,
    Grundlehren der Mathematischen Wissenschaften, 256, Springer, 1989.
    \bibitem[Hor85]{Hor85} Lars H\"ormander, 
    \textit{The Analysis of Linear Pertial Differential Operators III}, 
    Grundlehren der Mathematischen Wissenschaften, 274, Springer, 1985.
    \bibitem[K96]{K96} 金子晃, 新版 超関数入門, 東京大学出版会, 1996.
    \bibitem[L13]{L13} John M. Lee, 
    \textit{Introduction to Smooth Manifolds}, Second Edition,
    Graduate Texts in Mathematics, \textbf{218}, Springer, 2013.
    %\bibitem[KS90]{KS90} Masaki Kashiwara, Pierre Schapira, 
    %\textit{Sheaves on Manifolds}, 
    %Grundlehren der Mathematischen Wissenschaften, 292, Springer, 1990.
    %\bibitem[KS99]{KS99} Masaki Kashiwara, Pierre Schapira, 
    %\textit{Ind-Sheaves,distributions, and microlocalization}, 
    %Sem Ec. Polytechnique, May 18, 1999.
    
    %\bibitem[KS01]{KS01} Masaki Kashiwara, Pierre Schapira, 
    %\textit{Ind-sheaves}, 
    %Ast\`erisque, 271, Soci\`et\`e Math. de France, 2001.
    %\bibitem[KS06]{KS06} Masaki Kashiwara, Pierre Schapira, 
    %\textit{Categories and Sheaves}, 
    %Grundlehren der Mathematischen Wissenschaften, 332, Springer, 2006.
    %\bibitem[Og02]{Og02} 小木曽啓示, 代数曲線論, 朝倉書店, 2022.
    \bibitem[Y98]{Y98} 吉田朋好, ディラック作用素の指数定理, 共立講座 21世紀の数学 \textbf{22}, 共立出版, 1998.
\end{thebibliography}

%===============================================


\end{document}
