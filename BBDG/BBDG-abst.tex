\begin{abstract}
    本巻では偏屈層の理論について述べる.
    第1章で三角圏の\(t\)構造の定義と基本的な性質を与える.
    第2章では
    偏屈層と(任意の偏屈性に対する)中間拡大関手を
    滑層空間や概形の設定において導入する.
    第3章ではフィルター導来圏,実現関手,
    層の導来圏における局所化に関する内容の補足を行う.
    第4章ではmiddle perversityに対する
    偏屈層に関する基本的な事実をまとめる.
    第5章は本書の中核で,ここでは有限体上の代数多様体
    の上の混合\(\ell\)進偏屈層を考える.
    特に,中間拡大の純性に関する定理,分解定理,
    相対的な強レフシェッツ定理について述べる.
    第6章では,如何にして第5章の結果を
    複素代数幾何学に用いるかを述べる.
    この版では訂正と加筆の一覧を加え,
    追加の参考文献と
    いくつかの便利な関手の\(t\)完全性についての
    付録を載せた.
\end{abstract}