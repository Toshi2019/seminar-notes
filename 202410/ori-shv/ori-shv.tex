%\documentclass[12pt]{article}
\documentclass[uplatex,dvipdfmx,a4paper,10pt]{jsarticle} 





% ------------------------
% usepackage
% ------------------------
\usepackage{algorithm}
\usepackage{algorithmic}
\usepackage{amscd}
\usepackage{amsfonts}
\usepackage{amsmath}
\usepackage[psamsfonts]{amssymb}
\usepackage{amsthm}
\usepackage{ascmac}
\usepackage{bm}
\usepackage{color}
\usepackage{enumerate}
\usepackage{fancybox}
\usepackage[stable]{footmisc}
\usepackage{graphicx}
\usepackage{listings}
\usepackage{mathrsfs}
\usepackage{mathtools}
\usepackage{otf}
\usepackage{pifont}
\usepackage{proof}
\usepackage{subfigure}
\usepackage{tikz}
\usepackage{verbatim}
\usepackage[all]{xy}

\usetikzlibrary{cd}
\usetikzlibrary{arrows.meta}


% ================================
% パッケージを追加する場合のスペース 
\usepackage[dvipdfmx]{hyperref}
%\usepackage{xcolor}
\definecolor{darkgreen}{rgb}{0,0.45,0} 
\definecolor{darkred}{rgb}{0.75,0,0}
\definecolor{darkblue}{rgb}{0,0,0.6} 
\hypersetup{
    colorlinks=true,
    citecolor=darkgreen,
    linkcolor=darkred,
    urlcolor=darkblue,
}
\usepackage{pxjahyper}

%\usepackage[mathscr]{euscript} % mathscr のフォントを変更

%=================================





















\usepackage{tcolorbox}
\tcbuselibrary{listings,breakable, skins, theorems}
%\tcbuselibrary{breakable}




























% --------------------------
% theoremstyle
% --------------------------
\theoremstyle{definition}

% --------------------------
% newtheoem
% --------------------------

% 日本語で定理, 命題, 証明などを番号付きで用いるためのコマンドです. 
% If you want to use theorem environment in Japanece, 
% you can use these code. 
% Attention!
% All theorem enivironment numbers depend on 
% only section numbers.
\newtheorem{Axiom}{公理}[section]
\newtheorem{Definition}[Axiom]{定義}
\newtheorem{Theorem}[Axiom]{定理}
\newtheorem{Proposition}[Axiom]{命題}
\newtheorem{Lemma}[Axiom]{補題}
\newtheorem{Corollary}[Axiom]{系}
\newtheorem{Example}[Axiom]{例}
\newtheorem{Claim}[Axiom]{主張}
\newtheorem{Property}[Axiom]{性質}
\newtheorem{Attention}[Axiom]{注意}
\newtheorem{Question}[Axiom]{問}
\newtheorem{Problem}[Axiom]{問題}
\newtheorem{Consideration}[Axiom]{考察}
\newtheorem{Alert}[Axiom]{警告}
\newtheorem{Fact}[Axiom]{事実}


% 日本語で定理, 命題, 証明などを番号なしで用いるためのコマンドです. 
% If you want to use theorem environment with no number in Japanese, You can use these code.
\newtheorem*{Axiom*}{公理}
\newtheorem*{Definition*}{定義}
\newtheorem*{Theorem*}{定理}
\newtheorem*{Proposition*}{命題}
\newtheorem*{Lemma*}{補題}
\newtheorem*{Example*}{例}
\newtheorem*{Corollary*}{系}
\newtheorem*{Claim*}{主張}
\newtheorem*{Property*}{性質}
\newtheorem*{Attention*}{注意}
\newtheorem*{Question*}{問}
\newtheorem*{Problem*}{問題}
\newtheorem*{Consideration*}{考察}
\newtheorem*{Alert*}{警告}
\newtheorem{Fact*}{事実}


% 英語で定理, 命題, 証明などを番号付きで用いるためのコマンドです. 
% If you want to use theorem environment in English, You can use these code.
%all theorem enivironment number depend on only section number.
\newtheorem{Axiom+}{Axiom}[section]
\newtheorem{Definition+}[Axiom+]{Definition}
\newtheorem{Theorem+}[Axiom+]{Theorem}
\newtheorem{Proposition+}[Axiom+]{Proposition}
\newtheorem{Lemma+}[Axiom+]{Lemma}
\newtheorem{Example+}[Axiom+]{Example}
\newtheorem{Corollary+}[Axiom+]{Corollary}
\newtheorem{Claim+}[Axiom+]{Claim}
\newtheorem{Property+}[Axiom+]{Property}
\newtheorem{Attention+}[Axiom+]{Attention}
\newtheorem{Question+}[Axiom+]{Question}
\newtheorem{Problem+}[Axiom+]{Problem}
\newtheorem{Consideration+}[Axiom+]{Consideration}
\newtheorem{Alert+}{Alert}
\newtheorem{Fact+}[Axiom+]{Fact}
\newtheorem{Remark+}[Axiom+]{Remark}

% ----------------------------
% commmand
% ----------------------------
% 執筆に便利なコマンド集です. 
% コマンドを追加する場合は下のスペースへ. 

% 集合の記号 (黒板文字)
\newcommand{\NN}{\mathbb{N}}
\newcommand{\ZZ}{\mathbb{Z}}
\newcommand{\QQ}{\mathbb{Q}}
\newcommand{\RR}{\mathbb{R}}
\newcommand{\CC}{\mathbb{C}}
\newcommand{\PP}{\mathbb{P}}
\newcommand{\KK}{\mathbb{K}}


% 集合の記号 (太文字)
\newcommand{\nn}{\mathbf{N}}
\newcommand{\zz}{\mathbf{Z}}
\newcommand{\qq}{\mathbf{Q}}
\newcommand{\rr}{\mathbf{R}}
\newcommand{\cc}{\mathbf{C}}
\newcommand{\pp}{\mathbf{P}}
\newcommand{\kk}{\mathbf{K}}
\newcommand{\bk}{\mathbf{k}}


% 特殊な写像の記号
\newcommand{\ev}{\mathop{\mathrm{ev}}\nolimits} % 値写像
\newcommand{\pr}{\mathop{\mathrm{pr}}\nolimits} % 射影

% スクリプト体にするコマンド
%   例えば {\mcal C} のように用いる
\newcommand{\mcal}{\mathcal}

% 花文字にするコマンド 
%   例えば {\h C} のように用いる
\newcommand{\h}{\mathscr}

% ヒルベルト空間などの記号
\newcommand{\F}{\mcal{F}}
\newcommand{\X}{\mcal{X}}
\newcommand{\Y}{\mcal{Y}}
\newcommand{\Hil}{\mcal{H}}
\newcommand{\RKHS}{\Hil_{k}}
\newcommand{\Loss}{\mcal{L}_{D}}
\newcommand{\MLsp}{(\X, \Y, D, \Hil, \Loss)}

% 偏微分作用素の記号
\newcommand{\p}{\partial}

% 角カッコの記号 (内積は下にマクロがあります)
\newcommand{\lan}{\langle}
\newcommand{\ran}{\rangle}



% 圏の記号など
\newcommand{\Set}{{\bf Set}}
\newcommand{\Vect}{{\bf Vect}}
\newcommand{\FDVect}{{\bf FDVect}}
%\newcommand{\Ring}{{\bf Ring}}
\newcommand{\Ab}{{\bf Ab}}
\newcommand{\Mod}{\mathop{\mathrm{Mod}}\nolimits}
\newcommand{\Modf}{\mathop{\mathrm{Mod}^\mathrm{f}}\nolimits}
\newcommand{\CGA}{{\bf CGA}}
\newcommand{\GVect}{{\bf GVect}}
\newcommand{\Lie}{{\bf Lie}}
\newcommand{\dLie}{{\bf Liec}}



% 射の集合など
\newcommand{\Map}{\mathop{\mathrm{Map}}\nolimits} % 写像の集合
\newcommand{\Hom}{\mathop{\mathrm{Hom}}\nolimits} % 射集合
\newcommand{\End}{\mathop{\mathrm{End}}\nolimits} % 自己準同型の集合
\newcommand{\Aut}{\mathop{\mathrm{Aut}}\nolimits} % 自己同型の集合
\newcommand{\Mor}{\mathop{\mathrm{Mor}}\nolimits} % 射集合
\newcommand{\Ker}{\mathop{\mathrm{Ker}}\nolimits} % 核
\newcommand{\Img}{\mathop{\mathrm{Im}}\nolimits} % 像
\newcommand{\Cok}{\mathop{\mathrm{Coker}}\nolimits} % 余核
\newcommand{\Cim}{\mathop{\mathrm{Coim}}\nolimits} % 余像

% その他便利なコマンド
\newcommand{\dip}{\displaystyle} % 本文中で数式モード
\newcommand{\e}{\varepsilon} % イプシロン
\newcommand{\dl}{\delta} % デルタ
\newcommand{\pphi}{\varphi} % ファイ
\newcommand{\ti}{\tilde} % チルダ
\newcommand{\pal}{\parallel} % 平行
\newcommand{\op}{{\rm op}} % 双対を取る記号
\newcommand{\lcm}{\mathop{\mathrm{lcm}}\nolimits} % 最小公倍数の記号
\newcommand{\Probsp}{(\Omega, \F, \P)} 
\newcommand{\argmax}{\mathop{\rm arg~max}\limits}
\newcommand{\argmin}{\mathop{\rm arg~min}\limits}






\makeatletter
\renewenvironment{proof}[1][\proofname]{\par
  \pushQED{\qed}%
  \normalfont \topsep6\p@\@plus6\p@\relax
  \trivlist
  \item[\hskip\labelsep
%        \itshape
         \bfseries
%    #1\@addpunct{.}]\ignorespaces
    {#1}]\ignorespaces
}{%
  \popQED\endtrivlist\@endpefalse
}
\makeatother

\renewcommand{\proofname}{証明.}



%\renewcommand\proofname{\bf 証明} % 証明
\numberwithin{equation}{section}
\newcommand{\cTop}{\textsf{Top}}
%\newcommand{\cOpen}{\textsf{Open}}
\newcommand{\Op}{\mathop{\textsf{Op}}\nolimits}
\newcommand{\Ob}{\mathop{\textrm{Ob}}\nolimits}
\newcommand{\id}{\mathop{\mathrm{id}}\nolimits}
\newcommand{\pt}{\mathop{\mathrm{pt}}\nolimits}
\newcommand{\res}{\mathop{\rho}\nolimits}
\newcommand{\A}{\mcal{A}}
\newcommand{\B}{\mcal{B}}
\newcommand{\C}{\mcal{C}}
\newcommand{\D}{\mcal{D}}
\newcommand{\E}{\mcal{E}}
\newcommand{\G}{\mcal{G}}
%\newcommand{\H}{\mcal{H}}
\newcommand{\I}{\mcal{I}}
\newcommand{\J}{\mcal{J}}
\newcommand{\OO}{\mcal{O}}
\newcommand{\Ring}{\mathop{\textsf{Ring}}\nolimits}
\newcommand{\cAb}{\mathop{\textsf{Ab}}\nolimits}
%\newcommand{\Ker}{\mathop{\mathrm{Ker}}\nolimits}
\newcommand{\im}{\mathop{\mathrm{Im}}\nolimits}
\newcommand{\Coker}{\mathop{\mathrm{Coker}}\nolimits}
\newcommand{\Coim}{\mathop{\mathrm{Coim}}\nolimits}
\newcommand{\rank}{\mathop{\mathrm{rank}}\nolimits}
\newcommand{\Ht}{\mathop{\mathrm{Ht}}\nolimits}
\newcommand{\supp}{\mathop{\mathrm{supp}}\nolimits}
\newcommand{\colim}{\mathop{\mathrm{colim}}}
\newcommand{\Tor}{\mathop{\mathrm{Tor}}\nolimits}

\newcommand{\cat}{\mathscr{C}}

%筆記体
\newcommand{\cA}{\mcal{A}}
\newcommand{\cB}{\mcal{B}}
\newcommand{\cC}{\mcal{C}}
\newcommand{\cD}{\mcal{D}}
\newcommand{\cE}{\mcal{E}}
\newcommand{\cF}{\mcal{F}}
\newcommand{\cG}{\mcal{G}}
\newcommand{\cH}{\mcal{H}}
\newcommand{\cI}{\mcal{I}}
\newcommand{\cJ}{\mcal{J}}
\newcommand{\cK}{\mcal{K}}
\newcommand{\cL}{\mcal{L}}
\newcommand{\cM}{\mcal{M}}
\newcommand{\cN}{\mcal{N}}
\newcommand{\cO}{\mcal{O}}
\newcommand{\cP}{\mcal{P}}
\newcommand{\cQ}{\mcal{Q}}
\newcommand{\cR}{\mcal{R}}
\newcommand{\cS}{\mcal{S}}
\newcommand{\cT}{\mcal{T}}
\newcommand{\cU}{\mcal{U}}
\newcommand{\cV}{\mcal{V}}
\newcommand{\cW}{\mcal{W}}
\newcommand{\cX}{\mcal{X}}
\newcommand{\cY}{\mcal{Y}}
\newcommand{\cZ}{\mcal{Z}}


\newcommand{\scA}{\mathscr{A}}
\newcommand{\scB}{\mathscr{B}}
\newcommand{\scC}{\mathscr{C}}
\newcommand{\scD}{\mathscr{D}}
\newcommand{\scE}{\mathscr{E}}
\newcommand{\scF}{\mathscr{F}}
\newcommand{\scN}{\mathscr{N}}
\newcommand{\scO}{\mathscr{O}}
\newcommand{\scV}{\mathscr{V}}
\newcommand{\scU}{\mathscr{U}}


\newcommand{\ibA}{\mathop{\text{\textit{\textbf{A}}}}}
\newcommand{\ibB}{\mathop{\text{\textit{\textbf{B}}}}}
\newcommand{\ibC}{\mathop{\text{\textit{\textbf{C}}}}}
\newcommand{\ibD}{\mathop{\text{\textit{\textbf{D}}}}}
\newcommand{\ibE}{\mathop{\text{\textit{\textbf{E}}}}}
\newcommand{\ibF}{\mathop{\text{\textit{\textbf{F}}}}}
\newcommand{\ibG}{\mathop{\text{\textit{\textbf{G}}}}}
\newcommand{\ibH}{\mathop{\text{\textit{\textbf{H}}}}}
\newcommand{\ibI}{\mathop{\text{\textit{\textbf{I}}}}}
\newcommand{\ibJ}{\mathop{\text{\textit{\textbf{J}}}}}
\newcommand{\ibK}{\mathop{\text{\textit{\textbf{K}}}}}
\newcommand{\ibL}{\mathop{\text{\textit{\textbf{L}}}}}
\newcommand{\ibM}{\mathop{\text{\textit{\textbf{M}}}}}
\newcommand{\ibN}{\mathop{\text{\textit{\textbf{N}}}}}
\newcommand{\ibO}{\mathop{\text{\textit{\textbf{O}}}}}
\newcommand{\ibP}{\mathop{\text{\textit{\textbf{P}}}}}
\newcommand{\ibQ}{\mathop{\text{\textit{\textbf{Q}}}}}
\newcommand{\ibR}{\mathop{\text{\textit{\textbf{R}}}}}
\newcommand{\ibS}{\mathop{\text{\textit{\textbf{S}}}}}
\newcommand{\ibT}{\mathop{\text{\textit{\textbf{T}}}}}
\newcommand{\ibU}{\mathop{\text{\textit{\textbf{U}}}}}
\newcommand{\ibV}{\mathop{\text{\textit{\textbf{V}}}}}
\newcommand{\ibW}{\mathop{\text{\textit{\textbf{W}}}}}
\newcommand{\ibX}{\mathop{\text{\textit{\textbf{X}}}}}
\newcommand{\ibY}{\mathop{\text{\textit{\textbf{Y}}}}}
\newcommand{\ibZ}{\mathop{\text{\textit{\textbf{Z}}}}}

\newcommand{\ibx}{\mathop{\text{\textit{\textbf{x}}}}}

%\newcommand{\Comp}{\mathop{\mathrm{C}}\nolimits}
%\newcommand{\Komp}{\mathop{\mathrm{K}}\nolimits}
%\newcommand{\Domp}{\mathop{\mathsf{D}}\nolimits}%複体のホモトピー圏
%\newcommand{\Comp}{\mathrm{C}}
%\newcommand{\Komp}{\mathrm{K}}
%\newcommand{\Domp}{\mathsf{D}}%複体のホモトピー圏

\newcommand{\Comp}{\mathop{\mathrm{C}}\nolimits}
\newcommand{\Komp}{\mathop{\mathsf{K}}\nolimits}
\newcommand{\Domp}{\mathop{\mathsf{D}}\nolimits}
\newcommand{\Kompl}{\mathop{\mathsf{K}^\mathrm{+}}\nolimits}
\newcommand{\Kompu}{\mathop{\mathsf{K}^\mathrm{-}}\nolimits}
\newcommand{\Kompb}{\mathop{\mathsf{K}^\mathrm{b}}\nolimits}
\newcommand{\Dompl}{\mathop{\mathsf{D}^\mathrm{+}}\nolimits}
\newcommand{\Dompu}{\mathop{\mathsf{D}^\mathrm{-}}\nolimits}
\newcommand{\Dompb}{\mathop{\mathsf{D}^\mathrm{b}}\nolimits}
\newcommand{\Dompbf}{\mathop{\mathsf{D}_\mathrm{f}^\mathrm{b}}\nolimits}




\newcommand{\CCat}{\Comp(\cat)}
\newcommand{\KCat}{\Komp(\cat)}
\newcommand{\DCat}{\Domp(\cat)}%圏Cの複体のホモトピー圏
\newcommand{\HOM}{\mathop{\mathscr{H}\hspace{-2pt}om}\nolimits}%内部Hom
\newcommand{\RHOM}{\mathop{\mathrm{R}\hspace{-1.5pt}\HOM}\nolimits}

\newcommand{\muS}{\mathop{\mathrm{SS}}\nolimits}
\newcommand{\RG}{\mathop{\mathrm{R}\hspace{-0pt}\Gamma}\nolimits}
\newcommand{\RHom}{\mathop{\mathrm{R}\hspace{-1.5pt}\Hom}\nolimits}
\newcommand{\Rder}{\mathrm{R}}

\newcommand{\simar}{\mathrel{\overset{\sim}{\rightarrow}}}%同型右矢印
\newcommand{\simarr}{\mathrel{\overset{\sim}{\longrightarrow}}}%同型右矢印
\newcommand{\simra}{\mathrel{\overset{\sim}{\leftarrow}}}%同型左矢印
\newcommand{\simrra}{\mathrel{\overset{\sim}{\longleftarrow}}}%同型左矢印

\newcommand{\hocolim}{{\mathrm{hocolim}}}
\newcommand{\indlim}[1][]{\mathop{\varinjlim}\limits_{#1}}
\newcommand{\sindlim}[1][]{\smash{\mathop{\varinjlim}\limits_{#1}}\,}
\newcommand{\Pro}{\mathrm{Pro}}
\newcommand{\Ind}{\mathrm{Ind}}
\newcommand{\prolim}[1][]{\mathop{\varprojlim}\limits_{#1}}
\newcommand{\sprolim}[1][]{\smash{\mathop{\varprojlim}\limits_{#1}}\,}

\newcommand{\Sh}{\mathrm{Sh}}
\newcommand{\PSh}{\mathrm{PSh}}

\newcommand{\rmD}{\mathrm{D}}

\newcommand{\Lloc}[1][]{\mathord{\mathcal{L}^1_{\mathrm{loc},{#1}}}}
\newcommand{\ori}{\mathord{\mathrm{or}}}
\newcommand{\Db}{\mathord{\mathscr{D}b}}

\newcommand{\codim}{\mathop{\mathrm{codim}}\nolimits}



\newcommand{\gld}{\mathop{\mathrm{gld}}\nolimits}
\newcommand{\wgld}{\mathop{\mathrm{wgld}}\nolimits}


\newcommand{\tens}[1][]{\mathbin{\otimes_{\raise1.5ex\hbox to-.1em{}{#1}}}}
\newcommand{\ttens}[1][]{\mathbin{\mathop{\overset{\mathrm{}}{\tens}}_{#1}}}
\newcommand{\etens}{\mathbin{\boxtimes}}
\newcommand{\ltens}[1][]{\mathbin{\overset{\mathrm{L}}\tens}_{#1}}
\newcommand{\mtens}[1][]{\mathbin{\overset{\mathrm{\mu}}\tens}_{#1}}
\newcommand{\lltens}[1][]{{\mathop{\tens}\limits^{\mathrm{L}}_{#1}}}
\newcommand{\letens}{\overset{\mathrm{L}}{\etens}}
\newcommand{\detens}{\underline{\etens}}
\newcommand{\ldetens}{\overset{\mathrm{L}}{\underline{\etens}}}
\newcommand{\dtens}[1][]{{\overset{\mathrm{L}}{\underline{\otimes}}}_{#1}}

\newcommand{\blk}{\mathord{\ \cdot\ }}
\newcommand{\mres}[2][]{{\left.{#1}\right\rvert}_{#2}}


%\newcommand{\hocolim}{{\mathrm{hocolim}}}
%\newcommand{\indlim}[1][]{\mathop{\varinjlim}\limits_{#1}}
%\newcommand{\sindlim}[1][]{\smash{\mathop{\varinjlim}\limits_{#1}}\,}
%\newcommand{\Pro}{\mathrm{Pro}}
%\newcommand{\Ind}{\mathrm{Ind}}
%\newcommand{\prolim}[1][]{\mathop{\varprojlim}\limits_{#1}}
%\newcommand{\sprolim}[1][]{\smash{\mathop{\varprojlim}\limits_{#1}}\,}
\newcommand{\proolim}[1][]{\mathop{\text{\rm``{$\varprojlim$}''}}\limits_{#1}}
\newcommand{\sproolim}[1][]{\smash{\mathop{\rm``{\varprojlim}''}\limits_{#1}}}
\newcommand{\inddlim}[1][]{\mathop{\text{\rm``{$\varinjlim$}''}}\limits_{#1}}
\newcommand{\sinddlim}[1][]{\smash{\mathop{\text{\rm``{$\varinjlim$}''}}\limits_{#1}}\,}
\newcommand{\ooplus}{\mathop{\text{\rm``{$\oplus$}''}}\limits}
\newcommand{\bbigsqcup}{\mathop{``\bigsqcup''}\limits}
\newcommand{\bsqcup}{\mathop{``\sqcup''}\limits}
\newcommand{\dsum}[1][]{\mathbin{\oplus_{#1}}}

\newcommand{\Fct}{\mathop{\mathsf{Fct}}\nolimits}





%================================================
% 自前の定理環境
%   https://mathlandscape.com/latex-amsthm/
% を参考にした
\newtheoremstyle{mystyle}%   % スタイル名
    {5pt}%                   % 上部スペース
    {5pt}%                   % 下部スペース
    {}%              % 本文フォント
    {}%                  % 1行目のインデント量
    {\bfseries}%                      % 見出しフォント
    {.}%                     % 見出し後の句読点
    {12pt}%                     % 見出し後のスペース
    {\thmname{#1}\thmnumber{ #2}\thmnote{{\hspace{2pt}\normalfont (#3)}}}% % 見出しの書式

\theoremstyle{mystyle}
\newtheorem{AXM}{公理}%[section]
\newtheorem{DFN}[AXM]{定義}
\newtheorem{THM}[AXM]{定理}
\newtheorem*{THM*}{定理}
\newtheorem{PRP}[AXM]{命題}
\newtheorem{LMM}[AXM]{補題}
\newtheorem{CRL}[AXM]{系}
\newtheorem{EG}[AXM]{例}
\newtheorem*{EG*}{例}
\newtheorem{RMK}[AXM]{注意}
\newtheorem{CNV}[AXM]{約束}
\newtheorem{CMT}[AXM]{コメント}
\newtheorem*{CMT*}{コメント}
\newtheorem{NTN}[AXM]{記号}



% 定理環境ここまで
%====================================================

%====================================================
% 枠つき環境
% 参考 1:https://note.com/totomin/n/n0b02d6539728
% 参考 2:https://qiita.com/rityo_masu/items/5ef248024b294d72799a
%
\newtheorem{mythm}{定理}[section]
\newtheorem{mydfn}[mythm]{定義}
\newtheorem{mylmm}[mythm]{補題}
\newtheorem{myprp}[mythm]{命題}


\newtcolorbox{thmbox}{empty,% 一回全部消すよ〜
breakable,% 箱のページまたぎを許すよ〜
underlay={% 色塗って線引くよ〜
\fill[black!5] (frame.north west) rectangle (frame.south east);% 色塗ったよ〜
\draw[line width=2pt] ([xshift=1pt]frame.north west)--([xshift=1pt]frame.south west);% 線引いたよ〜幅 2pt の線だから 1pt ずらしておこうね〜
}}

\newtcolorbox{prpbox}{empty,% 一回全部消すよ〜
breakable,% 箱のページまたぎを許すよ〜
underlay={% 色塗って線引くよ〜
\fill[black!5] (frame.north west) rectangle (frame.south east);% 色塗ったよ〜
\draw[line width=2pt] ([xshift=1pt]frame.north west)--([xshift=1pt]frame.south west);% 線引いたよ〜幅 2pt の線だから 1pt ずらしておこうね〜
}}


\newtcolorbox{dfnbox}{empty,% 一回全部消すよ〜
breakable,% 箱のページまたぎを許すよ〜
underlay={% 色塗って線引くよ〜
%\fill[black!5] (frame.north west) rectangle (frame.south east);% 色塗ったよ〜
\draw[line width=2pt] ([xshift=1pt]frame.north west)--([xshift=1pt]frame.south west);% 線引いたよ〜幅 2pt の線だから 1pt ずらしておこうね〜
}}

\newtcolorbox{lmmbox}{empty,% 一回全部消すよ〜
breakable,% 箱のページまたぎを許すよ〜
underlay={% 色塗って線引くよ〜
\fill[black!5] (frame.north west) rectangle (frame.south east);% 色塗ったよ〜
%\draw[line width=2pt] ([xshift=1pt]frame.north west)--([xshift=1pt]frame.south west);% 線引いたよ〜幅 2pt の線だから 1pt ずらしておこうね〜
}}


\newenvironment{thm}{\begin{thmbox}\begin{mythm}}{\end{mythm}\end{thmbox}}
\newenvironment{dfn}{\begin{dfnbox}\begin{mydfn}}{\end{mydfn}\end{dfnbox}}
\newenvironment{lmm}{\begin{lmmbox}\begin{mylmm}}{\end{mylmm}\end{lmmbox}}
\newenvironment{prp}{\begin{prpbox}\begin{myprp}}{\end{myprp}\end{prpbox}}

% 枠つき環境ここまで
%=======================================================


































\title{向きづけ層}
\author{toshi2019}
\begin{document}
\maketitle

\section{局所自由層}
\subsection{局所的な条件について}\label{ssec:locality}

以後,「層が局所的に同型」という形の条件をよく用いる.
これについては次の2つの表現がある.
\begin{prp}
    \(F\)と\(G\)を\(X\)上の層とする.
    次の条件は同値である.
    \begin{enumerate}[(i)]
        \item 各点\(x\)に対し,開近傍\(U\)で\(\mres[F]{U}\cong\mres[G]{U}\)となるものが存在する.
        \item \(X\)の開被覆\((U_i)_{i\in I}\)で,各\(i\)に対し\(\mres[F]{U_{i}}\cong\mres[G]{U_{i}}\)となるものが存在する.
    \end{enumerate}
\end{prp}
\begin{proof}
    (i)\(\Rightarrow\)(ii):
    \(X\)の各点\(x\)に対し開近傍\(U_x\)で\(
        \mres[F]{U_{x}}\cong\mres[G]{U_{x}}
    \)となるものが存在する.
    \((U_x)_{x\in X}\)は\(X\)の開被覆である.

    (ii)\(\Rightarrow\)(i):
    \((U_i)_{i\in I}\)を\(X\)の開被覆で
    各\(U_i\)に対し\(\mres[F]{U_{i}}\cong\mres[G]{U_{i}}\)となるものとする.
    \(x\in X\)とすると\(x\in U_{i}\)となる\(i\in I\)が存在する.
\end{proof}
\subsection{局所自由層}
\(X\)を局所コンパクト空間とする.
\(\cA\)を\(X\)上の環とする.
まず,局所自由層を定義しよう.
\begin{dfn}[局所自由層]
    \(k\geqq 0\)を整数とし,
    \(\cL\)を\(\cA\)加群とする.
    \(X\)の開被覆\(X=\bigcup_{i\in I}U_i\)で,
    どの\(i\in I\)に対しても\(
        \mres[\cL]{U_{i}}
        \cong 
        \mres[\cA]{U_{i}}^{\oplus k}
    \)となるものが存在するとき,
    \(\cL\)は階数\(k\)の
    \textbf{局所自由層} (locally free sheaf) であるという.
    階数1の局所自由層のことを\textbf{可逆層} (invertible sheaf) とよぶ.
\end{dfn}

\ref{ssec:locality}節のことばを用いると,
\(\cL\)が局所自由層であるとは,
\(\cL\)が局所的に\(\cA^{\oplus k}\)と同型であるということである.

\(\bk\)を大域次元が有限な環とする.
\(\cA=\bk_X\)のとき,局所自由層は局所定数層である.
可逆層は局所的に\(\bk_X\)と同型な層である.

\(L\in\Mod(\bk_X)\)とする.
\(L^{\otimes -1}\coloneqq\RHOM_{\bk_X}(L,\bk_X)\)とおく.
\(\bk\)が体ならば,\(\bk_X\)は入射的なので,
\(\RHOM_{\bk_X}(L,\bk_X)=\HOM_{\bk_X}(L,\bk_X)\)が成り立つ.

\section{向きづけ層}

\subsection{対合律}

\(A_X\)を\(A\)をファイバーとする定数層とする.
\[
    \rmD'F\coloneqq \RHOM_{A_X}(F,A_X)
\]
とおく.
次の公式が成り立つ.

\begin{thm}[{\cite[演習III.3]{KS90}}]\label{thm:ori-dual}
    \(X\)を位相空間とする.
    \(F\)を\(\ZZ_X\)と局所的に同型な層とする.
    このとき,次の同型がある.\[
        F\ttens[]F\cong \ZZ_X,
        \quad
        \rmD'_{X} F\cong F.
    \]
\end{thm}
このノートだけの用語で定理\ref{thm:ori-dual}を
\(F\)の対合律 (involution law) とよぶことにする.

証明にあたり,次の事実に注意しよう.
\begin{lmm}\label{lmm:pm1}
    \(\varphi\colon\ZZ\to\ZZ\)をアーベル群の同型とする.
    このとき\(\varphi\)は\(\pm\id_{\ZZ}\)のどちらか一方である.
\end{lmm}

\begin{proof}
    \(\varphi(1)=m\)とおくと
    任意の整数\(n\)に対し\(\varphi(n)=n\varphi(1)=nm\)となるので,
    \(\ZZ\)から\(\ZZ\)への射は\(m\)倍写像である.
    \(\psi\colon\ZZ\to\ZZ\)を\(\varphi\)の逆射とすると
    \[
        1=\psi(\varphi(1))=\psi(m)=m\psi(1)
    \]
    となる.\(\ZZ\)の可逆元は\(\pm1\)のみであるから,
    \[
        m=\pm1,
        \quad
        \psi(1)=\pm1
        \quad
        \text{(複合同順)}
    \]
    である.
\end{proof}

\begin{proof}[{定理\ref{thm:ori-dual}の証明}]
    1つ目の主張を示す.
    \(X\)の開被覆\((U_i)_{i\in I}\)と各\(U_i\)上の層の射
    \[
        \theta_i\colon\mres[F]{U_{i}}\simar\mres[\ZZ_X]{U_{i}}
    \]で同型であるものが存在する.
    補題\ref{lmm:pm1}より,この\((\theta_i)_{i}\)は各\(i\)に対し
    \(\theta_i=\pm1\)とかける.実際,
    \(x\in U_i\)とすると,\(s_x\in F_x\cong\ZZ\)に対し\(
        (\theta_{i})_x(s_x)=\pm{s_x}
    \)となる.
    したがって,各\(i,j\)に対して    
    \begin{align*}
        \mres[\theta_{i}\otimes\theta_{i}]{{U_{i}\cap U_{j}}}
        &\cong
        (\pm1)\ttens[](\pm1)\\
        &\cong
        1\\
        &\cong
        \mres[\theta_{j}\otimes\theta_{j}]{{U_{i}\cap U_{j}}}
    \end{align*}
    となるので\((\theta_i\ttens[]\theta_i)_i\)から層の射
    \[
        \theta\tens[]\theta\colon F\tens[]F
        \to\ZZ_X\tens[]\ZZ_X
    \]
    がひきおこされる.これと同型
    \[
        \ZZ_X\tens[]\ZZ_X\to\ZZ_X
    \]
    の合成を\(\varphi\)で表す.
    各点\(x\in X\)に対して
    \[
        \varphi_x\colon F_x\ttens[]F_x\simar\ZZ\ttens[]\ZZ\cong(\ZZ_{X})_x
    \]が成り立つので,\(\varphi\)は同型である.

    1つ目の主張から2つ目の主張が従うことを示す.
    随伴\((\ttens[],\HOM)\)から
    \begin{align*}
        \Hom_{\ZZ_X}(F\ttens[]F,\ZZ_X)
        \cong
        \Hom_{\ZZ_X}(F,\HOM_{\ZZ_X}(F,\ZZ_X))
    \end{align*}
    である.
    同型は同型に送られるので同型\(
        F\ttens[]F\to \ZZ_X
    \)は同型\(F\to\HOM_{\ZZ_X}(F,\ZZ_X)=\rmD'F\)に送られる.
\end{proof}

定理\ref{thm:ori-dual}によると,\(\ZZ_X\)上の可逆層は対合律をみたす.
\subsection{向きづけ層}

\(A\)を可換環とし\(A_X\)で\(A\)をファイバーとする定数層とする.
\(\ori_{X}^{\ZZ}\)で\(\ZZ\)上の向きづけ層,
\(\ori_X\)で\(A\)上の向きづけ層を表す.
\(\ori_X\cong\ori^{\ZZ}_{X}\ttens[]A_X\)である.

\(\ori_X\)も対合律をみたす.

\begin{prp}[{\cite[命題3.3.4]{KS90}}]
    \(f\colon Y\to X\)をファイバー次元が\(l\)の位相的しずめ込みとする.
    \begin{enumerate}[(i)]
        \item \(\omega_{Y/X}^\zz\in\Dompl(\zz_Y)\)と
        \(\ori_{Y/X}^\zz\in\Mod(\zz_Y)\)を\(\zz\)上の
        双対化複体と向きづけ層とするとき,次の同型が成り立つ.\[
            \omega_{Y/X}\cong A_Y
            \mathop{\otimes}\limits_{\zz_Y}
            \omega_{Y/X}^\zz, 
            \quad
            \ori_{Y/X}\cong A_Y
            \mathop{\otimes}\limits_{\zz_Y}
            \ori_{Y/X}^\zz. 
        \]
        \item 次の自然な同型が成り立つ.
        \begin{align*}
                \ori_{Y/X}\otimes \ori_{Y/X}\cong A_Y,\\
                \HOM(\ori_{Y/X},A_Y)\cong A_Y.
        \end{align*}
        \item \(g\colon Z\to Y\)を連続写像で
        \(f\circ g\)がファイバー次元\(m\)の位相的しずめ込み
        になるものとする.\(F\in\Dompl(A_X)\)に対して,
        \[
            g^{!}\circ f^{-1}F
            \cong
            \left(f\circ g\right)^{-1}F
            \otimes \ori_{Z/X}
            \otimes g^{-1}\ori_{Y/X}[m-l]
        \]が成り立つ.
    \end{enumerate}
\end{prp}


\[
    \ori_{Y/X}\otimes \ori_{Y/X}
    \cong
    \left(\ori_{Y/X}^\zz\otimes \ori_{Y/X}^\zz\right)\otimes A_Y
    \cong Z_Y\otimes A_Y
    \cong A_Y
\]
である.

\paragraph{うめこみについて}
以降,\(A=\cc\)とし\(\cc\)上の向きづけ層を考える.
\(i\colon M\hookrightarrow X\)を閉埋め込みとし,
\(a_M\colon M\to\{\pt\}\)と\(a_X\colon X\to\{\pt\}\)を
一点への射とすると
\begin{align*}
    \ori_{M/X}[-n]
    &\cong \omega_{M/X} \cong i^{!}\cc_X\\
    &\cong i^{!}a_{X}^{-1}\cc\\
    &\cong (a_{X}\circ i)^{-1}\cc\ttens[\cc_M]\ori_{M}\ttens[\cc_M]i^{-1}\ori_{X}[n-2n]\\
    &\cong a_{M}^{-1}\cc\ttens[\cc_M]\ori_{M}\ttens[\cc_M]i^{-1}\ori_{X}[-n]\\
    &\cong \cc_{M}\ttens[\cc_M]\ori_{M}\ttens[\cc_M]i^{-1}\ori_{X}[-n]\\
    &\cong \ori_{M}\ttens[\cc_M]i^{-1}\ori_{X}[-n]
\end{align*}
が成り立つ.\(\omega_{M/X}\cong\ori_{M/X}[-n]\)なので,
\begin{equation}\label{eq:rel-ori1}
    \ori_{M/X}\cong \ori_{M}\ttens[\cc_M]i^{-1}\ori_{X}    
\end{equation}
である.\(X\)上の層\(\ori_X\)と制限\(
    \mres[\ori_{X}]{M}=i^{-1}\ori_{X}
\)を\(M\)上で同一視すると,上の式は
\begin{equation}\label{eq:rel-ori2}
    \ori_{M/X}\cong \ori_{M}\ttens \ori_{X}    
\end{equation}
とも書ける.(\cite[p.130]{KS90}の記法.)

\section{超関数}
\begin{equation}\label{eq:Bdefs}
    \RG_{M}(\cO_X)\ttens[]\ori_{M}[n]
    \cong 
    \RHOM_{\cc_X}(\rmD'_{M}{\cc_X}_M,\cO_X).
\end{equation}
\begin{proof}
    まず\(\rmD_{X}'{\cc_X}_{M}\cong\ori_{M}[-n]\)より,右辺は\[
        \RHOM_{\cc_X}(\rmD'_{M}{\cc_X}_M,\cO_X)
        \cong
        \RHOM_{\cc_X}(\ori_{M},\cO_X)[n]
    \]
    である.一方左辺は
    \[
        \RG_{M}(\cO_X)\ttens[]\ori_{M}
        \cong
        \RHOM({\cc_{X}}_{M},\cO_X)\ttens[]\ori_{M}
    \]である.よって\eqref{eq:Bdefs}が成り立つのは\[
        \RHOM({\cc_{X}}_{M},\cO_X)\ttens[]\ori_{M}
        \cong 
        \RHOM_{\cc_X}(\ori_{M},\cO_X)\tag{\(\ast\)}\label{eq:goal}
    \]となるときである.
\end{proof}

\paragraph{質問について}

可逆層で同型が\(\pm1\)なので,
\(\ori_M\cong\ori_{M}^{\otimes -1}\)

\eqref{eq:goal}は
\[
    \RHOM_{\cc_X}(\ori_{M}^{\otimes -1},\cO_X)%\tag{\(\ast\)}\label{eq:goal}
    \to
    \RHOM({\cc_{X}}_{M},\cO_X)\ttens[]\ori_{M}
\]
右は
\begin{align*}
    \RHOM({\cc_{X}}_{M},\cO_X)\ttens[]\ori_{M}
    &\to
    \RHOM({\cc_{X}}_{M},\cO_X\ttens[]\ori_{M}^{\otimes -1})\\
    &\to
    \RHOM({\cc_{X}}_{M},\RHOM(\ori_{M},\cO_X))\\
    &\cong
    \RHOM({\cc_{X}}_{M}\tens[]\ori_{M}^{\otimes -1},\cO_X)
\end{align*}















%===============================================
% 参考文献スペース
%===============================================
\begin{thebibliography}{20} 
    %\bibitem[GP74]{GP74} Victor Guillemin, Alan Pollack, \textit{Differential Topology}, Prentice-Hall, 1974.
    \bibitem[KS90]{KS90} Masaki Kashiwara, Pierre Schapira, 
      \textit{Sheaves on Manifolds}, 
      Grundlehren der Mathematischen Wissenschaften, 292, Springer, 1990.
    %\bibitem[Hat89]{Hat89} 服部晶夫, 多様体 増補版, 岩波全書 288, 岩波書店, 1989.
    \bibitem[Sch04]{Sch04} Pierre Schapira, 
      \textit{Sheaves: from Leray to Grothendieck and Sato}, 
      S\'eminaires et Congres 9, Soc. Math. France, pp.173--181, 2004.
    %\bibitem[Og02]{Og02} 小木曽啓示, 代数曲線論, 朝倉書店, 2022.
\end{thebibliography}
%===============================================
  

\end{document}