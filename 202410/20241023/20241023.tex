%\documentclass[12pt]{article}
\documentclass[uplatex,dvipdfmx,a4paper,10pt,draft]{jsarticle} 





% ------------------------
% usepackage
% ------------------------
\usepackage{algorithm}
\usepackage{algorithmic}
\usepackage{amscd}
\usepackage{amsfonts}
\usepackage{amsmath}
\usepackage[psamsfonts]{amssymb}
\usepackage{amsthm}
\usepackage{ascmac}
\usepackage{bm}
\usepackage{calligra}
\usepackage{color}
\usepackage{enumerate}
\usepackage[mathcal]{eucal}
\usepackage{fancybox}
\usepackage[stable]{footmisc}
\usepackage{graphicx}
\usepackage{listings}
\usepackage{mathrsfs}
\usepackage{mathtools}
\usepackage{otf}
\usepackage{pifont}
\usepackage{proof}
\usepackage{subfigure}
\usepackage{tikz}
\usepackage{verbatim}
\usepackage[all]{xy}

\usetikzlibrary{cd}
\usetikzlibrary{arrows.meta}


% ================================
% パッケージを追加する場合のスペース 
\usepackage[dvipdfmx]{hyperref}
%\usepackage{xcolor}
\definecolor{darkgreen}{rgb}{0,0.45,0} 
\definecolor{darkred}{rgb}{0.75,0,0}
\definecolor{darkblue}{rgb}{0,0,0.6} 
\hypersetup{
    colorlinks=true,
    citecolor=darkgreen,
    linkcolor=darkred,
    urlcolor=darkblue,
}
\usepackage{pxjahyper}

%\usepackage[mathscr]{euscript} % mathscr のフォントを変更

%=================================





















\usepackage{tcolorbox}
\tcbuselibrary{listings,breakable, skins, theorems}
%\tcbuselibrary{breakable}




























% --------------------------
% theoremstyle
% --------------------------
\theoremstyle{definition}

% --------------------------
% newtheoem
% --------------------------

% 日本語で定理, 命題, 証明などを番号付きで用いるためのコマンドです. 
% If you want to use theorem environment in Japanece, 
% you can use these code. 
% Attention!
% All theorem enivironment numbers depend on 
% only section numbers.
\newtheorem{Axiom}{公理}[section]
\newtheorem{Definition}[Axiom]{定義}
\newtheorem{Theorem}[Axiom]{定理}
\newtheorem{Proposition}[Axiom]{命題}
\newtheorem{Lemma}[Axiom]{補題}
\newtheorem{Corollary}[Axiom]{系}
\newtheorem{Example}[Axiom]{例}
\newtheorem{Claim}[Axiom]{主張}
\newtheorem{Property}[Axiom]{性質}
\newtheorem{Attention}[Axiom]{注意}
\newtheorem{Question}[Axiom]{問}
\newtheorem{Problem}[Axiom]{問題}
\newtheorem{Consideration}[Axiom]{考察}
\newtheorem{Alert}[Axiom]{警告}
\newtheorem{Fact}[Axiom]{事実}


% 日本語で定理, 命題, 証明などを番号なしで用いるためのコマンドです. 
% If you want to use theorem environment with no number in Japanese, You can use these code.
\newtheorem*{Axiom*}{公理}
\newtheorem*{Definition*}{定義}
\newtheorem*{Theorem*}{定理}
\newtheorem*{Proposition*}{命題}
\newtheorem*{Lemma*}{補題}
\newtheorem*{Example*}{例}
\newtheorem*{Corollary*}{系}
\newtheorem*{Claim*}{主張}
\newtheorem*{Property*}{性質}
\newtheorem*{Attention*}{注意}
\newtheorem*{Question*}{問}
\newtheorem*{Problem*}{問題}
\newtheorem*{Consideration*}{考察}
\newtheorem*{Alert*}{警告}
\newtheorem{Fact*}{事実}


% 英語で定理, 命題, 証明などを番号付きで用いるためのコマンドです. 
% If you want to use theorem environment in English, You can use these code.
%all theorem enivironment number depend on only section number.
\newtheorem{Axiom+}{Axiom}[section]
\newtheorem{Definition+}[Axiom+]{Definition}
\newtheorem{Theorem+}[Axiom+]{Theorem}
\newtheorem{Proposition+}[Axiom+]{Proposition}
\newtheorem{Lemma+}[Axiom+]{Lemma}
\newtheorem{Example+}[Axiom+]{Example}
\newtheorem{Corollary+}[Axiom+]{Corollary}
\newtheorem{Claim+}[Axiom+]{Claim}
\newtheorem{Property+}[Axiom+]{Property}
\newtheorem{Attention+}[Axiom+]{Attention}
\newtheorem{Question+}[Axiom+]{Question}
\newtheorem{Problem+}[Axiom+]{Problem}
\newtheorem{Consideration+}[Axiom+]{Consideration}
\newtheorem{Alert+}{Alert}
\newtheorem{Fact+}[Axiom+]{Fact}
\newtheorem{Remark+}[Axiom+]{Remark}

% ----------------------------
% commmand
% ----------------------------
% 執筆に便利なコマンド集です. 
% コマンドを追加する場合は下のスペースへ. 

% 集合の記号 (黒板文字)
\newcommand{\NN}{\mathbb{N}}
\newcommand{\ZZ}{\mathbb{Z}}
\newcommand{\QQ}{\mathbb{Q}}
\newcommand{\RR}{\mathbb{R}}
\newcommand{\CC}{\mathbb{C}}
\newcommand{\PP}{\mathbb{P}}
\newcommand{\KK}{\mathbb{K}}


% 集合の記号 (太文字)
\newcommand{\nn}{\mathbf{N}}
\newcommand{\zz}{\mathbf{Z}}
\newcommand{\qq}{\mathbf{Q}}
\newcommand{\rr}{\mathbf{R}}
\newcommand{\cc}{\mathbf{C}}
\newcommand{\pp}{\mathbf{P}}
\newcommand{\kk}{\mathbf{K}}
\newcommand{\bk}{\mathbf{k}}


% 特殊な写像の記号
\newcommand{\ev}{\mathop{\mathrm{ev}}\nolimits} % 値写像
\newcommand{\pr}{\mathop{\mathrm{pr}}\nolimits} % 射影

% スクリプト体にするコマンド
%   例えば {\mcal C} のように用いる
\newcommand{\mcal}{\mathcal}

% 花文字にするコマンド 
%   例えば {\h C} のように用いる
\newcommand{\h}{\mathscr}

% ヒルベルト空間などの記号
\newcommand{\F}{\mcal{F}}
\newcommand{\X}{\mcal{X}}
\newcommand{\Y}{\mcal{Y}}
\newcommand{\Hil}{\mcal{H}}
\newcommand{\RKHS}{\Hil_{k}}
\newcommand{\Loss}{\mcal{L}_{D}}
\newcommand{\MLsp}{(\X, \Y, D, \Hil, \Loss)}

% 偏微分作用素の記号
\newcommand{\p}{\partial}

% 角カッコの記号 (内積は下にマクロがあります)
\newcommand{\lan}{\langle}
\newcommand{\ran}{\rangle}



% 圏の記号など
\newcommand{\Set}{{\bf Set}}
\newcommand{\Vect}{{\bf Vect}}
\newcommand{\FDVect}{{\bf FDVect}}
%\newcommand{\Ring}{{\bf Ring}}
\newcommand{\Ab}{{\bf Ab}}
\newcommand{\Mod}{\mathop{\mathrm{Mod}}\nolimits}
\newcommand{\Modf}{\mathop{\mathrm{Mod}^\mathrm{f}}\nolimits}
\newcommand{\CGA}{{\bf CGA}}
\newcommand{\GVect}{{\bf GVect}}
\newcommand{\Lie}{{\bf Lie}}
\newcommand{\dLie}{{\bf Liec}}



% 射の集合など
\newcommand{\Map}{\mathop{\mathrm{Map}}\nolimits} % 写像の集合
\newcommand{\Hom}{\mathop{\mathrm{Hom}}\nolimits} % 射集合
\newcommand{\End}{\mathop{\mathrm{End}}\nolimits} % 自己準同型の集合
\newcommand{\Aut}{\mathop{\mathrm{Aut}}\nolimits} % 自己同型の集合
\newcommand{\Mor}{\mathop{\mathrm{Mor}}\nolimits} % 射集合
\newcommand{\Ker}{\mathop{\mathrm{Ker}}\nolimits} % 核
\newcommand{\Img}{\mathop{\mathrm{Im}}\nolimits} % 像
\newcommand{\Cok}{\mathop{\mathrm{Coker}}\nolimits} % 余核
\newcommand{\Cim}{\mathop{\mathrm{Coim}}\nolimits} % 余像

% その他便利なコマンド
\newcommand{\dip}{\displaystyle} % 本文中で数式モード
\newcommand{\e}{\varepsilon} % イプシロン
\newcommand{\dl}{\delta} % デルタ
\newcommand{\pphi}{\varphi} % ファイ
\newcommand{\ti}{\tilde} % チルダ
\newcommand{\pal}{\parallel} % 平行
\newcommand{\op}{{\rm op}} % 双対を取る記号
\newcommand{\lcm}{\mathop{\mathrm{lcm}}\nolimits} % 最小公倍数の記号
\newcommand{\Probsp}{(\Omega, \F, \P)} 
\newcommand{\argmax}{\mathop{\rm arg~max}\limits}
\newcommand{\argmin}{\mathop{\rm arg~min}\limits}






\makeatletter
\renewenvironment{proof}[1][\proofname]{\par
  \pushQED{\qed}%
  \normalfont \topsep6\p@\@plus6\p@\relax
  \trivlist
  \item[\hskip\labelsep
%        \itshape
         \bfseries
%    #1\@addpunct{.}]\ignorespaces
    {#1}]\ignorespaces
}{%
  \popQED\endtrivlist\@endpefalse
}
\makeatother

\renewcommand{\proofname}{証明.}



%\renewcommand\proofname{\bf 証明} % 証明
\numberwithin{equation}{section}
\newcommand{\cTop}{\textsf{Top}}
%\newcommand{\cOpen}{\textsf{Open}}
\newcommand{\Op}{\mathop{\textsf{Op}}\nolimits}
\newcommand{\Ob}{\mathop{\textrm{Ob}}\nolimits}
\newcommand{\id}{\mathop{\mathrm{id}}\nolimits}
\newcommand{\pt}{\mathop{\mathrm{pt}}\nolimits}
\newcommand{\res}{\mathop{\rho}\nolimits}
\newcommand{\A}{\mcal{A}}
\newcommand{\B}{\mcal{B}}
\newcommand{\C}{\mcal{C}}
\newcommand{\D}{\mcal{D}}
\newcommand{\E}{\mcal{E}}
\newcommand{\G}{\mcal{G}}
%\newcommand{\H}{\mcal{H}}
\newcommand{\I}{\mcal{I}}
\newcommand{\J}{\mcal{J}}
\newcommand{\OO}{\mcal{O}}
\newcommand{\Ring}{\mathop{\textsf{Ring}}\nolimits}
\newcommand{\cAb}{\mathop{\textsf{Ab}}\nolimits}
%\newcommand{\Ker}{\mathop{\mathrm{Ker}}\nolimits}
\newcommand{\im}{\mathop{\mathrm{Im}}\nolimits}
\newcommand{\Coker}{\mathop{\mathrm{Coker}}\nolimits}
\newcommand{\Coim}{\mathop{\mathrm{Coim}}\nolimits}
\newcommand{\rank}{\mathop{\mathrm{rank}}\nolimits}
\newcommand{\Ht}{\mathop{\mathrm{Ht}}\nolimits}
\newcommand{\supp}{\mathop{\mathrm{supp}}\nolimits}
\newcommand{\colim}{\mathop{\mathrm{colim}}}
\newcommand{\Tor}{\mathop{\mathrm{Tor}}\nolimits}

\newcommand{\cat}{\mathscr{C}}

%筆記体
\newcommand{\cA}{\mcal{A}}
\newcommand{\cB}{\mcal{B}}
\newcommand{\cC}{\mcal{C}}
\newcommand{\cD}{\mcal{D}}
\newcommand{\cE}{\mcal{E}}
\newcommand{\cF}{\mcal{F}}
\newcommand{\cG}{\mcal{G}}
\newcommand{\cH}{\mcal{H}}
\newcommand{\cI}{\mcal{I}}
\newcommand{\cJ}{\mcal{J}}
\newcommand{\cK}{\mcal{K}}
\newcommand{\cL}{\mcal{L}}
\newcommand{\cM}{\mcal{M}}
\newcommand{\cN}{\mcal{N}}
\newcommand{\cO}{\mcal{O}}
\newcommand{\cP}{\mcal{P}}
\newcommand{\cQ}{\mcal{Q}}
\newcommand{\cR}{\mcal{R}}
\newcommand{\cS}{\mcal{S}}
\newcommand{\cT}{\mcal{T}}
\newcommand{\cU}{\mcal{U}}
\newcommand{\cV}{\mcal{V}}
\newcommand{\cW}{\mcal{W}}
\newcommand{\cX}{\mcal{X}}
\newcommand{\cY}{\mcal{Y}}
\newcommand{\cZ}{\mcal{Z}}


\newcommand{\scA}{\mathscr{A}}
\newcommand{\scB}{\mathscr{B}}
\newcommand{\scC}{\mathscr{C}}
\newcommand{\scD}{\mathscr{D}}
\newcommand{\scE}{\mathscr{E}}
\newcommand{\scF}{\mathscr{F}}
\newcommand{\scN}{\mathscr{N}}
\newcommand{\scO}{\mathscr{O}}
\newcommand{\scV}{\mathscr{V}}
\newcommand{\scU}{\mathscr{U}}


\newcommand{\ibA}{\mathop{\text{\textit{\textbf{A}}}}}
\newcommand{\ibB}{\mathop{\text{\textit{\textbf{B}}}}}
\newcommand{\ibC}{\mathop{\text{\textit{\textbf{C}}}}}
\newcommand{\ibD}{\mathop{\text{\textit{\textbf{D}}}}}
\newcommand{\ibE}{\mathop{\text{\textit{\textbf{E}}}}}
\newcommand{\ibF}{\mathop{\text{\textit{\textbf{F}}}}}
\newcommand{\ibG}{\mathop{\text{\textit{\textbf{G}}}}}
\newcommand{\ibH}{\mathop{\text{\textit{\textbf{H}}}}}
\newcommand{\ibI}{\mathop{\text{\textit{\textbf{I}}}}}
\newcommand{\ibJ}{\mathop{\text{\textit{\textbf{J}}}}}
\newcommand{\ibK}{\mathop{\text{\textit{\textbf{K}}}}}
\newcommand{\ibL}{\mathop{\text{\textit{\textbf{L}}}}}
\newcommand{\ibM}{\mathop{\text{\textit{\textbf{M}}}}}
\newcommand{\ibN}{\mathop{\text{\textit{\textbf{N}}}}}
\newcommand{\ibO}{\mathop{\text{\textit{\textbf{O}}}}}
\newcommand{\ibP}{\mathop{\text{\textit{\textbf{P}}}}}
\newcommand{\ibQ}{\mathop{\text{\textit{\textbf{Q}}}}}
\newcommand{\ibR}{\mathop{\text{\textit{\textbf{R}}}}}
\newcommand{\ibS}{\mathop{\text{\textit{\textbf{S}}}}}
\newcommand{\ibT}{\mathop{\text{\textit{\textbf{T}}}}}
\newcommand{\ibU}{\mathop{\text{\textit{\textbf{U}}}}}
\newcommand{\ibV}{\mathop{\text{\textit{\textbf{V}}}}}
\newcommand{\ibW}{\mathop{\text{\textit{\textbf{W}}}}}
\newcommand{\ibX}{\mathop{\text{\textit{\textbf{X}}}}}
\newcommand{\ibY}{\mathop{\text{\textit{\textbf{Y}}}}}
\newcommand{\ibZ}{\mathop{\text{\textit{\textbf{Z}}}}}

\newcommand{\ibx}{\mathop{\text{\textit{\textbf{x}}}}}

%\newcommand{\Comp}{\mathop{\mathrm{C}}\nolimits}
%\newcommand{\Komp}{\mathop{\mathrm{K}}\nolimits}
%\newcommand{\Domp}{\mathop{\mathsf{D}}\nolimits}%複体のホモトピー圏
%\newcommand{\Comp}{\mathrm{C}}
%\newcommand{\Komp}{\mathrm{K}}
%\newcommand{\Domp}{\mathsf{D}}%複体のホモトピー圏

\newcommand{\Comp}{\mathop{\mathrm{C}}\nolimits}
\newcommand{\Komp}{\mathop{\mathsf{K}}\nolimits}
\newcommand{\Domp}{\mathop{\mathsf{D}}\nolimits}
\newcommand{\Kompl}{\mathop{\mathsf{K}^\mathrm{+}}\nolimits}
\newcommand{\Kompu}{\mathop{\mathsf{K}^\mathrm{-}}\nolimits}
\newcommand{\Kompb}{\mathop{\mathsf{K}^\mathrm{b}}\nolimits}
\newcommand{\Dompl}{\mathop{\mathsf{D}^\mathrm{+}}\nolimits}
\newcommand{\Dompu}{\mathop{\mathsf{D}^\mathrm{-}}\nolimits}
\newcommand{\Dompb}{\mathop{\mathsf{D}^\mathrm{b}}\nolimits}
\newcommand{\Dompbf}{\mathop{\mathsf{D}_\mathrm{f}^\mathrm{b}}\nolimits}




\newcommand{\CCat}{\Comp(\cat)}
\newcommand{\KCat}{\Komp(\cat)}
\newcommand{\DCat}{\Domp(\cat)}%圏Cの複体のホモトピー圏
%\DeclareMathOperator{\HOM}{\mathscr{H}\text{\kern -4.5pt {\calligra\large om}}\,}
%\newcommand{\HOM}{\mathop{\mathscr{H}\hspace{-2pt}om}\nolimits}%内部Hom
\newcommand{\HOM}{\mathop{\mathcal{H}\hspace{-0.5pt}om}\nolimits}%内部Hom
\newcommand{\RHOM}{\mathop{\mathrm{R}\hspace{-1.5pt}\HOM}\nolimits}

\newcommand{\muS}{\mathop{\mathrm{SS}}\nolimits}
\newcommand{\RG}{\mathop{\mathrm{R}\hspace{-0pt}\Gamma}\nolimits}
\newcommand{\RHom}{\mathop{\mathrm{R}\hspace{-1.5pt}\Hom}\nolimits}
\newcommand{\mhom}{\mathop{\mu\hspace{-0pt}hom}\nolimits}
\newcommand{\Rder}{\mathrm{R}}

\newcommand{\simar}{\mathrel{\overset{\sim}{\rightarrow}}}%同型右矢印
\newcommand{\simarr}{\mathrel{\overset{\sim}{\longrightarrow}}}%同型右矢印
\newcommand{\simra}{\mathrel{\overset{\sim}{\leftarrow}}}%同型左矢印
\newcommand{\simrra}{\mathrel{\overset{\sim}{\longleftarrow}}}%同型左矢印

\newcommand{\hocolim}{{\mathrm{hocolim}}}
\newcommand{\indlim}[1][]{\mathop{\varinjlim}\limits_{#1}}
\newcommand{\sindlim}[1][]{\smash{\mathop{\varinjlim}\limits_{#1}}\,}
\newcommand{\Pro}{\mathrm{Pro}}
\newcommand{\Ind}{\mathrm{Ind}}
\newcommand{\prolim}[1][]{\mathop{\varprojlim}\limits_{#1}}
\newcommand{\sprolim}[1][]{\smash{\mathop{\varprojlim}\limits_{#1}}\,}

\newcommand{\Sh}{\mathrm{Sh}}
\newcommand{\PSh}{\mathrm{PSh}}

\newcommand{\rmD}{\mathrm{D}}

\newcommand{\Lloc}[1][]{\mathord{\mathcal{L}^1_{\mathrm{loc},{#1}}}}
\newcommand{\ori}{\mathord{\mathrm{or}}}
\newcommand{\Db}{\mathord{\mathscr{D}b}}

\newcommand{\codim}{\mathop{\mathrm{codim}}\nolimits}



\newcommand{\gld}{\mathop{\mathrm{gld}}\nolimits}
\newcommand{\wgld}{\mathop{\mathrm{wgld}}\nolimits}


\newcommand{\tens}[1][]{\mathbin{\otimes_{\raise1.5ex\hbox to-.1em{}{#1}}}}
\newcommand{\ttens}[1][]{\mathbin{\mathop{\overset{\mathrm{}}{\tens}}_{#1}}}
\newcommand{\etens}{\mathbin{\boxtimes}}
\newcommand{\ltens}[1][]{\mathbin{\overset{\mathrm{L}}\tens}_{#1}}
\newcommand{\mtens}[1][]{\mathbin{\overset{\mathrm{\mu}}\tens}_{#1}}
\newcommand{\lltens}[1][]{{\mathop{\tens}\limits^{\mathrm{L}}_{#1}}}
%\newcommand{\letens}{\overset{\mathrm{L}}{\etens}}
\newcommand{\letens}[1][]{\mathbin{\mathop{\overset{\mathrm{L}}{\etens}}_{#1}}}
\newcommand{\detens}{\underline{\etens}}
\newcommand{\ldetens}{\overset{\mathrm{L}}{\underline{\etens}}}
\newcommand{\dtens}[1][]{{\overset{\mathrm{L}}{\underline{\otimes}}}_{#1}}

\newcommand{\blk}{\mathord{\ \cdot\ }}
\newcommand{\mres}[2][]{{\left.{#1}\right\rvert}_{#2}}




%\newcommand{\hocolim}{{\mathrm{hocolim}}}
%\newcommand{\indlim}[1][]{\mathop{\varinjlim}\limits_{#1}}
%\newcommand{\sindlim}[1][]{\smash{\mathop{\varinjlim}\limits_{#1}}\,}
%\newcommand{\Pro}{\mathrm{Pro}}
%\newcommand{\Ind}{\mathrm{Ind}}
%\newcommand{\prolim}[1][]{\mathop{\varprojlim}\limits_{#1}}
%\newcommand{\sprolim}[1][]{\smash{\mathop{\varprojlim}\limits_{#1}}\,}
\newcommand{\proolim}[1][]{\mathop{\text{\rm``{$\varprojlim$}''}}\limits_{#1}}
\newcommand{\sproolim}[1][]{\smash{\mathop{\rm``{\varprojlim}''}\limits_{#1}}}
\newcommand{\inddlim}[1][]{\mathop{\text{\rm``{$\varinjlim$}''}}\limits_{#1}}
\newcommand{\sinddlim}[1][]{\smash{\mathop{\text{\rm``{$\varinjlim$}''}}\limits_{#1}}\,}
\newcommand{\ooplus}{\mathop{\text{\rm``{$\oplus$}''}}\limits}
\newcommand{\bbigsqcup}{\mathop{``\bigsqcup''}\limits}
\newcommand{\bsqcup}{\mathop{``\sqcup''}\limits}
\newcommand{\dsum}[1][]{\mathbin{\oplus_{#1}}}

\newcommand{\Fct}{\mathop{\mathsf{Fct}}\nolimits}





%================================================
% 自前の定理環境
%   https://mathlandscape.com/latex-amsthm/
% を参考にした
\newtheoremstyle{mystyle}%   % スタイル名
    {5pt}%                   % 上部スペース
    {5pt}%                   % 下部スペース
    {}%              % 本文フォント
    {}%                  % 1行目のインデント量
    {\bfseries}%                      % 見出しフォント
    {.}%                     % 見出し後の句読点
    {12pt}%                     % 見出し後のスペース
    {\thmname{#1}\thmnumber{ #2}\thmnote{{\hspace{2pt}\normalfont (#3)}}}% % 見出しの書式

\theoremstyle{mystyle}
\newtheorem{AXM}{公理}%[section]
\newtheorem{DFN}[AXM]{定義}
\newtheorem{THM}[AXM]{定理}
\newtheorem*{THM*}{定理}
\newtheorem{PRP}[AXM]{命題}
\newtheorem{LMM}[AXM]{補題}
\newtheorem{CRL}[AXM]{系}
\newtheorem{EG}[AXM]{例}
\newtheorem*{EG*}{例}
\newtheorem{RMK}[AXM]{注意}
\newtheorem{CNV}[AXM]{約束}
\newtheorem{CMT}[AXM]{コメント}
\newtheorem*{CMT*}{コメント}
\newtheorem{NTN}[AXM]{記号}



% 定理環境ここまで
%====================================================

%====================================================
% 枠つき環境
% 参考 1:https://note.com/totomin/n/n0b02d6539728
% 参考 2:https://qiita.com/rityo_masu/items/5ef248024b294d72799a
%
\newtheorem{mythm}{定理}[section]
\newtheorem{mydfn}[mythm]{定義}
\newtheorem{mylmm}[mythm]{補題}
\newtheorem{myprp}[mythm]{命題}


\newtcolorbox{thmbox}{empty,% 一回全部消すよ〜
breakable,% 箱のページまたぎを許すよ〜
underlay={% 色塗って線引くよ〜
\fill[black!5] (frame.north west) rectangle (frame.south east);% 色塗ったよ〜
\draw[line width=2pt] ([xshift=1pt]frame.north west)--([xshift=1pt]frame.south west);% 線引いたよ〜幅 2pt の線だから 1pt ずらしておこうね〜
}}

\newtcolorbox{prpbox}{empty,% 一回全部消すよ〜
breakable,% 箱のページまたぎを許すよ〜
underlay={% 色塗って線引くよ〜
\fill[black!5] (frame.north west) rectangle (frame.south east);% 色塗ったよ〜
\draw[line width=2pt] ([xshift=1pt]frame.north west)--([xshift=1pt]frame.south west);% 線引いたよ〜幅 2pt の線だから 1pt ずらしておこうね〜
}}


\newtcolorbox{dfnbox}{empty,% 一回全部消すよ〜
breakable,% 箱のページまたぎを許すよ〜
underlay={% 色塗って線引くよ〜
%\fill[black!5] (frame.north west) rectangle (frame.south east);% 色塗ったよ〜
\draw[line width=2pt] ([xshift=1pt]frame.north west)--([xshift=1pt]frame.south west);% 線引いたよ〜幅 2pt の線だから 1pt ずらしておこうね〜
}}

\newtcolorbox{lmmbox}{empty,% 一回全部消すよ〜
breakable,% 箱のページまたぎを許すよ〜
underlay={% 色塗って線引くよ〜
\fill[black!5] (frame.north west) rectangle (frame.south east);% 色塗ったよ〜
%\draw[line width=2pt] ([xshift=1pt]frame.north west)--([xshift=1pt]frame.south west);% 線引いたよ〜幅 2pt の線だから 1pt ずらしておこうね〜
}}


\newenvironment{thm}{\begin{thmbox}\begin{mythm}}{\end{mythm}\end{thmbox}}
\newenvironment{dfn}{\begin{dfnbox}\begin{mydfn}}{\end{mydfn}\end{dfnbox}}
\newenvironment{lmm}{\begin{lmmbox}\begin{mylmm}}{\end{mylmm}\end{lmmbox}}
\newenvironment{prp}{\begin{prpbox}\begin{myprp}}{\end{myprp}\end{prpbox}}

% 枠つき環境ここまで
%=======================================================









\newcommand{\Dlcon}{\mathop{\mathsf{D}^{+}_{\rr_{>0}}}\nolimits}
%\newcommand{\Dlcon}{\mathop{\mathsf{D}^{+}_{\rr^+}}\nolimits}
\newcommand{\Dbcon}{\mathop{\mathsf{D}^{\mathrm{b}}_{\rr^+}}\nolimits}

\newcommand{\Int}[1][]{\mathop{\mathrm{Int}}\nolimits_{#1}}
\newcommand{\Cl}[1][]{\mathop{\mathrm{Cl}}\nolimits_{#1}}
\newcommand{\Bd}[1][]{\mathop{\mathrm{Bd}}\nolimits_{#1}}
\newcommand{\Fr}[1][]{\mathop{\mathrm{Fr}}\nolimits_{#1}}



\newcommand{\rrp}{\rr_{>0}}





% ---------------------------
% new definition macro
% ---------------------------
% 便利なマクロ集です

% 内積のマクロ
%   例えば \inner<\pphi | \psi> のように用いる
\def\inner<#1>{\langle #1 \rangle}

% ================================
% マクロを追加する場合のスペース 

%=================================















\title{ゼミノート(2024/10/23 発表分)}
\author{大柴寿浩}
\begin{document}
\maketitle

\section{フーリエ・佐藤変換(復習){\cite[\S 3.7]{KS90}}}

\begin{CRL}[{\cite[Corollary 3.7.3]{KS90}}]\label{373}
    \(U\)を\(X\)の開集合とする.
    \(X\)内の任意の\(\rr^{+}\)軌道\(b\)に対し,
    \(b\cap U\)が可縮である(よってとくに空でない)とする.
    さらに,任意の\(x\in X\)に対し,集合\(
        \{t\in\rr^{+};\mu(x,t)\in U\}
    \)が可縮であると仮定する.
    このとき,\(F\in\Dlcon(X)\)に対し,
    制限射\(\RG(X;F)\to\RG(U;F)\)は同型となる.
\end{CRL}

\begin{CMT}
    「さらに」から始まる文はSchapira のホームページにあるerrataによる.
\end{CMT}

\begin{thm}[{\cite[Theorem 3.7.7]{KS90}}]
    \(\Dlcon(E)\)から\(\Dlcon(E^\ast)\)への
    関手\(\widetilde{\varPhi}_{P'}\)と
    \(\widetilde{\varPsi}_{P}\)は自然に同型である.
\end{thm}
\begin{proof}
    \(
        \widetilde{\varPhi}_{P'}(F)
        =\Rder{p_2}_!\left(p_1^{-1}F\right)_{P'}
    \), \(
        \widetilde{\varPsi}_{P}(F)
        =\Rder{p_2}_{\ast}\RG_{P}(p_1^{-1}F)
    \)なので,これらを同型で結ぶ.
    \begin{align*}
        \widetilde{\varPhi}_{P'}(F)
        &=\Rder{p_2}_!\left(p_1^{-1}F\right)_{P'}\\
        &\cong\Rder{p_2}_!\RG_{P}\left(p_1^{-1}F\right)_{P'}\\
        &\cong\Rder{p_2}_!\left(\RG_{P}\left(p_1^{-1}F\right)_{P'}\right)\\
        &\cong\Rder{p_2}_{\ast}\left(
            \RG_{P}\left(p_1^{-1}F\right)_{P'}
        \right)\\
        &\cong\Rder{p_2}_{\ast}\RG_{P}(p_1^{-1}F).
    \end{align*}
\end{proof}
\begin{dfn}[{\cite[Definition 3.7.8]{KS90}}]
    \(F\in\Dlcon(E)\)とする.
    \(F\)の\textbf{フーリエ・佐藤変換} (Fourier-Sato transform) 
    \(F^{\wedge}\)を
    \begin{align*}
        F^{\wedge}
        \coloneqq \widetilde{\varPhi}_{P'}(F)\
        \left(=\Rder{p_2}_!\left(p_1^{-1}F\right)_{P'}\right)\\
        \left(
            \cong\widetilde{\varPsi}_{P}(F)
            =\Rder{p_2}_{\ast}\RG_{P}(p_1^{-1}F)
        \right)
    \end{align*}
    で定める.

    \(G\in\Dlcon(E^\ast)\)とする.
    \(G\)の\textbf{逆フーリエ・佐藤変換} (inverse 
    Fourier-Sato transform) \(G^{\vee}\)を
    \begin{align*}
        G^{\vee}
        \coloneqq \widetilde{\varPsi}_{P'}(F)\
        \left(=\Rder{p_1}_{\ast}\RG_{P'}(p_2^{!}G)\right)\\
        \left(
            \cong \widetilde{\varPhi}_{P}(G)
            =\Rder{p_1}_!\left(p_2^{!}G\right)_{P}
        \right)
    \end{align*}
    で定める.
\end{dfn}
\begin{thm}[{\cite[Theorem 3.7.9]{KS90}}]
    \(\Dlcon(E)\)から\(\Dlcon(E^\ast)\)への
    関手\({}^{\wedge}\)と
    \(\Dlcon(E^\ast)\)から\(\Dlcon(E)\)への
    関手\({}^{\vee}\)は圏同値であり,互いに準逆である.
    とくに,\(F\)と\(F'\)を\(\Dlcon(E)\)の対象とするとき,
    \[
        \Hom_{\Dlcon(E)}(F',F)\cong
        \Hom_{\Dlcon(E)}(F'^{\wedge},F^{\wedge})
    \]
    が成りたつ.
\end{thm}
\begin{lmm}[{\cite[Lemma 3.7.10]{KS90}}]
    \begin{enumerate}[(i)]
        \item \(\gamma\)を\(E\)の固有閉凸錐で零切断を含むものとする.
        このとき,次が成り立つ.\[
            \left(A_\gamma\right)^\wedge
            \cong A_{\Int{\gamma^\circ}}.
        \]
        \item \(U\)を\(E\)の開凸錐とする.このとき次が成り立つ.
        \[
            \left(A_U\right)^\wedge\cong
            A_{{U^{\circ}}^a}\tens \ori_{E^\ast/Z}[-n].
        \]
    \end{enumerate}
\end{lmm}
\begin{RMK}[{\cite[Remark 3.7.11]{KS90}}]
    RMK
\end{RMK}
\begin{prp}[{\cite[Proposition 3.7.12]{KS90}}]
    \(F\in\Dlcon(E)\)とする.
    \begin{enumerate}[(i)]
        \item \(F^{\wedge\wedge}\cong F^a\tens \ori_{E/Z}[-n]\).
        \item \(U\)を\(E^\ast\)の開凸集合とすると\[
            \RG(U;F^\wedge)
            \cong\RG_{U^\circ}(\tau^{-1}\pi(U);F)
            \cong\RG_{U^\circ}(E;F).
        \]
        \item \(\gamma\)を\(E^\ast\)の固有閉凸錐で
        零切断を含むものとすると\[
            \RG_{\gamma}(E^\ast;F^\wedge)
            \cong
            \RG(\Int{\gamma^{\circ a}};F\tens\ori_{E/Z}[-n]).
        \]
        \item 次が成り立つ.\[
            \left(\rmD'F\right)^\vee\cong\rmD'(F^\wedge),
            \quad
            \left(\rmD F\right)^\vee\cong\rmD(F^\wedge).
            \]
    \end{enumerate}
\end{prp}
\begin{proof}
    (ii) 
    \begin{align*}
        \RG(U;F^{\wedge})
        &\cong \RHom_{A_{E^{\ast}}}(A_{U},F^{\wedge})\\
        &\cong \dots\\
        &\cong \RG_{U^{\circ}}(\tau^{-1}(\pi(U));F)\\
        &\cong \RG_{U^{\circ}}(E;F)
    \end{align*}
    後半の\[
        \RG_{U^{\circ}}(\tau^{-1}(\pi(U));F)
        \cong 
        \RG_{U^{\circ}}(E;F)
    \]について,\(\tau^{-1}(\pi(U))\)は次の2条件
    \begin{itemize}
        \item 各\(\rrp\)軌道\(b\)に対し,\(b\cap \tau^{-1}(\pi(U))\)が可縮である
        \item 任意の\(v\in E\)に対し,集合\(\{t\in\rrp;\mu(v,t)\in\tau^{-1}(\pi(U))\}\)が可縮である
    \end{itemize}
    をみたす.したがって,系\ref{373}から,制限射が同型であることが従う.

    (iii) 
    \begin{align*}
        \RG_{\gamma}(E^{\ast};F^{\wedge})
        &\cong\RHom(A_{\gamma},F^{\wedge})\\
        &\underset{\text{Thm.3.7.9}}{\cong}\RHom((A_{\gamma})^{\wedge},F^{\wedge\wedge})\\
        &\underset{\text{Thm.3.7.10(i);(i)}}{\cong}\RHom(A_{\Int{\gamma^{\circ}}},F^a\tens \ori_{E/Z}[-n])\\
        &\cong\RHom(A_{\Int{\gamma^{\circ a}}},F\tens \ori_{E/Z}[-n])\\
        &\cong\RG({\Int{\gamma^{\circ a}}};F\tens \ori_{E/Z}[-n]).
    \end{align*}

    (iv) 
    \(\rmD\)の方について示す.
    \begin{align*}
        \RHOM_{A_{E}}(F^{\wedge},\omega_{E^{\ast}/Z})
        &=\RHOM_{A_{E}}(\Rder{{p_{2}}_{!}}(p_{1}^{-1}F)_{P'},\omega_{E^{\ast}/Z})\\
        &=\Rder{{p_{2}}_{\ast}}\RHOM_{A_{E}}((p_{1}^{-1}F)_{P'},p_{2}^{!}\omega_{E^{\ast}/Z})\\
        &=\Rder{{p_{2}}_{\ast}}\RG_{P'}\RHOM_{A_{E}}(p_{1}^{-1}F,p_{2}^{!}\omega_{E^{\ast}/Z})\\
        &=\Rder{{p_{2}}_{\ast}}\RG_{P'}\RHOM_{A_{E}}(p_{1}^{-1}F,p_{1}^{!}\omega_{E/Z})\\
        &=\Rder{{p_{2}}_{\ast}}\RG_{P'}p_{1}^{!}\RHOM_{A_{E}}(F,\omega_{E/Z})\\
        &=(\rmD{F})^{\vee}
    \end{align*}
\end{proof}

\begin{prp}[{\cite[Proposition 3.7.13]{KS90}}]
    \begin{enumerate}[(i)]
        \item \(F\in\Dlcon(E)\)とする.このとき次が成り立つ.\[
            (f_\tau^!F)^\wedge\cong f_\pi^!(F^\wedge),
            \quad
            (f_\tau^{-1}F)^\wedge\cong f_\pi^{-1}(F^\wedge).
            \]
        \item \(G\in\Dlcon(E')\)とする.このとき次が成り立つ.\[
            (\Rder{f_\tau}_\ast G)^\wedge\cong \Rder{f_\pi}_\ast(G^\wedge),
            \quad
            (\Rder{f_\tau}_! G)^\wedge\cong \Rder{f_\pi}_!(G^\wedge),
            \]
    \end{enumerate}
\end{prp}

\begin{prp}[{\cite[Proposition 3.7.14]{KS90}}]
    \begin{enumerate}[(i)]
        \item \(F\in\Dlcon(E_1)\)とする.このとき次が成り立つ.
        \begin{align*}
            {}^tf^{-1}(F^\wedge)&\cong (\Rder{f}_!F)^\wedge,\\
            {}^tf^{!}(F^\vee)&\cong (\Rder{f}_\ast F)^\vee,\\
            {}^tf^{!}(F^\wedge)&\cong (\Rder{f}_{\ast}F)^\wedge\tens\omega_{E_2^\ast/E_1^\ast},\\
            {}^tf^{!}(F^\vee)&\cong (\Rder{f}_{!}F)^\vee\tens\omega_{E_2^\ast/E_1^\ast}.
        \end{align*}
        \item \(G\in\Dlcon(E_2)\)とする.このとき次が成り立つ.
        \begin{align*}
            (f^{-1}F)^\vee&\cong \Rder^t{f}_!(G^\vee),\\
            (f^{!}F)^\wedge&\cong \Rder^t{f}_\ast(G^\wedge),\\
            (\omega_{E_1/E_2}\tens f^!G)^\vee&\cong \Rder^t{f}_\ast(G^\vee),\\
            (\omega_{E_1/E_2}\tens f^{-1}G)^\wedge&\cong \Rder^t{f}_!(G^\wedge).
        \end{align*}
    \end{enumerate}
\end{prp}

\begin{prp}[{\cite[Proposition 3.7.15]{KS90}}]
    \(F_i\in\Dlcon(E_i)\), \(i=1,2\)とする.このとき次が成り立つ.
    \[
        F_1^\wedge\letens[Z]F_2^\wedge
        \cong
        \left(F_1\letens[Z]F_2\right)^\wedge.
    \]
\end{prp}

\section{特殊化(復習){\cite[\S 4.2]{KS90}}}

\begin{thm}[{\cite[Thm 4.2.3]{KS90}}]
    \(F\in\Dompb(X)\)とする.
    \begin{enumerate}[(i)]
        \item \(\nu_{M}(F)\in\Dbcon(T_{M}X)\)であり,
        \(\supp(\nu_{M}(F))\subset C_{M}(\supp(F))\)である.
        \item \(V\)を\(T_{M}X\)の錐状開集合とする.このとき\[
            H^{j}(V;\nu_{M}(F))
            \cong
            \indlim[U]H^{j}(U;F)
        \]である.
        ただし\(U\)は\(
            C_{M}(X-U)\cap V=\varnothing
        \)となる\(X\)の開集合の族を走る.
        とくに\(v\in T_{M}X\)ならば,
        \[
            H^{j}(\nu_{M}(F))_{v}
            \cong
            \indlim[U]H^{j}(U;F)
        \]である.
        ただし\(U\)は\(
            v\notin C_{M}(X-U)
        \)となる\(X\)の開集合の族を走る.
        \item \(A\)を\(T_{M}X\)の錐状閉集合とする.このとき\[
            H^{j}_{A}(T_{M}X;\nu_{M}(F))
            \cong
            \indlim[Z,U]H^{j}_{Z\cap U}(U;F)
        \]である.
        ただし\(U\)は\(
            M
        \)の\(X\)における開近傍の族を走り,
        \(Z\)は\(C_{M}(Z)\subset A\)となる\(X\)の閉集合を走る.
        \item 次の同型が成り立つ.\begin{align*}
            \mres[\nu_{M}(F)]{M}&\cong \Rder{\tau}_{\ast}(\nu_{M}(F))\cong \mres[F]{M},\\
            \mres[(\RG_{M}(\nu_{M}(F)))]{M}&\cong \Rder{\tau}_{!}(\nu_{M}(F))\cong \mres[\RG_{M}(F)]{M}.
        \end{align*}
        \item \(\Rder{\dot{\tau}}_{\ast}(\mres[\nu_{M}(F)]{\dot{T}_{M}X})\cong\mres[\RG_{X-M}(F)]{M}\)である.
    \end{enumerate}
\end{thm}

\begin{proof}
    (i):
    \(\tilde{p}^{-1}F\)が錐状層であることを示せば,
    \(\Rder{j}_{\ast}\)と\(s^{-1}\)が
    錐状層を錐状層に送ることから結果が従う.
    各次数\(j\in \zz\)に対し,
    \(H^{j}(\tilde{p}^{-1}F)\)が各ファイバー\(
        (x',x'',t)\in \varOmega
    \)の\(\rr^{+}\)軌道\(b=\rr^{+}(x',x'',t)\)上で
    局所定数であることを示す.
    \((cx',x'',c^{-1}t)\in b\)とする.このとき
    \begin{align*}
        \left(
            \mres[H^{j}(\tilde{p}^{-1}F)]{b}
        \right)_{(cx',x'',c^{-1}t)}
        &\cong
        \left(
            H^{j}(\tilde{p}^{-1}F)
        \right)_{(cx',x'',c^{-1}t)}\\
        &\cong
        \left(
            \tilde{p}^{-1}H^{j}(F)
        \right)_{(cx',x'',c^{-1}t)}\\
        &\cong
        \left(
            H^{j}(F)
        \right)_{\tilde{p}(cx',x'',c^{-1}t)}\\
        &\cong
        H^{j}(F)_{\tilde{p}(x',x'')}
    \end{align*}
    であり,\(b\)の全ての点での茎が同型となる.
    よって各実数\(c>0\)に対し,\((cx',x'',c^{-1}t)\)の
    十分小さい近傍\(B_c\subset \varOmega\)で\(
        \mres[H^{j}(\tilde{p}^{-1}F)]{b\cap B_c}
    \)が定数層\(\left(
        H^{j}(F)_{\tilde{p}(x',x'')}
    \right)_{b\cap B_c}\)となるものが存在する.
    \(b=\bigcup_{c\in\rr^+}B_{c}\cap b\)である.
    よって\(H^{j}(\tilde{p}^{-1}F)\)は錐状層である.

    \((x'';v)\in T_{M}X-{C_{M}(\supp(F))}\)とする.
    各\(j\in\zz\)に対し
    \[
        H^{j}(F)_{(x'';v)}=0
    \]
    であるから\((x,0)=(x'';v)\in T_{M}X=t^{-1}(0)\)として,
    \begin{align*}
        \left(
            H^{j}(s^{-1}\Rder{j}_{\ast}\tilde{p}^{-1}F)
        \right)_{(x,0)}
        =s^{-1}\Rder{j}_{\ast}\tilde{p}^{-1}H^{j}(F)_{(x'';v)}
        =0
    \end{align*}
    である.したがって\((x'';v)\in T_{M}X-\supp(\nu_{M}(F))\)である.

    (ii):
    \(U\)を\(X\)の開集合で\(C_{M}(X-U)\cap V=\varnothing\)となるものとする.
    次の射がある.
    \begin{equation}
        \begin{cases}
            \RG(U;F)
            &\underset{(1)}{\to}\RG(p^{-1}(U);p^{-1}F)\\
            &\underset{(2)}{\to}\RG(p^{-1}(U)\cap \varOmega;p^{-1}F)\\
            &\underset{(3)}{\to}\RG(\tilde{p}^{-1}(U)\cup V;\Rder{j}_{\ast}j^{-1}p^{-1}F)\\
            &\underset{(4)}{\to}\RG(V;\nu_{M}(F)).
        \end{cases}
    \end{equation}
    ただし,それぞれの射は次のように定まる.
    \begin{enumerate}[(1)]
        \item \(F\to\Rder{p}_{\ast}p^{-1}F\)から定まる.
        \item 制限射.もっというと\(A_{p^{-1}(U)\cap\varOmega}\to A_{p^{-1}(U)}\)から定まる.
        \item まず\(
            p^{-1}F\to\Rder{j}_{\ast}j^{-1}p^{-1}F
        \)から\(
            \RG(\tilde{p}^{-1}(U);\Rder{j}_{\ast}j^{-1}p^{-1}F)
        \)への射が定まる.
        \(\varOmega\)で台を切り落としているので\(
            \RG(\tilde{p}^{-1}(U);\Rder{j}_{\ast}j^{-1}p^{-1}F)
        \)を\(V\)の\(\bar{\varOmega}\)での開近傍\(
            \tilde{p}^{-1}(U)\cup V
        \)に広げてもよい.
        \item \(
            \Rder{j}_{\ast}j^{-1}p^{-1}F
            \to
            \Rder{s}_{\ast}s^{-1}\Rder{j}_{\ast}j^{-1}p^{-1}F
        \)と\(
            s^{-1}A_{\tilde{p}^{-1}(U)\cup V}
            =A_{s^{-1}(\tilde{p}^{-1}(U)\cup V)}
            =A_V
        \)から定まる.
    \end{enumerate}
    コホモロジーをとって,帰納極限の普遍性を考えると次の射
    が定まる.
    \[
        \indlim[U]H^{k}(U;F)\to H^{k}(V;\nu_{M}(F)).
    \]
    この射が同型であることを示す.いま
    \begin{align*}
        H^{k}(V;\nu_{M}(F))
        &\cong
        \indlim[W\in I_V]H^{k}(W,\Rder{j}_{\ast}j^{-1}p^{-1}F)
        \quad (\because\text{注意2.6.9})\\
        &\cong\indlim[W\in I_V]H^{k}(W\cap\varOmega,p^{-1}F)
        \quad (\because j^{-1}\dashv \Rder{j}_{\ast},
        ~ j_{!}\dashv j^{-1})
    \end{align*}
    である.
    命題\ref{prp414}より,
    \(p\colon W\cap\varOmega \to p(W\cap\varOmega)\)の
    各ファイバーは連結,すなわち\(\rr\)と同相としてよい.
    このとき,\begin{enumerate}[(i)]
        \item \(p\)は位相的沈めこみである.
        \item \(
            \Rder{p}_{!}p^{!}\zz_{p(W\cap\varOmega)}
            \to\zz_{W\cap\varOmega}
        \)は同型である.
        実際,ファイバーが\(\rr\)と同相であることから,
        注意3.3.10の式(3.3.13)\[
            \zz\simar\RG(p^{-1}(x);\zz_{p^{-1}(x)})
        \]が成り立つ.
    \end{enumerate}したがって命題3.3.9を\(p\)に適用すると\(
        F\to \Rder{p}_{!}p^{!}F
    \)となることから\[
        H^{k}(W\cap\varOmega;p^{-1}F)
        \cong 
        H^{k}(p(W\cap\varOmega);F)
    \]である.命題\ref{prp413} (i) より,\(W\)を走らせたときの
    \(U=p(W\cap\varOmega)\)は\(
        C_{M}(X-U)\cap V=\varnothing
    \)となる\(X\)の開集合を走る.したがって上の同型が示された.

    (iii):
    \(U\)を\(M\)の\(X\)における開近傍,
    \(Z\)を\(C_{M}(Z)\subset A\)となる\(X\)の閉集合とする.
    次の射の列がある.
    \begin{align*}
        \RG_{Z\cap U}(U;F)
        &\underset{(1)}{\to}\RG_{p^{-1}(Z\cap U)}(p^{-1}(U);p^{-1}F)\\
        &\underset{(2)}{\to}\RG_{p^{-1}(Z\cap U)\cap \varOmega}(p^{-1}(U)\cap \varOmega;p^{-1}F)\\
        &\underset{(3)}{\to}\RG_{(p^{-1}(Z\cap U)\cap \varOmega)\cup A}(p^{-1}(U);\Rder{j}_{\ast}j^{-1}p^{-1}F)\\
        &\underset{(4)}{\to}\RG_{A}(T_{M}X;\nu_{M}(F)).
    \end{align*}
    ただし,それぞれの射は次のように定まる.
    \begin{enumerate}[(1)]
        \item \(F\to \Rder{p}_{\ast}p^{-1}F\)に\(\RG_{Z\cap U}(U;\blk)\)を適用.
        \item \(p^{-1}F\to \Rder{j}_{\ast}j^{-1}p^{-1}F\)に\(\RG_{p^{-1}(Z\cap U)}(p^{-1}(U);\blk)\)を適用.
        \item 切り落としと台の随伴と微妙なやつ.
        \item \(s^{-1}\)を当てる.\(T_{M}X\)上の切断は\(A\)に台を持つことから.
    \end{enumerate}
    以上の合成と切除の三角を考えると,次の可換図式が得られる.
    \[    
        \vcenter{\xymatrix@C=18pt@R=18pt{
        \cdots
        \ar[r]
        &
        \indlim[U]H^{k-1}(U-Z;F)
        \ar[d]^-{\gamma_{k-1}}
        \ar[r]
        %\ar@{}[dr]|\square
        &
        \indlim[U]H^{k}_{U-Z}(U;F)
        \ar[r]
        \ar[d]^-{\alpha_k}
        &
        \indlim[U]H^{k}(U;F)
        \ar[r]
        \ar[d]^-{\beta_k}
        &
        \cdots
        \\
        \cdots
        \ar[r]
        &
        H^{k-1}(T_{M}X-A;\nu_{M}F)
        \ar[r]
        %\ar@{}[dr]|\square
        &
        H^{k}_{A}(T_{M}X;\nu_{M}F)
        \ar[r]
        &
        H^{k}(T_{M}X;\nu_{M}F)
        \ar[r]
        &
        \cdots.
      }}
    \]
    各行は完全であり,\(\gamma_k\)と\(\beta_k\)は (ii) より同型である.
    したがって,\(\alpha_k\)も同型である.

    (iv):
    \(k\colon M\hookrightarrow T_{M}X\)を零切断とみなす閉埋め込みとする.
    このとき,\begin{align*}
        \mres[F]{M}
        =i^{-1}F
        &\cong k^{-1}s^{-1}p^{-1}F\\
        &\to k^{-1}s^{-1}\Rder{j}_{\ast}j^{-1}p^{-1}F\\
        &\cong k^{-1}\nu_{M}F\\
        &=\mres[\nu_{M}F]{M}
    \end{align*}
    と\begin{align*}
        k^{!}\nu_{M}F
        &= k^{!}s^{!}j_{!}j^{!}p^{!}F\\
        &\to k^{!}s^{!}p^{!}F\\
        &= i^{!}F=i^{-1}\RG_{M}F
    \end{align*}
    という射が得られる.
    これらは (ii), (iii) から同型である.(\(v\in T_{M}X\)として\(0\)を取ればよい.)
    残りの同型は\(i^{-1}\cong \Rder{\tau}_{\ast}\)と\(i^{!}\cong \Rder{\tau}_{!}\)から従う.

    (v):
    次の三角の射がある.
    \[    
        \vcenter{\xymatrix@C=26pt@R=26pt{
        \mres[\RG_{M}(F)]{M}    
        \ar[d]
        \ar[r]
        %\ar@{}[dr]|\square
        &
        \mres[F]{M}    
        \ar[r]
        \ar[d]
        &
        \mres[\RG_{X-M}(F)]{M}    
        \ar[r]
        \ar[d]
        &
        +1
        \\
        \mres[\RG_{M}(\nu_{M}F)]{M}    
        \ar[r]
        %\ar@{}[dr]|\square
        &
        \Rder{\tau_{\ast}}\nu_{M}F    
        \ar[r]
        &
        \Rder{\dot{\tau}}_{\ast}(\mres[\nu_{M}F]{\dot{T}_{M}X})    
        \ar[r]
        &
        +1.
      }}
    \]
    左と真ん中が同型なので右も同型である.
\end{proof}

%\subsection*{前回飛ばしたところ}
\begin{prp}[{\cite[Prop.4.2.4]{KS90}}]
    \(G\in\Dompb(Y)\)とする.
    \begin{enumerate}[(i)]
        \item 標準的な射による可換図式    \[    
            \vcenter{\xymatrix@C=36pt@R=36pt{
            \Rder(T_{N}f)_{!}\nu_{N}(G)
            \ar[d]
            \ar[r]
            %\ar@{}[dr]|\square
            &
            \Rder({f_{!}}G)
            %\ar[r]
            \ar[d]
            \\
            \Rder(T_{N}f)_{\ast}\nu_{N}(G)
            %\ar@{}[dr]|\square
            &
            \Rder({f_{\ast}}G)
            \ar[l]
          }}
        \]が存在する.
        \item さらに,\(\supp(G)\to X\)と\(
            C_N(\supp(G))\to T_{M}X
        \)が固有かつ\(\supp(G)\cap f^{-1}(M)\subset N\)ならば,
        以上の射はすべて同型である.
        
        とくに\(f^{-1}(M)=N\)かつ,
        \(f\)が\(M\)に関して斉かつ\(\supp(G)\)上固有のとき,
        以上の射はすべて同型である.
    \end{enumerate}
\end{prp}

\begin{proof}
    (ii) 
    \(\overline{\tilde{p}_{Y}^{-1}(\supp(G))}\)が\(
        \tilde{X}_M
    \)上固有ならば,
    \(\Rder{\tilde{f'}_{\ast}}\)と\(\Rder(T_{N}f)_{\ast}\)を
    それぞれ\(\Rder{\tilde{f'}_{!}}\)と\(\Rder(T_{N}f)_{!}\)に
    置き換えることができるので,射はすべて同型となる.
    したがって,\(Y\)の閉部分集合\(Z\)に対し,
    \(Z\)が\(X\)上固有であること,
    \(C_{N}(Z)\)が\(T_{M}X\)上固有であること,
    そして\(Z\cap f^{-1}(M)\subset N\)であることを仮定したとき,
    \(\overline{\tilde{p}_{Y}^{-1}(Z)}\)が\(\tilde{X}_{M}\)上
    固有であることを示せばよい.
    \(
        \overline{\tilde{p}_{Y}^{-1}(Z)}\to\tilde{X}_{M}
    \)のファイバーはコンパクトなので,閉写像であることを示せば十分である.
    \(\{u_{n}\}_{n}\)を\(
        \overline{\tilde{p}_{Y}^{-1}(Z)}
    \)の点列で\(\{\tilde{f}'(u_{n})\}_{n}\)が収束するものとする.
    \(\{u_{n}\}_{n}\)の部分列で収束するものが存在することを示せばよい.
    \(\{\tilde{p}_{Y}(u_{n})\}_{n}\)が収束すると仮定してよい.
    \(\tilde{p}^{-1}_{Y}(Z)-T_{N}Y\to \tilde{X}_{M}-T_{M}X\)は固有なので,
    \(\{\tilde{f}'(u_{n})\}_{n}\)は\(T_{M}X\)の点に収束するとしてよい.
    このとき,\(\{\tilde{p}_{Y}(u_{n})\}_{n}\)の
    極限点は\(Z\cap f^{-1}(M)\)に含まれるので\(N\)に含まれる.

    \(X\)と\(Y\)の局所座標系を(4.1.10)のようにとり,
    \(u_{n}=(y'_{n},y''_{n},t_n)\)とおく.
    このとき,\(
        t_{n}\underset{n}{\to}0
    \), \(t_{n}y'_{n}\underset{n}{\to}0\)である.
    さらに\(t_{n}>0\)(すなわち\(
        u_{n}\in\tilde{p}_{Y}^{-1}(z)
    \))としてよい.
    \(\{y''_{n}\}_{n}\)は収束するので,
    \(\{\lvert{y'_n}\rvert\}_{n}\)が有界であることを示せばよい.
    いま,\(\lvert{y'_{n}}\rvert\underset{n}{\to}\infty\)だったと仮定して矛盾を導く.
    部分列を取り出すことで,\(
        \{y'_{n}/\lvert{y'_{n}}\rvert\}_{n}
    \)は\(0\)でないベクトル\(v\)に収束する.
    このとき\(
        \{(y'_{n}/\lvert{y'_{n}}\rvert,
        y''_{n},
        t_{n}\lvert{y'_{n}}\rvert)\}_{n}
    \)は\(\tilde{p}^{-1}_{Y}(Z)\)に属し,
    点\(p\in T_{N}Y\)で零切断に属さないものに収束する.
    他方,\(
        \tilde{f'}(u_{n})
        =\left(
            \dfrac{1}{t_{n}}f_1(t_n,y'_n,y''_n),
            f_2(t_n,y'_n,y''_n)
        \right)
    \)は収束し\(
        \left\{
            \dfrac{1}{t_{n}\lvert{y'_{n}}\rvert}
            f_{1}(t_{n},y'_n,y''_n)
        \right\}_{n}
    \)は\(0\)に収束する.
    これより\(T_{N}f(p)\)は\(T_{M}X\)の零切断に属することが従う.
    したがって\(
        C_{N}(Z)\cap (T_{N}f)^{-1}(T_{N}f(p))
        \supset\rr_{\geqq0}p
    \)となる.
    これは\(C_{N}(Z)\to T_{M}X\)が固有であるという
    事実にムジュンする.
\end{proof}

\section{超局所化{\cite[\S 4.3]{KS90}}}

\begin{dfn}[{\cite[Def 4.3.1]{KS90}}]
    \(F\in\Dompb(X)\)とする.\(\nu_{M}(F)\)のフーリエ佐藤変換
    \[
        \mu_{M}(F)
        \coloneqq 
        \nu_{M}(F)^{\wedge}
    \]
    を\(F\)の\(M\)に沿った\textbf{超局所化} (microlocalization) と
    いう.
\end{dfn}

定理4.2.3とIII \S7の結果を適用することで次が得られる.

\begin{thm}[{\cite[Thm 4.3.2]{KS90}}]
    \(F\in\Dompb(X)\)とする.
    \begin{enumerate}[(i)]
        \item \(\mu_{M}(F)\in\Dbcon(T_{M}X)\)である.
        \item \(V\)を\(T^{\ast}_{M}X\)の錐状開集合とする.
        このとき\[
            H^{j}(V;\mu_{M}(F))
            \cong
            \indlim[U,Z]H_{Z\cap U}^{j}(U;F)
        \]である.
        ただし\(U\)は\(
            U\cap M=\pi(V)
        \)となる\(X\)の開集合の族を走り,
        \(Z\)は\(X\)の閉集合で\(
            C_{M}(Z)_{\pi(p)}
            \subset 
            \{v\in (T_{M}X)_{\pi(p)};\inner<v,p>>0\}\cup\{0\}
        \)となるものの族を走る.
        \item \(Z\)を\(T_{M}X\)の固有閉凸錐で零切断を含むものとする.
        このとき\[
            H^{j}_{Z}(T^{\ast}_{M}X;\mu_{M}(F)\tens[]\ori_{M/X})
            \cong
            \indlim[U]H^{j-l}(U;F)
        \]である.
        ただし\(U\)は\(X\)の開集合で\(
            C_{M}(X-U)\cap\Int[]{Z^{\circ a}}=\varnothing
        \)となるものの族を走る.
        \item 次の同型が成り立つ.\begin{align*}
            \mres[\mu_{M}(F)]{M}
            &\cong \Rder{\pi}_{\ast}(\mu_{M}(F))
            \cong \mres[\RG_{M}(F)]{M}
            \cong i^{!}F,\\
            \RG_{M}(\mu_{M}(F))
            &\cong \Rder{\pi}_{!}(\mu_{M}(F))
            \cong i^{-1}F\tens[]\omega_{M/X}.
        \end{align*}
    \end{enumerate}
\end{thm}

次が成り立つ.
\[
    i^{-1}F\tens[]\omega_{M/X}
    =\mres[F]{M}\tens[]\ori_{M/X}[-l].
\]

最後の結果を完全三角
\[
    \RG_M(\mu_{M}(F))\to
    \Rder{\pi_{\ast}}\mu_{M}(F)\to
    \Rder{\dot{\pi}_{\ast}\mu_{M}(F)\to+1}
\]
に適用することで,完全三角
\[
    \mres[F]{M}\tens[]\omega_{M/X}\to
    \mres[\RG_M(F)]{M}\to
    \Rder{\dot{\pi}_{\ast}\mu_{M}(F)\to+1}
\]
を得る.
いま,\(f\colon Y\to X\)を多様体の射とし,
\(N\)を\(Y\)の,余次元\(k\)の閉部分多様体で\(f(N)\subset M\)を
みたすものとする.
写像\(Tf\)は














%\section{\(\mhom\){\cite[\S 4.4]{KS90}}}




%===============================================
% 参考文献スペース
%===============================================
\begin{thebibliography}{20} 
    %\bibitem[GP74]{GP74} Victor Guillemin, Alan Pollack, \textit{Differential Topology}, Prentice-Hall, 1974.
    \bibitem[KS90]{KS90} Masaki Kashiwara, Pierre Schapira, 
      \textit{Sheaves on Manifolds}, 
      Grundlehren der Mathematischen Wissenschaften, 292, Springer, 1990.
    %\bibitem[Hat89]{Hat89} 服部晶夫, 多様体 増補版, 岩波全書 288, 岩波書店, 1989.
    \bibitem[Sch04]{Sch04} Pierre Schapira, 
      \textit{Sheaves: from Leray to Grothendieck and Sato}, 
      S\'eminaires et Congres 9, Soc. Math. France, pp.173--181, 2004.
    %\bibitem[Og02]{Og02} 小木曽啓示, 代数曲線論, 朝倉書店, 2022.
\end{thebibliography}
%===============================================
  

\end{document}