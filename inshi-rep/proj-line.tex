\section{複素射影直線}
\subsection{複素射影直線の定義}
$\cc^{2}$から原点$0=(0,0)$を除いた集合$\cc^{2}-\{0\}$
の点$(a_{0},a_{1}), (b_{0},b_{1})$に対し次の関係を考える.
\begin{align}\label{eq:sim1}
    (a_{0},a_{1})\sim (b_{0},b_{1})
    \Longleftrightarrow
    (a_{0},a_{1})= c\cdot(b_{0},b_{1})
    \text{となる複素数}c\neq0\text{が存在する.}
\end{align}
これは同値関係である.
$(a_0,a_1)$の同値類$\{c\cdot(a_0,a_1); c\in\cc-\{0\}\}$を
$[a_0\colon a_1]$とかく.

同値関係${\sim}$の定める商写像を用いて次の集合を定義する.

\begin{Definition}
    $\pp^{1} \coloneqq \left(\cc^{2}-\{0\}\right)/{\sim}$
    を\textbf{複素射影直線} (complex projective line) という.
\end{Definition}

\begin{Definition}\label{def:coord1}
    次の写像の組を考える.
    $\begin{tikzcd}
      {\cc^{2}-\{0\}}
        \arrow[r, shift left ,"\pr_1=X_0"]
        \arrow[r, shift right,"\pr_2=X_1"']
      & {\cc}
    \end{tikzcd}; (a_0,a_1)\mapsto a_0,a_1.$
    この組を$\cc^{2}-\{0\}$の標準座標,$\pp^1$の同次座標という.
\end{Definition}

$\pp^{1}$は商写像$\pi \colon \cc^{2}-\{0\}
\rightarrow\left(\cc^{2}-\{0\}\right)/{\sim}$
による商位相により位相空間になる.この定義から$\pi$の連続性が従う.

$\pp^1$の位相空間としての性質を調べるために,次の部分集合を定義する.
\begin{align*}
    U_0=\{[a_0\colon a_1]\in\pp^1; a_0\neq0\},\quad
    U_1=\{[a_0\colon a_1]\in\pp^1; a_1\neq0\}.
\end{align*}
このとき次が成り立つ.
\begin{align}
    U_0\cup U_1 &= \pp^1, \label{eq:cov1}\\
    U_0\cap U_1 
    &= \{[a_0\colon a_1]\in\pp^1; a_0, a_1\neq0\} \label{eq:inter-p1}\\
    &= U_0 - \{[1\colon 0]\}  \notag\\
    &= U_1 - \{[0\colon 1]\}. \notag
\end{align}

\begin{Lemma}\label{mnf:p1}
    1. 
    商写像$\pi \colon \cc^{2}-\{0\}
    \rightarrow\left(\cc^{2}-\{0\}\right)/{\sim}$
    は開写像である.

    2. 
    $U_0$と$U_1$は$\pp^1$の開集合であり,
    \begin{align*}
        \pphi_0&\colon U_0\overset{{\sim}}{\longrightarrow}\cc;\ [a_0\colon a_1]\mapsto a_1/a_0,\\
        \pphi_1&\colon U_1\overset{{\sim}}{\longrightarrow}\cc;\ [a_0\colon a_1]\mapsto a_0/a_1
    \end{align*}
    はともに同相写像である.

    3. 
    任意の$\dip A = \begin{bmatrix}
        a&b\\c&d
    \end{bmatrix}\in GL(2,\cc)$は自己同相写像
    \begin{align*}
        p_A\colon \pp^1\overset{{\sim}}{\longrightarrow}\pp^1;
        \begin{bmatrix}
            a_0\\a_1
        \end{bmatrix}
        \mapsto
        \begin{bmatrix}
            a&b\\c&d
        \end{bmatrix}
        \begin{bmatrix}
            a_0\\a_1
        \end{bmatrix}
    \end{align*}
    を引き起こす.

    4. 
    $\pp^1$は第2可算公理をみたす連結なコンパクトハウスドルフ空間である.
\end{Lemma}

\begin{proof}[\textbf{証明}]
    1. 
    $U$を$\cc^2-\{0\}$の開集合とする.$\pi(U)$が$\pp^1$の
    開集合であること,すなわち$\pi^{-1}(\pi(U))$が$\cc^2-\{0\}$の
    開集合であることを示す.
    いま,任意の開集合$U\subset\cc^2-\{0\}$に対し,
    複素数$c\neq0$を用いて
    \begin{align*}
        cU = \left\{(ca_0,ca_1); (a_0,a_1)\in\cc^2-\{0\}\right\}
    \end{align*}
    とおくと,$cU$は$\cc^2-\{0\}$の開集合であり,
    \begin{align*}\label{eq:proj}
        \pi^{-1}(\pi(U)) = \bigcup_{c\in\cc-\{0\}} cU %\tag{$\ast$}
    \end{align*}
    なので,$\pi^{-1}(\pi(U))$は$\cc^2-\{0\}$の
    開集合である.

    2. 
    まず$U_0, U_1$が$\pp^1$の開集合であることを示す.
    $U_0=\{[a_0:a_1]; a_0\neq0\}$は$V_0=\{(a_0,a_1);a_0\neq0\}$
    の$\pi$による像であり,$V_0$は
    $\cc^2-\{0\}$の開集合であるから,
    $U_0$は$\pp^1$の開集合である.同様に$U_1$も$\pp^1$の開集合である.

    $\pphi_0\colon U_0\to\cc$が連続であることを示す.$V$を$\cc$の
    開集合とする.
    $V=\pphi_0\circ\pi(V_0) (= \widetilde{\pphi_0}(V_0)$
    とおく)である.
    $\widetilde{\pphi_0}^{-1}(V) 
    = \pi^{-1}\left(\pphi_0^{-1}(V)\right)$は$V_0$の開集合である.
    したがって,これは$\cc^2-\{0\}$の開集合であり,商位相の定義から$\pphi_0^{-1}(V)\subset U_0$
    は開集合である.
    
    $\pphi_0$が同相であることを示す.$\psi_0\colon \cc \to U_0$を
    $\psi_0(z)=[1\colon z]$で定める.このとき
    $\psi_0\circ\pphi_0\left([a_0\colon a_1]\right)
    =\psi_0\left( a_1/a_0 \right)
    =[1\colon a_1/a_0]=[a_0\colon a_1]$である.
    また$\pphi_0\circ\psi_0(z)=\pphi_0([1\colon z])=z/1=z$.
    したがって,$\psi_0\circ\pphi_0=\id_{U_0}$かつ
    $\pphi_0\circ\psi_0=\id_{\cc}$であり,$\psi_0=\pphi_0^{-1}$である.
    $\psi_0=\pphi_0^{-1}$は自然な単射$\cc 
    \hookrightarrow \cc^2-\{0\}$と$\pi$の合成であり,
    これらは連続なので,その合成である$\psi_0$も連続である.
    以上より$\pphi_0$は同相である.

    3. 
    $\dip A = \begin{bmatrix}
        a&b\\c&d
    \end{bmatrix}$を可逆な行列とする.
    $A$を自己同形$\cc^2\to\cc^2$とみたとき,
    それを$\cc^2-\{0\}$に制限した
    $\mapres{A}{\cc^2-\{0\}}\colon\cc^2-\{0\}\to\cc^2-\{0\}$
    は自己同相であり,
    逆写像は$\mapres{A^{-1}}{\cc^2-\{0\}}$で与えられる.
    一般に$A(cx)=cAx$なので,$A$から可逆な写像$p_A$が不備なく定まり,
    逆写像は$p_{A^{-1}}$で与えられる.

    $p_A$が連続であることを示す.$V$を$\pp^1$の開集合とする.
    次の図式が可換であり,$\pi$と$A$は連続写像であるから,
    $\pi^{-1}\left(p_A^{-1}(V)\right)
    =A^{-1}\left(\pi^{-1}(V)\right)$は$\cc^2-\{0\}$の開集合
    である.
    \begin{equation*}
        \vcenter{\xymatrix@C=36pt@R=36pt{
        \cc^2-\{0\} 
        \ar@{{}->}[d]_{\pi} 
        \ar@{{}->}[r]^-{A} 
        & \cc^2-\{0\} 
        \ar@{{}->}[d]^{\pi} 
        \\
        \pp^1 \ar[r]_{p_A}  
        & \pp^1 \ar@{}[lu]
        }}
    \end{equation*}
    $\pp^{1}$の商位相の定義より$\pi^{-1}(V)$は$\pp^{1}$の開集合である.
    したがって$p_A$で連続である.$p_A^{-1}$が連続であることも同様である.

    4. 
    第2可算公理をみたすこと:
    \begin{align*}
        \qq(\sqrt{-1})=\{a+b\sqrt{-1};a,b\in\qq\}
    \end{align*}
    に属する点$z$と有理数$p$に対し
    $U_{p}(z)$を考えると
    $\left(U_{p}(z)\right)_{p\in\qq,z\in\cc}$
    は$\cc$の位相空間としての基底になる.
    したがって$\cc$は第2可算公理をみたす.
    直積集合$\cc^2$も第2可算であるから,1点を除いた$\cc^2-\{0\}$もそうであり,
    これに全射$\pi$を適用した$\pp^1$も第2可算公理をみたす.

    連結かつコンパクトであること:
    $S^3=\{P=(a_0,a_1)\in\cc^2;|a_0|^2+|a_1|^2=1\}\subset\cc^2-\{0\}$
    であり,$\cc^2-\{0\}$の相対位相により,$S^3$は有界閉集合
    つまりコンパクト集合であり,連結である.
    全射連続写像$\mapres{\pi}{S^3}\colon S^3\to\pp^1$により 
    $\pp^1$は連結かつコンパクトである.
    $\mapres{\pi}{S^3}$が全射であることは
    \begin{align*}
        [a_0\colon a_1]
        =
        \left[
            \frac{a_0}{\sqrt{a_0^2+a_1^2}}\colon \frac{a_1}{\sqrt{a_0^2+a_1^2}}
        \right]
    \end{align*}
    であることからしたがう.

    ハウスドルフであること:
    $P\neq Q$を$\pp^1$の点とする.
    $p\colon GL(2,\cc)\to \Aut_{\tTop}(\pp^1)$は
    全射.したがって,$U_0\subset \pp^1$から,任意の
    $p_{A}\in \Aut_{\tTop}(U_0)$に対し$A\in GL(2,\cc)$が存在する.
    つまり$p_A(P),p_A(Q)\in U_0$となる$A\in GL(2,\cc)$が存在する.
    $U_0\cong \cc$であり$\cc$はハウスドルフなので,$p_A(P)$の
    開近傍$U_P$と$p_A(Q)$の開近傍$U_Q$で$U_P\cap U_Q=\emp$を
    みたすものが存在する.
    $U_P$と$U_Q$は$U_0\subset\pp^1$の開集合であり,$p_{A}$が
    同相なので$p_A^{-1}(U_P)$, $p_A^{-1}(U_Q)$は$\pp^1$に
    おける$P$, $Q$の開近傍で
    $p_A^{-1}(U_P)\cap p_A^{-1}(U_Q)=\emp$
    をみたす.よって$\pp^1$はハウスドルフである.
\end{proof}

\subsection{貼りあわせ関数}\label{ssec:patch}

補題\ref{mnf:p1}.2 から
$\pphi_0 \colon U_0\isom\cc$, $\pphi_1\colon {U_1} \isom\cc$
である.
ここで,$\pphi_0(U_0)$の標準座標を$w$, 
$\pphi_1(U_1)$の標準座標を$z$で表すことにする.
定義\ref{def:coord1}のようにかくと
\begin{align*}
    z&\colon \pphi_1(U_1)= \cc \to \cc; (a)\mapsto a\\
    w&\colon \pphi_0(U_0)= \cc \to \cc; (b)\mapsto b
\end{align*}
のようになる.複素数の一つ組に対し第一成分を対応させるということである.
これによって点$(a)$と座標値$z(a)$を同一視し,点を単に$z$と書いたりする.
ガウス平面$\cc$に,そこでの標準座標をつけて$\cc_z$, $\cc_w$のように表すと,
$\cc_w\subset\pp^1$, $\cc_z\subset\pp^1$とみなせる.
$z$も$w$も0でないとき,$\cc_z$と$\cc_w$の間には,
\begin{align}\label{eq:patch1}
    z=\frac{1}{w}
\end{align}
の関係がある.
$z,w\ne 0$は\eqref{eq:inter-p1}より
$[z\colon w]\in U_0\cap U_1$ということである.
$[z\colon w]\in U_0\cap U_1$のとき$z$は$w$の正則関数になっている.
$\pphi_0 (U_0\cap U_1) = \cc_w-\{0\} 
= \cc_w \cap \cc_z 
= \cc_z-\{0\} 
= \pphi_1 (U_0\cap U_1)$
なので,
この正則関数を
$\pphi_{10}\colon \cc_w-\{0\}=\pphi_0 (U_0\cap U_1) 
\to \cc_z-\{0\} = \pphi_1 (U_0\cap U_1)$
とかくことにすると,次の図式が可換になる.
\begin{equation*}
    \vcenter{\xymatrix@C=35pt@R=35pt{
    {U_0\cap U_1} 
        \ar@{<-{}}[d]^{\pphi_0^{-1}} 
        \ar@{{}=}[r]^-{[1\colon w]=[z\colon 1]}
    & {U_0 \cap U_1} 
        %
        \\
    \cc_w -\{0\}
        \ar[r]_{\pphi_{10}}
    & \cc_z -\{0\}
        \ar@{<-{}}[u]_{\pphi_1}
        %\ar@{}[lu]
    }}
\end{equation*}
つまり,$\pphi_{10} = \pphi_1\circ \pphi_0^{-1}$である.
また,$\pphi_{01} = \pphi_0\circ \pphi_1^{-1}\colon \pphi_1 (U_0\cap U_1) \to \pphi_0 (U_0\cap U_1)$
も$w=1/z$として同様に定まる.これは正則であり$\pphi_{10}$の逆関数でもある.

以上から次が従う.

\begin{Proposition}
    複素射影直線$\pp^1$は,$\pp^1$を台集合とし,
    $(\pphi_0\colon U_0\to\cc_w, \pphi_1\colon U_1\to\cc_z)$を
    座標近傍系とするコンパクトリーマン面である.    
\end{Proposition}

\begin{proof}[\bf{証明}]
    コンパクト性は補題\ref{mnf:p1}.4 で示した.
    定義\ref{def:mnf}の(1)--(4)で$n=1$としたものが成り立つことを示す.

    (1) 
    補題\ref{mnf:p1}.4 からしたがう.

    (2) 
    \eqref{eq:cov1}と\eqref{eq:inter-p1} からしたがう.

    (3) 
    補題\ref{mnf:p1}.2 からしたがう.

    (4) 
    上で説明した.
\end{proof}
ここでは2枚の被覆で座標近傍系を定めたが,
以下断りなく極大座標近傍系を考える.