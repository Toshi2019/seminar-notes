
%Don't forget to delete
%showkeys
%overfullrule
%\date \ber \er \cmt

%------------------------
\documentclass[11pt, a4paper, dvipdfmx]{jsarticle}


%usepackage
%------------------------
\usepackage{amsmath}
\usepackage{amsthm}
%\usepackage[psamsfonts]{amssymb}
\usepackage{color}
\usepackage{ascmac}
\usepackage{amsfonts}
\usepackage{mathrsfs}
\usepackage{mathtools}
\usepackage{amssymb}
\usepackage{graphicx}
\usepackage{fancybox}
%\usepackage{enumerate}
\usepackage{enumitem}
\usepackage{verbatim}
\usepackage{subfigure}
\usepackage{proof}
\usepackage{listings}
\usepackage{otf}
\usepackage{algorithm}
\usepackage{algorithmic}
\usepackage{tikz}
\usetikzlibrary{cd}
\usepackage[all]{xy}
\usepackage{amscd}

\usepackage{pb-diagram}

\usepackage[dvipdfmx]{hyperref}
\usepackage{xcolor}
\definecolor{darkgreen}{rgb}{0,0.45,0} 
\definecolor{darkred}{rgb}{0.75,0,0}
\definecolor{darkblue}{rgb}{0,0,0.6} 
\hypersetup{
    colorlinks=true,
    citecolor=darkgreen,
    linkcolor=darkred,
    urlcolor=darkblue,
}
\usepackage{pxjahyper}

\usepackage{enumitem}

\usepackage{bbm}

% ================================
% パッケージを追加する場合のスペース 
\usepackage{latexsym}
\usepackage{wrapfig}
\usepackage{layout}
\usepackage{url}

\usepackage{okumacro}
%=================================


% --------------------------
% theoremstyle
% --------------------------
\theoremstyle{definition}


% --------------------------
% newtheoem
% --------------------------

% 日本語で定理, 命題, 証明などを番号付きで用いるためのコマンドです. 
% If you want to use theorem environment in Japanece, 
% you can use these code. 
% Attention!
% All theorem enivironment numbers depend on 
% only section numbers.
\newtheorem{Axiom}{公理}[section]
\newtheorem{Definition}[Axiom]{定義}
\newtheorem{Theorem}[Axiom]{定理}
\newtheorem{Proposition}[Axiom]{命題}
\newtheorem{Lemma}[Axiom]{補題}
\newtheorem{Corollary}[Axiom]{系}
\newtheorem{Example}[Axiom]{例}
\newtheorem{Claim}[Axiom]{主張}
\newtheorem{Property}[Axiom]{性質}
\newtheorem{Attention}[Axiom]{注意}
\newtheorem{Question}[Axiom]{問}
\newtheorem{Problem}[Axiom]{問題}
\newtheorem{Consideration}[Axiom]{考察}
\newtheorem{Alert}[Axiom]{警告}
\newtheorem{Fact}[Axiom]{事実}


% 日本語で定理, 命題, 証明などを番号なしで用いるためのコマンドです. 
% If you want to use theorem environment with no number in Japanese, You can use these code.
\newtheorem*{Axiom*}{公理}
\newtheorem*{Definition*}{定義}
\newtheorem*{Theorem*}{定理}
\newtheorem*{Proposition*}{命題}
\newtheorem*{Lemma*}{補題}
\newtheorem*{Example*}{例}
\newtheorem*{Corollary*}{系}
\newtheorem*{Claim*}{主張}
\newtheorem*{Property*}{性質}
\newtheorem*{Attention*}{注意}
\newtheorem*{Question*}{問}
\newtheorem*{Problem*}{問題}
\newtheorem*{Consideration*}{考察}
\newtheorem*{Alert*}{警告}
\newtheorem{Fact*}{事実}


% 英語で定理, 命題, 証明などを番号付きで用いるためのコマンドです. 
% If you want to use theorem environment in English, You can use these code.
%all theorem enivironment number depend on only section number.
\newtheorem{Axiom+}{Axiom}[section]
\newtheorem{Definition+}[Axiom+]{Definition}
\newtheorem{Theorem+}[Axiom+]{Theorem}
\newtheorem{Proposition+}[Axiom+]{Proposition}
\newtheorem{Lemma+}[Axiom+]{Lemma}
\newtheorem{Example+}[Axiom+]{Example}
\newtheorem{Corollary+}[Axiom+]{Corollary}
\newtheorem{Claim+}[Axiom+]{Claim}
\newtheorem{Property+}[Axiom+]{Property}
\newtheorem{Attention+}[Axiom+]{Attention}
\newtheorem{Question+}[Axiom+]{Question}
\newtheorem{Problem+}[Axiom+]{Problem}
\newtheorem{Consideration+}[Axiom+]{Consideration}
\newtheorem{Alert+}{Alert}
\newtheorem{Fact+}[Axiom+]{Fact}
\newtheorem{Remark+}[Axiom+]{Remark}

% ----------------------------
% commmand
% ----------------------------
% 執筆に便利なコマンド集です. 
% コマンドを追加する場合は下のスペースへ. 

% 集合の記号 (黒板文字)
\newcommand{\NN}{\mathbb{N}}
\newcommand{\ZZ}{\mathbb{Z}}
\newcommand{\QQ}{\mathbb{Q}}
\newcommand{\RR}{\mathbb{R}}
\newcommand{\CC}{\mathbb{C}}
\newcommand{\PP}{\mathbb{P}}
\newcommand{\KK}{\mathbb{K}}


% 集合の記号 (太文字)
\newcommand{\nn}{\mathbf{N}}
\newcommand{\zz}{\mathbf{Z}}
\newcommand{\qq}{\mathbf{Q}}
\newcommand{\rr}{\mathbf{R}}
\newcommand{\cc}{\mathbf{C}}
\newcommand{\pp}{\mathbf{P}}
\newcommand{\kk}{\mathbf{K}}

% 特殊な写像の記号
\newcommand{\ev}{\mathop{\mathrm{ev}}\nolimits} % 値写像
\newcommand{\pr}{\mathop{\mathrm{pr}}\nolimits} % 射影

% スクリプト体にするコマンド
%   例えば {\mcal C} のように用いる
\newcommand{\mcal}{\mathcal}

% 花文字にするコマンド 
%   例えば {\h C} のように用いる
\newcommand{\h}{\mathscr}

% ヒルベルト空間などの記号
\newcommand{\F}{\mcal{F}}
\newcommand{\X}{\mcal{X}}
\newcommand{\Y}{\mcal{Y}}
\newcommand{\HH}{\mcal{H}}
\newcommand{\RKHS}{\Hil_{k}}
\newcommand{\Loss}{\mcal{L}_{D}}
\newcommand{\MLsp}{(\X, \Y, D, \Hil, \Loss)}

% 偏微分作用素の記号
\newcommand{\p}{\partial}

% 角カッコの記号 (内積は下にマクロがあります)
\newcommand{\lan}{\langle}
\newcommand{\ran}{\rangle}



% 圏の記号など
\newcommand{\Set}{{\bf Set}}
\newcommand{\Vect}{{\bf Vect}}
\newcommand{\FDVect}{{\bf FDVect}}
\newcommand{\Ring}{{\bf Ring}}
\newcommand{\Ab}{{\bf Ab}}
\newcommand{\Mod}{\mathop{\mathrm{Mod}}\nolimits}
\newcommand{\CGA}{{\bf CGA}}
\newcommand{\GVect}{{\bf GVect}}
\newcommand{\Lie}{{\bf Lie}}
\newcommand{\dLie}{{\bf Liec}}



% 射の集合など
\newcommand{\Map}{\mathop{\mathrm{Map}}\nolimits}
\newcommand{\Hom}{\mathop{\mathrm{Hom}}\nolimits}
\newcommand{\End}{\mathop{\mathrm{End}}\nolimits}
\newcommand{\Aut}{\mathop{\mathrm{Aut}}\nolimits}
\newcommand{\Mor}{\mathop{\mathrm{Mor}}\nolimits}

% その他便利なコマンド
\newcommand{\dip}{\displaystyle} % 本文中で数式モード
\newcommand{\e}{\varepsilon} % イプシロン
\newcommand{\dl}{\delta} % デルタ
\newcommand{\pphi}{\varphi} % ファイ
\newcommand{\ti}{\tilde} % チルダ
\newcommand{\pal}{\parallel} % 平行
\newcommand{\op}{{\rm op}} % 双対を取る記号
\newcommand{\lcm}{\mathop{\mathrm{lcm}}\nolimits} % 最小公倍数の記号
\newcommand{\Probsp}{(\Omega, \F, \P)} 
\newcommand{\argmax}{\mathop{\rm arg~max}\limits}
\newcommand{\argmin}{\mathop{\rm arg~min}\limits}





% ================================
% コマンドを追加する場合のスペース 
\newcommand{\UU}{\mcal{U}}
\newcommand{\OO}{\mcal{O}}
\newcommand{\emp}{\varnothing}
\newcommand{\ceq}{\coloneqq}
\newcommand{\sbs}{\subset}
\newcommand{\mapres}[2]{\left. #1 \right|_{#2}}
\newcommand{\ded}{\hfill $\blacksquare$}
\newcommand{\id}{\mathrm{id}}
\newcommand{\isom}{\overset{\sim}{\longrightarrow}}
\newcommand{\tTop}{\textsf{Top}}
\newcommand{\pfb}{\textbf{証明}}
\newcommand{\Int}{\mathop{\mathrm{Int}}\nolimits} % 内部


% 自前の定理環境
%   https://mathlandscape.com/latex-amsthm/
% を参考にした
\newtheoremstyle{mystyle}%   % スタイル名
    {5pt}%                   % 上部スペース
    {5pt}%                   % 下部スペース
    {}%              % 本文フォント
    {}%                  % 1行目のインデント量
    {\bfseries}%                      % 見出しフォント
    {.}%                     % 見出し後の句読点
    {12pt}%                     % 見出し後のスペース
    {\thmname{#1}\thmnumber{ #2 }\thmnote{{\normalfont (#3)}}}% % 見出しの書式

\theoremstyle{mystyle}
\newtheorem{AXM}{公理}[section]
\newtheorem{DFN}[Axiom]{定義}
\newtheorem{THM}[Axiom]{定理}
\newtheorem*{THM*}{定理}
\newtheorem{PRP}[Axiom]{命題}
\newtheorem{LMM}[Axiom]{補題}
\newtheorem{CRL}[Axiom]{系}
\newtheorem{EG}[Axiom]{例}

%\newtheorem{}{Axiom}[]
\numberwithin{equation}{section} % 式番号を「(3.5)」のように印刷

\newcommand{\MM}{\mcal{M}}

% =================================


% ---------------------------
% new definition macro
% ---------------------------
% 便利なマクロ集です

% 内積のマクロ
%   例えば \inner<\pphi | \psi> のように用いる
\def\inner<#1>{\langle #1 \rangle}

% ================================
% マクロを追加する場合のスペース 

%=================================





% ----------------------------
% documenet 
% ----------------------------
% 以下, 本文の執筆スペースです. 
% Your main code must be written between 
% begin document and end document.
% ---------------------------




\begin{document}

\title{リーマン面}
\author{大柴 寿浩}
\date{}

%\date{December 13, 2021}
\begin{titlepage}
\maketitle
\begin{abstract}
    北大の院試用レポート.
    複素トーラス(楕円曲線)からリーマン球面(複素射影直線)への正則射が
    4点で分岐する2重被覆であることを示すことが目的である.
\end{abstract}
\thispagestyle{empty}
\end{titlepage}
%\setcounter{page}{1}
\section*{記号}
次の記号について断りなく用いることがある.
\begin{itemize}
    \item 添字:$I$を添字集合とする
    何らかの族$(x_i)_{i\in I}$を$(x_i)_i$や$(x_i)$のように
    略記することがある.
    \item 位相空間$X$に対し
    $\Aut_{\tTop}(X)\coloneqq\{X\text{上の自己同相写像}\}$とかく.
    \item ガウス平面に含まれる,点$\alpha$を中心とする半径$r$の開円板を
    $D(\alpha;r)\coloneqq\{z\in\cc; |z-\alpha|<r\}$で表す.
\end{itemize}

% enumerate の番号を(1)のようにする
%\renewcommand{\theenumii}{(\arabic{enumii})}
\section{リーマン面}

\subsection{複素多様体とリーマン面}

$\cc^{n}$での座標が$z=(z^{1},\dots,z^{n})$であるとき,
複素数空間$\cc^{n}$を
$\cc^{n}_{z}$とか$\cc^{n}_{(z^{1},\dots,z^{n})}$とかく.
$\UU$を$\cc^n$の空でない開集合とする.このとき,$\UU$で定義された
複素数値関数$f$は標準座標を用いて$f(z)=f(z^{1},\dots,z^{n})$とかける.

$f$が$\UU$で\textbf{正則}であるとは,
$f(z)$が$\UU$で連続であり,
各変数$z^{j}\ (j=1,\dots,n)$について正則であることをいう.

\begin{Definition}[$n$次元複素多様体,リーマン面]\label{def:mnf}
    $X$を位相空間とする.
    $(\pphi_{i}\colon U_{i}\to \UU_{i})_{i\in I}$を
    写像の族とする.このとき,対$(X,(\pphi_{i})_{i})$が
    次の条件(1)--(4)をみたすとき,$X$を台集合とし$(\pphi_{i})_i$を
    座標近傍系とする\textbf{$n$次元複素多様体} ($n$-dimensional complex manifold) という.
    \begin{enumerate}
        \renewcommand{\theenumi}{\arabic{enumi}}
        \renewcommand{\labelenumi}{(\theenumi)}
        \item $X$は空集合でなく,第2可算公理を満たす連結なハウスドルフ空間である.
        \item すべての$i\in I$に対して$U_i$は$X$の空でない開集合であり,
        $(U_i)_i$は$X$の開披覆である.
        \item すべての$i\in I$に対して,
        $\UU_i$は$\cc^{n}_{(z^{1},\dots,z^{n})}$の
        空でない開集合であり$\pphi_{i}\colon U_i\to\UU_i$は同相である.
        \item 任意の$i\neq j \in I$で$U_i\cap U_j \neq \emp$をみたすもの
        に対して$\UU_{ij}\ceq \pphi_j(U_i\cap U_j)\sbs \UU_j$と
        おくとき,$\pphi_{ij}\ceq 
        \mapres{\pphi_{i}\circ\pphi_{j}^{-1}}{\UU_{ij}}
        \colon 
        \UU_{ij}\to\UU_{ji}$は正則である.
    \end{enumerate}
    とくに,1次元複素多様体を\textbf{リーマン面} (Riemann surface) という.
\end{Definition}

\begin{Example}\label{ex:openR}
    1. 
    $\cc^n$の領域$\UU$は$(\id_{\UU}\colon\UU\to\UU)$を
    座標近傍系とする$n$次元複素多様体である.

    2. 
    $X=(X, (\pphi_{i}\colon U_{i}\to \UU_{i})_{i\in I})$を
    $n$次元複素多様体とし,$U$を$X$の領域とする.
    $J\coloneqq\{i\in I; U\cap U_i \neq \emp\}$とおく.
    $U$は$(\mapres{\pphi_{j}}{U\cap U_{j}})_{j\in J}$を
    座標近傍系とする$n$次元複素多様体になる.
    この多様体$U$を開部分(複素)多様体という.
\end{Example}

台空間がコンパクトなリーマン面を
とくにコンパクトリーマン面とか閉リーマン面という.
例\ref{ex:openR}.2 のように,リーマン面$X$の領域$U$は
リーマン面になる.この$U$を$X$の開リーマン面という.

\subsection{複素多様体とリーマン面の射}
\begin{Definition}
    $X$を$n$次元複素多様体, $Y$を$m$次元複素多様体とする.
    $f\colon X\to Y$を$X$から$Y$への連続写像とする.
    
    1. 
    $P$を$X$の点とする.$P$, $f(P)$の近傍での$f$のある座標表示
    $w_j=f_{ij}(z_i)$, あるいは
    $(w_{j}^{1},\dots,w_{j}^{m})
    =\left(f_{ij}^{1}(z_{i}^{1},\dots,z_{i}^{n}),\dots,f_{ij}^{m}(z_{i}^{1},\dots,z_{i}^{n})\right)$
    が$z_i(P)=(z_{i}^{1}(P),\dots,z_{i}^{n}(P))$で
    正則であるとき,$f$は$P$で\textbf{正則}であるという.
    
    2. 
    $f$がすべての点$P\in X$で正則であるとき
    $f$を\textbf{正則写像} (holomorphic mapping) とか
    \textbf{正則射} (holomorphism) という.
    また$\cc$への正則写像を\textbf{正則関数} (holomorphic 
    function) という.

    $X$と$Y$がともにリーマン面であるとき,$f$をリーマン面の\textbf{射} (morphism) ともいう.

    3. 
    $U$を$X$の空でない開集合とする.
    $U$上の関数$f$は$U$の各連結成分上正則であるとき$U$上の
    正則関数という.ここで,複素多様体の領域は例\ref{ex:openR}.2 の方法で
    複素多様体とみなしている.
\end{Definition}
\begin{Definition}
    $X$をリーマン面とし,$U$を$X$の領域,$S$を$X$の離散閉部分集合とする.
    $S$の各点でしか極をもたない$U-S$上の
    正則関数を\textbf{有理型関数} (meromorphic function) という.
\end{Definition}

リーマン面$X$上の有理型関数は$X$から$\pp^1$へのリーマン面の射と見做せる.

\begin{Definition}
    $X$と$Y$を$n$次元複素多様体とする.$f\colon X\to Y$を正則写像とする.
    正則写像$g\colon Y\to X$で$g\circ f=\id_X$
    かつ$f\circ g=\id_Y$をみたすものが存在するとき,
    $f$を\textbf{双正則写像} (biholomorphic mapping) 
    %とか\textbf{双正則射} (biholomorphism) 
    という.
    $X$から$Y$への双正則写像が存在するとき,
    $X$と$Y$は\textbf{同形} (isomorphic) 
%    とか 
%    \textbf{双正則同値} (biholomorphically equivalent), 
%    またはたんに\textbf{双正則} (biholomorhic) 
    であるという.
\end{Definition}

\section{複素射影直線}
\subsection{複素射影直線の定義}
$\cc^{2}$から原点$0=(0,0)$を除いた集合$\cc^{2}-\{0\}$
の点$(a_{0},a_{1}), (b_{0},b_{1})$に対し次の関係を考える.
\begin{align}\label{eq:sim1}
    (a_{0},a_{1})\sim (b_{0},b_{1})
    \Longleftrightarrow
    (a_{0},a_{1})= c\cdot(b_{0},b_{1})
    \text{となる複素数}c\neq0\text{が存在する.}
\end{align}
これは同値関係である.
$(a_0,a_1)$の同値類$\{c\cdot(a_0,a_1); c\in\cc-\{0\}\}$を
$[a_0\colon a_1]$とかく.

同値関係${\sim}$の定める商写像を用いて次の集合を定義する.

\begin{Definition}
    $\pp^{1} \coloneqq \left(\cc^{2}-\{0\}\right)/{\sim}$
    を\textbf{複素射影直線} (complex projective line) という.
\end{Definition}

\begin{Definition}\label{def:coord1}
    次の写像の組を考える.
    $\begin{tikzcd}
      {\cc^{2}-\{0\}}
        \arrow[r, shift left ,"\pr_1=X_0"]
        \arrow[r, shift right,"\pr_2=X_1"']
      & {\cc}
    \end{tikzcd}; (a_0,a_1)\mapsto a_0,a_1.$
    この組を$\cc^{2}-\{0\}$の標準座標,$\pp^1$の同次座標という.
\end{Definition}

$\pp^{1}$は商写像$\pi \colon \cc^{2}-\{0\}
\rightarrow\left(\cc^{2}-\{0\}\right)/{\sim}$
による商位相により位相空間になる.この定義から$\pi$の連続性が従う.

$\pp^1$の位相空間としての性質を調べるために,次の部分集合を定義する.
\begin{align*}
    U_0=\{[a_0\colon a_1]\in\pp^1; a_0\neq0\},\quad
    U_1=\{[a_0\colon a_1]\in\pp^1; a_1\neq0\}.
\end{align*}
このとき次が成り立つ.
\begin{align}
    U_0\cup U_1 &= \pp^1, \label{eq:cov1}\\
    U_0\cap U_1 
    &= \{[a_0\colon a_1]\in\pp^1; a_0, a_1\neq0\} \label{eq:inter-p1}\\
    &= U_0 - \{[1\colon 0]\}  \notag\\
    &= U_1 - \{[0\colon 1]\}. \notag
\end{align}

\begin{Lemma}\label{mnf:p1}
    1. 
    商写像$\pi \colon \cc^{2}-\{0\}
    \rightarrow\left(\cc^{2}-\{0\}\right)/{\sim}$
    は開写像である.

    2. 
    $U_0$と$U_1$は$\pp^1$の開集合であり,
    \begin{align*}
        \pphi_0&\colon U_0\overset{{\sim}}{\longrightarrow}\cc;\ [a_0\colon a_1]\mapsto a_1/a_0,\\
        \pphi_1&\colon U_1\overset{{\sim}}{\longrightarrow}\cc;\ [a_0\colon a_1]\mapsto a_0/a_1
    \end{align*}
    はともに同相写像である.

    3. 
    任意の$\dip A = \begin{bmatrix}
        a&b\\c&d
    \end{bmatrix}\in GL(2,\cc)$は自己同相写像
    \begin{align*}
        p_A\colon \pp^1\overset{{\sim}}{\longrightarrow}\pp^1;
        \begin{bmatrix}
            a_0\\a_1
        \end{bmatrix}
        \mapsto
        \begin{bmatrix}
            a&b\\c&d
        \end{bmatrix}
        \begin{bmatrix}
            a_0\\a_1
        \end{bmatrix}
    \end{align*}
    を引き起こす.

    4. 
    $\pp^1$は第2可算公理をみたす連結なコンパクトハウスドルフ空間である.
\end{Lemma}

\begin{proof}[\textbf{証明}]
    1. 
    $U$を$\cc^2-\{0\}$の開集合とする.$\pi(U)$が$\pp^1$の
    開集合であること,すなわち$\pi^{-1}(\pi(U))$が$\cc^2-\{0\}$の
    開集合であることを示す.
    いま,任意の開集合$U\subset\cc^2-\{0\}$に対し,
    複素数$c\neq0$を用いて
    \begin{align*}
        cU = \left\{(ca_0,ca_1); (a_0,a_1)\in\cc^2-\{0\}\right\}
    \end{align*}
    とおくと,$cU$は$\cc^2-\{0\}$の開集合であり,
    \begin{align*}\label{eq:proj}
        \pi^{-1}(\pi(U)) = \bigcup_{c\in\cc-\{0\}} cU %\tag{$\ast$}
    \end{align*}
    なので,$\pi^{-1}(\pi(U))$は$\cc^2-\{0\}$の
    開集合である.

    2. 
    まず$U_0, U_1$が$\pp^1$の開集合であることを示す.
    $U_0=\{[a_0:a_1]; a_0\neq0\}$は$V_0=\{(a_0,a_1);a_0\neq0\}$
    の$\pi$による像であり,$V_0$は
    $\cc^2-\{0\}$の開集合であるから,
    $U_0$は$\pp^1$の開集合である.同様に$U_1$も$\pp^1$の開集合である.

    $\pphi_0\colon U_0\to\cc$が連続であることを示す.$V$を$\cc$の
    開集合とする.
    $V=\pphi_0\circ\pi(V_0) (= \widetilde{\pphi_0}(V_0)$
    とおく)である.
    $\widetilde{\pphi_0}^{-1}(V) 
    = \pi^{-1}\left(\pphi_0^{-1}(V)\right)$は$V_0$の開集合である.
    したがって,これは$\cc^2-\{0\}$の開集合であり,商位相の定義から$\pphi_0^{-1}(V)\subset U_0$
    は開集合である.
    
    $\pphi_0$が同相であることを示す.$\psi_0\colon \cc \to U_0$を
    $\psi_0(z)=[1\colon z]$で定める.このとき
    $\psi_0\circ\pphi_0\left([a_0\colon a_1]\right)
    =\psi_0\left( a_1/a_0 \right)
    =[1\colon a_1/a_0]=[a_0\colon a_1]$である.
    また$\pphi_0\circ\psi_0(z)=\pphi_0([1\colon z])=z/1=z$.
    したがって,$\psi_0\circ\pphi_0=\id_{U_0}$かつ
    $\pphi_0\circ\psi_0=\id_{\cc}$であり,$\psi_0=\pphi_0^{-1}$である.
    $\psi_0=\pphi_0^{-1}$は自然な単射$\cc 
    \hookrightarrow \cc^2-\{0\}$と$\pi$の合成であり,
    これらは連続なので,その合成である$\psi_0$も連続である.
    以上より$\pphi_0$は同相である.

    3. 
    $\dip A = \begin{bmatrix}
        a&b\\c&d
    \end{bmatrix}$を可逆な行列とする.
    $A$を自己同形$\cc^2\to\cc^2$とみたとき,
    それを$\cc^2-\{0\}$に制限した
    $\mapres{A}{\cc^2-\{0\}}\colon\cc^2-\{0\}\to\cc^2-\{0\}$
    は自己同相であり,
    逆写像は$\mapres{A^{-1}}{\cc^2-\{0\}}$で与えられる.
    一般に$A(cx)=cAx$なので,$A$から可逆な写像$p_A$が不備なく定まり,
    逆写像は$p_{A^{-1}}$で与えられる.

    $p_A$が連続であることを示す.$V$を$\pp^1$の開集合とする.
    次の図式が可換であり,$\pi$と$A$は連続写像であるから,
    $\pi^{-1}\left(p_A^{-1}(V)\right)
    =A^{-1}\left(\pi^{-1}(V)\right)$は$\cc^2-\{0\}$の開集合
    である.
    \begin{equation*}
        \vcenter{\xymatrix@C=36pt@R=36pt{
        \cc^2-\{0\} 
        \ar@{{}->}[d]_{\pi} 
        \ar@{{}->}[r]^-{A} 
        & \cc^2-\{0\} 
        \ar@{{}->}[d]^{\pi} 
        \\
        \pp^1 \ar[r]_{p_A}  
        & \pp^1 \ar@{}[lu]
        }}
    \end{equation*}
    $\pp^{1}$の商位相の定義より$\pi^{-1}(V)$は$\pp^{1}$の開集合である.
    したがって$p_A$で連続である.$p_A^{-1}$が連続であることも同様である.

    4. 
    第2可算公理をみたすこと:
    \begin{align*}
        \qq(\sqrt{-1})=\{a+b\sqrt{-1};a,b\in\qq\}
    \end{align*}
    に属する点$z$と有理数$p$に対し
    $U_{p}(z)$を考えると
    $\left(U_{p}(z)\right)_{p\in\qq,z\in\cc}$
    は$\cc$の位相空間としての基底になる.
    したがって$\cc$は第2可算公理をみたす.
    直積集合$\cc^2$も第2可算であるから,1点を除いた$\cc^2-\{0\}$もそうであり,
    これに全射$\pi$を適用した$\pp^1$も第2可算公理をみたす.

    連結かつコンパクトであること:
    $S^3=\{P=(a_0,a_1)\in\cc^2;|a_0|^2+|a_1|^2=1\}\subset\cc^2-\{0\}$
    であり,$\cc^2-\{0\}$の相対位相により,$S^3$は有界閉集合
    つまりコンパクト集合であり,連結である.
    全射連続写像$\mapres{\pi}{S^3}\colon S^3\to\pp^1$により 
    $\pp^1$は連結かつコンパクトである.
    $\mapres{\pi}{S^3}$が全射であることは
    \begin{align*}
        [a_0\colon a_1]
        =
        \left[
            \frac{a_0}{\sqrt{a_0^2+a_1^2}}\colon \frac{a_1}{\sqrt{a_0^2+a_1^2}}
        \right]
    \end{align*}
    であることからしたがう.

    ハウスドルフであること:
    $P\neq Q$を$\pp^1$の点とする.
    $p\colon GL(2,\cc)\to \Aut_{\tTop}(\pp^1)$は
    全射.したがって,$U_0\subset \pp^1$から,任意の
    $p_{A}\in \Aut_{\tTop}(U_0)$に対し$A\in GL(2,\cc)$が存在する.
    つまり$p_A(P),p_A(Q)\in U_0$となる$A\in GL(2,\cc)$が存在する.
    $U_0\cong \cc$であり$\cc$はハウスドルフなので,$p_A(P)$の
    開近傍$U_P$と$p_A(Q)$の開近傍$U_Q$で$U_P\cap U_Q=\emp$を
    みたすものが存在する.
    $U_P$と$U_Q$は$U_0\subset\pp^1$の開集合であり,$p_{A}$が
    同相なので$p_A^{-1}(U_P)$, $p_A^{-1}(U_Q)$は$\pp^1$に
    おける$P$, $Q$の開近傍で
    $p_A^{-1}(U_P)\cap p_A^{-1}(U_Q)=\emp$
    をみたす.よって$\pp^1$はハウスドルフである.
\end{proof}

\subsection{貼りあわせ関数}\label{ssec:patch}

補題\ref{mnf:p1}.2 から
$\pphi_0 \colon U_0\isom\cc$, $\pphi_1\colon {U_1} \isom\cc$
である.
ここで,$\pphi_0(U_0)$の標準座標を$w$, 
$\pphi_1(U_1)$の標準座標を$z$で表すことにする.
定義\ref{def:coord1}のようにかくと
\begin{align*}
    z&\colon \pphi_1(U_1)= \cc \to \cc; (a)\mapsto a\\
    w&\colon \pphi_0(U_0)= \cc \to \cc; (b)\mapsto b
\end{align*}
のようになる.複素数の一つ組に対し第一成分を対応させるということである.
これによって点$(a)$と座標値$z(a)$を同一視し,点を単に$z$と書いたりする.
ガウス平面$\cc$に,そこでの標準座標をつけて$\cc_z$, $\cc_w$のように表すと,
$\cc_w\subset\pp^1$, $\cc_z\subset\pp^1$とみなせる.
$z$も$w$も0でないとき,$\cc_z$と$\cc_w$の間には,
\begin{align}\label{eq:patch1}
    z=\frac{1}{w}
\end{align}
の関係がある.
$z,w\ne 0$は\eqref{eq:inter-p1}より
$[z\colon w]\in U_0\cap U_1$ということである.
$[z\colon w]\in U_0\cap U_1$のとき$z$は$w$の正則関数になっている.
$\pphi_0 (U_0\cap U_1) = \cc_w-\{0\} 
= \cc_w \cap \cc_z 
= \cc_z-\{0\} 
= \pphi_1 (U_0\cap U_1)$
なので,
この正則関数を
$\pphi_{10}\colon \cc_w-\{0\}=\pphi_0 (U_0\cap U_1) 
\to \cc_z-\{0\} = \pphi_1 (U_0\cap U_1)$
とかくことにすると,次の図式が可換になる.
\begin{equation*}
    \vcenter{\xymatrix@C=35pt@R=35pt{
    {U_0\cap U_1} 
        \ar@{<-{}}[d]^{\pphi_0^{-1}} 
        \ar@{{}=}[r]^-{[1\colon w]=[z\colon 1]}
    & {U_0 \cap U_1} 
        %
        \\
    \cc_w -\{0\}
        \ar[r]_{\pphi_{10}}
    & \cc_z -\{0\}
        \ar@{<-{}}[u]_{\pphi_1}
        %\ar@{}[lu]
    }}
\end{equation*}
つまり,$\pphi_{10} = \pphi_1\circ \pphi_0^{-1}$である.
また,$\pphi_{01} = \pphi_0\circ \pphi_1^{-1}\colon \pphi_1 (U_0\cap U_1) \to \pphi_0 (U_0\cap U_1)$
も$w=1/z$として同様に定まる.これは正則であり$\pphi_{10}$の逆関数でもある.

以上から次が従う.

\begin{Proposition}
    複素射影直線$\pp^1$は,$\pp^1$を台集合とし,
    $(\pphi_0\colon U_0\to\cc_w, \pphi_1\colon U_1\to\cc_z)$を
    座標近傍系とするコンパクトリーマン面である.    
\end{Proposition}

\begin{proof}[\bf{証明}]
    コンパクト性は補題\ref{mnf:p1}.4 で示した.
    定義\ref{def:mnf}の(1)--(4)で$n=1$としたものが成り立つことを示す.

    (1) 
    補題\ref{mnf:p1}.4 からしたがう.

    (2) 
    \eqref{eq:cov1}と\eqref{eq:inter-p1} からしたがう.

    (3) 
    補題\ref{mnf:p1}.2 からしたがう.

    (4) 
    上で説明した.
\end{proof}
ここでは2枚の被覆で座標近傍系を定めたが,
以下断りなく極大座標近傍系を考える.

\section{複素トーラス}
\subsection{複素トーラスの定義}\label{ssec:torus}
$\omega_1$, $\omega_2$を$\rr$上一次独立な複素数とする.
$\omega_1$, $\omega_2$に対し,
ガウス平面$\cc$の加法部分群$\Omega$を
\begin{align*}
    \Omega\coloneqq\{n_1\omega_1+n_2\omega_2; n_1, n_2\in\zz\}
\end{align*}
で定める.
$E\coloneqq \cc/\Omega$とおく.商写像を$p\colon\cc\to E$とかく.
また,$S\coloneqq\{a\omega_1+b\omega_2; 0\leqq a,b<1\}$とおく.
このとき,$p$は$E$と$S$の間の1対1対応を定める.
実際,$x=x_1\omega_1+x_2\omega_2$, 
$y=y_1\omega_1+y_2\omega_2\in S$とし,$p(x)=p(y)$とする.このとき,
$p(x-y)=[0]$, つまり,$x-y=n_1\omega_1+n_2\omega_2$となる
整数$n_1$, $n_2$が存在する.
$0\leqq x_1, x_2, y_1,y_2<1$なので$n_1=n_2=0$となることが
必要である.
したがって,$x=y$となる.つまり$p$は単射である.
$p$が全射であることは,次の補題\ref{lem:elliptic-is-mnf}.1 から従う.

\begin{Lemma}\label{lem:elliptic-is-mnf}
    1. 
    $p$は全射かつ連続な開写像である.

    2. 
    $E$は第2可算公理をみたす連結なコンパクトハウスドルフ空間である.
\end{Lemma}

\begin{proof}[\pfb]
    1. 
    $p$は全射かつ連続であること:
    $\alpha$を$E$の点とする.$\alpha$に対し,$\alpha+0\omega_1+0\omega_2$は
    $\alpha=p(\alpha+0\omega_1+0\omega_2)$をみたす.
    $p$の連続性は商位相の定義より従う.

    $p$が開写像であること:
    $U$を$\cc$の空でない開集合とする.
    \begin{align*}
        p^{-1}(p(U))=\bigcup_{\omega\in\Omega}(U+\omega)
    \end{align*}
    であり,$U+\omega$は$\cc$の開集合なので,
    その合併である$p^{-1}(p(U))$も$\cc$の開集合である.
    したがって,$p(U)$は$E$の開集合である.

    2. 
    第2可算であること:
    $p$が全射かつ連続な開写像なので,$\cc$の位相空間としての
    基底の$p$による像は$E$の基底になる.
    実際,$x\in E$とし,$U$を$E$における$x$の開近傍とする.
    このとき,$p$は全射なので$p^{-1}(x)$は空でなく,
    $p^{-1}(U)\subset \cc$は$p$の連続性から
    $p^{-1}(x)$の元たちの開近傍である.
    $\cc$の基底の元$V$で$p^{-1}(U)$に含まれ,
    $p^{-1}(x)$を含むものが存在する.
    この$V$に対し,$p$が連続な開写像であることから,$p(V)$は$E$の開集合であり,
    $x\in p(V)\subset U$が成り立つ.
    よって,$\cc$が第2可算であることから$E$も第2可算である.
    
    連結かつコンパクトであること:
    $S$の閉包$\overline{S}$は連結かつコンパクトである.
    また,$p(\overline{S})=E$でもある.
    $p$の連続性によって,$E$は連結かつコンパクトである.

    ハウスドルフであること:
    $P\neq Q$を$E$の点とする.$P$, $Q$に対し,
    $S$の点$x$, $y$で
    $p(x)=P$, $p(y)=Q$となるものが存在する.
    $x,y\in \partial S$のとき,複素数$\e$を適当にとって,
    $x,y\in\Int(S+\e)$となるようにできるので,
    $x$, $y$は$S$の内点としてよい.
    このとき,$S\subset\cc$がハウスドルフであることから,
    実数$r>0$で,
    $D(x;r), D(y;r)\subset S$かつ$D(x;r)\cap D(y;r)=\varnothing$を
    みたすものが存在する.この$r$に対し,
    $P\in p(D(x;r))$かつ$Q\in p(D(y;r))$であり,
    $p(D(x;r))\cap p(D(y;r))=\varnothing$が成り立つ.
\end{proof}

$E$の複素構造を定める.$P$を$E$の点とする.
複素数$\e$と$P=p(x)$となる点$x\in \cc$と$x$の開近傍$\UU_x$で
$\UU_x\subset S+\e$となるものが存在する.
$U_P\coloneqq p(\UU_x)$とおくと,
$U_P$は$P$の$E$における開近傍である.
$p^{-1}(U_P)=\bigsqcup_{\omega\in\Omega}\UU_x+\omega$であり,
任意の$\omega\in\Omega$に対し,
$\mapres{p}{\UU_x+\omega}\colon \UU_x+\omega\to U_P$は
同相写像である.
$\mapres{p}{\UU_x}$の逆写像を$\pphi_{P,x}\colon U_P\to\UU_x$とおく.
このとき,$\left(\pphi_{P,x}\right)_{P\in E, x\in p^{-1}(P)}$は
$E$の座標近傍系である.
実際,$P$, $Q$を$E$の点とし,$\cc$の点$x$, $y$を$P=p(x)$, $Q=p(y)$を
みたすものとする.このとき,
$\pphi_{Q,y}\circ \pphi_{P,x}^{-1}\colon \pphi_{P,x}(U_P\cap U_Q)\to\pphi_{Q,y}(U_P\cap U_Q)$
は$x$に何らかの$\omega\in\Omega$を足して$y$に並行移動させる
写像$y=x+\omega$なので正則である.

したがって,次が成り立つ.

\begin{Proposition}
    $\left(E, 
    \left(\pphi_{P,x}\right)_{P\in E, x\in p^{-1}(P)}
    \right)$はコンパクトリーマン面である.
\end{Proposition}
\begin{proof}[\pfb]
    (1) 
    補題\ref{lem:elliptic-is-mnf}.2 で示した.

    (2)
    $E=\bigcup_{P\in E}U_P$と
    補題\ref{lem:elliptic-is-mnf}.1 から従う.

    (3), (4)
    上で示した.
\end{proof}
コンパクトリーマン面$E$を複素トーラスという.

\subsection{楕円関数}
\ref{ssec:torus}節の記号を用いる.
\begin{Definition}
    $f$を$\cc$上定義された有理型関数とする.
    $f$が$\omega_1$と$\omega_2$を周期とするとき,
    $f$は2重周期$\omega_1$と$\omega_2$をもつ楕円関数であるとか,
    $\Omega$を周期とする楕円関数という.
\end{Definition}
\begin{Lemma}\label{lem:corr}
    商写像$p\colon\cc\to E$の
    引き戻し$p^{\ast}\colon f\mapsto f\circ p$は
    $\{E\text{上の有理型関数}\}$から
    $\{\Omega\text{を周期とする}\cc\text{上の楕円関数}\}$への
    1対1対応を定める.
\end{Lemma}

\begin{proof}[\pfb]
    $f$を$E$上の有理型関数とする.
    $p^{\ast}f$は$\cc$上の有理型関数である.
    $p^{\ast}f$の2重周期性を示す.$z\in\cc$, $\omega\in\Omega$とする.
    \begin{align*}
        p^{\ast}f(z+\omega)=f(p(z+\omega))=f(p(z))=p^{\ast}f(z)
    \end{align*}
    である.

    $p^{\ast}$が1対1対応となることを示す.
    $f$,$g$を$E$上の有理型関数で$p^{\ast}f=p^{\ast}g$を
    みたすものとする.
    $p$は全射なので,
    $f(p(z))=g(p(z))$から$f=g$である.よって,$p^{\ast}$は単射である.

    $g$を$\Omega$を周期とする楕円関数とする.$P$を$E$の点とする.
    $\pphi_{P,x}\colon U_P\to\UU_x$に対し,
    $f^P$を$\mapres{g}{\UU_x}$を局所座標表示とする$U_P\subset E$上の
    有理型関数とする.$g$の2重周期性から,$f^P$は$x$の取り方によらない.
    $Q\in E$を$U_P\cap U_Q\neq\varnothing$となる点とすると,
    $\mapres{f^P}{U_P\cap U_Q}=\mapres{f^Q}{U_P\cap U_Q}$が
    成り立つ.$(f^P)_{P\in E}$を貼り合わせることで,
    $E$上の有理型関数$f$が定まる.($\mapres{f}{U_P}\coloneqq f^P$)
    この$f$に対し,$p^{\ast}f=g$が成り立つ.
\end{proof}

\begin{Theorem}
    \begin{align}
        \wp(u)\coloneqq 
        \sum_{\substack{\omega\in\Omega,\\\omega\ne 0}}
        \left(\frac{1}{(u-\omega)^2}-\frac{1}{\omega}\right)
    \end{align}
    は$\Omega$にのみ2位の極を持つ楕円関数である.
\end{Theorem}$\wp(u)$をWeierstrassの$\wp$関数という.
\begin{proof}[\pfb]
    $\wp(u)$が$\cc-\Omega$で正則であり,
    $\Omega$では2位の極をもつこと:
    $M\geqq0$を実数とする.
    $D(0;2M)$は$\cc$のコンパクト集合であり,
    $\Omega$は離散閉集合なので,
    $\Omega\cap D(0;2M)$は有限集合である.
    \begin{align*}
        \wp(z)
        =\left(
            \frac{1}{z^2}
            +\sum_{
                \substack{\omega\in\Omega-\{0\},\\|\omega|\leqq2M}
            }
            \left(
                \frac{1}{(z-\omega)^2}-\frac{1}{\omega^2}
            \right)
        \right)
        +\sum_{
            \substack{\omega\in\Omega-\{0\},\\|\omega|>2M}
        }\left(
            \frac{1}{(z-\omega)^2}-\frac{1}{\omega^2}
        \right)
    \end{align*}
    と2つの和に分解する.
    第1項は有限和なので$\overline{D(0;M)}-\Omega$
    で正則かつ$\overline{D(0;M)}\cap\Omega$で
    2位の極をもつ有理型関数である.

    第2項が$\overline{D(0;M)}$で一様収束することを示す.
    $z$を$|z|\leqq M$をみたす複素数とする.
    $|\omega|>2M\geqq2|z|$である.したがって
    \begin{align*}
        \left|\frac{1}{(z-\omega)^2}-\frac{1}{\omega^2}\right|
        =\frac{|z|}{|\omega|^2}\frac{|z-2\omega|}{|z-\omega|^2}
        =\frac{|z|}{|\omega|^3}\frac{|z/\omega-2|}{|z/\omega-1|^2}
        \leqq\frac{M|-1/2-2|}{|\omega|^3|1/2-1|^2}
        %=\frac{M\cdot 5/2}{|\omega|^3\cdot 1/4}
        =\frac{10M}{|\omega|^3}    
    \end{align*}
    が成り立つ.ここで次の補題\ref{lem:conv}を用いると
    $\dip \sum_{\substack{\omega\in\Omega,\\|\omega|>2M}}
    \frac{10M}{|\omega|^{3}}$が収束することがわかる.
    したがって,第2項も収束する.
    \begin{Lemma}\label{lem:conv}
        実数$s>1$に対し
        $\dip \sum_{\omega\in\Omega-\{0\}}\frac{1}{|\omega|^{2s}}$は
        収束する.
    \end{Lemma}
    \begin{proof}[\textbf{補題の証明}]
        $\pphi(x,y)\coloneqq |x\omega_1+y\omega_2|^2$とおく.
        このとき,
        \begin{align*}
            \pphi(x,y)
            &=(x\omega_1+y\omega_2)\overline{(x\omega_1+y\omega_2)}\\
            &=x^2|\omega_1|^2+xy\left(
                \omega_2\overline{\omega_1}+\omega_1\overline{\omega_2}
                \right)+y^2|\omega_2|^2\\
            &=|\omega_1|^2x^2+\left(
                \overline{\omega_1\overline{\omega_2}}
                \right)\left(\omega_1\overline{\omega_2}
                \right)xy+|\omega_2|^2y^2\\
            &=|\omega_1|^2x^2+|\omega_1\overline{\omega_2}|^2xy+|\omega_2|^2y^2
        \end{align*}
        となり,実数$a$, $b$, $c$を用いて,$ax^2+2bxy+cy^2$とかける.
        すなわち,$\pphi(x,y)$ は実係数2次形式であり,
        $\omega_1$と$\omega_2$が独立であることから正定値である.
        2次形式に対応する実対称行列は,実数の固有値を持つ.
        いま,$\pphi$は正定値なので固有値を$0< m_1\leqq m_2$とおいてよい.
        実対称行列は直交行列を用いて対角化でき,任意の$x,y\in\rr$に対し,
        \begin{align*}
            m_1(x^2+y^2)\leqq\pphi(x,y)\leqq m_2(x^2+y^2)
        \end{align*}
        が成り立つ.よって,2つの級数
        \begin{align*}
            \sum_{\omega\in\Omega-\{0\}}\frac{1}{|\omega|^{2s}}
            =\sum_{(n_1,n_2)\in\zz^2-\{(0,0)\}}\frac{1}{\pphi(n_1,n_2)^s},\quad
            \sum_{(n_1,n_2)\in\zz^2-\{(0,0)\}}\frac{1}{(n_1+n_2)^s}
        \end{align*}
        の一方が収束するとき,他方も収束し,発散も共にする.
        十分小さい実数$\e>0$に対し,2つ目の級数と積分
        \begin{align*}
            \lim_{R\to\infty}\iint_{\e\leqq\sqrt{x^2+y^2}\leqq R}\,\frac{1}{(x^2+y^2)^s}\,dxdy
        \end{align*}
        を比較すればよい.
        $r\cos\theta=x$, $r\sin\theta=y$とすると,
        $x^2+y^2=r^2$, $dxdy=rdrd\theta$から,
        \begin{align*}
            \lim_{R\to\infty}\iint_{\e\leqq\sqrt{x^2+y^2}\leqq R}\,\frac{1}{(x^2+y^2)^s}\,dxdy
            &=
            \lim_{R\to\infty}
            \iint_{\e\leqq r\leqq R}\frac{1}{r^{2s}}rdrd\theta\\
            &=
            \lim_{R\to\infty} 2\pi\int_{\e}^{R}\frac{1}{r^{2s-1}}dr\\
            &=
            \lim_{R\to\infty}\frac{\pi}{1-s}\left(\frac{1}{R^{2(s-1)}}-\frac{1}{\e^{2(s-1)}}\right)
        \end{align*}
        となる.したがって,$s>1$のとき収束し,そうでないときは発散する.
    \end{proof}


    $\wp(u)$が2重周期関数であること:
    $\omega\in\{\omega_1,\omega_2\}$とする.$\wp'(z)=-2\sum_{\omega\in\Omega}\frac{1}{(z-\omega)^3}$は
    楕円関数なので,$\wp'(z+\omega)=\wp'(z)$をみたす.
    よって,$(\wp(z+\omega)-\wp(z))'=0$なので,
    $\wp(z+\omega)-\wp(z)\eqqcolon c_{\omega}$は$\cc-\Omega$上の定数関数である.
    $\omega/2\notin\Omega$であり,$\wp(z)=\wp(-z)$なので,
    $\wp(\omega/2)=\wp(-\omega/2)$である.$\omega/2=-\omega/2+\omega$なので,
    $c_\omega=\wp(-\omega/2+\omega)-\wp(\omega/2)=\wp(\omega/2)-\wp(\omega/2)=0$. 
    したがって,$\cc-\Omega$上$\wp(z+\omega)-\wp(z)=0$が成り立つ.有理型関数の剛性から$\cc$上$\wp(z)=\wp(z+\omega)$が成り立つ.
\end{proof}

\section{2重被覆}
\subsection{分岐と被覆}
証明はしないが次の事実がある.
\begin{Fact}\label{fact:branch}
    $X$と$Y$をリーマン面とする.
    $f\colon X\to Y$を定値でないリーマン面の射とする.
    $P\in X$, $Q=f(P)\in Y$とおく.
    このとき,$P$のまわりの局所座標$t$と$Q$のまわりの局所座標$s$と
    正の整数$n\geqq 1$で,$f$
    局所座標表示が$s=t^n$となるものが存在する.
    また,この$n$は座標の取り方によらない.    
\end{Fact}
この$n$を$P$における$f$の分岐指数といい,$e_P$とかく.
$e_P>1$のとき,$P$を$f$の分岐点という.
\begin{Fact}\label{fact:deg}
    $X$と$Y$をコンパクトリーマン面とする.
    $f\colon X\to Y$を定値でない射とする.
    このとき,次が成り立つ.
    
    1.
    任意の$Q\in Y$に対し$f^{-1}(Q)\neq\varnothing$かつ$\#f^{-1}(Q)<\infty$である.
    
    2. 
    $f$の分岐点は高々有限個である.
    
    3. 
    $Q$を$Y$の点とする.
    このとき,$d(Q)\coloneqq \sum_{P\in f^{-1}(Q)}e_P$は一定である.
    これを$\deg f$とかく.
    
    4. 
    分岐点でない点$Q\in Y$に対し$\# f^{-1}(Q)=\deg f$である.
    分岐点$Q\in Y$に対し,$\# f^{-1}(Q)<\deg f$である.
\end{Fact}

$d=\deg f$を$f$の写像度といい,$f$を$d$重被覆写像という.

\subsection{複素トーラスから複素射影直線への2重被覆}
補題\ref{lem:corr}より,$\wp$は$E$上の有理型関数と見做せる.
さらに,有理型関数は$\pp^1$への射と見做せる.
\begin{Theorem}
    $\wp$の定めるリーマン面の射$\wp\colon E\to\pp^1$; 
    $\wp([z])=[\wp(z):1]$は4点$[0]$, 
    $[\omega_1/2]$, $[\omega_2/2]$, 
    $[(\omega_1+\omega_2)/2]$で分岐する2重被覆である.
    これらの点を$E$の2分点と呼ぶ.
\end{Theorem}
\begin{proof}[\pfb]
    $\wp$は$\Omega$にのみ2位の極をもつ楕円関数であったから,
    $[0]$のみに2位の極をもつ$E$上の有理型関数というのと同じである.
    したがって,$\wp^{-1}(\infty)=\{[0]\}$であり,
    事実\ref{fact:deg}より,$\deg\wp=2$である.
    いま,$\wp$は偶関数なので,$[a]\in E$に対し,
    $\wp([a])=\wp([-a])$が成り立つ.
    $[a]$が$E$の2分点でなければ,$[a]\neq[-a]$である.
    $\deg\wp=2$なので,
    このとき,$\wp^{-1}(\wp([a]))=\{[a],[-a]\}$と確定する.

    $[\omega_1/2]$の近傍で$\wp$を局所座標表示する.
    $\wp$は$\omega_1$を周期にもつ偶関数なので
    $\wp(-z)=\wp(z)=\wp(z+\omega_1)$をみたす.
    両辺を微分して,
    $-\wp'(-z)=\wp'(z+\omega_1)$となるが,$z=-\omega_1/2$のとき,
    $-\wp'(\omega_1/2)=\wp'(\omega_1/2)$となる.
    したがって,$\wp'(\omega_1/2)=0$となる.
    よって,$\wp(z)$の$\omega_1/2$のまわりでの
    展開における1次の項の係数は0である.
    したがって,$e_{[\omega_1/2]}>1$であり,
    $[\omega_1/2]$は$\wp$の分岐点である.
    $[\omega_2/2]$と$[(\omega_1+\omega_2)/2]$についても同様に,
    $e_{[\omega_2/2]}>1$, $e_{[(\omega_1+\omega_2)/2]}>1$となるので,
    $\wp$は$E$の2分点で分岐する2重被覆であることが示せた.
\end{proof}
%\providecommand{\bysame}{\leavevmode\hbox to3em{\hrulefill}\thinspace}
\begin{thebibliography}{15}

\bibitem[Og02]{ogs} 小木曽啓示, 代数曲線論, 朝倉書店, 2002. 

\end{thebibliography}

\end{document}


