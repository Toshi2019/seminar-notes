\section{2重被覆}
\subsection{分岐と被覆}
証明はしないが次の事実がある.
\begin{Fact}\label{fact:branch}
    $X$と$Y$をリーマン面とする.
    $f\colon X\to Y$を定値でないリーマン面の射とする.
    $P\in X$, $Q=f(P)\in Y$とおく.
    このとき,$P$のまわりの局所座標$t$と$Q$のまわりの局所座標$s$と
    正の整数$n\geqq 1$で,$f$
    局所座標表示が$s=t^n$となるものが存在する.
    また,この$n$は座標の取り方によらない.    
\end{Fact}
この$n$を$P$における$f$の分岐指数といい,$e_P$とかく.
$e_P>1$のとき,$P$を$f$の分岐点という.
\begin{Fact}\label{fact:deg}
    $X$と$Y$をコンパクトリーマン面とする.
    $f\colon X\to Y$を定値でない射とする.
    このとき,次が成り立つ.
    
    1.
    任意の$Q\in Y$に対し$f^{-1}(Q)\neq\varnothing$かつ$\#f^{-1}(Q)<\infty$である.
    
    2. 
    $f$の分岐点は高々有限個である.
    
    3. 
    $Q$を$Y$の点とする.
    このとき,$d(Q)\coloneqq \sum_{P\in f^{-1}(Q)}e_P$は一定である.
    これを$\deg f$とかく.
    
    4. 
    分岐点でない点$Q\in Y$に対し$\# f^{-1}(Q)=\deg f$である.
    分岐点$Q\in Y$に対し,$\# f^{-1}(Q)<\deg f$である.
\end{Fact}

$d=\deg f$を$f$の写像度といい,$f$を$d$重被覆写像という.

\subsection{複素トーラスからリーマン球面への2重被覆}
補題\ref{lem:corr}より,$\wp$は$E$上の有理形関数と見做せる.
さらに,有理形関数は$\pp^1$への射と見做せる.
\begin{Theorem}
    $\wp$の定めるリーマン面の射$\wp\colon E\to\pp^1$; 
    $\wp([z])=[\wp(z):1]$は4点$[0]$, 
    $[\omega_1/2]$, $[\omega_2/2]$, 
    $[(\omega_1+\omega_2)/2]$で分岐する2重被覆である.
    これらの点を$E$の2分点と呼ぶ.
\end{Theorem}
\begin{proof}[\pfb]
    $\wp$は$\Omega$にのみ2位の極をもつ楕円関数であったから,
    $[0]$のみに2位の極をもつ$E$上の有理形関数というのと同じである.
    したがって,$\wp^{-1}(\infty)=\{[0]\}$であり,
    事実\ref{fact:deg}より,$\deg\wp=2$である.
    いま,$\wp$は偶関数なので,$[a]\in E$に対し,
    $\wp([a])=\wp([-a])$が成り立つ.
    $[a]$が$E$の2分点でなければ,$[a]\neq[-a]$である.
    $\deg\wp=2$なので,
    このとき,$\wp^{-1}(\wp([a]))=\{[a],[-a]\}$と確定する.

    $[\omega_1/2]$の近傍で$\wp$を局所座標表示する.
    $\wp$は$\omega_1$を周期にもつ偶関数なので
    $\wp(-z)=\wp(z)=\wp(z+\omega_1)$をみたす.
    両辺を微分して,
    $-\wp'(-z)=\wp'(z+\omega_1)$となるが,$z=-\omega_1/2$のとき,
    $-\wp'(\omega_1/2)=\wp'(\omega_1/2)$となる.
    したがって,$\wp'(\omega_1/2)=0$となる.
    よって,$\wp(z)$の$\omega_1/2$のまわりでの
    展開における1次の項の係数は0である.
    したがって,$e_{[\omega_1/2]}>1$であり,
    $[\omega_1/2]$は$\wp$の分岐点である.
    $[\omega_2/2]$と$[(\omega_1+\omega_2)/2]$についても同様に,
    $e_{[\omega_2/2]}>1$, $e_{[(\omega_1+\omega_2)/2]}>1$となるので,
    $\wp$は$E$の2分点で分岐する2重被覆であることが示せた.
\end{proof}