
% enumerate の番号を(1)のようにする
%\renewcommand{\theenumii}{(\arabic{enumii})}
\section{リーマン面}

\subsection{複素多様体とリーマン面}

$\cc^{n}$での座標が$z=(z^{1},\dots,z^{n})$であるとき,
複素数空間$\cc^{n}$を
$\cc^{n}_{z}$とか$\cc^{n}_{(z^{1},\dots,z^{n})}$とかく.
$\UU$を$\cc^n$の空でない開集合とする.このとき,$\UU$で定義された
複素数値関数$f$は標準座標を用いて$f(z)=f(z^{1},\dots,z^{n})$とかける.

$f$が$\UU$で\textbf{正則}であるとは,
$f(z)$が$\UU$で連続であり,
各変数$z^{j}\ (j=1,\dots,n)$について正則であることをいう.

\begin{Definition}[$n$次元複素多様体,リーマン面]\label{def:mnf}
    $X$を位相空間とする.
    $(\pphi_{i}\colon U_{i}\to \UU_{i})_{i\in I}$を
    写像の族とする.このとき,対$(X,(\pphi_{i})_{i})$が
    次の条件(1)--(4)をみたすとき,$X$を台集合とし$(\pphi_{i})_i$を
    座標近傍系とする\textbf{$n$次元複素多様体} ($n$-dimensional complex manifold) という.
    \begin{enumerate}
        \renewcommand{\theenumi}{\arabic{enumi}}
        \renewcommand{\labelenumi}{(\theenumi)}
        \item $X$は空集合でなく,第2可算公理を満たす連結なハウスドルフ空間である.
        \item すべての$i\in I$に対して$U_i$は$X$の空でない開集合であり,
        $(U_i)_i$は$X$の開披覆である.
        \item すべての$i\in I$に対して,
        $\UU_i$は$\cc^{n}_{(z^{1},\dots,z^{n})}$の
        空でない開集合であり$\pphi_{i}\colon U_i\to\UU_i$は同相である.
        \item 任意の$i\neq j \in I$で$U_i\cap U_j \neq \emp$をみたすもの
        に対して$\UU_{ij}\ceq \pphi_j(U_i\cap U_j)\sbs \UU_j$と
        おくとき,$\pphi_{ij}\ceq 
        \mapres{\pphi_{i}\circ\pphi_{j}^{-1}}{\UU_{ij}}
        \colon 
        \UU_{ij}\to\UU_{ji}$は正則である.
    \end{enumerate}
    とくに,1次元複素多様体を\textbf{リーマン面} (Riemann surface) という.
\end{Definition}

\begin{Example}\label{ex:openR}
    1. 
    $\cc^n$の領域$\UU$は$(\id_{\UU}\colon\UU\to\UU)$を
    座標近傍系とする$n$次元複素多様体である.

    2. 
    $X=(X, (\pphi_{i}\colon U_{i}\to \UU_{i})_{i\in I})$を
    $n$次元複素多様体とし,$U$を$X$の領域とする.
    $J\coloneqq\{i\in I; U\cap U_i \neq \emp\}$とおく.
    $U$は$(\mapres{\pphi_{j}}{U\cap U_{j}})_{j\in J}$を
    座標近傍系とする$n$次元複素多様体になる.
    この多様体$U$を開部分(複素)多様体という.
\end{Example}

台空間がコンパクトなリーマン面を
とくにコンパクトリーマン面とか閉リーマン面という.
例\ref{ex:openR}.2 のように,リーマン面$X$の領域$U$は
リーマン面になる.この$U$を$X$の開リーマン面という.

\subsection{複素多様体とリーマン面の射}
\begin{Definition}
    $X$を$n$次元複素多様体, $Y$を$m$次元複素多様体とする.
    $f\colon X\to Y$を$X$から$Y$への連続写像とする.
    
    1. 
    $P$を$X$の点とする.$P$, $f(P)$の近傍での$f$のある座標表示
    $w_j=f_{ij}(z_i)$, あるいは
    $(w_{j}^{1},\dots,w_{j}^{m})
    =\left(f_{ij}^{1}(z_{i}^{1},\dots,z_{i}^{n}),\dots,f_{ij}^{m}(z_{i}^{1},\dots,z_{i}^{n})\right)$
    が$z_i(P)=(z_{i}^{1}(P),\dots,z_{i}^{n}(P))$で
    正則であるとき,$f$は$P$で\textbf{正則}であるという.
    
    2. 
    $f$がすべての点$P\in X$で正則であるとき
    $f$を\textbf{正則写像} (holomorphic mapping) とか
    \textbf{正則射} (holomorphism) という.
    また$\cc$への正則写像を\textbf{正則関数} (holomorphic 
    function) という.

    $X$と$Y$がともにリーマン面であるとき,$f$をリーマン面の\textbf{射} (morphism) ともいう.

    3. 
    $U$を$X$の空でない開集合とする.
    $U$上の関数$f$は$U$の各連結成分上正則であるとき$U$上の
    正則関数という.ここで,複素多様体の領域は例\ref{ex:openR}.2 の方法で
    複素多様体とみなしている.
\end{Definition}

\begin{Definition}
    $X$と$Y$を$n$次元複素多様体とする.$f\colon X\to Y$を正則写像とする.
    正則写像$g\colon Y\to X$で$g\circ f=\id_X$
    かつ$f\circ g=\id_Y$をみたすものが存在するとき,
    $f$を\textbf{双正則写像} (biholomorphic mapping) とか\textbf{双正則射} (biholomorphism) という.
    $X$から$Y$への双正則写像が存在するとき,
    $X$と$Y$は\textbf{同形} (isomorphic) とか 
    \textbf{双正則同値} (biholomorphically equivalent), 
    またはたんに\textbf{双正則} (biholomorhic) であるという.
\end{Definition}
