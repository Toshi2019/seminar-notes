
\section{楕円曲線}
\subsection{楕円関数}
$\omega_1$, $\omega_2$を$\rr$上一次独立な複素数とする.
$\omega_1$, $\omega_2$に対し,
ガウス平面$\cc$の加法部分群$\Omega$を
\begin{align*}
    \Omega\coloneqq\{n_1\omega_1+n_2\omega_2; n_1, n_2\in\zz\}
\end{align*}
で定める.
$E\coloneqq \cc/\Omega$とおく.商写像を$p\colon\cc\to E$とかく.
また,$S\coloneqq\{a\omega_1+b\omega_2; 0\leqq a,b<1\}$とおく.
このとき,$E$と$S$の間には1対1対応が存在する.
$E$の点$\alpha$に対し,$\Omega$の点$n_1\omega_1+n_2\omega_2$で
$\alpha+n_1\omega_1+n_2\omega_2\in S$となるものが存在する.

\begin{Lemma}
    1. 
    $p$は全射かつ連続な開写像である.

    2. 
    $E$は第2可算公理をみたす連結なコンパクトハウスドルフ空間である.
\end{Lemma}

\begin{proof}[\pfb]
    1. 
    $p$は全射かつ連続であること:
    $\alpha$を$E$の点とする.$\alpha$に対し,$\alpha+0\omega_1+0\omega_2$は
    $\alpha=p(\alpha+0\omega_1+0\omega_2)$をみたす.
    $p$の連続性は商位相の定義より従う.

    $p$が開写像であること:
    $U$を$\cc$の空でない開集合とする.
    \begin{align*}
        p^{-1}(p(U))=\bigcup_{\omega\in\Omega}(U+\omega)
    \end{align*}
    であり,$U+\omega$は$\cc$の開集合なので,
    その合併である$p^{-1}(p(U))$も$\cc$の開集合である.
    したがって,$p(U)$は$E$の開集合である.

    2. 
    $P\neq Q$を$E$の点とする.$P$, $Q$に対し,
    $S$の点$x$, $y$で
    $p(x)=P$, $p(y)=Q$となるものが存在する.
\end{proof}

\subsection{Weierstrassの$\wp$関数}
\begin{Theorem}
    \begin{align}
        \wp(u)\coloneqq 
        \sum_{\substack{\omega\in\Omega,\\\omega\ne 0}}
        \left(\frac{1}{(u-\omega)^2}-\frac{1}{\omega}\right)
    \end{align}
    は$\Omega$にのみ2位の極を持つ楕円関数である.
\end{Theorem}$\wp(u)$をWeierstrassの$\wp$関数という.
\begin{proof}[\pfb]
    $\wp(u)$が$\cc-\Omega$で正則であり,
    $\Omega$では2位の極をもつこと:
    $M\geqq0$を実数とする.
    $D(0;2M)$は$\cc$のコンパクト集合であり,
    $\Omega$は離散閉集合なので,
    $\Omega\cap D(0;2M)$は有限集合である.
    \begin{align*}
        \wp(z)
        =\left(
            \frac{1}{z^2}
            +\sum_{
                \substack{\omega\in\Omega-\{0\},\\|\omega|\leqq2M}
            }
            \left(
                \frac{1}{(z-\omega)^2}-\frac{1}{\omega^2}
            \right)
        \right)
        +\sum_{
            \substack{\omega\in\Omega-\{0\},\\|\omega|>2M}
        }\left(
            \frac{1}{(z-\omega)^2}-\frac{1}{\omega^2}
        \right)
    \end{align*}
    と2つの和に分解する.
    第1項は有限和なので$\overline{D(0;M)}-\Omega$
    で正則かつ$\overline{D(0;M)}\cap\Omega$で
    2位の極をもつ有理形関数である.

    第2項が$\overline{D(0;M)}$で一様収束することを示す.
    $z$を$|z|\leqq M$をみたす複素数とする.
    $|\omega|>2M\geqq2|z|$である.したがって
    \begin{align*}
        \left|\frac{1}{(z-\omega)^2}-\frac{1}{\omega^2}\right|
        =\frac{|z|}{|\omega|^2}\frac{|z-2\omega|}{|z-\omega|^2}
        =\frac{|z|}{|\omega|^3}\frac{|z/\omega-2|}{|z/\omega-1|^2}
        \leqq\frac{M|-1/2-2|}{|\omega|^3|1/2-1|^2}
        %=\frac{M\cdot 5/2}{|\omega|^3\cdot 1/4}
        =\frac{10M}{|\omega|^3}    
    \end{align*}
    が成り立つ.ここで次の補題\ref{lem:conv}を用いると
    $\dip \sum_{\substack{\omega\in\Omega,\\|\omega|>2M}}
    \frac{10M}{|\omega|^{3}}$が収束することがわかる.
    したがって,第2項も収束する.
    \begin{Lemma}\label{lem:conv}
        実数$s>1$に対し
        $\dip \sum_{\omega\in\Omega-\{0\}}\frac{1}{|\omega|^{2s}}$は
        収束する.
    \end{Lemma}
    %\begin{proof}[\pfb]
        
    %\end{proof}


    $\wp(u)$が2重周期関数であること:
\end{proof}
