
\section{複素トーラス}
\subsection{複素トーラスの定義}\label{ssec:torus}
$\omega_1$, $\omega_2$を$\rr$上一次独立な複素数とする.
$\omega_1$, $\omega_2$に対し,
ガウス平面$\cc$の加法部分群$\Omega$を
\begin{align*}
    \Omega\coloneqq\{n_1\omega_1+n_2\omega_2; n_1, n_2\in\zz\}
\end{align*}
で定める.
$E\coloneqq \cc/\Omega$とおく.商写像を$p\colon\cc\to E$とかく.
また,$S\coloneqq\{a\omega_1+b\omega_2; 0\leqq a,b<1\}$とおく.
このとき,$p$は$E$と$S$の間の1対1対応を定める.
実際,$x=x_1\omega_1+x_2\omega_2$, 
$y=y_1\omega_1+y_2\omega_2\in S$とし,$p(x)=p(y)$とする.このとき,
$p(x-y)=[0]$, つまり,$x-y=n_1\omega_1+n_2\omega_2$となる
整数$n_1$, $n_2$が存在する.
$0\leqq x_1, x_2, y_1,y_2<1$なので$n_1=n_2=0$となることが
必要である.
したがって,$x=y$となる.つまり$p$は単射である.
$p$が全射であることは,次の補題\ref{lem:elliptic-is-mnf}.1 から従う.

\begin{Lemma}\label{lem:elliptic-is-mnf}
    1. 
    $p$は全射かつ連続な開写像である.

    2. 
    $E$は第2可算公理をみたす連結なコンパクトハウスドルフ空間である.
\end{Lemma}

\begin{proof}[\pfb]
    1. 
    $p$は全射かつ連続であること:
    $\alpha$を$E$の点とする.$\alpha$に対し,$\alpha+0\omega_1+0\omega_2$は
    $\alpha=p(\alpha+0\omega_1+0\omega_2)$をみたす.
    $p$の連続性は商位相の定義より従う.

    $p$が開写像であること:
    $U$を$\cc$の空でない開集合とする.
    \begin{align*}
        p^{-1}(p(U))=\bigcup_{\omega\in\Omega}(U+\omega)
    \end{align*}
    であり,$U+\omega$は$\cc$の開集合なので,
    その合併である$p^{-1}(p(U))$も$\cc$の開集合である.
    したがって,$p(U)$は$E$の開集合である.

    2. 
    第2可算であること:
    $p$が全射かつ連続な開写像なので,$\cc$の位相空間としての
    基底の$p$による像は$E$の基底になる.
    実際,$x\in E$とし,$U$を$E$における$x$の開近傍とする.
    このとき,$p$は全射なので$p^{-1}(x)$は空でなく,
    $p^{-1}(U)\subset \cc$は$p$の連続性から
    $p^{-1}(x)$の元たちの開近傍である.
    $\cc$の基底の元$V$で$p^{-1}(U)$に含まれ,
    $p^{-1}(x)$を含むものが存在する.
    この$V$に対し,$p$が連続な開写像であることから,$p(V)$は$E$の開集合であり,
    $x\in p(V)\subset U$が成り立つ.
    よって,$\cc$が第2可算であることから$E$も第2可算である.
    
    連結かつコンパクトであること:
    $S$の閉包$\overline{S}$は連結かつコンパクトである.
    また,$p(\overline{S})=E$でもある.
    $p$の連続性によって,$E$は連結かつコンパクトである.

    ハウスドルフであること:
    $P\neq Q$を$E$の点とする.$P$, $Q$に対し,
    $S$の点$x$, $y$で
    $p(x)=P$, $p(y)=Q$となるものが存在する.
    $x,y\in \partial S$のとき,複素数$\e$を適当にとって,
    $x,y\in\Int(S+\e)$となるようにできるので,
    $x$, $y$は$S$の内点としてよい.
    このとき,$S\subset\cc$がハウスドルフであることから,
    実数$r>0$で,
    $D(x;r), D(y;r)\subset S$かつ$D(x;r)\cap D(y;r)=\varnothing$を
    みたすものが存在する.この$r$に対し,
    $P\in p(D(x;r))$かつ$Q\in p(D(y;r))$であり,
    $p(D(x;r))\cap p(D(y;r))=\varnothing$が成り立つ.
\end{proof}

$E$の複素構造を定める.$P$を$E$の点とする.
複素数$\e$と$P=p(x)$となる点$x\in \cc$と$x$の開近傍$\UU_x$で
$\UU_x\subset S+\e$となるものが存在する.
$U_P\coloneqq p(\UU_x)$とおくと,
$U_P$は$P$の$E$における開近傍である.
$p^{-1}(U_P)=\bigsqcup_{\omega\in\Omega}\UU_x+\omega$であり,
任意の$\omega\in\Omega$に対し,
$\mapres{p}{\UU_x+\omega}\colon \UU_x+\omega\to U_P$は
同相写像である.
$\mapres{p}{\UU_x}$の逆写像を$\pphi_{P,x}\colon U_P\to\UU_x$とおく.
このとき,$\left(\pphi_{P,x}\right)_{P\in E, x\in p^{-1}(P)}$は
$E$の座標近傍系である.
実際,$P$, $Q$を$E$の点とし,$\cc$の点$x$, $y$を$P=p(x)$, $Q=p(y)$を
みたすものとする.このとき,
$\pphi_{Q,y}\circ \pphi_{P,x}^{-1}\colon \pphi_{P,x}(U_P\cap U_Q)\to\pphi_{Q,y}(U_P\cap U_Q)$
は$x$に何らかの$\omega\in\Omega$を足して$y$に並行移動させる
写像$y=x+\omega$なので正則である.

したがって,次が成り立つ.

\begin{Proposition}
    $\left(E, 
    \left(\pphi_{P,x}\right)_{P\in E, x\in p^{-1}(P)}
    \right)$はコンパクトリーマン面である.
\end{Proposition}
\begin{proof}[\pfb]
    (1) 
    補題\ref{lem:elliptic-is-mnf}.2 で示した.

    (2)
    $E=\bigcup_{P\in E}U_P$と
    補題\ref{lem:elliptic-is-mnf}.1 から従う.

    (3), (4)
    上で示した.
\end{proof}
コンパクトリーマン面$E$を複素トーラスという.

\subsection{楕円関数}
\ref{ssec:torus}節の記号を用いる.
\begin{Definition}
    $f$を$\cc$上定義された有理型関数とする.
    $f$が$\omega_1$と$\omega_2$を周期とするとき,
    $f$は2重周期$\omega_1$と$\omega_2$をもつ楕円関数であるとか,
    $\Omega$を周期とする楕円関数という.
\end{Definition}
\begin{Lemma}\label{lem:corr}
    商写像$p\colon\cc\to E$の
    引き戻し$p^{\ast}\colon f\mapsto f\circ p$は
    $\{E\text{上の有理型関数}\}$から
    $\{\Omega\text{を周期とする}\cc\text{上の楕円関数}\}$への
    1対1対応を定める.
\end{Lemma}

\begin{proof}[\pfb]
    $f$を$E$上の有理型関数とする.
    $p^{\ast}f$は$\cc$上の有理型関数である.
    $p^{\ast}f$の2重周期性を示す.$z\in\cc$, $\omega\in\Omega$とする.
    \begin{align*}
        p^{\ast}f(z+\omega)=f(p(z+\omega))=f(p(z))=p^{\ast}f(z)
    \end{align*}
    である.

    $p^{\ast}$が1対1対応となることを示す.
    $f$,$g$を$E$上の有理型関数で$p^{\ast}f=p^{\ast}g$を
    みたすものとする.
    $p$は全射なので,
    $f(p(z))=g(p(z))$から$f=g$である.よって,$p^{\ast}$は単射である.

    $g$を$\Omega$を周期とする楕円関数とする.$P$を$E$の点とする.
    $\pphi_{P,x}\colon U_P\to\UU_x$に対し,
    $f^P$を$\mapres{g}{\UU_x}$を局所座標表示とする$U_P\subset E$上の
    有理型関数とする.$g$の2重周期性から,$f^P$は$x$の取り方によらない.
    $Q\in E$を$U_P\cap U_Q\neq\varnothing$となる点とすると,
    $\mapres{f^P}{U_P\cap U_Q}=\mapres{f^Q}{U_P\cap U_Q}$が
    成り立つ.$(f^P)_{P\in E}$を貼り合わせることで,
    $E$上の有理型関数$f$が定まる.($\mapres{f}{U_P}\coloneqq f^P$)
    この$f$に対し,$p^{\ast}f=g$が成り立つ.
\end{proof}

\begin{Theorem}
    \begin{align}
        \wp(u)\coloneqq 
        \sum_{\substack{\omega\in\Omega,\\\omega\ne 0}}
        \left(\frac{1}{(u-\omega)^2}-\frac{1}{\omega}\right)
    \end{align}
    は$\Omega$にのみ2位の極を持つ楕円関数である.
\end{Theorem}$\wp(u)$をWeierstrassの$\wp$関数という.
\begin{proof}[\pfb]
    $\wp(u)$が$\cc-\Omega$で正則であり,
    $\Omega$では2位の極をもつこと:
    $M\geqq0$を実数とする.
    $D(0;2M)$は$\cc$のコンパクト集合であり,
    $\Omega$は離散閉集合なので,
    $\Omega\cap D(0;2M)$は有限集合である.
    \begin{align*}
        \wp(z)
        =\left(
            \frac{1}{z^2}
            +\sum_{
                \substack{\omega\in\Omega-\{0\},\\|\omega|\leqq2M}
            }
            \left(
                \frac{1}{(z-\omega)^2}-\frac{1}{\omega^2}
            \right)
        \right)
        +\sum_{
            \substack{\omega\in\Omega-\{0\},\\|\omega|>2M}
        }\left(
            \frac{1}{(z-\omega)^2}-\frac{1}{\omega^2}
        \right)
    \end{align*}
    と2つの和に分解する.
    第1項は有限和なので$\overline{D(0;M)}-\Omega$
    で正則かつ$\overline{D(0;M)}\cap\Omega$で
    2位の極をもつ有理型関数である.

    第2項が$\overline{D(0;M)}$で一様収束することを示す.
    $z$を$|z|\leqq M$をみたす複素数とする.
    $|\omega|>2M\geqq2|z|$である.したがって
    \begin{align*}
        \left|\frac{1}{(z-\omega)^2}-\frac{1}{\omega^2}\right|
        =\frac{|z|}{|\omega|^2}\frac{|z-2\omega|}{|z-\omega|^2}
        =\frac{|z|}{|\omega|^3}\frac{|z/\omega-2|}{|z/\omega-1|^2}
        \leqq\frac{M|-1/2-2|}{|\omega|^3|1/2-1|^2}
        %=\frac{M\cdot 5/2}{|\omega|^3\cdot 1/4}
        =\frac{10M}{|\omega|^3}    
    \end{align*}
    が成り立つ.ここで次の補題\ref{lem:conv}を用いると
    $\dip \sum_{\substack{\omega\in\Omega,\\|\omega|>2M}}
    \frac{10M}{|\omega|^{3}}$が収束することがわかる.
    したがって,第2項も収束する.
    \begin{Lemma}\label{lem:conv}
        実数$s>1$に対し
        $\dip \sum_{\omega\in\Omega-\{0\}}\frac{1}{|\omega|^{2s}}$は
        収束する.
    \end{Lemma}
    \begin{proof}[\textbf{補題の証明}]
        $\pphi(x,y)\coloneqq |x\omega_1+y\omega_2|^2$とおく.
        このとき,
        \begin{align*}
            \pphi(x,y)
            &=(x\omega_1+y\omega_2)\overline{(x\omega_1+y\omega_2)}\\
            &=x^2|\omega_1|^2+xy\left(
                \omega_2\overline{\omega_1}+\omega_1\overline{\omega_2}
                \right)+y^2|\omega_2|^2\\
            &=|\omega_1|^2x^2+\left(
                \overline{\omega_1\overline{\omega_2}}
                \right)\left(\omega_1\overline{\omega_2}
                \right)xy+|\omega_2|^2y^2\\
            &=|\omega_1|^2x^2+|\omega_1\overline{\omega_2}|^2xy+|\omega_2|^2y^2
        \end{align*}
        となり,実数$a$, $b$, $c$を用いて,$ax^2+2bxy+cy^2$とかける.
        すなわち,$\pphi(x,y)$ は実係数2次形式であり,
        $\omega_1$と$\omega_2$が独立であることから正定値である.
        2次形式に対応する実対称行列は,実数の固有値を持つ.
        いま,$\pphi$は正定値なので固有値を$0< m_1\leqq m_2$とおいてよい.
        実対称行列は直交行列を用いて対角化でき,任意の$x,y\in\rr$に対し,
        \begin{align*}
            m_1(x^2+y^2)\leqq\pphi(x,y)\leqq m_2(x^2+y^2)
        \end{align*}
        が成り立つ.よって,2つの級数
        \begin{align*}
            \sum_{\omega\in\Omega-\{0\}}\frac{1}{|\omega|^{2s}}
            =\sum_{(n_1,n_2)\in\zz^2-\{(0,0)\}}\frac{1}{\pphi(n_1,n_2)^s},\quad
            \sum_{(n_1,n_2)\in\zz^2-\{(0,0)\}}\frac{1}{(n_1+n_2)^s}
        \end{align*}
        の一方が収束するとき,他方も収束し,発散も共にする.
        十分小さい実数$\e>0$に対し,2つ目の級数と積分
        \begin{align*}
            \lim_{R\to\infty}\iint_{\e\leqq\sqrt{x^2+y^2}\leqq R}\,\frac{1}{(x^2+y^2)^s}\,dxdy
        \end{align*}
        を比較すればよい.
        $r\cos\theta=x$, $r\sin\theta=y$とすると,
        $x^2+y^2=r^2$, $dxdy=rdrd\theta$から,
        \begin{align*}
            \lim_{R\to\infty}\iint_{\e\leqq\sqrt{x^2+y^2}\leqq R}\,\frac{1}{(x^2+y^2)^s}\,dxdy
            &=
            \lim_{R\to\infty}
            \iint_{\e\leqq r\leqq R}\frac{1}{r^{2s}}rdrd\theta\\
            &=
            \lim_{R\to\infty} 2\pi\int_{\e}^{R}\frac{1}{r^{2s-1}}dr\\
            &=
            \lim_{R\to\infty}\frac{\pi}{1-s}\left(\frac{1}{R^{2(s-1)}}-\frac{1}{\e^{2(s-1)}}\right)
        \end{align*}
        となる.したがって,$s>1$のとき収束し,そうでないときは発散する.
    \end{proof}


    $\wp(u)$が2重周期関数であること:
    $\omega\in\{\omega_1,\omega_2\}$とする.$\wp'(z)=-2\sum_{\omega\in\Omega}\frac{1}{(z-\omega)^3}$は
    楕円関数なので,$\wp'(z+\omega)=\wp'(z)$をみたす.
    よって,$(\wp(z+\omega)-\wp(z))'=0$なので,
    $\wp(z+\omega)-\wp(z)\eqqcolon c_{\omega}$は$\cc-\Omega$上の定数関数である.
    $\omega/2\notin\Omega$であり,$\wp(z)=\wp(-z)$なので,
    $\wp(\omega/2)=\wp(-\omega/2)$である.$\omega/2=-\omega/2+\omega$なので,
    $c_\omega=\wp(-\omega/2+\omega)-\wp(\omega/2)=\wp(\omega/2)-\wp(\omega/2)=0$. 
    したがって,$\cc-\Omega$上$\wp(z+\omega)-\wp(z)=0$が成り立つ.有理型関数の剛性から$\cc$上$\wp(z)=\wp(z+\omega)$が成り立つ.
\end{proof}
