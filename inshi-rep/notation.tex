\section*{凡例}
次の記号・用語について断りなく用いることがある.
\begin{itemize}
    \item 添字:$I$を添字集合とする
    何らかの族$(x_i)_{i\in I}$を$(x_i)_i$や$(x_i)$のように
    略記することがある.
    \item 位相空間$X$に対し
    $\Aut_{\tTop}(X)\coloneqq\{X\text{上の自己同相写像}\}$とかく.
    \item $\cc$内の,点$\alpha$を中心とする半径$r$の開円板を
    $D(\alpha;r)\coloneqq\{z\in\cc; |z-\alpha|<r\}$で表す.
    \item 領域:位相空間$X$の部分集合$U$が
    空でない連結開集合であるとき,$U$を領域という.
\end{itemize}