%================================================
%    sheaves on manifolds ノート
%================================================

% -----------------------
% preamble
% -----------------------
% ここから本文 (\begin{document}) までの
% ソースコードに変更を加えた場合は
% 編集者まで連絡してください. 
% Don't change preamble code yourself. 
% If you add something
% (usepackage, newtheorem, newcommand, renewcommand),
% please tell it 
% to the editor of institutional paper of RUMS.

% ------------------------
% documentclass
% ------------------------
\documentclass[11pt, a4paper, dvipdfmx, leqno]{jsbook}

% ------------------------
% usepackage
% ------------------------
\usepackage{algorithm}
\usepackage{algorithmic}
\usepackage{amscd}
\usepackage{amsfonts}
\usepackage{amsmath}
\usepackage[psamsfonts]{amssymb}
\usepackage{amsthm}
\usepackage{ascmac}
\usepackage{color}
\usepackage{enumerate}
\usepackage{fancybox}
\usepackage[stable]{footmisc}
\usepackage{graphicx}
\usepackage{listings}
\usepackage{mathrsfs}
\usepackage{mathtools}
\usepackage{otf}
\usepackage{pifont}
\usepackage{proof}
\usepackage{subfigure}
\usepackage{tikz}
\usepackage{verbatim}
\usepackage[all]{xy}
\usepackage{url}
\usetikzlibrary{cd}



% ================================
% パッケージを追加する場合のスペース 
%\usepackage{calligra}
\usepackage[dvipdfmx]{hyperref}
\usepackage{xcolor}
\definecolor{darkgreen}{rgb}{0,0.45,0} 
\definecolor{darkred}{rgb}{0.75,0,0}
\definecolor{darkblue}{rgb}{0,0,0.6} 
\hypersetup{
    colorlinks=true,
    citecolor=darkgreen,
    linkcolor=darkred,
    urlcolor=darkblue,
}
\usepackage{pxjahyper}
\usepackage[mathscr]{euscript}
\usepackage{layout}
\usepackage{framed}
\definecolor{lightgray}{rgb}{0.75,0.75,0.75}
\renewenvironment{leftbar}{%
  \def\FrameCommand{\textcolor{lightgray}{\vrule width 0.7zw} \hspace{10pt}}% 
  \MakeFramed {\advance\hsize-\width \FrameRestore}}%
{\endMakeFramed}
\newenvironment{redleftbar}{%
  \def\FrameCommand{\textcolor{red}{\vrule width 1pt} \hspace{10pt}}% 
  \MakeFramed {\advance\hsize-\width \FrameRestore}}%
 {\endMakeFramed}

%=================================


% --------------------------
% theoremstyle
% --------------------------
\theoremstyle{definition}

% --------------------------
% newtheoem
% --------------------------

% 日本語で定理, 命題, 証明などを番号付きで用いるためのコマンドです. 
% If you want to use theorem environment in Japanece, 
% you can use these code. 
% Attention!
% All theorem enivironment numbers depend on 
% only section numbers.
\newtheorem{Axiom}{公理}[section]
\newtheorem{Definition}[Axiom]{定義}
\newtheorem{Theorem}[Axiom]{定理}
\newtheorem{Proposition}[Axiom]{命題}
\newtheorem{Lemma}[Axiom]{補題}
\newtheorem{Corollary}[Axiom]{系}
\newtheorem{Example}[Axiom]{例}
\newtheorem{Claim}[Axiom]{主張}
\newtheorem{Property}[Axiom]{性質}
\newtheorem{Attention}[Axiom]{注意}
\newtheorem{Question}[Axiom]{問}
\newtheorem{Problem}[Axiom]{問題}
\newtheorem{Consideration}[Axiom]{考察}
\newtheorem{Alert}[Axiom]{警告}
\newtheorem{Fact}[Axiom]{事実}
\newtheorem{com}[Axiom]{コメント}


% 日本語で定理, 命題, 証明などを番号なしで用いるためのコマンドです. 
% If you want to use theorem environment with no number in Japanese, You can use these code.
\newtheorem*{Axiom*}{公理}
\newtheorem*{Definition*}{定義}
\newtheorem*{Theorem*}{定理}
\newtheorem*{Proposition*}{命題}
\newtheorem*{Lemma*}{補題}
\newtheorem*{Example*}{例}
\newtheorem*{Corollary*}{系}
\newtheorem*{Claim*}{主張}
\newtheorem*{Property*}{性質}
\newtheorem*{Attention*}{注意}
\newtheorem*{Question*}{問}
\newtheorem*{Problem*}{問題}
\newtheorem*{Consideration*}{考察}
\newtheorem*{Alert*}{警告}
\newtheorem*{Fact*}{事実}
\newtheorem*{com*}{コメント}



% 英語で定理, 命題, 証明などを番号付きで用いるためのコマンドです. 
% If you want to use theorem environment in English, You can use these code.
%all theorem enivironment number depend on only section number.
\newtheorem{Axiom+}{Axiom}[section]
\newtheorem{Definition+}[Axiom+]{Definition}
\newtheorem{Theorem+}[Axiom+]{Theorem}
\newtheorem{Proposition+}[Axiom+]{Proposition}
\newtheorem{Lemma+}[Axiom+]{Lemma}
\newtheorem{Example+}[Axiom+]{Example}
\newtheorem{Corollary+}[Axiom+]{Corollary}
\newtheorem{Claim+}[Axiom+]{Claim}
\newtheorem{Property+}[Axiom+]{Property}
\newtheorem{Attention+}[Axiom+]{Attention}
\newtheorem{Question+}[Axiom+]{Question}
\newtheorem{Problem+}[Axiom+]{Problem}
\newtheorem{Consideration+}[Axiom+]{Consideration}
\newtheorem{Alert+}{Alert}
\newtheorem{Fact+}[Axiom+]{Fact}
\newtheorem{Remark+}[Axiom+]{Remark}

% ----------------------------
% commmand
% ----------------------------
% 執筆に便利なコマンド集です. 
% コマンドを追加する場合は下のスペースへ. 

% 集合の記号 (黒板文字)
\newcommand{\NN}{\mathbb{N}}
\newcommand{\ZZ}{\mathbb{Z}}
\newcommand{\QQ}{\mathbb{Q}}
\newcommand{\RR}{\mathbb{R}}
\newcommand{\CC}{\mathbb{C}}
\newcommand{\PP}{\mathbb{P}}
\newcommand{\KK}{\mathbb{K}}


% 集合の記号 (太文字)
\newcommand{\nn}{\mathbf{N}}
\newcommand{\zz}{\mathbf{Z}}
\newcommand{\qq}{\mathbf{Q}}
\newcommand{\rr}{\mathbf{R}}
\newcommand{\cc}{\mathbf{C}}
\newcommand{\pp}{\mathbf{P}}
\newcommand{\kk}{\mathbf{K}}

% 特殊な写像の記号
\newcommand{\ev}{\mathop{\mathrm{ev}}\nolimits} % 値写像
\newcommand{\pr}{\mathop{\mathrm{pr}}\nolimits} % 射影

% スクリプト体にするコマンド
%   例えば {\mcal C} のように用いる
\newcommand{\mcal}{\mathcal}

% 花文字にするコマンド 
%   例えば {\h C} のように用いる
\newcommand{\h}{\mathscr}

% ヒルベルト空間などの記号
\newcommand{\F}{\mcal{F}}
\newcommand{\X}{\mcal{X}}
\newcommand{\Y}{\mcal{Y}}
\newcommand{\Hil}{\mcal{H}}
\newcommand{\RKHS}{\Hil_{k}}
\newcommand{\Loss}{\mcal{L}_{D}}
\newcommand{\MLsp}{(\X, \Y, D, \Hil, \Loss)}

% 偏微分作用素の記号
\newcommand{\p}{\partial}

% 角カッコの記号 (内積は下にマクロがあります)
\newcommand{\lan}{\langle}
\newcommand{\ran}{\rangle}



% 圏の記号など
\newcommand{\Set}{{\bf Set}}
\newcommand{\Vect}{{\bf Vect}}
\newcommand{\FDVect}{{\bf FDVect}}
\newcommand{\Mod}{\mathop{\mathrm{Mod}}\nolimits}
\newcommand{\CGA}{{\bf CGA}}
\newcommand{\GVect}{{\bf GVect}}
\newcommand{\Lie}{{\bf Lie}}
\newcommand{\dLie}{{\bf Liec}}



% 射の集合など
\newcommand{\Map}{\mathop{\mathrm{Map}}\nolimits}
\newcommand{\Hom}{\mathop{\mathrm{Hom}}\nolimits}
\newcommand{\End}{\mathop{\mathrm{End}}\nolimits}
\newcommand{\Aut}{\mathop{\mathrm{Aut}}\nolimits}
\newcommand{\Mor}{\mathop{\mathrm{Mor}}\nolimits}

% その他便利なコマンド
\newcommand{\dip}{\displaystyle} % 本文中で数式モード
\newcommand{\e}{\varepsilon} % イプシロン
\newcommand{\dl}{\delta} % デルタ
\newcommand{\pphi}{\varphi} % ファイ
\newcommand{\ti}{\tilde} % チルダ
\newcommand{\pal}{\parallel} % 平行
\newcommand{\op}{{\rm op}} % 双対を取る記号
\newcommand{\lcm}{\mathop{\mathrm{lcm}}\nolimits} % 最小公倍数の記号
\newcommand{\Probsp}{(\Omega, \F, \P)} 
\newcommand{\argmax}{\mathop{\rm arg~max}\limits}
\newcommand{\argmin}{\mathop{\rm arg~min}\limits}





% ================================
% コマンドを追加する場合のスペース 
\renewcommand\proofname{\bf 証明} % 証明
\numberwithin{equation}{section}
\newcommand{\cTop}{\textsf{Top}}
%\newcommand{\cOpen}{\textsf{Open}}
\newcommand{\Op}{\mathop{\textsf{Open}}\nolimits}
\newcommand{\Ob}{\mathop{\textrm{Ob}}\nolimits}
\newcommand{\id}{\mathop{\mathrm{id}}\nolimits}
\newcommand{\pt}{\mathop{\mathrm{pt}}\nolimits}
\newcommand{\res}{\mathop{\rho}\nolimits}
\newcommand{\A}{\mcal{A}}
\newcommand{\B}{\mcal{B}}
\newcommand{\C}{\mcal{C}}
\newcommand{\D}{\mcal{D}}
\newcommand{\E}{\mcal{E}}
\newcommand{\G}{\mcal{G}}
%\newcommand{\H}{\mcal{H}}
\newcommand{\I}{\mcal{I}}
\newcommand{\J}{\mcal{J}}
\newcommand{\OO}{\mcal{O}}
\newcommand{\Ring}{\mathop{\textsf{Ring}}\nolimits}
\newcommand{\cAb}{\mathop{\textsf{Ab}}\nolimits}
\newcommand{\Ker}{\mathop{\mathrm{Ker}}\nolimits}
\newcommand{\im}{\mathop{\mathrm{Im}}\nolimits}
\newcommand{\Coker}{\mathop{\mathrm{Coker}}\nolimits}
\newcommand{\Coim}{\mathop{\mathrm{Coim}}\nolimits}
\newcommand{\Ht}{\mathop{\mathrm{Ht}}\nolimits}
\newcommand{\supp}{\mathop{\mathrm{supp}}\nolimits}
\newcommand{\colim}{\mathop{\mathrm{colim}}}
\newcommand{\Tor}{\mathop{\mathrm{Tor}}\nolimits}

\newcommand{\cat}{\mathscr{C}}

\newcommand{\scA}{\mathscr{A}}
\newcommand{\scB}{\mathscr{B}}
\newcommand{\scC}{\mathscr{C}}
\newcommand{\scD}{\mathscr{D}}
\newcommand{\scE}{\mathscr{E}}
\newcommand{\scF}{\mathscr{F}}

\newcommand{\ibA}{\mathop{\text{\textit{\textbf{A}}}}}
\newcommand{\ibB}{\mathop{\text{\textit{\textbf{B}}}}}
\newcommand{\ibC}{\mathop{\text{\textit{\textbf{C}}}}}
\newcommand{\ibD}{\mathop{\text{\textit{\textbf{D}}}}}
\newcommand{\ibE}{\mathop{\text{\textit{\textbf{E}}}}}
\newcommand{\ibF}{\mathop{\text{\textit{\textbf{F}}}}}
\newcommand{\ibG}{\mathop{\text{\textit{\textbf{G}}}}}
\newcommand{\ibH}{\mathop{\text{\textit{\textbf{H}}}}}
\newcommand{\ibI}{\mathop{\text{\textit{\textbf{I}}}}}
\newcommand{\ibJ}{\mathop{\text{\textit{\textbf{J}}}}}
\newcommand{\ibK}{\mathop{\text{\textit{\textbf{K}}}}}
\newcommand{\ibL}{\mathop{\text{\textit{\textbf{L}}}}}
\newcommand{\ibM}{\mathop{\text{\textit{\textbf{M}}}}}
\newcommand{\ibN}{\mathop{\text{\textit{\textbf{N}}}}}
\newcommand{\ibO}{\mathop{\text{\textit{\textbf{O}}}}}
\newcommand{\ibP}{\mathop{\text{\textit{\textbf{P}}}}}
\newcommand{\ibQ}{\mathop{\text{\textit{\textbf{Q}}}}}
\newcommand{\ibR}{\mathop{\text{\textit{\textbf{R}}}}}
\newcommand{\ibS}{\mathop{\text{\textit{\textbf{S}}}}}
\newcommand{\ibT}{\mathop{\text{\textit{\textbf{T}}}}}
\newcommand{\ibU}{\mathop{\text{\textit{\textbf{U}}}}}
\newcommand{\ibV}{\mathop{\text{\textit{\textbf{V}}}}}
\newcommand{\ibW}{\mathop{\text{\textit{\textbf{W}}}}}
\newcommand{\ibX}{\mathop{\text{\textit{\textbf{X}}}}}
\newcommand{\ibY}{\mathop{\text{\textit{\textbf{Y}}}}}
\newcommand{\ibZ}{\mathop{\text{\textit{\textbf{Z}}}}}

\newcommand{\ibx}{\mathop{\text{\textit{\textbf{x}}}}}

\newcommand{\Comp}{\mathop{\mathrm{C}}\nolimits}
\newcommand{\Komp}{\mathop{\mathrm{K}}\nolimits}
\newcommand{\Domp}{\mathop{\mathrm{D}}\nolimits}%複体のホモトピー圏

\newcommand{\CCat}{\Comp(\cat)}
\newcommand{\KCat}{\Komp(\cat)}
\newcommand{\DCat}{\Domp(\cat)}%圏Cの複体のホモトピー圏
\newcommand{\HOM}{\mathop{\mathscr{H}\hspace{-2pt}om}\nolimits}%内部Hom
\newcommand{\RHOM}{\mathop{\mathrm{R}\hspace{-1.5pt}\HOM}\nolimits}

\newcommand{\muS}{\mathop{\mathrm{SS}}\nolimits}
\newcommand{\RG}{\mathop{\mathrm{R}\hspace{-0pt}\Gamma}\nolimits}

\newcommand{\simar}{\mathrel{\overset{\sim}{\longrightarrow}}}%内部Hom
\newcommand{\simra}{\mathrel{\overset{\sim}{\longleftarrow}}}%内部Hom

\newcommand{\hocolim}{{\mathrm{hocolim}}}
\newcommand{\indlim}[1][]{\mathop{\varinjlim}\limits_{#1}}
\newcommand{\sindlim}[1][]{\smash{\mathop{\varinjlim}\limits_{#1}}\,}
\newcommand{\Pro}{\mathrm{Pro}}
\newcommand{\Ind}{\mathrm{Ind}}
\newcommand{\prolim}[1][]{\mathop{\varprojlim}\limits_{#1}}
\newcommand{\sprolim}[1][]{\smash{\mathop{\varprojlim}\limits_{#1}}\,}

% =================================



%================================================
% 自前の定理環境
%   https://mathlandscape.com/latex-amsthm/
% を参考にした
\newtheoremstyle{mystyle}%   % スタイル名
    {5pt}%                   % 上部スペース
    {5pt}%                   % 下部スペース
    {}%              % 本文フォント
    {}%                  % 1行目のインデント量
    {\bfseries}%                      % 見出しフォント
    {.}%                     % 見出し後の句読点
    {12pt}%                     % 見出し後のスペース
    {\thmname{#1}\thmnumber{ #2}\thmnote{{\hspace{2pt}\normalfont (#3)}}}% % 見出しの書式

\theoremstyle{mystyle}
\newtheorem{AXM}{公理}[section]
\newtheorem{DFN}[Axiom]{定義}
\newtheorem{THM}[Axiom]{定理}
\newtheorem*{THM*}{定理}
\newtheorem{PRP}[Axiom]{命題}
\newtheorem{LMM}[Axiom]{補題}
\newtheorem{CRL}[Axiom]{系}
\newtheorem{EG}[Axiom]{例}
\newtheorem{CNV}[Axiom]{規約}


% 定理環境ここまで
%====================================================

% ---------------------------
% new definition macro
% ---------------------------
% 便利なマクロ集です

% 内積のマクロ
%   例えば \inner<\pphi | \psi> のように用いる
\def\inner<#1>{\langle #1 \rangle}

% ================================
% マクロを追加する場合のスペース 

%=================================





% ----------------------------
% documenet 
% ----------------------------
% 以下, 本文の執筆スペースです. 
% Your main code must be written between 
% begin document and end document.
% ---------------------------

\title{Notes on Sheaves on Manifolds}
\author{大柴寿浩}
\date{}
\begin{document}
\maketitle
\frontmatter
\layout
\chapter*{はじめに}
2023年度から始めた\cite{KS90}のセミナーのノート.

\section*{記号}
次の記号は断りなく使う.
\begin{itemize}
    \item 添字:
    なんらかの族$(a_i)_{i\in I}$を$(a_i)_i$とか$(a_i)$と
    略記することがある.
    \item 近傍:位相空間\(X\)の点\(x\)や部分集合\(Z\)に対し,
    その開近傍系をそれぞれ\(I_x\)や\(I_Z\)で表す.
    これらは,包含関係の逆で有向順序集合をなす.
\end{itemize}
\mainmatter
\chapter{ホモロジー代数}
\setcounter{section}{2}

\section{複体の圏}

$\cat$を加法圏とする.
\begin{Attention*}
    加法圏とは次の3つの条件(\ref{additive:bilin})--(\ref{additive:biproduct})をみたす圏のことである.
    \begin{enumerate}
        \renewcommand{\labelenumi}{({\arabic{enumi}})}
        \item どの対象$X, Y\in\cat$に対しても$\Hom_{\cat}(X,Y)$が
        加法群になり,どの対象$X, Y, Z\in\cat$に対しても
        合成$\circ\colon\Hom_{\cat}(Y,Z)\times\Hom_{\cat}(X,Y)
        \to\Hom_{\cat}(X,Z)$が双線型である.\label{additive:bilin}
        \item 零対象$0\in\cat$が存在する.
        さらに$\Hom_{\cat}(0,0)=0$が成り立つ.\label{additive:zero}
        \item 任意の対象$X, Y\in\cat$に対して積と余積が存在し,
        さらにそれらは同型になる.(それらを複積といい$X\oplus Y$とかく.)\label{additive:biproduct}
    \end{enumerate}
\end{Attention*}
圏$\cat$から,
$\cat$の対象の複体の圏$\Comp(\cat)$を作ることができる.
まず複体の定義をする.
圏$\cat$の対象のと射の列
\begin{equation}\label{eq:complex}
    \begin{tikzcd}[column sep=0.6cm]
        \cdots
        \arrow[r]
        &
        X^{n-1}
        \arrow[r,"d_X^{n-1}"] 
        &[+0.3cm]%[-0.03\linewidth]
        X^n
        \arrow[r,"d_X^{n}"]
        &%[0.12\linewidth]
        X^{n+1}
        \arrow[r, ""] 
        &%[0.12\linewidth]
        \cdots
    \end{tikzcd}
\end{equation}
を考える.
この列$X=\left((X^n)_{n\in\zz},(d_X^{n})_{n\in\zz}\right)$が
\textbf{複体} (complex) であるとは,任意の$n\in\zz$に対し
\begin{equation}
    d_X^{n+1}\circ d_X^{n}=0
\end{equation}
が成り立つことをいう.

圏$\cat$の
対象の複体$X=((X^n),(d_X^{n}))$, $Y=((Y^n),(d_Y^{n}))$の間の射を,
$\cat$の射の族$(f^n\colon X^n\to Y^n)_{n\in\zz}$で,
図式
\begin{equation*}
    \vcenter{\xymatrix@C=26pt@R=26pt{
    \cdots \ar[r]
    & 
    X^{n}
    \ar[r]^-{d_X^n}
    \ar[d]^-{f^n}
    &
    X^{n+1}
    \ar[r]
    \ar[d]^-{f^{n+1}} 
    &\cdots
    \\
    \cdots \ar[r]
    & 
    Y^{n}
    \ar[r]^-{d_Y^n}
    &
    Y^{n+1}
    \ar[r]
    &\cdots
    }}
\end{equation*}
を可換にする,すなわちどの番号$n\in\zz$に対しても
\begin{equation}
    d_Y^n\circ f^n=f^{n+1}\circ d_X^n 
\end{equation}
が成り立つものとして定める.

以上の準備のもとで,$\cat$の複体の圏$\Comp(\cat)$を次のように定める.
\begin{itemize}
    \item 対象:$\Ob(\Comp(\cat))=\{\text{$\cat$の複体}\}$
    \item 射:$\Hom_{\Comp(\cat)}(X,Y)=\{\text{$\cat$の複体の射}\}$
\end{itemize}
このとき,$\CCat$は加法圏になる.
\begin{proof}[\textbf{圏になることの証明}]
    $f\colon X\to Y$と$g\colon Y\to Z$を$\CCat$の射とする.
    $f$と$g$の合成$g\circ f$は$(g^n\circ f^n)_n$で与えられる.
    これがうまくいくことは
    \begin{equation*}
        \vcenter{\xymatrix@C=26pt@R=26pt{
        \cdots \ar[r]
        & 
        X^{n}
        \ar[r]^-{d_X^n}
        \ar[d]^-{f^n}
        &
        X^{n+1}
        \ar[r]
        \ar[d]^-{f^{n+1}} 
        &\cdots
        \\
        \cdots \ar[r]
        & 
        Y^{n}
        \ar[r]^-{d_Y^n}
        \ar[d]^-{g^n}
        &
        Y^{n+1}
        \ar[r]
        \ar[d]^-{g^{n+1}} 
        &\cdots
        \\
        \cdots \ar[r]
        & 
        Z^{n}
        \ar[r]^-{d_Z^n}
        &
        Z^{n+1}
        \ar[r]
        &\cdots
        }}
    \end{equation*}
    が可換になることからわかる.

    $X$の恒等射は$(\id_{X^n})_n$で与えられる.
\end{proof}
\begin{proof}[\textbf{加法圏になることの証明}]
    $X$と$Y$を$\cat$の複体とする.
    \subparagraph*{(\ref{additive:bilin})射の集合のアーベル群構造}
    $f,g\in\Hom_{\CCat}(X,Y)$に対し,$f+g$が$(f^n+g^n)_n$で定まる.
    \subparagraph*{(\ref{additive:zero})零対象の存在}
    $\CCat$の零対象0は
    \begin{equation*}%\label{eq:complex}
        \begin{tikzcd}[column sep=0.4cm]
            \cdots
            \arrow[r]
            &
            0
            \arrow[r,"0"] 
            &%[+0.3cm]
            0
            \arrow[r,"0"]
            &%[0.12\linewidth]
            0
            \arrow[r, ""] 
            &%[0.12\linewidth]
            \cdots
        \end{tikzcd}
    \end{equation*}
    で与えられる.
    \subparagraph*{(\ref{additive:biproduct})複積の存在}    
    $X$と$Y$の複積$X\oplus Y$は
    \begin{equation*}%\label{eq:complex}
        \begin{tikzcd}[column sep=1.2cm]
            \cdots
            \arrow[r]
            &[-0.6cm]
            X^{n-1}\oplus Y^{n-1}
            \arrow[r,"d_X^{n-1}\oplus d_Y^{n-1}"] 
            &[+0.3cm]%[-0.03\linewidth]
            X^n\oplus Y^{n}
            \arrow[r,"d_X^{n}\oplus d_Y^{n}"]
            &%[0.12\linewidth]
            X^{n+1}\oplus Y^{n+1}
            \arrow[r, ""] 
            &[-0.6cm]%[0.12\linewidth]
            \cdots
        \end{tikzcd}
    \end{equation*}
    で与えられる.
\end{proof}

さらに$\cat$がアーベル圏ならば,$\CCat$もアーベル圏になる.
\begin{Attention*}
    加法圏$\cat$がアーベル圏であるとは
    次の条件(\ref{abel:ker}), (\ref{abel:homTh})をみたすことをいう.
    \begin{enumerate}
        \renewcommand{\labelenumi}{({\arabic{enumi}})}
        \setcounter{enumi}{3}
        \item 任意の$\cat$の射$f\colon X\to Y$に対し,
        $f$の核$\Ker{f}$と余核$\Coker f$が存在する.\label{abel:ker}
        \item 任意の$\cat$の射$f\colon X\to Y$に対し,
        自然に定まる射$\Coim f\to\im f$は同型である.\label{abel:homTh}
    \end{enumerate}
\end{Attention*}

\begin{proof}$X$と$Y$を$\cat$の複体とする.
    \subparagraph*{(\ref{abel:ker})核と余核の存在}
    複体の射$f\colon X\to Y$に対し,
    核$\Ker{f}$は$(\Ker{f^n})_n$で,
    余核$\Coker{f}$は$(\Coker{f^n})_n$で与えられる.
    \begin{com*}[4/24]
        「$\Ker{f}$のdifferentialの構成はどうなっていますか?」

        次の図式を考える.
        \begin{equation*}
            \vcenter{\xymatrix@C=26pt@R=18pt{
            \cdots \ar[r]
            & 
            \Ker{f^{n}}
            \ar[r]^-{\overline{d}_X^n}
            \ar[d]^-{\iota^n}
            &
            \Ker{f^{n+1}}
            \ar[r]
            \ar[d]^-{\iota^{n+1}} 
            &\cdots
            \\    
            \cdots \ar[r]
            & 
            X^{n}
            \ar[r]^-{d_X^n}
            \ar[d]^-{f^n}
            &
            X^{n+1}
            \ar[r]
            \ar[d]^-{f^{n+1}} 
            &\cdots
            \\
            \cdots \ar[r]
            & 
            Y^{n}
            \ar[r]^-{d_Y^n}
            &
            Y^{n+1}
            \ar[r]
            &\cdots
            }}
        \end{equation*}
        ここで,$\iota^n$は$\Ker{f^n}$の普遍性から自然に定まる射である.
        $\overline{d}_X^n\colon\Ker{f^n}\to\Ker{f^{n+1}}$が
        $d_X^n\circ\iota^n$によって定められることを示せば良い.
        \begin{align*}
            f^{n+1}\circ d_X^n\circ\iota^n
            =d_Y^n\circ f^{n+1}\circ\iota^n
            =d_Y^n\circ0=0
        \end{align*}
        より,$d_X^n\circ\iota^n$は$\Ker{f^{n+1}}$に
        値を取る.したがって,
        $\overline{d}_X^n\colon\Ker{f^n}\to\Ker{f^{n+1}}$が定まる.
    \end{com*}

    \subparagraph*{(\ref{abel:homTh})余像と像が同型になること}
    各次数$n$ごとに$\Coim f^n\cong \im f^n$が成り立つことから従う.
\end{proof}
圏$\CCat$の充満部分圏$\Comp^+(\cat)$, 
$\Comp^-(\cat)$, $\Comp^\mathrm{b}(\cat)$を
\begin{align*}
    \Ob(\Comp^+(\cat))
    &=\left\{
        0\rightarrow
        X^{n}\overset{d_X^{n}}{\longrightarrow}
        X^{n+1}\rightarrow\cdots
        \quad (n\ll 0)
    \right\}, \\
    \Ob(\Comp^-(\cat))
    &=\left\{
        \cdots\rightarrow 
        X^{n-1}\overset{d_X^{n-1}}{\longrightarrow}
        X^{n}\rightarrow
        0
        \quad (n\gg 0)
    \right\}, \\
    \Ob(\Comp^\mathrm{b}(\cat))
    &=\left\{ 
        0\rightarrow
        X^{n}\rightarrow
        \cdots\rightarrow 
        X^{m}\rightarrow
        0
        \quad (n\ll 0, m\gg0)
    \right\}
\end{align*}
で定める.

$\cat$の対象$X$に対し$\CCat$の対象
\begin{equation*}
    \cdots\to
    0\rightarrow
    X\rightarrow
    0\to\cdots
\end{equation*}
を対応させることによって,
忠実充満な関手$\cat\hookrightarrow\CCat$が定まる.

$k$を整数とする.$\cat$の複体
\begin{equation*}
    \begin{tikzcd}[column sep=0.6cm]
        X\colon\cdots
        \arrow[r]
        &
        X^{n-1}
        \arrow[r,"d_X^{n-1}"] 
        &[+0.3cm]%[-0.03\linewidth]
        X^n
        \arrow[r,"d_X^{n}"]
        &%[0.12\linewidth]
        X^{n+1}
        \arrow[r, ""] 
        &%[0.12\linewidth]
        \cdots
    \end{tikzcd}
\end{equation*}
に対し,$X[k]$を
$X[k]^n=X^{n+k}$, $d_{X[k]}^n=(-1)^{k}d_X^{n+k}$
で定める.図式でかくと
\begin{equation*}
    \begin{tikzcd}[column sep=1.4cm]
        X[k]\colon\cdots
        \arrow[r]
        &[-1cm]
        X^{n+k-1}
        \arrow[r,"(-1)^kd_X^{n+k-1}"] 
        &[+0.5cm]%[-0.03\linewidth]
        X^{n+k}
        \arrow[r,"(-1)^kd_X^{n+k}"]
        &%[0.12\linewidth]
        X^{n+k+1}
        \arrow[r, ""] 
        &[-1cm]%[0.12\linewidth]
        \cdots
    \end{tikzcd}
\end{equation*}
のようになる.
$X$から$Y$への射$f\colon X\to Y$に対し,
$f[k]\colon X[k]\to Y[k]$を$f[k]^n=f^{n+k}$で定める.
$X$を$X[k]$に対応させることで
関手$[k]\colon\CCat\to\CCat$が定まる.
この関手を次数$k$のシフト関手と呼ぶ.

\begin{proof}[\textbf{$[k]$が関手になることの証明}]
    $X[k]$が複体になること:
    \begin{align*}
        (-1)^kd_X^{n+k}\circ (-1)^kd_X^{n+k-1}
        =(-1)^{2k}d_X^{n+k}\circ d_X^{n+k-1}=0.
    \end{align*} 

    $f[k]$が複体の射になること:
    \begin{equation*}
        \vcenter{\xymatrix
        @C=20pt@R=26pt
        {
        \cdots \ar[r]
        & 
        X^{n+k}
        \ar[rr]^-{(-1)^kd_X^{n+k}}
        \ar[d]^-{f^{n+k}}
        &&
        X^{n+k+1}
        \ar[r]
        \ar[d]^-{f^{n+k+1}} 
        &\cdots
        \\
        \cdots \ar[r]
        & 
        Y^{n+k}
        \ar[rr]^-{(-1)^kd_Y^{n+k}}
        &&
        Y^{n+k+1}
        \ar[r]
        &\cdots
        }}
    \end{equation*}
    が可換になることを示せばよい.
    \begin{align*}
        f^{n+k+1}\circ (-1)^{k}d_X^{n+k+1}
        =(-1)^{k}f^{n+k+1}\circ d_X^{n+k+1}
        =(-1)^{k}d_Y^{n+k+1}\circ f^{n+k}. 
    \end{align*}

    $[k]$が合成を保つこと:
    $f\colon X\to Y$, $g\colon Y\to Z$を複体の射とする.
    このとき
    \begin{align*}
        (g[k]\circ f[k])^n
        =g[k]^n\circ f[k]^n
        =g^{n+k}\circ f^{n+k}
        =(g\circ f)^{n+k}
        =(g\circ f)[k]^n        
    \end{align*}
    が成り立つ.

    $[k]$が恒等射を保つこと:
    $\id_X[k]^n=\id_X^{n+k}=\id_{X[k]}^{n}$.
\end{proof}

\paragraph{ホモトピー}
$\cat$の複体の圏$\CCat$から,ホモトピックな射を同一視することによって,
新たな圏$\KCat$が得られる.まず準備.

$\CCat$を圏$\cat$の複体の圏とする.
$X,Y\in\CCat$とする.
$f\colon X\to Y$が0にホモトピックであるとは,
$\cat$の射の族$(s^n\colon X^n\to Y^{n-1})$で,
\begin{equation}
    f^n=d_Y^{n-1}\circ s^n+s^{n+1}\circ d_X^{n}\quad(n\in\zz)
\end{equation}
となるものが存在することをいう.

$f,g\colon X\to Y$に対し,
$f-g$が0にホモトピックであるとき,
$f$と$g$はホモトピックであるといい,$f\simeq g$とかく.
$f$が0とホモトピックであることを$f\simeq0$で表す.
このとき$s=(s^n)$を$f$と$g$の間のホモトピーという.
$\simeq$は同値関係である.
\begin{proof}$f,g,h$を$X$から$Y$への$\cat$の複体の射とする.
    \subparagraph*{反射律}$(s^n=0)$が$f$と$f$の間の
    ホモトピーを与える.

    \subparagraph*{対称律}
    $f$と$g$の間のホモトピーを$s$とするとき,
    $-s$が$g$と$f$の間のホモトピーを与える.

    \subparagraph*{推移律}$f$と$g$の間のホモトピーを$s$, 
    $g$と$h$の間のホモトピーを$t$とする.
    このとき,$s+t$が$f$と$h$の間のホモトピーを与える.
\end{proof}

\begin{Proposition}\label{prop:ht}
    $X,Y\in\CCat$に対し,
    $\Hom_{\CCat}(X,Y)$の加法部分群$\Ht(X,Y)$を
    \begin{equation}
        \Ht(X,Y)\coloneqq\{f\in\Hom_{\CCat}(X,Y)\mid f\simeq 0\}
    \end{equation}
    で定める.複体の射$f\colon X\to Y$と$g\colon Y\to Z$のどちらかが
    0にホモトピックならば,合成$g\circ f$は$0$にホモトピックになる.
    したがって,射の合成は次の写像をひきおこす.
    \begin{align*}
        \Hom_{\CCat}(Y,Z)\times\Ht(X,Y)\to\Ht(X,Z),\\
        \Ht(Y,Z)\times\Hom_{\CCat}(X,Y)\to\Ht(X,Z).
    \end{align*}        
\end{Proposition}
\begin{proof}$f\in\Hom_{\CCat}(X,Y)$, 
    $g\in\Hom_{\CCat}(Y,Z)$とする.
    
    $f\simeq 0$のとき,$s$を0とのホモトピーとすると,
    $g\circ f$と0との間のホモトピーは
    \begin{align*}
        (g^{n-1}\circ s^{n}\colon X^n\to Y^{n-1}\to Z^{n-1})_n
    \end{align*}
    で与えられる.

    $g\simeq 0$のとき,$t$を0とのホモトピーとすると,
    $g\circ f$と0との間のホモトピーは
    \begin{align*}
        (t^{n}\circ f^{n}\colon X^n\to Y^{n}\to Z^{n-1})_n
    \end{align*}
    で与えられる.
\end{proof}

以上の準備のもとで,圏$\cat$のホモトピー圏$\KCat$を次のように定める.
\begin{itemize}
    \item 対象:$\Ob(\KCat)=\Ob(\CCat)$
    \item 射:$\Hom_{\KCat}(X,Y)=\Hom_{\CCat}(X,Y)/\Ht(X,Y)$
\end{itemize}
$\KCat$は加法圏になる.
\begin{proof}[\textbf{$\KCat$が加法圏になることの証明}]
    命題\ref{prop:ht}より,射の合成がきちんと定まる.

    各$X,Y\in\KCat$に対する$\Hom_{\KCat}(X,Y)$のアーベル群構造は
    $\Ht(X,Y)$による剰余群の構造として得られ,
    さらに命題\ref{prop:ht}より,合成の双線型性が得られる.

    零対象と複積は$\CCat$と同様である.
\end{proof}

圏$\KCat$の充満部分圏$\Komp^+(\cat)$, 
$\Komp^-(\cat)$, $\Komp^\mathrm{b}(\cat)$を,
それぞれ$\Comp^+(\cat)$, 
$\Comp^-(\cat)$, $\Comp^\mathrm{b}(\cat)$と同じ対象をとって定める.

\paragraph*{コホモロジー}
$\cat$をアーベル圏とする.
$X\in\CCat$に対し,
\begin{align*}
    Z^k(X)&\coloneqq \Ker d_X^k,\\
    B^k(X)&\coloneqq \im d_X^{k-1},\\
    H^k(X)&\coloneqq \Ker d_X^k/\im d_X^{k-1}
\end{align*}
とおく.$H^k(X)$を複体$X$の$k$次のコホモロジーという.

\begin{Attention*}
    完全列$0\to X\rightarrow Y\rightarrow Z\to0$に対し,
    $Z$を$Y$の商対象といい,$Y/X$とかく.
    一般に単射$i\colon X\hookrightarrow Y$の
    余核$\Coker{i}$を$Y/X$とかける.
\end{Attention*}

任意の$k$に対し$H^k$は$\CCat$から$\cat$への加法関手を定める.
\begin{equation}
    H^k(X)=H^0(X[k])
\end{equation}
$f\colon X\to Y$が0とホモトピックならば,
$H^k(f)\colon H^k(X)\to H^k(Y)$は0. 
よって$H^k$は$\KCat$から$\cat$への関手を定める.

完全列たち
\begin{align*}
    &X^{k-1}\to Z^{k}(X)\to H^{k}(X)\to 0,\\
    &0\to H^{k}(X)\to \Coker{d_X^{k-1}}\to X^{k+1},\\
    &0\to Z^{k-1}(X)\to X^{k-1}\to B^{k}(X)\to 0,\\
    &0\to B^{k}(X)\to X^{k}\to \Coker{d_X^{k-1}}\to 0,\\
    &0\to H^{k}(X)\to \Coker{d_X^{k-1}}\overset{d_X^{k}}{\longrightarrow}Z^{k+1}(X)\to H^{k+1}(X)\to 0.
\end{align*}

\begin{Proposition}
    $0\to X\to Y\to Z\to 0$を$\CCat$の完全列とする.
    このとき,$\cat$における次の長完全列が存在する.
    \begin{equation*}
        \cdots\to 
        H^{n}(X)\to H^{n}(Y)\to H^{n}(Z)
        \overset{\delta}{\longrightarrow} 
        H^{n+1}(X)\to \cdots.
    \end{equation*}
\end{Proposition}

\paragraph*{切り落とし}
$X\in\CCat$と整数$n$に対し,
$\tau^{\leqq n}(X),\tau^{\geqq n}(X)\in\CCat$を
\begin{align}
    \tau^{\leqq n}(X)\colon 
    \cdots\to X^{n-2}\to X^{n-1}\to \Ker{d^n}\to0\to\cdots,\\
    \tau^{\geqq n}(X)\colon 
    \cdots 0\to \Coker{d^{n-1}}\to X^{n+1}\to X^{n+2}\to\cdots
\end{align}
で定める.
このとき,$\CCat$における次の射が得られる.
\begin{align*}
    \tau^{\leqq n}(X)\to X,\quad X\to\tau^{\geqq n}(X),
\end{align*}
また,$n'\leqq n$ならば
\begin{align*}
    \tau^{\leqq n'}(X)\to \tau^{\leqq n}(X),
    \quad 
    \tau^{\geqq n'}(X)\to\tau^{\geqq n}(X).
\end{align*}

\begin{Proposition}
    \begin{enumerate}
        \item 自然な射$H^k(\tau^{\leqq n}(X))\to H^k(X)$は$k\leqq n$ならば同型であり,$k>n$では$H^k(X)=0$である.
        \item 自然な射$H^k(X)\to H^k(\tau^{\geqq n}(X))$は$k\geqq n$ならば同型であり,$k<n$では$H^k(X)=0$である.
    \end{enumerate}
\end{Proposition}

\begin{Attention}
    ホモトピー同値    
\end{Attention}

\section{写像錐}
$\cat$を加法圏とし$f\colon X\to Y$を$\CCat$の射とする.
\begin{Definition}
    $f$の写像錐$M(f)$とは次で定まる$\CCat$の対象である.
    \begin{align*}
        \begin{cases}
            M(f)^n=X^{n+1}\oplus Y^{n},\\
            d^{n}_{M(f)}=\begin{bmatrix*}
                d_{X[1]}^n & 0\\
                f^{n+1} & d_Y^n\\
            \end{bmatrix*}
        \end{cases}
    \end{align*}
\end{Definition}

射$\alpha(f)\colon Y\to M(f)$と$\beta(f)\colon M(f)\to X[1]$を
次で定める.
\begin{align}
    \alpha(f)^{n}=\begin{bmatrix*}
        0\\\id_{Y^n}
    \end{bmatrix*},\\
    \beta(f)^{n}=\begin{bmatrix*}
        \id_{X^{n+1}}&0
    \end{bmatrix*}.
\end{align}

\begin{com*}[4/24]
    「どうして逆に$X\to M(f)$や$M(f)\to Y$じゃないんですか?」
    
    例えば,逆に$\gamma^n \colon M(f)^n\to Y^n$を$\begin{bmatrix*}
        0&\id_{Y^n}
    \end{bmatrix*}$で定めようとしても,
    \begin{align*}
        \gamma^{n+1}\circ d_{M(f)}^{n}
        &=\begin{bmatrix*}
            0&\id_{Y^n}
        \end{bmatrix*}\begin{bmatrix*}
            d_{X[1]}^n & 0\\
            f^{n+1} & d_Y^n\\
        \end{bmatrix*}
        =\begin{bmatrix*}
            f^{n+1}&d_Y^n
        \end{bmatrix*},\\
        d_{Y}^n\circ \gamma^{n}
        &=d_Y^n\circ
        \begin{bmatrix*}
            0&\id_{Y^n}
        \end{bmatrix*}
        =
        \begin{bmatrix*}
            0&d_Y^n
        \end{bmatrix*}
    \end{align*}
    となり,両者は一致しない.
    したがって,$\gamma$は複体の射にならない.
    $X\to M(f)$も同様である.
    したがって,$M(f)$に対して定まる自然な射は
    $\alpha, \beta$のようにせざるを得ない.
\end{com*}

\begin{Lemma}
    任意の$\CCat$の射$f\colon N\to Y$に対し,
    $\phi\colon X[1]\to M(\alpha(f))$で次の条件をみたすものが存在する.
    \begin{enumerate}
        \item $\phi$は$\KCat$で同型である,
        \item 次の図式は$\KCat$で可換になる:    
        \begin{equation*}
            \vcenter{\xymatrix@C=30pt@R=30pt{
            Y \ar[r]^-{\alpha(f)}
            \ar[d]^-{\id_Y}
            & 
            M(f)
            \ar[r]^-{\beta(f)}
            \ar[d]^-{\id_{M(f)}}
            &
            X[1]
            \ar[r]^-{-f[1]}
            \ar[d]^-{\phi} 
            &
            Y[1]
            \ar[d]^-{\id_{Y[1]}}
            \\
            Y 
            \ar[r]^-{\alpha(f)}
            & 
            M(f)
            \ar[r]^-{\alpha(\alpha(f))}
            &
            M(\alpha(f))
            \ar[r]^-{\beta(\alpha(f))}
            &
            Y[1].
            }}
        \end{equation*}
    \end{enumerate}
\end{Lemma}

%\begin{proof}
%    $\phi$と逆$\psi$を次で定める.
%\end{proof}

\newpage
%\thispagestyle{myheadings}
{\Large{2023/05/01}}
\section{三角圏}\label{ssec:tricat}
$\cat$を加法圏とし,$T\colon\cat\to\cat$を自己関手とする.
$\cat$の三角とは射の列
\begin{equation*}
    X\to Y\to Z\to T(X)
\end{equation*}
のことである.

\begin{Definition}
    三角圏$\cat$は次のデータ\eqref{eq:tricat}, \eqref{eq:family-dt}と
    規則(TR\ref{TR0})--(TR\ref{TR5})からなる.\ref{des:data}
    \begin{description}
        \item[データ] \begin{align}
            %\renewcommand{\labelenumi}{(\ref{ssec:tricat}.{\arabic{enumi}})}
            \text{加法圏}\cat\text{と自己関手}T\colon\cat\to\cat\text{の組},\label{eq:tricat}\\
            \text{特三角 (distinguished triangle) の族.}\label{eq:family-dt}
        \end{align}    
        \item[規則] \label{des:data}
        \begin{quote}
            \begin{enumerate}
                %\setlength{\leftskip}{11pt}
                \setcounter{enumi}{-1}
                \renewcommand{\labelenumi}{(TR{\arabic{enumi}})}
                \item 特三角に同形な三角は特三角である.\label{TR0}
                \item 任意の対象$X\in\cat$に対し,
                    $X\overset{\id_X}{\longrightarrow}X\longrightarrow
                    0\longrightarrow T(X)$は特三角である.\label{TR1}
                \item $\cat$の任意の射$f\colon X\to Y$は
                    特三角$X\overset{f}{\to}Y\to Z\to T(X)$に埋め込める.
                    つまり$Z\in\cat$で$X\overset{f}{\to}Y\to Z\to T(X)$が
                    特三角となるものが存在する.\label{TR2}
                \item $X\overset{f}{\to}
                    Y\overset{g}{\to} 
                    Z\overset{h}{\to} T(X)$が
                    特三角であることと$Y\overset{g}{\longrightarrow}
                    Z\overset{h}{\longrightarrow} T(X)\overset{-T(f)}{\longrightarrow} T(Y)$が
                    特三角であることは同値である.\label{TR3}
                \item 2つの特三角$X\overset{f}{\to}
                Y\to Z\to T(X)$, $X'\overset{f'}{\to}
                Y'\to Z'\to T(X')$に対し,可換図式
                \begin{equation*}
                    \vcenter{\xymatrix
                    @C=26pt@R=26pt
                    {
                    X\ar[r]^-{f}\ar[d]^-{u}
                    &
                    Y\ar[d]^-{v} 
                    \\
                    X'\ar[r]^-{f'}
                    &
                    Y'
                    }}
                \end{equation*}
                は特三角の射に埋め込める.\label{TR4}
                \item(八面体公理).3つの特三角
                \begin{align*}
                    X\overset{f}{\to}Y\to Z'\to T(X), \\
                    Y\overset{g}{\to}Z\to X'\to T(Y),\\
                    X\overset{g\circ f}{\longrightarrow}
                    Z\to Y'\to T(X)
                \end{align*}
                に対し, \label{TR5}
            \end{enumerate}    
        \end{quote}
    \end{description}
\end{Definition}

\section{圏の局所化}

%\begin{equation}\label{eq:complex}
%    \cdots\rightarrow 
%    X^{n-1}\overset{d_X^{n-1}}{\longrightarrow}
%    X^{n}\overset{d_X^{n}}{\longrightarrow}
%    X^{n+1}\rightarrow\cdots
%\end{equation}

%$\Ob(\Comp(\cat))=\left\{\right\}$

\clearpage
\chapter{層}
アーベル層の圏はアーベル圏になる.
したがって層の導来圏が考えられる.

\(\cdot\otimes\cdot\)の導来関手を考えたいが,
テンソルに関する複体が有界になるとは限らないので,
平坦分解の長さが有限になるという仮定をおく.

\section{弱大域次元}
\begin{PRP}
    \(A\)を環とする.
    \begin{enumerate}
        \item 自由加群は射影加群である.
        
        \item 射影加群は自由加群の自由加群の直和因子である.
        
        \item 射影加群は平坦加群である.
        
        \item \(n\geqq0\)を整数とする.
            次の条件(a)--\((\text{b})^{\op}\)は同値である.
            \begin{enumerate}[(a)]
                \item 任意の\(j>n\), \(N\in\Mod(A^\op)\), 
                \(M\in\Mod(A)\)に対し,\(\Tor_j^A(N,M)=0\)
                \item 任意の\(M\in\Mod(A)\)に対し,分解\[
                    0\to P^n\to\dots\to P^0\to M\to0\quad\text{(\(P^j\)は平坦)}
                    \]が存在する.
                \item[\((\text{b})^{\op}\)]\setlength{\leftskip}{10pt}
                任意の\(M\in\Mod(A^\op)\)に対し,分解\[
                    0\to P^n\to\dots\to P^0\to M\to0\quad\text{(\(P^j\)は平坦)}
                \]が存在する.
            \end{enumerate}
    \end{enumerate}
\end{PRP}
\begin{proof}
    \begin{enumerate}
        \item \(M\)を自由加群とする.左\(A\)加群の全射
        \(g\colon N\twoheadrightarrow N'\)に対し,\[
            g_\ast\colon\Hom_A(M,N)\to\Hom_A(M,N') 
            \quad\text{in \(\Mod(\zz)\)}
        \]が全射であることを示す.
        \(\psi\colon M\to N'\)を\(A\)加群の射とする.
        \(I\)を\(M\cong A^{\oplus I}\)となる添字集合とすると
        任意の\(m\in M\)は,
        \(M\)の生成系\((m_i)\)と\((a_i)_i\in A^{\oplus I}\)を
        用いて,\(m=\sum_{i\in I}a_im_i\)とかける.
        このとき,\[
            \psi(m)=\sum_{i}a_i\psi(m_i)\in N'
        \]であり,\(g\)が全射なので,\(n\in N\)で\[
            g(n)=\psi(m)=\sum_{i}a_i\psi(m_i),
            \quad
            \psi(m_i)=g(n_i)
        \]となるものがある.
        この\((n_i)_i\)に対して,\(\phi\colon M\to N\)を
        \[
            \phi(m_i)=n_i
        \]で定めると,
        \[
            \left(g_\ast(\phi)\right)(m_i)=g\circ\phi(m_i)=g(n_i)=\psi(m_i)
        \]となる.
        \item \(P\)を射影加群とする.
        自由加群\(A^{\oplus I}\)と
        全射\(p\colon A^{\oplus I}\twoheadrightarrow P\)が存在する.
        実際,\(I= P\)として,
        \(p\)を\(p((a_x)_{x\in P})=\sum_{x\in P}a_xx\)と
        定めればよい.\(Q=\Ker p\)とすると,\[
            0\to Q\hookrightarrow A^{\oplus I}\twoheadrightarrow P\to0
        \]は完全列である.
        このとき,\(P\)が射影加群であることから,
        \(\id_P\)に対して,\(u\colon P\to A^{\oplus I}\)で
        \[p_\ast(u)=p\circ u=\id_P\]となる者が存在する.
        したがって,上の完全列は分裂し,
        \(A^{\oplus I}\cong P\oplus Q\)となる.
        \item 
    \end{enumerate}
\end{proof}

\clearpage
\section{非特性変形補題}
\begin{leftbar}
\begin{PRP}[{\cite[Prop. 2.5.1]{KS90}}]\label{PRP2.5.1}
    \(X\)を位相空間とし,\(Z\)を部分空間とする.
    \(F\)を\(X\)上の層とし,自然な射
    \[
        \psi\colon\indlim[U\in I_Z]\Gamma(U;F)\to\Gamma(Z;F)
    \]を考える.

    (i) 
    \(\psi\)は単射である.

    (ii)
    \(X\)がハウスドルフで\(Z\)がコンパクトならば,\(\psi\)は同型である.
\end{PRP}
\end{leftbar}
\begin{leftbar}
    \begin{PRP}[{\cite[Prop. 1.12.4]{KS90}}]\label{PRP1.12.4}
        \[
            \phi_k\colon H^k(\indlim X)\to\prolim H^k(X_n)
        \]
        について,\(H^{i-1}(X_n)\)がML条件を満たすならば,\(\phi_k\)は一対一対応である.
    \end{PRP}
\end{leftbar}
    
\begin{leftbar}
\begin{PRP}[{\cite[Prop. 1.12.6]{KS90}}]\label{PRP1.12.6}
    \((X_s,\rho_{s,t})\)を実数を添字とする射影系とする.
    \[
        \lambda_s\colon X_s\to\prolim[r<s]X_r,
        \quad
        \mu_s\colon\indlim[t>s]X_t\to X_s
    \]
    がどちらも単射(全射)ならば,すべての実数\(s_0\leqq s_1\)に対し,
    \(\rho_{s_0,s_1}\colon X_{s_1}\to X_{s_0}\)は単射(全射)となる.
\end{PRP}
\end{leftbar}

\begin{leftbar}    
\begin{PRP}[{\cite[Prop. 2.7.2, 非特性変形補題]{KS90}}]
    \(X\)をハウスドルフ空間とし,\(F\in\Domp^+(\zz_X)\)とする.
    また,\((U_t)_{t\in\rr}\)を\(X\)の開集合の族で次の条件(i)--(iii)をみたすものとする.
    \begin{enumerate}[(i)]
        \item 任意の実数\(t\)に対し,\(\bigcup_{s<t}U_s=U_t\)が成り立つ.
        \item 任意の実数\(s\leqq t\)に対し,\(\overline{U_t-U_s}\cap\supp F\)はコンパクト集合である.
        \item 実数\(s\)に対して\(Z_s=\bigcap_{t>s}\overline{U_t-U_s}\)とおくとき,
        任意の実数\(s\leqq t\)と任意の点\(x\in Z_s-U_t\)に対して\(\left(\RG_{X-U_t}(F)\right)_x=0\)が成り立つ.
    \end{enumerate}
    このとき,任意の実数\(t\)に対して,次の同型が成り立つ.
    \[
        \RG\left(\bigcup_{s\in\rr}U_s;F\right)\overset{\sim}{\longrightarrow}\RG(U_t;F)
    \]
\end{PRP}
\end{leftbar}
\begin{proof}
    次の条件を考える.
    \begin{align*}
        (a)_k^s\colon\quad \indlim[t>s]H^k(U_t;F)\simar H^k(U_s;F)\\
        (b)_k^t\colon\quad \prolim[s<t]H^k(U_s;F)\simra H^k(U_t;F)        
    \end{align*}
    任意の実数\(s\)と任意の整数\(k\)に対して\((a)_k^s\)が,
    任意の実数\(t\)と任意の整数\(k<k_0\)に対して\((b)_k^t\)が成り立つとする.
    このとき,\(k_0\)に対し,\((b)_{k_0}^t\)が成り立つことを示す.
    命題\ref{PRP1.12.6}より,
    (\((a)_k^s\)の方が\(\mu_s\),\((b)_k^t\)の方が\(\lambda_t\)として)
    各次数\(k<k_0\)と各実数\(s\leqq t\)に対し,
    \begin{equation}
        H^k(U_t;F)\simar H^k(U_s;F)
    \end{equation}
    が成り立つ.
    このとき,
    \(t\)を固定して,
    射影系\(\left(
        H^{k_0-1}\left(U_{t-\frac{1}{n}};F\right)
    \right)_{n\in\nn}\)を考えると,これはML条件をみたす.
    \begin{center}
        \begin{minipage}{.9\textwidth}
            \begin{redleftbar}
                \(\because)\) 
                任意の\(n\in\nn\)に対し,
                \[
                    \rho_{n,p}
                    \left(
                        H^{k_0-1}\left(U_{t-\frac{1}{p}};F\right)
                        \to 
                        H^{k_0-1}\left(U_{t-\frac{1}{n}};F\right)
                    \right)
                \]
                はすべて同形なので,当然安定.
            \end{redleftbar}
        \end{minipage}
     \end{center}        
    よって,命題\ref{PRP1.12.4}より\((b)_{k_0}^t\)が従う.
    \(k\)に関する帰納法により,
    どの\(t\in\rr\)と\(k\in\zz\)に対しても\((b)_{k}^t\)が成り立つ.
    \begin{center}
        命題2.7.1を
        \(\left(H^{k}\left(U_{n};F\right)\right)_{n\in\nn}\)に
        用いると←わかってない
    \end{center}
    \(k\)に関する帰納法で,定理の結論
    \[
        \RG\left(\bigcup_{s\in\rr}U_s;F\right)\overset{\sim}{\longrightarrow}\RG(U_t;F)
    \]
    が従う.

    \subparagraph*{\((a)_k^s\)の証明}
    \(X\)を\(\supp{F}\)におきかえて,どの実数\(s\leqq t\)に対しても
    \(\overline{U_t-U_s}\)はコンパクトとしてよい.
    次のd.t.を考える\footnote{
        \cite[(2.6.32)]{KS90}のd.t.
        \[
            \RG_{Z'}(F)\to\RG_{Z}(F)\to\RG_{Z-Z'}(F)\underset{+1}{\longrightarrow}
        \]
        を用いる.但し,\(Z\)は\(X\)の局所閉集合,\(Z'\)は\(Z\)の閉集合である.
    }.
    \[
        \RG_{(X-U_t)}(F)\rvert_{Z_s}\to
        \RG_{(X-U_s)}(F)\rvert_{Z_s}\to
        \RG_{(U_t-U_s)}(F)\rvert_{Z_s}
        \overset{+1}{\longrightarrow}.
    \]
    仮定(iii)より,左と真ん中の2つは0なので,
    d.t.の性質から,\(\RG_{(U_t-U_s)}(F)\rvert_{Z_s}=0\)となる.
    したがって,任意の\(k\in\zz\)と\(t\geqq s\)に対し,
    \begin{align*}
        0&=H^k(Z_s;\RG_{(U_t-U_s)}(F))\\
        &=\indlim[U\supset Z_s]H^k(U\cap U_t;\RG_{X-U_s}(F))
    \end{align*}
    となる.
    \begin{center}
        \begin{minipage}{.9\textwidth}
        \begin{redleftbar}
            \(\RG_{U_t-U_s}(F)\)は\(X\)上の層で,
            それを\(Z_s\)に制限した\(\RG_{U_t-U_s}(F)\rvert_{Z_s}\)は\(Z_s\)上の層である.
            \(Z_s\)での大域切断\(\RG(Z_s;\RG_{U_t-U_s}(F)\rvert_{Z_s})\)のコホモロジーをとっているので,
            \cite[Notations 2.6.8]{KS90}の2番目の記号を用いることになる.

            \(Z_s\)はハウスドルフ空間\(X\)の
            コンパクト集合\(\overline{U_t-U_s}\)の
            共通部分として表されているので,コンパクトである(\(X\)の置き換えがここに効いている).
            したがって,\cite[Remark 2.6.9 (ii)]{KS90}の場合に当てはまり,
            そこでの記号を用いて書くと
            \[H^j(Z;F)\simeq\indlim[U\in I_Z]H^j(U;F)\]が成り立つ.
            これが上の式の2つ目の変形.
            詳しく書くと,
            \begin{align*}
                H^k(Z_s;\RG_{U_t-U_s}(F))
                &=\indlim[U\in I_Z]H^k(U;\RG_{U_t-U_s}(F))\\
                &=\indlim[U\in I_Z]H^k(U\cap U_t;\RG_{X-U_s}(F))
            \end{align*}
            ここで,2つ目の変形は次のように考える.
            \(U_t-U_s\)に台を持つ層の\(U\)上の切断は\(U\cap U_t\)上で切断を考えても同じ.
            台の方も,\(U\)が\(Z_s\)に十分近ければ\(X-U_s\)で考えても同じ.
        \end{redleftbar}
    \end{minipage}
        
    \end{center}
\end{proof}
\backmatter
%===============================================
% 参考文献スペース
%===============================================
\begin{thebibliography}{20} 
    \bibitem[KS90]{KS90} Masaki Kashiwara, Pierre Schapira, 
        \textit{Sheaves on Manifolds}, 
        Grundlehren der Mathematischen Wissenschaften, 292, Springer, 1990.
        \bibitem[KS06]{KS06} Masaki Kashiwara, Pierre Schapira, 
        \textit{Categories and Sheaves}, 
        Grundlehren der Mathematischen Wissenschaften, 332, Springer, 2006.
        \bibitem[Sh16]{Sh16} 志甫淳, 層とホモロジー代数, 共立出版, 2016.
    %\bibitem[Og02]{Og02} 小木曽啓示, 代数曲線論, 朝倉書店, 2022.
\end{thebibliography}

%===============================================


\end{document}
