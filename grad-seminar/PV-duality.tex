%=====================================
%   guest-oshiba.tex
%   2023年度方程 大柴寿浩
%   2023/11/23 執筆開始
%=====================================

% -----------------------
% preamble
% -----------------------
% ここから本文 (\begin{document}) までの
% ソースコードに変更を加えた場合は
% 編集者まで連絡してください. 
% Don't change preamble code yourself. 
% If you add something
% (usepackage, newtheorem, newcommand, renewcommand),
% please tell it 
% to the editor of institutional paper of RUMS.

% ------------------------
% documentclass
% ------------------------
\documentclass[11pt, a4paper, dvipdfmx]{jsarticle}

% ------------------------
% usepackage
% ------------------------
\usepackage{algorithm}
\usepackage{algorithmic}
\usepackage{amscd}
\usepackage{amsfonts}
\usepackage{amsmath}
\usepackage[psamsfonts]{amssymb}
\usepackage{amsthm}
\usepackage{ascmac}
\usepackage{color}
\usepackage{enumerate}
\usepackage{fancybox}
\usepackage[stable]{footmisc}
\usepackage{graphicx}
\usepackage{listings}
\usepackage{mathrsfs}
\usepackage{mathtools}
\usepackage{otf}
\usepackage{pifont}
\usepackage{proof}
\usepackage{subfigure}
\usepackage{tikz}
\usepackage{verbatim}
\usepackage[all]{xy}

\usetikzlibrary{cd}



% ================================
% パッケージを追加する場合のスペース 

%=================================


% --------------------------
% theoremstyle
% --------------------------
\theoremstyle{definition}

% --------------------------
% newtheoem
% --------------------------

% 日本語で定理, 命題, 証明などを番号付きで用いるためのコマンドです. 
% If you want to use theorem environment in Japanece, 
% you can use these code. 
% Attention!
% All theorem enivironment numbers depend on 
% only section numbers.
\newtheorem{Axiom}{公理}[section]
\newtheorem{Definition}[Axiom]{定義}
\newtheorem{Theorem}[Axiom]{定理}
\newtheorem{Proposition}[Axiom]{命題}
\newtheorem{Lemma}[Axiom]{補題}
\newtheorem{Corollary}[Axiom]{系}
\newtheorem{Example}[Axiom]{例}
\newtheorem{Claim}[Axiom]{主張}
\newtheorem{Property}[Axiom]{性質}
\newtheorem{Attention}[Axiom]{注意}
\newtheorem{Question}[Axiom]{問}
\newtheorem{Problem}[Axiom]{問題}
\newtheorem{Consideration}[Axiom]{考察}
\newtheorem{Alert}[Axiom]{警告}
\newtheorem{Fact}[Axiom]{事実}


% 日本語で定理, 命題, 証明などを番号なしで用いるためのコマンドです. 
% If you want to use theorem environment with no number in Japanese, You can use these code.
\newtheorem*{Axiom*}{公理}
\newtheorem*{Definition*}{定義}
\newtheorem*{Theorem*}{定理}
\newtheorem*{Proposition*}{命題}
\newtheorem*{Lemma*}{補題}
\newtheorem*{Example*}{例}
\newtheorem*{Corollary*}{系}
\newtheorem*{Claim*}{主張}
\newtheorem*{Property*}{性質}
\newtheorem*{Attention*}{注意}
\newtheorem*{Question*}{問}
\newtheorem*{Problem*}{問題}
\newtheorem*{Consideration*}{考察}
\newtheorem*{Alert*}{警告}
\newtheorem{Fact*}{事実}


% 英語で定理, 命題, 証明などを番号付きで用いるためのコマンドです. 
% If you want to use theorem environment in English, You can use these code.
%all theorem enivironment number depend on only section number.
\newtheorem{Axiom+}{Axiom}[section]
\newtheorem{Definition+}[Axiom+]{Definition}
\newtheorem{Theorem+}[Axiom+]{Theorem}
\newtheorem{Proposition+}[Axiom+]{Proposition}
\newtheorem{Lemma+}[Axiom+]{Lemma}
\newtheorem{Example+}[Axiom+]{Example}
\newtheorem{Corollary+}[Axiom+]{Corollary}
\newtheorem{Claim+}[Axiom+]{Claim}
\newtheorem{Property+}[Axiom+]{Property}
\newtheorem{Attention+}[Axiom+]{Attention}
\newtheorem{Question+}[Axiom+]{Question}
\newtheorem{Problem+}[Axiom+]{Problem}
\newtheorem{Consideration+}[Axiom+]{Consideration}
\newtheorem{Alert+}{Alert}
\newtheorem{Fact+}[Axiom+]{Fact}
\newtheorem{Remark+}[Axiom+]{Remark}

% ----------------------------
% commmand
% ----------------------------
% 執筆に便利なコマンド集です. 
% コマンドを追加する場合は下のスペースへ. 

% 集合の記号 (黒板文字)
\newcommand{\NN}{\mathbb{N}}
\newcommand{\ZZ}{\mathbb{Z}}
\newcommand{\QQ}{\mathbb{Q}}
\newcommand{\RR}{\mathbb{R}}
\newcommand{\CC}{\mathbb{C}}
\newcommand{\PP}{\mathbb{P}}
\newcommand{\KK}{\mathbb{K}}


% 集合の記号 (太文字)
\newcommand{\nn}{\mathbf{N}}
\newcommand{\zz}{\mathbf{Z}}
\newcommand{\qq}{\mathbf{Q}}
\newcommand{\rr}{\mathbf{R}}
\newcommand{\cc}{\mathbf{C}}
\newcommand{\pp}{\mathbf{P}}
\newcommand{\kk}{\mathbf{K}}

% 特殊な写像の記号
\newcommand{\ev}{\mathop{\mathrm{ev}}\nolimits} % 値写像
\newcommand{\pr}{\mathop{\mathrm{pr}}\nolimits} % 射影

% スクリプト体にするコマンド
%   例えば {\mcal C} のように用いる
\newcommand{\mcal}{\mathcal}

% 花文字にするコマンド 
%   例えば {\h C} のように用いる
\newcommand{\h}{\mathscr}

% ヒルベルト空間などの記号
\newcommand{\F}{\mcal{F}}
\newcommand{\X}{\mcal{X}}
\newcommand{\Y}{\mcal{Y}}
\newcommand{\Hil}{\mcal{H}}
\newcommand{\RKHS}{\Hil_{k}}
\newcommand{\Loss}{\mcal{L}_{D}}
\newcommand{\MLsp}{(\X, \Y, D, \Hil, \Loss)}

% 偏微分作用素の記号
\newcommand{\p}{\partial}

% 角カッコの記号 (内積は下にマクロがあります)
\newcommand{\lan}{\langle}
\newcommand{\ran}{\rangle}



% 圏の記号など
\newcommand{\Set}{{\bf Set}}
\newcommand{\Vect}{{\bf Vect}}
\newcommand{\FDVect}{{\bf FDVect}}
%\newcommand{\Ring}{{\bf Ring}}
\newcommand{\Ab}{{\bf Ab}}
\newcommand{\Mod}{\mathop{\mathrm{Mod}}\nolimits}
\newcommand{\CGA}{{\bf CGA}}
\newcommand{\GVect}{{\bf GVect}}
\newcommand{\Lie}{{\bf Lie}}
\newcommand{\dLie}{{\bf Liec}}



% 射の集合など
\newcommand{\Map}{\mathop{\mathrm{Map}}\nolimits} % 写像の集合
\newcommand{\Hom}{\mathop{\mathrm{Hom}}\nolimits} % 射集合
\newcommand{\End}{\mathop{\mathrm{End}}\nolimits} % 自己準同型の集合
\newcommand{\Aut}{\mathop{\mathrm{Aut}}\nolimits} % 自己同型の集合
\newcommand{\Mor}{\mathop{\mathrm{Mor}}\nolimits} % 射集合
\newcommand{\Ker}{\mathop{\mathrm{Ker}}\nolimits} % 核
\newcommand{\Img}{\mathop{\mathrm{Im}}\nolimits} % 像
\newcommand{\Cok}{\mathop{\mathrm{Coker}}\nolimits} % 余核
\newcommand{\Cim}{\mathop{\mathrm{Coim}}\nolimits} % 余像

% その他便利なコマンド
\newcommand{\dip}{\displaystyle} % 本文中で数式モード
\newcommand{\e}{\varepsilon} % イプシロン
\newcommand{\dl}{\delta} % デルタ
\newcommand{\pphi}{\varphi} % ファイ
\newcommand{\ti}{\tilde} % チルダ
\newcommand{\pal}{\parallel} % 平行
\newcommand{\op}{{\rm op}} % 双対を取る記号
\newcommand{\lcm}{\mathop{\mathrm{lcm}}\nolimits} % 最小公倍数の記号
\newcommand{\Probsp}{(\Omega, \F, \P)} 
\newcommand{\argmax}{\mathop{\rm arg~max}\limits}
\newcommand{\argmin}{\mathop{\rm arg~min}\limits}





% ================================
% コマンドを追加する場合のスペース 
%\newcommand{\OO}{\mcal{O}}



\renewcommand\proofname{\bf 証明} % 証明
\numberwithin{equation}{section}
\newcommand{\cTop}{\textsf{Top}}
%\newcommand{\cOpen}{\textsf{Open}}
\newcommand{\Op}{\mathop{\textsf{Open}}\nolimits}
\newcommand{\Ob}{\mathop{\textrm{Ob}}\nolimits}
\newcommand{\id}{\mathop{\mathrm{id}}\nolimits}
\newcommand{\pt}{\mathop{\mathrm{pt}}\nolimits}
\newcommand{\res}{\mathop{\rho}\nolimits}
\newcommand{\A}{\mcal{A}}
\newcommand{\B}{\mcal{B}}
\newcommand{\C}{\mcal{C}}
\newcommand{\D}{\mcal{D}}
\newcommand{\E}{\mcal{E}}
\newcommand{\G}{\mcal{G}}
%\newcommand{\H}{\mcal{H}}
\newcommand{\I}{\mcal{I}}
\newcommand{\J}{\mcal{J}}
\newcommand{\OO}{\mcal{O}}
\newcommand{\Ring}{\mathop{\textsf{Ring}}\nolimits}
\newcommand{\cAb}{\mathop{\textsf{Ab}}\nolimits}
%\newcommand{\Ker}{\mathop{\mathrm{Ker}}\nolimits}
\newcommand{\im}{\mathop{\mathrm{Im}}\nolimits}
\newcommand{\Coker}{\mathop{\mathrm{Coker}}\nolimits}
\newcommand{\Coim}{\mathop{\mathrm{Coim}}\nolimits}
\newcommand{\rank}{\mathop{\mathrm{rank}}\nolimits}
\newcommand{\Ht}{\mathop{\mathrm{Ht}}\nolimits}
\newcommand{\supp}{\mathop{\mathrm{supp}}\nolimits}
\newcommand{\colim}{\mathop{\mathrm{colim}}}
\newcommand{\Tor}{\mathop{\mathrm{Tor}}\nolimits}

\newcommand{\cat}{\mathscr{C}}

\newcommand{\scA}{\mathscr{A}}
\newcommand{\scB}{\mathscr{B}}
\newcommand{\scC}{\mathscr{C}}
\newcommand{\scD}{\mathscr{D}}
\newcommand{\scE}{\mathscr{E}}
\newcommand{\scF}{\mathscr{F}}

\newcommand{\ibA}{\mathop{\text{\textit{\textbf{A}}}}}
\newcommand{\ibB}{\mathop{\text{\textit{\textbf{B}}}}}
\newcommand{\ibC}{\mathop{\text{\textit{\textbf{C}}}}}
\newcommand{\ibD}{\mathop{\text{\textit{\textbf{D}}}}}
\newcommand{\ibE}{\mathop{\text{\textit{\textbf{E}}}}}
\newcommand{\ibF}{\mathop{\text{\textit{\textbf{F}}}}}
\newcommand{\ibG}{\mathop{\text{\textit{\textbf{G}}}}}
\newcommand{\ibH}{\mathop{\text{\textit{\textbf{H}}}}}
\newcommand{\ibI}{\mathop{\text{\textit{\textbf{I}}}}}
\newcommand{\ibJ}{\mathop{\text{\textit{\textbf{J}}}}}
\newcommand{\ibK}{\mathop{\text{\textit{\textbf{K}}}}}
\newcommand{\ibL}{\mathop{\text{\textit{\textbf{L}}}}}
\newcommand{\ibM}{\mathop{\text{\textit{\textbf{M}}}}}
\newcommand{\ibN}{\mathop{\text{\textit{\textbf{N}}}}}
\newcommand{\ibO}{\mathop{\text{\textit{\textbf{O}}}}}
\newcommand{\ibP}{\mathop{\text{\textit{\textbf{P}}}}}
\newcommand{\ibQ}{\mathop{\text{\textit{\textbf{Q}}}}}
\newcommand{\ibR}{\mathop{\text{\textit{\textbf{R}}}}}
\newcommand{\ibS}{\mathop{\text{\textit{\textbf{S}}}}}
\newcommand{\ibT}{\mathop{\text{\textit{\textbf{T}}}}}
\newcommand{\ibU}{\mathop{\text{\textit{\textbf{U}}}}}
\newcommand{\ibV}{\mathop{\text{\textit{\textbf{V}}}}}
\newcommand{\ibW}{\mathop{\text{\textit{\textbf{W}}}}}
\newcommand{\ibX}{\mathop{\text{\textit{\textbf{X}}}}}
\newcommand{\ibY}{\mathop{\text{\textit{\textbf{Y}}}}}
\newcommand{\ibZ}{\mathop{\text{\textit{\textbf{Z}}}}}

\newcommand{\ibx}{\mathop{\text{\textit{\textbf{x}}}}}

%\newcommand{\Comp}{\mathop{\mathrm{C}}\nolimits}
%\newcommand{\Komp}{\mathop{\mathrm{K}}\nolimits}
%\newcommand{\Domp}{\mathop{\mathsf{D}}\nolimits}%複体のホモトピー圏
\newcommand{\Comp}{\mathrm{C}}
\newcommand{\Komp}{\mathrm{K}}
\newcommand{\Domp}{\mathsf{D}}%複体のホモトピー圏

\newcommand{\CCat}{\Comp(\cat)}
\newcommand{\KCat}{\Komp(\cat)}
\newcommand{\DCat}{\Domp(\cat)}%圏Cの複体のホモトピー圏
\newcommand{\HOM}{\mathop{\mathscr{H}\hspace{-2pt}om}\nolimits}%内部Hom
\newcommand{\RHOM}{\mathop{\mathrm{R}\hspace{-1.5pt}\HOM}\nolimits}

\newcommand{\muS}{\mathop{\mathrm{SS}}\nolimits}
\newcommand{\RG}{\mathop{\mathrm{R}\hspace{-0pt}\Gamma}\nolimits}
\newcommand{\RHom}{\mathop{\mathrm{R}\hspace{-1.5pt}\Hom}\nolimits}
\newcommand{\Rder}{\mathop{\mathrm{R}}\nolimits}

\newcommand{\simar}{\mathrel{\overset{\sim}{\longrightarrow}}}%内部Hom
\newcommand{\simra}{\mathrel{\overset{\sim}{\longleftarrow}}}%内部Hom

\newcommand{\hocolim}{{\mathrm{hocolim}}}
\newcommand{\indlim}[1][]{\mathop{\varinjlim}\limits_{#1}}
\newcommand{\sindlim}[1][]{\smash{\mathop{\varinjlim}\limits_{#1}}\,}
\newcommand{\Pro}{\mathrm{Pro}}
\newcommand{\Ind}{\mathrm{Ind}}
\newcommand{\prolim}[1][]{\mathop{\varprojlim}\limits_{#1}}
\newcommand{\sprolim}[1][]{\smash{\mathop{\varprojlim}\limits_{#1}}\,}

\newcommand{\Sh}{\mathrm{Sh}}
\newcommand{\PSh}{\mathrm{PSh}}














%================================================
% 自前の定理環境
%   https://mathlandscape.com/latex-amsthm/
% を参考にした
\newtheoremstyle{mystyle}%   % スタイル名
    {5pt}%                   % 上部スペース
    {5pt}%                   % 下部スペース
    {}%              % 本文フォント
    {}%                  % 1行目のインデント量
    {\bfseries}%                      % 見出しフォント
    {.}%                     % 見出し後の句読点
    {12pt}%                     % 見出し後のスペース
    {\thmname{#1}\thmnumber{ #2}\thmnote{{\hspace{2pt}\normalfont (#3)}}}% % 見出しの書式

\theoremstyle{mystyle}
\newtheorem{AXM}{公理}[section]
\newtheorem{DFN}[Axiom]{定義}
\newtheorem{THM}[Axiom]{定理}
\newtheorem*{THM*}{定理}
\newtheorem{PRP}[Axiom]{命題}
\newtheorem{LMM}[Axiom]{補題}
\newtheorem{CRL}[Axiom]{系}
\newtheorem{EG}[Axiom]{例}
\newtheorem{CNV}[Axiom]{規約}
\newtheorem{CMT}[Axiom]{コメント}


% 定理環境ここまで
%====================================================

% =================================





% ---------------------------
% new definition macro
% ---------------------------
% 便利なマクロ集です

% 内積のマクロ
%   例えば \inner<\pphi | \psi> のように用いる
\def\inner<#1>{\langle #1 \rangle}

% ================================
% マクロを追加する場合のスペース 

%=================================





% ----------------------------
% documenet 
% ----------------------------
% 以下, 本文の執筆スペースです. 
% Your main code must be written between 
% begin document and end document.
% ---------------------------

\title{Poincar\'e-Verdier双対性}
\author{大柴寿浩
%\thanks{北海道大学大学院理学院数学専攻修士1年}
}
\date{}
\begin{document}
\maketitle
\section*{はじめに}
%本稿では,筆者の専門である代数解析学について述べる.
\(X\), \(Y\)を局所コンパクト空間とし,
\(f\colon{Y}\to{X}\)を連続写像とする.
\(A\)を大域次元が有限な可換環とする.
\(F\in\Domp^+(A_X)\), \(G\in\Domp^+(A_Y)\)とする.

\(\mathrm{R}f_!\colon\Domp^+(A_Y)\to\Domp^+(A_X)\)の右随伴関手
\(f^!\colon\Domp^+(A_X)\to\Domp^+(A_Y)\)を構成するのが目標.

\section{例}
随伴を仮定した場合,層を具体的に設定するとどのような結果が従うか見てみる.
\(X=\{\pt\}\), \(F=A_{\{\pt\}}\), \(G=A_Y\)のとき,
\begin{align*}
    \RHom_{A_X}(\mathrm{R}{a_Y}_!A_Y,A)
    &=\RHom_{A_Y}(A_Y,a_Y^!A)\\
    &=\RG\left(Y;\RHOM_{A_Y}(A_Y,a_Y^!A)\right)\\
    &=\RG\left(Y;a_Y^!A\right)
\end{align*}
である.\(\omega_Y\colon a_Y^!A\)とおく.
開集合\(U\subset{Y}\)に対し,
\begin{align*}
    \RG(U;\omega_Y)
    &\cong\RG\left(U;\RHOM_{A}(A_U,a_Y^!A)\right)\\
    &\cong\RHom_{A_U}({A_U},A)
\end{align*}
なので,
\begin{align*}
    \RHom_{A_U}({A_U},a_Y^!A)
    \cong\RHom_A(\RG_c(Y;A_U),A)\\
    \cong\RHom_A(\RG_c(U;A_U),A).
\end{align*}
ここで,\(U\)が\(\rr^n\)と同相であるとすると,
\[\RG_c(U;A_U)\cong A[-n]\]
したがって,\[
    \cong\RHom_A(\RG_c(U;A_U),A)\cong A[n]
\]
である.
つまり,\(\omega_Y=a_Y^!A\)は
\[
    H^k(\omega_Y)=\begin{cases}
        0 & (k\neq -n)\\
        \rank =1\text{の局所定数層} & (k=-n)
    \end{cases}
\]
\section{構成}
\subsection{構成}

\(X\), \(Y\)を局所コンパクト空間とし,
\(f\colon{Y}\to{X}\)を連続写像とする.
\(A\)を大域次元が有限な可換環とする.
\(F\in\Domp^+(A_X)\), \(G\in\Domp^+(A_Y)\)とする.

\(\Rder{f_!}\colon\Domp^+(A_Y)\to\Domp^+(A_X)\)の右随伴関手
\(f^!\colon\Domp^+(A_X)\to\Domp^+(A_Y)\)を構成する.
まず,開集合\(V\subset Y\)に対し,\(f^!{F}\)の\(V\)上の切断
に関する条件を見てみる.
\[
    \RG(V;f^!{F})
    =\RHom_{A_Y}(A_V,f^!F)
    =\RHom_{A_X}(\Rder{f_!}A_V,F)
\]
となることから,\(f^!F\)は
\(V\mapsto\RHom_{A_X}(\Rder{f_!}A_V,F)\)
という対応でなければならない.
\(\Rder{f_!}\)を計算するにはc柔軟分解
\(A_V\sim K\)を
取ればよく,さらに\(F\)が入射的であれば,
\[
    \RHom_{A_X}(\Rder{f_!}A_V,F)
    =\Hom_{A_X}({f_!}K_V,F)
\]
となって,結局
\[
    \RG(V;f^!{F})
    =\Hom_{A_X}({f_!}K_V,F)
\]
とできる.

\paragraph{\(f\)に関する仮定}
\begin{DFN}
    \(Y\)上の層\(G\)が\(f\)柔軟であるとは,各点\(x\in{X}\)に対し,
    \(G\rvert_{f^{-1}(x)}\)がc柔軟であることをいう.
\end{DFN}
\(G\)が\(f\)柔軟であることと,
任意の開部分集合\(V\subset{Y}\)と\(j\ne0\)に対し,
\(\Rder^jf_!G_V=0\)となることと同値である.

次を仮定する.
\begin{equation}\label{eq:fdim}
    f_!\colon\Mod(\zz_Y)\to\Mod(\zz_X)\text{のコホモロジー次元は有限である.}
\end{equation}
つまり,整数\(r\geqq0\)で,全ての\(j>r\)に対し\(\Rder^jf_!=0\)となるものが存在する.
\eqref{eq:fdim}は次の条件と同値である.
\begin{equation}\label{eq:fdim1}
\begin{cases}
    \text{任意の}G\in\Sh(Y)\text{に対し,完全列}\\
    0\to{G}\to{G^0}\to\dots\to{G^r}\to0\\
    \text{で,どの}G^j\text{も\(f\)柔軟であるものが存在する.}
\end{cases}    
\end{equation}
\begin{equation}\label{eq:fdim2}%\tag{3.1.4\('\)}
\begin{cases}
    \text{完全列}
    {G^0}\to\dots\to{G^r}\to0\\
    \text{において,\(j<r\)に対し\(G^j\)が\(f\)柔軟ならば,}\\
    \text{\(G^r\)が\(f\)柔軟となる.}
\end{cases}
\end{equation}
\(f_!\)のコホモロジー次元が\(\leqq{r}\)となるのは,
任意の\(x\in{X}\)に対し,
\(\Gamma_c(f^{-1}(x);\cdot)\)の
コホモロジー次元が\(\leqq{r}\)となるときである.
実際,\(f_!\rvert_{f^{-1}(x)}F=\Gamma_c(f^{-1}(x);F)=0\)となるので.

\paragraph{構成}
以上の仮定は,
\[
    \RG(V;f^!{F})
    =\Hom_{A_X}({f_!}K_V,F)
\]
の構成をするためだった.
\(f_!K_V\)の分解をしたくて,
その長さが有限になるという仮定である.

さて,\(K\)を\(\zz_Y\)加群,\(F\)を\(A_X\)加群とする.
このとき,\(A\)加群の前層\(f^!_KF\)を次で定める.
\(V\in\Op(Y)\)に対し,
\[
    (f^!_KF)(V)
    \coloneqq
    \Hom_{A_X}\left(f_!\left(
        A_Y\mathop{\otimes}\limits_{\zz_Y}K_V
    \right),F\right)
\]
とする.
制限射は\(K_{V'}\to{K_V}\)から引き起こされるもの.

\begin{LMM}\label{LMM:PV-flat}
    \(K\)を平坦かつ\(f\)柔軟な\(\zz_Y\)加群とする.
    \begin{enumerate}[(i)]
        \item \(Y\)上の任意の層\(G\)に対し\(G\otimes_{\zz_Y}K\)は\(f\)柔軟である.
        \item \(G\mapsto f_!(G\otimes_{\zz_Y}K)\)は
        \(\Mod(\zz_Y)\)から\(\Mod(\zz_X)\)への完全関手である.
    \end{enumerate}
\end{LMM}
\begin{proof}
    (i) 
    \(Y\)上の任意の層\(G\)に対し,
    \cite[Prop.2.4.12]{KS90}の証明から,
    分解
    \[
        \to{G^{-r}}\to\dots\to{G^{0}}\to{G}\to0
    \]
    で,各\(G^{j}\)が\(\zz_V\)の直和となるものが存在する.

    \subparagraph*{復習:}
    \(G\in\Mod(\zz_Y)\)に対し,
    \[
        \mathfrak{S}\coloneqq\{(V,s);V\in\Op(Y),s\in\Gamma(V;G)\}
    \]とし,
    各\((V,s)\in\mathfrak{S}\)に対し,
    \(\zz_Y(V,s)\coloneqq\zz_V\in\Mod(\zz_Y)\)とおき,
    \(P\coloneqq\bigoplus_{(V,s)\in\mathfrak{S}}\zz_Y(V,s)\)とおく.
    \(\Hom_{\zz_Y}(\zz_V,G)\cong\Gamma(V;G)\)より,
    各\(\varphi\colon\zz_V\to{G}\)に対し,\(s\in G(V)\)がただ一つある.
    これにより全射\(P\twoheadrightarrow G\)が得られる.
    各\(y\in Y\)に対し,
    \(P_y=\bigoplus_{(V,s),\\y\in V}\zz\)である.
    \rightline{(復習終わり)}

    \(\Ker(P\to G)\)に対し,同様に\(\zz_Y\)の直和からの全射が構成できる.
    これを繰り返して,分解
    \[
        \to{G^{-r}}\to\dots\to{G^{0}}\to{G}\to0
    \]
    で,\(\zz_V\)の直和となるものが得られる.
    よって,どの\(j\)についても,
    \(G^j\otimes_{\zz_Y}K\)は\(f\)柔軟である.
    実際,\(
        (\bigoplus\zz_V)\otimes_{\zz_Y}K\cong 
        \bigoplus(\zz_V\otimes_{\zz_Y}K)\cong\bigoplus{K_V}
    \)であり,茎ごとにc柔軟なので.
    この分解に\(-\otimes_{\zz_Y}K\)をあてた
    \[
        \to{G^{-r}}\otimes_{\zz_Y}K\to\dots
        \to{G^{0}}\otimes_{\zz_Y}K\to{G}\otimes_{\zz_Y}K\to0
    \]
    は\(K\)が平坦なので完全である.
    したがって,\(r\)を十分大きくとれば,
    \(f_!\)のコホモロジー次元が有限であるという仮定から,
    \eqref{eq:fdim2}を用いて,
    \({G}\otimes_{\zz_Y}K\)も\(f\)柔軟であることが従う.

    (ii) 
    \(\Sh(Y)\)の完全列\[
        0\to G_1\to G_2\to G_3\to0
    \]に対し,
    \[
        0\to G_1\otimes_{\zz_Y}K\to G_2\otimes_{\zz_Y}K\to G_3\otimes_{\zz_Y}K\to0
    \]
    は\(f\)柔軟である.いま,
    \[
        0\to f_!(G_1\otimes_{\zz_Y}K)
        \to f_!(G_2\otimes_{\zz_Y}K)
        \to f_!(G_3\otimes_{\zz_Y}K)\to0
    \]
    のストーク
    \[\tag{\(\natural\)}\label{eq:stalk-nat}
        0\to (f_!(G_1\otimes_{\zz_Y}K))_x
        \to (f_!(G_2\otimes_{\zz_Y}K))_x
        \to (f_!(G_3\otimes_{\zz_Y}K))_x\to0
    \]
    を考えると,これらは,\(W=f^{-1}(x)\)とおくとき
    \[
        0\to \Gamma_c(W;(G_1\mathop{\otimes}\limits_{\zz_Y}K)\rvert_{W})
        \to \Gamma_c(W;(G_2\mathop{\otimes}\limits_{\zz_Y}K)\rvert_{W})
        \to \Gamma_c(W;(G_3\mathop{\otimes}\limits_{\zz_Y}K)\rvert_{W})
        \to0
    \]
    である.\(G_j\mathop{\otimes}\limits_{\zz_Y}K\)たちは
    \(f\)柔軟なので,これらは\(W\)でc柔軟であり,
    \(f_!\)が完全であることから,\eqref{eq:stalk-nat}は完全である.
    よって,もとの層の系列も完全である.
\end{proof}

\begin{LMM}
    \(K\)を平坦かつ\(f\)柔軟な\(\zz_Y\)加群とし,
    \(F\)を\(A_X\)入射加群とする.
    \begin{enumerate}[(i)]
        \item 前層\(f^!_KF\)は層である.
        \item \(G\in\Mod(A_Y)\)に関して関手的な自然同型\[
            \Hom_{A_X}(f_!(G\otimes_{\zz_Y}K),F)\overset{\sim}{\rightarrow}\Hom_{A_Y}(G,f^!_KF)
        \]が存在する.また,\(f^!_KF\)は\(A_Y\)入射加群である.
    \end{enumerate}
\end{LMM}
\begin{proof}
    (i) 
    \(-\otimes_{\zz_Y}K\)をたんに\(-\otimes K\)で表す.
    \(V\)を\(Y\)の開集合とし,\((V_j)_j\)を\(V\)の開被覆とする.
    \[
        \bigoplus_{j,k}A_{V_{j}\cap V_{k}}\to\bigoplus_{j}A_{V_j}\to A_V\to0
    \]は完全である.\footnote{
        開集合は小さい方から大きい方へ随伴射が生える.
        \(A_{jk}\to{A_{j},A_{k}}\)は
        それぞれ\(s_{jk}\mapsto s_{jk}\)と
        \(s_{jk}\mapsto -s_{jk}\)で定める.
        \(A_{V_j}\)の台は\(V\)に含まれるので,
        各切断の和の台も被覆の条件から\(V\)に含まれるので
        全射の方も成り立つ.
    }これに完全関手\(f_!(-\otimes{K})\)をあてて完全列
    \[
        \bigoplus_{j,k}f_!(A_{V_{j}\cap V_{k}}\otimes{K})
        \to\bigoplus_{j}f_!(A_{V_j}\otimes{K})
        \to f_!(A_V\otimes{K})\to0
    \]
    を得る.
    さらに左完全な反変関手\(\Hom_{A_X}(-,F)\)をあてて,完全列
    \[
        0
        \to 
        \Hom_{A_X}(f_!(A_V\otimes{K}),F)\to
        \Hom_{A_X}(\bigoplus_{j}f_!(A_{V_j}\otimes{K}),F)\to
        \Hom_{A_X}(\bigoplus_{j,k}f_!(A_{V_{j}\cap V_{k}}\otimes{K}),F)
    \]を得る.これは極限を交換すれば
    \[
        0\to
        f_K^!F(V)\to
        \prod_{j}f_K^!F(V_j)\to
        \prod_{j,k}f_K^!F(V_j\cap V_k)
    \]であり,\(f_K^!F\)が層であることが示された.

    (ii) 
    アーベル群の射\[
        \alpha(G)\colon\Hom_{A_X}(f_!(G\otimes{K}),F)
        \to\Hom_{A_Y}(G,f^!_KF)
    \]を定める.
    \(\phi\in\Hom_{A_X}(f_!(G\otimes{K}),F)\)とする.
    \(V\in\Op(Y)\)に対し,次\(A_X\)加群の射がある.
    \begin{align*}
        G(V)\otimes f_!(A_Y\otimes{K_V})
        &\underset{?}{\textcolor{red}{\longrightarrow}}
        f_!(G\otimes{K_V})\\
        &\underset{\text{adj.}}{\longrightarrow}f_!(G\otimes{K})\\
        &\underset{\phi}{\longrightarrow}F.
    \end{align*}
    よって,Hom・テンソル随伴から,
    \(G(V)\)から\(f^!_KF(V)=\Hom(f_!(A_Y\otimes{K_V}),F)\)への
    射を得る.
    この射は\(V\in\Op(Y)\)について関手的.
    よって,\(\alpha(G)(\phi)\in\Hom_{A_Y}(G,f^!_KF).\)

    \(\alpha(G)\)が同型であることを
    \begin{enumerate}[(a)]\setlength{\leftskip}{22pt}
        \item \(G=A_V\)の場合,
        \item \(G=\bigoplus_jA_{V_j}\)の場合,
        \item 一般の場合
    \end{enumerate}
    の3段階で示す.

    (a) 
    \(G=A_V\)のとき,
    \begin{align*}
        \Hom_{A_X}(f_!(G\otimes{K}),F)
        &\cong
        \Hom_{A_X}(f_!(A_V\otimes{K}),F)\\
        &\cong
        (f^!_KF)(V)\\
        &\cong
        \Hom_{A_Y}(A_V,f^!_KF)\\
        &=\Hom_{A_Y}(G,f^!_KF)
    \end{align*}
    なので同型.

    (b) 
    \(G=\bigoplus_jA_{V_j}\)のとき,
    \(\alpha(G)=\prod_j\alpha(A_{V_j})\)
    なので,(a) の結果からこれも同型.

    (c) 
    \(G\)が一般のとき,\cite[Prop.2.4.12]{KS90}より,完全列
    \[
        0\to G''\to G'\to G\to0
    \]で,\(G'\cong\bigoplus_jA_{V_j}\)となるものがある.
    (\(G''\)は全射\(G'\twoheadrightarrow G\)の核として得られる.)
    よって,(b)より\(\alpha(G')\)は同型.
    次の可換図式を考える.
    \begin{equation*}
        \vcenter{\xymatrix
        @C=26pt@R=26pt
        {
        0\ar[r]
        &
        \Hom(f_!(G\otimes{K}),F)
        \ar[r]
        \ar[d]^-{\alpha(G)}
        &
        \Hom(f_!(G'\otimes{K}),F)
        \ar[r]
        \ar[d]^-{\alpha(G')}
        &
        \Hom(f_!(G''\otimes{K}),F)
        \ar[d]^-{\alpha(G'')} 
        \\
        0\ar[r]
        &
        \Hom(G,f_K^!F)
        \ar[r]
        &
        \Hom(G',f_K^!F)
        \ar[r]
        &
        \Hom(G'',f_K^!F).
        }}
    \end{equation*}
    補題\ref{LMM:PV-flat}より,2つの行はどちらも完全.
    \(\alpha(G')\)は同型なので,左の可換性から\(\alpha(G)\)は単射.
    同じことを\(\alpha(G'')\)にやれば\(\alpha(G'')\)は単射.
    よって,五項補題から\(\alpha(G)\)は同型.(左に0の同型を追加する.)

    \(\Hom_{A_Y}(-,f_K^!F)\)が完全関手であることから,
    \(f_K^!F\)が入射加群であることもわかる.
\end{proof}

\(f_!\)のコホモロジー次元を\(\leqq r\)とする.

\begin{LMM}
    \(\zz_Y\)に対し,分解
    \(0\to \zz_Y\to K^0\to\dots\to K^r\to0\)
    で,各\(K^j\)が平坦かつ\(f\)柔軟な\(\zz_Y\)加群となるものが存在する.
\end{LMM}
\begin{proof}
    入射分解と同じように分解を構成する.

    \subparagraph*{入射分解の復習}
    \(\hat{Y}\coloneqq\left(Y,P(Y)\right)\)を\(Y\)に
    離散位相を入れた空間とし,
    \(p\colon \hat{Y}\to Y\)を自然な連続写像とする.
    \(F\)を\(\zz_Y\)加群とする.
    \(p^{-1}F \in\Mod(\zz_{\hat{Y}})\)に対し,
    入射加群\(I\in\Mod(\zz_{\hat{Y}})\)への
    単射\[0\to p^{-1}F\to I\]があったとする.
    この\(I\)と単射に対し,左完全関手\(p_\ast\)を適用すると,
    \(0\to F\to p_\ast{I}\)は\(\Mod(\zz_Y)\)における完全列である.
    (ここに\(p_\ast p^{-1}F=(\id_Y)_\ast(\id_Y)^{-1}=F\)である.)
    \(p_{\ast}\)は入射的対象を保つので,\(p_\ast{I}\)は入射加群である.

    最初の完全列を構成する.
    \(F'\in\Mod(\zz_{\hat{Y}})\)とすると,各点\(y\in Y\)に対し,
    \(\Mod(\zz_{\hat{Y},y})=\Mod(\zz)\)における完全列
    \[
        0\to F'_y\to I_y
    \]が存在する.(\(\Mod(\zz)\)は充分入射的対象を持つのだった.)
    単射の積は単射なので,\(Y\)全体で積を取った
    \[
        0\to \prod_{y\in Y}F'_y\to \prod_{y\in Y}I_y
    \]
    も完全である.
    したがって,\(\Mod(\zz_{\hat{Y}})\)での
    入射加群\(I=\prod_{y\in Y}I_y\)への単射\(F\to I\)が得られた.

    この操作を余核に対して繰り返し行えば,
    \(F\in\Mod(\zz_Y)\)の入射分解が得られる.
    
    {\rightline{(復習終わり)}}

    \(\zz_Y\)の入射分解を
    \[
        0\to \zz_Y\to K^0\to\cdots\to K^r\to0
    \]
    とする.入射加群は\(c\)柔軟であり,したがって\(f\)柔軟である.
    したがって,この分解は\(c\)柔軟分解である.
    %茎を見れば平坦.

    各項の平坦性を見る.
\end{proof}


%===============================================
% 参考文献スペース
%===============================================
\begin{thebibliography}{20} 
    \bibitem[KS90]{KS90} Masaki Kashiwara, Pierre Schapira, 
    \textit{Sheaves on Manifolds}, 
    Grundlehren der Mathematischen Wissenschaften, 292, Springer, 1990.
\end{thebibliography}

%===============================================


\end{document}
