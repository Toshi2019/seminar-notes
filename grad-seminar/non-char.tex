% -----------------------
% preamble
% -----------------------
% ここから本文 (\begin{document}) までの
% ソースコードに変更を加えた場合は
% 編集者まで連絡してください. 
% Don't change preamble code yourself. 
% If you add something
% (usepackage, newtheorem, newcommand, renewcommand),
% please tell it 
% to the editor of institutional paper of RUMS.

% ------------------------
% documentclass
% ------------------------
\documentclass[11pt, a4paper, dvipdfmx, leqno]{jsarticle}

% ------------------------
% usepackage
% ------------------------
\usepackage{algorithm}
\usepackage{algorithmic}
\usepackage{amscd}
\usepackage{amsfonts}
\usepackage{amsmath}
\usepackage[psamsfonts]{amssymb}
\usepackage{amsthm}
\usepackage{ascmac}
\usepackage{color}
\usepackage{enumerate}
\usepackage{fancybox}
\usepackage[stable]{footmisc}
\usepackage{graphicx}
\usepackage{listings}
\usepackage{mathrsfs}
\usepackage{mathtools}
\usepackage{otf}
\usepackage{pifont}
\usepackage{proof}
\usepackage{subfigure}
\usepackage{tikz}
\usepackage{verbatim}
\usepackage[all]{xy}
\usepackage{url}
\usetikzlibrary{cd}



% ================================
% パッケージを追加する場合のスペース 
%\usepackage{calligra}
\usepackage[dvipdfmx]{hyperref}
\usepackage{xcolor}
\definecolor{darkgreen}{rgb}{0,0.45,0} 
\definecolor{darkred}{rgb}{0.75,0,0}
\definecolor{darkblue}{rgb}{0,0,0.6} 
\hypersetup{
    colorlinks=true,
    citecolor=darkgreen,
    linkcolor=darkred,
    urlcolor=darkblue,
}
\usepackage{pxjahyper}
\usepackage[mathscr]{euscript}
\usepackage{layout}
\usepackage{framed}
\definecolor{lightgray}{rgb}{0.75,0.75,0.75}
\renewenvironment{leftbar}{%
  \def\FrameCommand{\textcolor{lightgray}{\vrule width 0.4zw} \hspace{10pt}}% 
  \MakeFramed {\advance\hsize-\width \FrameRestore}}%
{\endMakeFramed}
\newenvironment{redleftbar}{%
  \def\FrameCommand{\textcolor{red}{\vrule width 1pt} \hspace{10pt}}% 
  \MakeFramed {\advance\hsize-\width \FrameRestore}}%
 {\endMakeFramed}

%=================================


% --------------------------
% theoremstyle
% --------------------------
\theoremstyle{definition}

% --------------------------
% newtheoem
% --------------------------

% 日本語で定理, 命題, 証明などを番号付きで用いるためのコマンドです. 
% If you want to use theorem environment in Japanece, 
% you can use these code. 
% Attention!
% All theorem enivironment numbers depend on 
% only section numbers.
\newtheorem{Axiom}{公理}[section]
\newtheorem{Definition}[Axiom]{定義}
\newtheorem{Theorem}[Axiom]{定理}
\newtheorem{Proposition}[Axiom]{命題}
\newtheorem{Lemma}[Axiom]{補題}
\newtheorem{Corollary}[Axiom]{系}
\newtheorem{Example}[Axiom]{例}
\newtheorem{Claim}[Axiom]{主張}
\newtheorem{Property}[Axiom]{性質}
\newtheorem{Attention}[Axiom]{注意}
\newtheorem{Question}[Axiom]{問}
\newtheorem{Problem}[Axiom]{問題}
\newtheorem{Consideration}[Axiom]{考察}
\newtheorem{Alert}[Axiom]{警告}
\newtheorem{Fact}[Axiom]{事実}
\newtheorem{com}[Axiom]{コメント}


% 日本語で定理, 命題, 証明などを番号なしで用いるためのコマンドです. 
% If you want to use theorem environment with no number in Japanese, You can use these code.
\newtheorem*{Axiom*}{公理}
\newtheorem*{Definition*}{定義}
\newtheorem*{Theorem*}{定理}
\newtheorem*{Proposition*}{命題}
\newtheorem*{Lemma*}{補題}
\newtheorem*{Example*}{例}
\newtheorem*{Corollary*}{系}
\newtheorem*{Claim*}{主張}
\newtheorem*{Property*}{性質}
\newtheorem*{Attention*}{注意}
\newtheorem*{Question*}{問}
\newtheorem*{Problem*}{問題}
\newtheorem*{Consideration*}{考察}
\newtheorem*{Alert*}{警告}
\newtheorem*{Fact*}{事実}
\newtheorem*{com*}{コメント}



% 英語で定理, 命題, 証明などを番号付きで用いるためのコマンドです. 
% If you want to use theorem environment in English, You can use these code.
%all theorem enivironment number depend on only section number.
\newtheorem{Axiom+}{Axiom}[section]
\newtheorem{Definition+}[Axiom+]{Definition}
\newtheorem{Theorem+}[Axiom+]{Theorem}
\newtheorem{Proposition+}[Axiom+]{Proposition}
\newtheorem{Lemma+}[Axiom+]{Lemma}
\newtheorem{Example+}[Axiom+]{Example}
\newtheorem{Corollary+}[Axiom+]{Corollary}
\newtheorem{Claim+}[Axiom+]{Claim}
\newtheorem{Property+}[Axiom+]{Property}
\newtheorem{Attention+}[Axiom+]{Attention}
\newtheorem{Question+}[Axiom+]{Question}
\newtheorem{Problem+}[Axiom+]{Problem}
\newtheorem{Consideration+}[Axiom+]{Consideration}
\newtheorem{Alert+}{Alert}
\newtheorem{Fact+}[Axiom+]{Fact}
\newtheorem{Remark+}[Axiom+]{Remark}

% ----------------------------
% commmand
% ----------------------------
% 執筆に便利なコマンド集です. 
% コマンドを追加する場合は下のスペースへ. 

% 集合の記号 (黒板文字)
\newcommand{\NN}{\mathbb{N}}
\newcommand{\ZZ}{\mathbb{Z}}
\newcommand{\QQ}{\mathbb{Q}}
\newcommand{\RR}{\mathbb{R}}
\newcommand{\CC}{\mathbb{C}}
\newcommand{\PP}{\mathbb{P}}
\newcommand{\KK}{\mathbb{K}}


% 集合の記号 (太文字)
\newcommand{\nn}{\mathbf{N}}
\newcommand{\zz}{\mathbf{Z}}
\newcommand{\qq}{\mathbf{Q}}
\newcommand{\rr}{\mathbf{R}}
\newcommand{\cc}{\mathbf{C}}
\newcommand{\pp}{\mathbf{P}}
\newcommand{\kk}{\mathbf{K}}

% 特殊な写像の記号
\newcommand{\ev}{\mathop{\mathrm{ev}}\nolimits} % 値写像
\newcommand{\pr}{\mathop{\mathrm{pr}}\nolimits} % 射影

% スクリプト体にするコマンド
%   例えば {\mcal C} のように用いる
\newcommand{\mcal}{\mathcal}

% 花文字にするコマンド 
%   例えば {\h C} のように用いる
\newcommand{\h}{\mathscr}

% ヒルベルト空間などの記号
\newcommand{\F}{\mcal{F}}
\newcommand{\X}{\mcal{X}}
\newcommand{\Y}{\mcal{Y}}
\newcommand{\Hil}{\mcal{H}}
\newcommand{\RKHS}{\Hil_{k}}
\newcommand{\Loss}{\mcal{L}_{D}}
\newcommand{\MLsp}{(\X, \Y, D, \Hil, \Loss)}

% 偏微分作用素の記号
\newcommand{\p}{\partial}

% 角カッコの記号 (内積は下にマクロがあります)
\newcommand{\lan}{\langle}
\newcommand{\ran}{\rangle}



% 圏の記号など
\newcommand{\Set}{{\bf Set}}
\newcommand{\Vect}{{\bf Vect}}
\newcommand{\FDVect}{{\bf FDVect}}
\newcommand{\Mod}{\mathop{\mathrm{Mod}}\nolimits}
\newcommand{\CGA}{{\bf CGA}}
\newcommand{\GVect}{{\bf GVect}}
\newcommand{\Lie}{{\bf Lie}}
\newcommand{\dLie}{{\bf Liec}}



% 射の集合など
\newcommand{\Map}{\mathop{\mathrm{Map}}\nolimits}
\newcommand{\Hom}{\mathop{\mathrm{Hom}}\nolimits}
\newcommand{\End}{\mathop{\mathrm{End}}\nolimits}
\newcommand{\Aut}{\mathop{\mathrm{Aut}}\nolimits}
\newcommand{\Mor}{\mathop{\mathrm{Mor}}\nolimits}

% その他便利なコマンド
\newcommand{\dip}{\displaystyle} % 本文中で数式モード
\newcommand{\e}{\varepsilon} % イプシロン
\newcommand{\dl}{\delta} % デルタ
\newcommand{\pphi}{\varphi} % ファイ
\newcommand{\ti}{\tilde} % チルダ
\newcommand{\pal}{\parallel} % 平行
\newcommand{\op}{{\rm op}} % 双対を取る記号
\newcommand{\lcm}{\mathop{\mathrm{lcm}}\nolimits} % 最小公倍数の記号
\newcommand{\Probsp}{(\Omega, \F, \P)} 
\newcommand{\argmax}{\mathop{\rm arg~max}\limits}
\newcommand{\argmin}{\mathop{\rm arg~min}\limits}





% ================================
% コマンドを追加する場合のスペース 
\renewcommand\proofname{\bf 証明} % 証明
\numberwithin{equation}{section}
\newcommand{\cTop}{\textsf{Top}}
%\newcommand{\cOpen}{\textsf{Open}}
\newcommand{\Op}{\mathop{\textsf{Open}}\nolimits}
\newcommand{\Ob}{\mathop{\textrm{Ob}}\nolimits}
\newcommand{\id}{\mathop{\mathrm{id}}\nolimits}
\newcommand{\pt}{\mathop{\mathrm{pt}}\nolimits}
\newcommand{\res}{\mathop{\rho}\nolimits}
\newcommand{\A}{\mcal{A}}
\newcommand{\B}{\mcal{B}}
\newcommand{\C}{\mcal{C}}
\newcommand{\D}{\mcal{D}}
\newcommand{\E}{\mcal{E}}
\newcommand{\G}{\mcal{G}}
%\newcommand{\H}{\mcal{H}}
\newcommand{\I}{\mcal{I}}
\newcommand{\J}{\mcal{J}}
\newcommand{\OO}{\mcal{O}}
\newcommand{\Ring}{\mathop{\textsf{Ring}}\nolimits}
\newcommand{\cAb}{\mathop{\textsf{Ab}}\nolimits}
\newcommand{\Ker}{\mathop{\mathrm{Ker}}\nolimits}
\newcommand{\im}{\mathop{\mathrm{Im}}\nolimits}
\newcommand{\Coker}{\mathop{\mathrm{Coker}}\nolimits}
\newcommand{\Coim}{\mathop{\mathrm{Coim}}\nolimits}
\newcommand{\Ht}{\mathop{\mathrm{Ht}}\nolimits}
\newcommand{\supp}{\mathop{\mathrm{supp}}\nolimits}
\newcommand{\colim}{\mathop{\mathrm{colim}}}
\newcommand{\Tor}{\mathop{\mathrm{Tor}}\nolimits}

\newcommand{\cat}{\mathscr{C}}

\newcommand{\scA}{\mathscr{A}}
\newcommand{\scB}{\mathscr{B}}
\newcommand{\scC}{\mathscr{C}}
\newcommand{\scD}{\mathscr{D}}
\newcommand{\scE}{\mathscr{E}}
\newcommand{\scF}{\mathscr{F}}

\newcommand{\ibA}{\mathop{\text{\textit{\textbf{A}}}}}
\newcommand{\ibB}{\mathop{\text{\textit{\textbf{B}}}}}
\newcommand{\ibC}{\mathop{\text{\textit{\textbf{C}}}}}
\newcommand{\ibD}{\mathop{\text{\textit{\textbf{D}}}}}
\newcommand{\ibE}{\mathop{\text{\textit{\textbf{E}}}}}
\newcommand{\ibF}{\mathop{\text{\textit{\textbf{F}}}}}
\newcommand{\ibG}{\mathop{\text{\textit{\textbf{G}}}}}
\newcommand{\ibH}{\mathop{\text{\textit{\textbf{H}}}}}
\newcommand{\ibI}{\mathop{\text{\textit{\textbf{I}}}}}
\newcommand{\ibJ}{\mathop{\text{\textit{\textbf{J}}}}}
\newcommand{\ibK}{\mathop{\text{\textit{\textbf{K}}}}}
\newcommand{\ibL}{\mathop{\text{\textit{\textbf{L}}}}}
\newcommand{\ibM}{\mathop{\text{\textit{\textbf{M}}}}}
\newcommand{\ibN}{\mathop{\text{\textit{\textbf{N}}}}}
\newcommand{\ibO}{\mathop{\text{\textit{\textbf{O}}}}}
\newcommand{\ibP}{\mathop{\text{\textit{\textbf{P}}}}}
\newcommand{\ibQ}{\mathop{\text{\textit{\textbf{Q}}}}}
\newcommand{\ibR}{\mathop{\text{\textit{\textbf{R}}}}}
\newcommand{\ibS}{\mathop{\text{\textit{\textbf{S}}}}}
\newcommand{\ibT}{\mathop{\text{\textit{\textbf{T}}}}}
\newcommand{\ibU}{\mathop{\text{\textit{\textbf{U}}}}}
\newcommand{\ibV}{\mathop{\text{\textit{\textbf{V}}}}}
\newcommand{\ibW}{\mathop{\text{\textit{\textbf{W}}}}}
\newcommand{\ibX}{\mathop{\text{\textit{\textbf{X}}}}}
\newcommand{\ibY}{\mathop{\text{\textit{\textbf{Y}}}}}
\newcommand{\ibZ}{\mathop{\text{\textit{\textbf{Z}}}}}

\newcommand{\ibx}{\mathop{\text{\textit{\textbf{x}}}}}

\newcommand{\Comp}{\mathop{\mathrm{C}}\nolimits}
\newcommand{\Komp}{\mathop{\mathrm{K}}\nolimits}
\newcommand{\Domp}{\mathop{\mathrm{D}}\nolimits}%複体のホモトピー圏

\newcommand{\CCat}{\Comp(\cat)}
\newcommand{\KCat}{\Komp(\cat)}
\newcommand{\DCat}{\Domp(\cat)}%圏Cの複体のホモトピー圏
\newcommand{\HOM}{\mathop{\mathscr{H}\hspace{-2pt}om}\nolimits}%内部Hom
\newcommand{\RHOM}{\mathop{\mathrm{R}\hspace{-1.5pt}\HOM}\nolimits}

\newcommand{\muS}{\mathop{\mathrm{SS}}\nolimits}
\newcommand{\RG}{\mathop{\mathrm{R}\hspace{-0pt}\Gamma}\nolimits}

\newcommand{\simar}{\mathrel{\overset{\sim}{\longrightarrow}}}%内部Hom
\newcommand{\simra}{\mathrel{\overset{\sim}{\longleftarrow}}}%内部Hom

\newcommand{\hocolim}{{\mathrm{hocolim}}}
\newcommand{\indlim}[1][]{\mathop{\varinjlim}\limits_{#1}}
\newcommand{\sindlim}[1][]{\smash{\mathop{\varinjlim}\limits_{#1}}\,}
\newcommand{\Pro}{\mathrm{Pro}}
\newcommand{\Ind}{\mathrm{Ind}}
\newcommand{\prolim}[1][]{\mathop{\varprojlim}\limits_{#1}}
\newcommand{\sprolim}[1][]{\smash{\mathop{\varprojlim}\limits_{#1}}\,}

% =================================



%================================================
% 自前の定理環境
%   https://mathlandscape.com/latex-amsthm/
% を参考にした
\newtheoremstyle{mystyle}%   % スタイル名
    {5pt}%                   % 上部スペース
    {5pt}%                   % 下部スペース
    {}%              % 本文フォント
    {}%                  % 1行目のインデント量
    {\bfseries}%                      % 見出しフォント
    {.}%                     % 見出し後の句読点
    {12pt}%                     % 見出し後のスペース
    {\thmname{#1}\thmnumber{ #2}\thmnote{{\hspace{2pt}\normalfont (#3)}}}% % 見出しの書式

\theoremstyle{mystyle}
\newtheorem{AXM}{公理}[section]
\newtheorem{DFN}[Axiom]{定義}
\newtheorem{THM}[Axiom]{定理}
\newtheorem*{THM*}{定理}
\newtheorem{PRP}[Axiom]{命題}
\newtheorem{LMM}[Axiom]{補題}
\newtheorem{CRL}[Axiom]{系}
\newtheorem{EG}[Axiom]{例}
\newtheorem{CNV}[Axiom]{規約}


% 定理環境ここまで
%====================================================

% ---------------------------
% new definition macro
% ---------------------------
% 便利なマクロ集です

% 内積のマクロ
%   例えば \inner<\pphi | \psi> のように用いる
\def\inner<#1>{\langle #1 \rangle}

% ================================
% マクロを追加する場合のスペース 

%=================================





% ----------------------------
% documenet 
% ----------------------------
% 以下, 本文の執筆スペースです. 
% Your main code must be written between 
% begin document and end document.
% ---------------------------

\title{非特性変形}
\author{大柴寿浩}
\date{}
\begin{document}
\maketitle



\section*{記号}
次の記号は断りなく使う.
\begin{itemize}
    \item 添字:
    なんらかの族$(a_i)_{i\in I}$を$(a_i)_i$とか$(a_i)$と
    略記することがある.
    \item 近傍:位相空間\(X\)の点\(x\)や部分集合\(Z\)に対し,
    その開近傍系をそれぞれ\(I_x\)や\(I_Z\)で表す.
    これらは,包含関係の逆で有向順序集合をなす.
\end{itemize}



\section{非特性変形補題}
\begin{leftbar}
\begin{PRP}[{\cite[Prop. 2.5.1]{KS90}}]\label{PRP2.5.1}
    \(X\)を位相空間とし,\(Z\)を部分空間とする.
    \(F\)を\(X\)上の層とし,自然な射
    \[
        \psi\colon\indlim[U\in I_Z]\Gamma(U;F)\to\Gamma(Z;F)
    \]を考える.

    (i) 
    \(\psi\)は単射である.

    (ii)
    \(X\)がハウスドルフで\(Z\)がコンパクトならば,\(\psi\)は同型である.
\end{PRP}
\end{leftbar}
\begin{leftbar}
    \begin{PRP}[{\cite[Prop. 1.12.4]{KS90}}]\label{PRP1.12.4}
        \[
            \phi_k\colon H^k(\indlim X)\to\prolim H^k(X_n)
        \]
        について,\(H^{i-1}(X_n)\)がML条件を満たすならば,\(\phi_k\)は一対一対応である.
    \end{PRP}
\end{leftbar}
    
\begin{leftbar}
\begin{PRP}[{\cite[Prop. 1.12.6]{KS90}}]\label{PRP1.12.6}
    \((X_s,\rho_{s,t})\)を実数を添字とする射影系とする.
    \[
        \lambda_s\colon X_s\to\prolim[r<s]X_r,
        \quad
        \mu_s\colon\indlim[t>s]X_t\to X_s
    \]
    がどちらも単射(全射)ならば,すべての実数\(s_0\leqq s_1\)に対し,
    \(\rho_{s_0,s_1}\colon X_{s_1}\to X_{s_0}\)は単射(全射)となる.
\end{PRP}
\end{leftbar}

\begin{leftbar}    
\begin{PRP}[{\cite[Prop. 2.7.2, 非特性変形補題]{KS90}}]
    \(X\)をハウスドルフ空間とし,\(F\in\Domp^+(\zz_X)\)とする.
    また,\((U_t)_{t\in\rr}\)を\(X\)の開集合の族で次の条件(i)--(iii)をみたすものとする.
    \begin{enumerate}[(i)]
        \item 任意の実数\(t\)に対し,\(\bigcup_{s<t}U_s=U_t\)が成り立つ.
        \item 任意の実数\(s\leqq t\)に対し,\(\overline{U_t-U_s}\cap\supp F\)はコンパクト集合である.
        \item 実数\(s\)に対して\(Z_s=\bigcap_{t>s}\overline{U_t-U_s}\)とおくとき,
        任意の実数\(s\leqq t\)と任意の点\(x\in Z_s-U_t\)に対して\(\left(\RG_{X-U_t}(F)\right)_x=0\)が成り立つ.
    \end{enumerate}
    このとき,任意の実数\(t\)に対して,次の同型が成り立つ.
    \[
        \RG\left(\bigcup_{s\in\rr}U_s;F\right)\overset{\sim}{\longrightarrow}\RG(U_t;F)
    \]
\end{PRP}
\end{leftbar}
\begin{proof}
    次の条件を考える.
    \begin{align*}
        (a)_k^s\colon\quad \indlim[t>s]H^k(U_t;F)\simar H^k(U_s;F)\\
        (b)_k^t\colon\quad \prolim[s<t]H^k(U_s;F)\simra H^k(U_t;F)        
    \end{align*}
    任意の実数\(s\)と任意の整数\(k\)に対して\((a)_k^s\)が,
    任意の実数\(t\)と任意の整数\(k<k_0\)に対して\((b)_k^t\)が成り立つとする.
    このとき,\(k_0\)に対し,\((b)_{k_0}^t\)が成り立つことを示す.
    命題\ref{PRP1.12.6}より,
    (\((a)_k^s\)の方が\(\mu_s\),\((b)_k^t\)の方が\(\lambda_t\)として)
    各次数\(k<k_0\)と各実数\(s\leqq t\)に対し,
    \begin{equation}
        H^k(U_t;F)\simar H^k(U_s;F)
    \end{equation}
    が成り立つ.
    このとき,
    \(t\)を固定して,
    射影系\(\left(
        H^{k_0-1}\left(U_{t-\frac{1}{n}};F\right)
    \right)_{n\in\nn}\)を考えると,これはML条件をみたす.
    \begin{center}
        \begin{minipage}{.9\textwidth}
            \begin{redleftbar}
                \(\because)\) 
                任意の\(n\in\nn\)に対し,
                \[
                    \rho_{n,p}
                    \left(
                        H^{k_0-1}\left(U_{t-\frac{1}{p}};F\right)
                        \to 
                        H^{k_0-1}\left(U_{t-\frac{1}{n}};F\right)
                    \right)
                \]
                はすべて同形なので,当然安定.
            \end{redleftbar}
        \end{minipage}
     \end{center}        
    よって,命題\ref{PRP1.12.4}より\((b)_{k_0}^t\)が従う.
    \(k\)に関する帰納法により,
    どの\(t\in\rr\)と\(k\in\zz\)に対しても\((b)_{k}^t\)が成り立つ.
        命題2.7.1を
        \(\left(H^{k}\left(U_{n};F\right)\right)_{n\in\nn}\)に
        用いると,
    \(k\)に関する帰納法から,定理の結論
    \[
        \RG\left(\bigcup_{s\in\rr}U_s;F\right)\overset{\sim}{\longrightarrow}\RG(U_t;F)
    \]
    が従う.

    \subparagraph*{\((a)_k^s\)の証明}
    \(X\)を\(\supp{F}\)におきかえて,どの実数\(s\leqq t\)に対しても
    \(\overline{U_t-U_s}\)はコンパクトとしてよい.
    次のd.t.を考える\footnote{
        \cite[(2.6.32)]{KS90}のd.t.
        \[
            \RG_{Z'}(F)\to\RG_{Z}(F)\to\RG_{Z-Z'}(F)\underset{+1}{\longrightarrow}
        \]
        を用いる.但し,\(Z\)は\(X\)の局所閉集合,\(Z'\)は\(Z\)の閉集合である.
    }.
    \[
        \RG_{(X-U_t)}(F)\rvert_{Z_s}\to
        \RG_{(X-U_s)}(F)\rvert_{Z_s}\to
        \RG_{(U_t-U_s)}(F)\rvert_{Z_s}
        \overset{+1}{\longrightarrow}.
    \]
    仮定(iii)より,左と真ん中の2つは0なので,
    d.t.の性質から,\(\RG_{(U_t-U_s)}(F)\rvert_{Z_s}=0\)となる.
    したがって,任意の\(k\in\zz\)と\(t\geqq s\)に対し,
    \begin{align*}
        0&=H^k(Z_s;\RG_{(U_t-U_s)}(F))\\
        &=\indlim[U\supset Z_s]H^k(U\cap U_t;\RG_{X-U_s}(F))
    \end{align*}
    となる.
    \begin{center}
        \begin{minipage}{.9\textwidth}
        \begin{redleftbar}
            \(\RG_{U_t-U_s}(F)\)は\(X\)上の層で,
            それを\(Z_s\)に制限した\(\RG_{U_t-U_s}(F)\rvert_{Z_s}\)は\(Z_s\)上の層である.
            \(Z_s\)での大域切断\(\RG(Z_s;\RG_{U_t-U_s}(F)\rvert_{Z_s})\)のコホモロジーをとっているので,
            \cite[Notations 2.6.8]{KS90}の2番目の記号を用いることになる.

            \(Z_s\)はハウスドルフ空間\(X\)の
            コンパクト集合\(\overline{U_t-U_s}\)の
            共通部分として表されているので,コンパクトである(\(X\)の置き換えがここに効いている).
            したがって,\cite[Remark 2.6.9 (ii)]{KS90}の場合に当てはまり,
            そこでの記号を用いて書くと
            \[H^j(Z;F)\simeq\indlim[U\in I_Z]H^j(U;F)\]が成り立つ.
            これが上の式の2つ目の変形.
            詳しく書くと,
            \begin{align*}
                H^k(Z_s;\RG_{U_t-U_s}(F))
                &=\indlim[U\in I_Z]H^k(U;\RG_{U_t-U_s}(F))\\
                &=\indlim[U\in I_Z]H^k(U\cap U_t;\RG_{X-U_s}(F))
            \end{align*}
            ここで,2つ目の変形は次のように考える.
            \(U_t-U_s\)に台を持つ層の\(U\)上の切断は\(U\cap U_t\)上で切断を考えても同じ.
            台の方も,\(U\)が\(Z_s\)に十分近ければ\(X-U_s\)で考えても同じ.
        \end{redleftbar}
    \end{minipage}
    \end{center}
    
    どの開集合\(U\)に対しても,実数\(s<t'\leqq t\)で
    \(U\cap U_t\supset U_{t'}-U_s\)となるものが存在する.
    したがって,
    \[
        \indlim[t>s]H^k(U_t;\RG_{X-U_s}(F))=0
    \]
    が成り立つ\footnote{
        \(X-U_s\)に台をもつ層\(F\)の\(U\cap U_t\)での切断を考えたとき,
        \(U\cap U_t\supset U_{t'}-U_s\)から,
        \(U_{t'}-U_s\)での切断を考えればよいことになる.
        台が\(U_s\)を含まないので,最初から\(U_{t'}\)を走らせてよい.
    }.
    条件(i)より,\((U_t)\)は増大族なので,これから,\((a)_k^s\)が従う.
\end{proof}
%\backmatter
%===============================================
% 参考文献スペース
%===============================================
\begin{thebibliography}{20} 
    \bibitem[KS90]{KS90} Masaki Kashiwara, Pierre Schapira, 
        \textit{Sheaves on Manifolds}, 
        Grundlehren der Mathematischen Wissenschaften, 292, Springer, 1990.
        \bibitem[KS06]{KS06} Masaki Kashiwara, Pierre Schapira, 
        \textit{Categories and Sheaves}, 
        Grundlehren der Mathematischen Wissenschaften, 332, Springer, 2006.
        \bibitem[Sh16]{Sh16} 志甫淳, 層とホモロジー代数, 共立出版, 2016.
    %\bibitem[Og02]{Og02} 小木曽啓示, 代数曲線論, 朝倉書店, 2022.
\end{thebibliography}

\end{document}