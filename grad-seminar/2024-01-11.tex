%=====================================
%   guest-oshiba.tex
%   2023年度方程 大柴寿浩
%   2023/11/23 執筆開始
%=====================================

% -----------------------
% preamble
% -----------------------
% ここから本文 (\begin{document}) までの
% ソースコードに変更を加えた場合は
% 編集者まで連絡してください. 
% Don't change preamble code yourself. 
% If you add something
% (usepackage, newtheorem, newcommand, renewcommand),
% please tell it 
% to the editor of institutional paper of RUMS.

% ------------------------
% documentclass
% ------------------------
\documentclass[11pt, a4paper, dvipdfmx]{jsarticle}

% ------------------------
% usepackage
% ------------------------
\usepackage{algorithm}
\usepackage{algorithmic}
\usepackage{amscd}
\usepackage{amsfonts}
\usepackage{amsmath}
\usepackage[psamsfonts]{amssymb}
\usepackage{amsthm}
\usepackage{ascmac}
\usepackage{color}
\usepackage{enumerate}
\usepackage{fancybox}
\usepackage[stable]{footmisc}
\usepackage{graphicx}
\usepackage{listings}
\usepackage{mathrsfs}
\usepackage{mathtools}
\usepackage{otf}
\usepackage{pifont}
\usepackage{proof}
\usepackage{subfigure}
\usepackage{tikz}
\usepackage{verbatim}
\usepackage[all]{xy}

\usetikzlibrary{cd}



% ================================
% パッケージを追加する場合のスペース 
\usepackage[dvipdfmx]{hyperref}
\usepackage{xcolor}
\definecolor{darkgreen}{rgb}{0,0.45,0} 
\definecolor{darkred}{rgb}{0.75,0,0}
\definecolor{darkblue}{rgb}{0,0,0.6} 
\hypersetup{
    colorlinks=true,
    citecolor=darkgreen,
    linkcolor=darkred,
    urlcolor=darkblue,
}
\usepackage{pxjahyper}

%=================================


% --------------------------
% theoremstyle
% --------------------------
\theoremstyle{definition}

% --------------------------
% newtheoem
% --------------------------

% 日本語で定理, 命題, 証明などを番号付きで用いるためのコマンドです. 
% If you want to use theorem environment in Japanece, 
% you can use these code. 
% Attention!
% All theorem enivironment numbers depend on 
% only section numbers.
\newtheorem{Axiom}{公理}[section]
\newtheorem{Definition}[Axiom]{定義}
\newtheorem{Theorem}[Axiom]{定理}
\newtheorem{Proposition}[Axiom]{命題}
\newtheorem{Lemma}[Axiom]{補題}
\newtheorem{Corollary}[Axiom]{系}
\newtheorem{Example}[Axiom]{例}
\newtheorem{Claim}[Axiom]{主張}
\newtheorem{Property}[Axiom]{性質}
\newtheorem{Attention}[Axiom]{注意}
\newtheorem{Question}[Axiom]{問}
\newtheorem{Problem}[Axiom]{問題}
\newtheorem{Consideration}[Axiom]{考察}
\newtheorem{Alert}[Axiom]{警告}
\newtheorem{Fact}[Axiom]{事実}


% 日本語で定理, 命題, 証明などを番号なしで用いるためのコマンドです. 
% If you want to use theorem environment with no number in Japanese, You can use these code.
\newtheorem*{Axiom*}{公理}
\newtheorem*{Definition*}{定義}
\newtheorem*{Theorem*}{定理}
\newtheorem*{Proposition*}{命題}
\newtheorem*{Lemma*}{補題}
\newtheorem*{Example*}{例}
\newtheorem*{Corollary*}{系}
\newtheorem*{Claim*}{主張}
\newtheorem*{Property*}{性質}
\newtheorem*{Attention*}{注意}
\newtheorem*{Question*}{問}
\newtheorem*{Problem*}{問題}
\newtheorem*{Consideration*}{考察}
\newtheorem*{Alert*}{警告}
\newtheorem{Fact*}{事実}


% 英語で定理, 命題, 証明などを番号付きで用いるためのコマンドです. 
% If you want to use theorem environment in English, You can use these code.
%all theorem enivironment number depend on only section number.
\newtheorem{Axiom+}{Axiom}[section]
\newtheorem{Definition+}[Axiom+]{Definition}
\newtheorem{Theorem+}[Axiom+]{Theorem}
\newtheorem{Proposition+}[Axiom+]{Proposition}
\newtheorem{Lemma+}[Axiom+]{Lemma}
\newtheorem{Example+}[Axiom+]{Example}
\newtheorem{Corollary+}[Axiom+]{Corollary}
\newtheorem{Claim+}[Axiom+]{Claim}
\newtheorem{Property+}[Axiom+]{Property}
\newtheorem{Attention+}[Axiom+]{Attention}
\newtheorem{Question+}[Axiom+]{Question}
\newtheorem{Problem+}[Axiom+]{Problem}
\newtheorem{Consideration+}[Axiom+]{Consideration}
\newtheorem{Alert+}{Alert}
\newtheorem{Fact+}[Axiom+]{Fact}
\newtheorem{Remark+}[Axiom+]{Remark}

% ----------------------------
% commmand
% ----------------------------
% 執筆に便利なコマンド集です. 
% コマンドを追加する場合は下のスペースへ. 

% 集合の記号 (黒板文字)
\newcommand{\NN}{\mathbb{N}}
\newcommand{\ZZ}{\mathbb{Z}}
\newcommand{\QQ}{\mathbb{Q}}
\newcommand{\RR}{\mathbb{R}}
\newcommand{\CC}{\mathbb{C}}
\newcommand{\PP}{\mathbb{P}}
\newcommand{\KK}{\mathbb{K}}


% 集合の記号 (太文字)
\newcommand{\nn}{\mathbf{N}}
\newcommand{\zz}{\mathbf{Z}}
\newcommand{\qq}{\mathbf{Q}}
\newcommand{\rr}{\mathbf{R}}
\newcommand{\cc}{\mathbf{C}}
\newcommand{\pp}{\mathbf{P}}
\newcommand{\kk}{\mathbf{K}}

% 特殊な写像の記号
\newcommand{\ev}{\mathop{\mathrm{ev}}\nolimits} % 値写像
\newcommand{\pr}{\mathop{\mathrm{pr}}\nolimits} % 射影

% スクリプト体にするコマンド
%   例えば {\mcal C} のように用いる
\newcommand{\mcal}{\mathcal}

% 花文字にするコマンド 
%   例えば {\h C} のように用いる
\newcommand{\h}{\mathscr}

% ヒルベルト空間などの記号
\newcommand{\F}{\mcal{F}}
\newcommand{\X}{\mcal{X}}
\newcommand{\Y}{\mcal{Y}}
\newcommand{\Hil}{\mcal{H}}
\newcommand{\RKHS}{\Hil_{k}}
\newcommand{\Loss}{\mcal{L}_{D}}
\newcommand{\MLsp}{(\X, \Y, D, \Hil, \Loss)}

% 偏微分作用素の記号
\newcommand{\p}{\partial}

% 角カッコの記号 (内積は下にマクロがあります)
\newcommand{\lan}{\langle}
\newcommand{\ran}{\rangle}



% 圏の記号など
\newcommand{\Set}{{\bf Set}}
\newcommand{\Vect}{{\bf Vect}}
\newcommand{\FDVect}{{\bf FDVect}}
%\newcommand{\Ring}{{\bf Ring}}
\newcommand{\Ab}{{\bf Ab}}
\newcommand{\Mod}{\mathop{\mathrm{Mod}}\nolimits}
\newcommand{\CGA}{{\bf CGA}}
\newcommand{\GVect}{{\bf GVect}}
\newcommand{\Lie}{{\bf Lie}}
\newcommand{\dLie}{{\bf Liec}}



% 射の集合など
\newcommand{\Map}{\mathop{\mathrm{Map}}\nolimits} % 写像の集合
\newcommand{\Hom}{\mathop{\mathrm{Hom}}\nolimits} % 射集合
\newcommand{\End}{\mathop{\mathrm{End}}\nolimits} % 自己準同型の集合
\newcommand{\Aut}{\mathop{\mathrm{Aut}}\nolimits} % 自己同型の集合
\newcommand{\Mor}{\mathop{\mathrm{Mor}}\nolimits} % 射集合
\newcommand{\Ker}{\mathop{\mathrm{Ker}}\nolimits} % 核
\newcommand{\Img}{\mathop{\mathrm{Im}}\nolimits} % 像
\newcommand{\Cok}{\mathop{\mathrm{Coker}}\nolimits} % 余核
\newcommand{\Cim}{\mathop{\mathrm{Coim}}\nolimits} % 余像

% その他便利なコマンド
\newcommand{\dip}{\displaystyle} % 本文中で数式モード
\newcommand{\e}{\varepsilon} % イプシロン
\newcommand{\dl}{\delta} % デルタ
\newcommand{\pphi}{\varphi} % ファイ
\newcommand{\ti}{\tilde} % チルダ
\newcommand{\pal}{\parallel} % 平行
\newcommand{\op}{{\rm op}} % 双対を取る記号
\newcommand{\lcm}{\mathop{\mathrm{lcm}}\nolimits} % 最小公倍数の記号
\newcommand{\Probsp}{(\Omega, \F, \P)} 
\newcommand{\argmax}{\mathop{\rm arg~max}\limits}
\newcommand{\argmin}{\mathop{\rm arg~min}\limits}





% ================================
% コマンドを追加する場合のスペース 
%\newcommand{\OO}{\mcal{O}}



\renewcommand\proofname{\bf 証明} % 証明
\numberwithin{equation}{section}
\newcommand{\cTop}{\textsf{Top}}
%\newcommand{\cOpen}{\textsf{Open}}
\newcommand{\Op}{\mathop{\textsf{Open}}\nolimits}
\newcommand{\Ob}{\mathop{\textrm{Ob}}\nolimits}
\newcommand{\id}{\mathop{\mathrm{id}}\nolimits}
\newcommand{\pt}{\mathop{\mathrm{pt}}\nolimits}
\newcommand{\res}{\mathop{\rho}\nolimits}
\newcommand{\A}{\mcal{A}}
\newcommand{\B}{\mcal{B}}
\newcommand{\C}{\mcal{C}}
\newcommand{\D}{\mcal{D}}
\newcommand{\E}{\mcal{E}}
\newcommand{\G}{\mcal{G}}
%\newcommand{\H}{\mcal{H}}
\newcommand{\I}{\mcal{I}}
\newcommand{\J}{\mcal{J}}
\newcommand{\OO}{\mcal{O}}
\newcommand{\Ring}{\mathop{\textsf{Ring}}\nolimits}
\newcommand{\cAb}{\mathop{\textsf{Ab}}\nolimits}
%\newcommand{\Ker}{\mathop{\mathrm{Ker}}\nolimits}
\newcommand{\im}{\mathop{\mathrm{Im}}\nolimits}
\newcommand{\Coker}{\mathop{\mathrm{Coker}}\nolimits}
\newcommand{\Coim}{\mathop{\mathrm{Coim}}\nolimits}
\newcommand{\rank}{\mathop{\mathrm{rank}}\nolimits}
\newcommand{\Ht}{\mathop{\mathrm{Ht}}\nolimits}
\newcommand{\supp}{\mathop{\mathrm{supp}}\nolimits}
\newcommand{\colim}{\mathop{\mathrm{colim}}}
\newcommand{\Tor}{\mathop{\mathrm{Tor}}\nolimits}

\newcommand{\cat}{\mathscr{C}}

\newcommand{\scA}{\mathscr{A}}
\newcommand{\scB}{\mathscr{B}}
\newcommand{\scC}{\mathscr{C}}
\newcommand{\scD}{\mathscr{D}}
\newcommand{\scE}{\mathscr{E}}
\newcommand{\scF}{\mathscr{F}}

\newcommand{\ibA}{\mathop{\text{\textit{\textbf{A}}}}}
\newcommand{\ibB}{\mathop{\text{\textit{\textbf{B}}}}}
\newcommand{\ibC}{\mathop{\text{\textit{\textbf{C}}}}}
\newcommand{\ibD}{\mathop{\text{\textit{\textbf{D}}}}}
\newcommand{\ibE}{\mathop{\text{\textit{\textbf{E}}}}}
\newcommand{\ibF}{\mathop{\text{\textit{\textbf{F}}}}}
\newcommand{\ibG}{\mathop{\text{\textit{\textbf{G}}}}}
\newcommand{\ibH}{\mathop{\text{\textit{\textbf{H}}}}}
\newcommand{\ibI}{\mathop{\text{\textit{\textbf{I}}}}}
\newcommand{\ibJ}{\mathop{\text{\textit{\textbf{J}}}}}
\newcommand{\ibK}{\mathop{\text{\textit{\textbf{K}}}}}
\newcommand{\ibL}{\mathop{\text{\textit{\textbf{L}}}}}
\newcommand{\ibM}{\mathop{\text{\textit{\textbf{M}}}}}
\newcommand{\ibN}{\mathop{\text{\textit{\textbf{N}}}}}
\newcommand{\ibO}{\mathop{\text{\textit{\textbf{O}}}}}
\newcommand{\ibP}{\mathop{\text{\textit{\textbf{P}}}}}
\newcommand{\ibQ}{\mathop{\text{\textit{\textbf{Q}}}}}
\newcommand{\ibR}{\mathop{\text{\textit{\textbf{R}}}}}
\newcommand{\ibS}{\mathop{\text{\textit{\textbf{S}}}}}
\newcommand{\ibT}{\mathop{\text{\textit{\textbf{T}}}}}
\newcommand{\ibU}{\mathop{\text{\textit{\textbf{U}}}}}
\newcommand{\ibV}{\mathop{\text{\textit{\textbf{V}}}}}
\newcommand{\ibW}{\mathop{\text{\textit{\textbf{W}}}}}
\newcommand{\ibX}{\mathop{\text{\textit{\textbf{X}}}}}
\newcommand{\ibY}{\mathop{\text{\textit{\textbf{Y}}}}}
\newcommand{\ibZ}{\mathop{\text{\textit{\textbf{Z}}}}}

\newcommand{\ibx}{\mathop{\text{\textit{\textbf{x}}}}}

%\newcommand{\Comp}{\mathop{\mathrm{C}}\nolimits}
%\newcommand{\Komp}{\mathop{\mathrm{K}}\nolimits}
%\newcommand{\Domp}{\mathop{\mathsf{D}}\nolimits}%複体のホモトピー圏
\newcommand{\Comp}{\mathrm{C}}
\newcommand{\Komp}{\mathrm{K}}
\newcommand{\Domp}{\mathsf{D}}%複体のホモトピー圏

\newcommand{\CCat}{\Comp(\cat)}
\newcommand{\KCat}{\Komp(\cat)}
\newcommand{\DCat}{\Domp(\cat)}%圏Cの複体のホモトピー圏
\newcommand{\HOM}{\mathop{\mathscr{H}\hspace{-2pt}om}\nolimits}%内部Hom
\newcommand{\RHOM}{\mathop{\mathrm{R}\hspace{-1.5pt}\HOM}\nolimits}

\newcommand{\muS}{\mathop{\mathrm{SS}}\nolimits}
\newcommand{\RG}{\mathop{\mathrm{R}\hspace{-0pt}\Gamma}\nolimits}
\newcommand{\RHom}{\mathop{\mathrm{R}\hspace{-1.5pt}\Hom}\nolimits}
\newcommand{\Rder}{\mathrm{R}}

\newcommand{\simar}{\mathrel{\overset{\sim}{\longrightarrow}}}%内部Hom
\newcommand{\simra}{\mathrel{\overset{\sim}{\longleftarrow}}}%内部Hom

\newcommand{\hocolim}{{\mathrm{hocolim}}}
\newcommand{\indlim}[1][]{\mathop{\varinjlim}\limits_{#1}}
\newcommand{\sindlim}[1][]{\smash{\mathop{\varinjlim}\limits_{#1}}\,}
\newcommand{\Pro}{\mathrm{Pro}}
\newcommand{\Ind}{\mathrm{Ind}}
\newcommand{\prolim}[1][]{\mathop{\varprojlim}\limits_{#1}}
\newcommand{\sprolim}[1][]{\smash{\mathop{\varprojlim}\limits_{#1}}\,}

\newcommand{\Sh}{\mathrm{Sh}}
\newcommand{\PSh}{\mathrm{PSh}}

\newcommand{\rmD}{\mathrm{D}}













%================================================
% 自前の定理環境
%   https://mathlandscape.com/latex-amsthm/
% を参考にした
\newtheoremstyle{mystyle}%   % スタイル名
    {5pt}%                   % 上部スペース
    {5pt}%                   % 下部スペース
    {}%              % 本文フォント
    {}%                  % 1行目のインデント量
    {\bfseries}%                      % 見出しフォント
    {.}%                     % 見出し後の句読点
    {12pt}%                     % 見出し後のスペース
    {\thmname{#1}\thmnumber{ #2}\thmnote{{\hspace{2pt}\normalfont (#3)}}}% % 見出しの書式

\theoremstyle{mystyle}
\newtheorem{AXM}{公理}[section]
\newtheorem{DFN}[AXM]{定義}
\newtheorem{THM}[AXM]{定理}
\newtheorem*{THM*}{定理}
\newtheorem{PRP}[AXM]{命題}
\newtheorem{LMM}[AXM]{補題}
\newtheorem{CRL}[AXM]{系}
\newtheorem{EG}[AXM]{例}
\newtheorem*{EG*}{例}
\newtheorem{CNV}[AXM]{規約}
\newtheorem{CMT}[AXM]{コメント}


% 定理環境ここまで
%====================================================

\usepackage{framed}
\definecolor{lightgray}{rgb}{0.75,0.75,0.75}
\renewenvironment{leftbar}{%
  \def\FrameCommand{\textcolor{lightgray}{\vrule width 0.7zw} \hspace{10pt}}% 
  \MakeFramed {\advance\hsize-\width \FrameRestore}}%
{\endMakeFramed}
\newenvironment{redleftbar}{%
  \def\FrameCommand{\textcolor{lightgray}{\vrule width 1pt} \hspace{10pt}}% 
  \MakeFramed {\advance\hsize-\width \FrameRestore}}%
 {\endMakeFramed}


% =================================





% ---------------------------
% new definition macro
% ---------------------------
% 便利なマクロ集です

% 内積のマクロ
%   例えば \inner<\pphi | \psi> のように用いる
\def\inner<#1>{\langle #1 \rangle}

% ================================
% マクロを追加する場合のスペース 

%=================================





% ----------------------------
% documenet 
% ----------------------------
% 以下, 本文の執筆スペースです. 
% Your main code must be written between 
% begin document and end document.
% ---------------------------

\title{2024/01/11 セミナー資料}
\author{大柴寿浩}
\date{2024/01/11}
\begin{document}
\maketitle
\section{{\cite[3.1]{KS90}}の続き}
\(X\)と\(Y\)を局所コンパクト空間とし,
\(X\times Y\)から\(X\)と\(Y\)への射影を
それぞれ\(q_1\), \(q_2\)で表す.
\begin{PRP}[{\cite[Proposition 3.1.15]{KS90}}]
    \(Y\)の\(c\)柔軟次元は有限であるとする.
    \(F\in\Domp^+(A_X)\), 
    \(G\in\Domp^{\mathrm{b}}(A_X)\)とする.このとき,
    次の自然な同型が成り立つ.
    \[
        \RG(X\times Y; \RHOM(q_2^{-1}G,q_1^!F))
        \cong
        \RHOM\left(
            \RG_c\left(Y;G\right),\RG\left(X;F\right)
        \right).    
    \]
\end{PRP}
\begin{proof}
    次の図式を考える.
    \[\begin{tikzcd}
        X\times Y
        \arrow[rr,"q_{2}"]
        \arrow[dd,"q_{1}"']
        &&
        Y
        \arrow[dd,"\mathrm{a}_Y"]
        \\
        {}&\square&{} 
        \\
        X\arrow[rr,"\mathrm{a}_X"']
        &&
        \{\pt\}.
    \end{tikzcd}\]
    このとき,
    \begin{align*}
        &\RG(X\times Y; \RHOM(q_2^{-1}G,q_1^!F))\\
        &\cong\Rder {\mathrm{a}_{X\times Y}}_\ast 
        \RHOM(q_2^{-1}G,q_1^!F)\quad 
        &\text{(切断を1点への射影で表す.)}\\
        &\cong \Rder {\mathrm{a}_{X}}_\ast\Rder{q_{1}}_\ast
        \RHOM(q_2^{-1}G,q_1^!F)\quad 
        &\text{(\(\mathrm{a}_{X\times Y}=\mathrm{a}_X\circ q_1.\))}\\
        &\cong \Rder {\mathrm{a}_{X}}_\ast
        \RHOM(\Rder{q_{1}}_! q_2^{-1}G,F)\quad 
        &\text{(ポアンカレ・ヴェルディエ双対.)}\\
        &\cong \Rder {\mathrm{a}_{X}}_\ast
        \RHOM(\mathrm{a}_{X}^{-1}\Rder {\mathrm{a}_{Y}}_!G,F)\quad 
        &\text{(固有基底変換.)}\\
        &\cong 
        \RHOM(\Rder {\mathrm{a}_{Y}}_!G,{\Rder\mathrm{a}_{X}}_{\ast}F)\quad 
        &\text{(順像と逆像の随伴.)}\\
        &\cong 
        \RHOM(\RG_c\left(Y;G\right),\RG\left(X;F\right)).\quad 
        &\text{(順像を切断で表す.)}
    \end{align*}
\end{proof}
\begin{comment}
\begin{CMT}
    最後から2行目の変形(順像と逆像の随伴の行)で
    \({\Rder\mathrm{a}_Y}_!G\)を考えている.
    これが\(\Domp^+(A_{\{\pt\}})\)に属すために,
    \(Y\)の次元に関する条件が要る.
\end{CMT}
\end{comment}
\begin{DFN}[{\cite[Definition 3.1.16]{KS90}}]
    \begin{enumerate}[(i)]
        \item \(f\colon Y\to X\)を連続写像とし,
        \(f_!\)のコホモロジー次元は有限であるとする.
        \[
            \omega_{Y/X}\coloneqq f^!A_X \in \Domp^+(A_Y)
        \]
        とおき,
        \textbf{相対双対化複体} (relative dualizing complex) という.
        \(\omega_f\)ともかく.
        \(X=\{\pt\}\)のとき,
        \(\omega_Y=\omega_{Y/\{\pt\}}\)とおき,双対化複体とよぶ.
        \item \(X\)のc柔軟次元が有限であるとする.
        \(F\in\Domp^{\mathrm{b}}(X)\)とする.このとき,
        \begin{align*}
            \rmD_XF=\RHOM(F,\omega_X), \\
            \rmD'_XF=\RHOM(F,A_X)
        \end{align*}
        とおく.
        \(\rmD_XF\)を\(F\)の
        \textbf{ヴェルディエ双対} (Verdier dual) という.
    \end{enumerate}
\end{DFN}

次がある.
\begin{equation}
    f^{-1}(\cdot)\otimes\omega_{Y/X}\to f^!(\cdot).\label{eq:VerdierDual}
\end{equation}
\begin{proof}
    \cite[Proposition 3.1.11]{KS90}の式
    \[
        f^!(\cdot)\otimes^\mathrm{L}_{A_Y}f^{-1}(\cdot)
        \to
        f^!\left(\cdot\otimes^\mathrm{L}_{A_X}\cdot\right)
    \]の第1引数に\(A_X\)を入れると次の射が得られる.
    \[
        f^!A_X\otimes^\mathrm{L}_{A_Y}f^{-1}(\cdot)
        \to
        f^!\left(A_X\otimes^\mathrm{L}_{A_X}\cdot\right)
        \cong f^!(\cdot).
    \]        
\end{proof}

ポアンカレ・ヴェルディエ双対定理から,
\(F\in\Domp^\mathrm{b}(A_X)\)に対して次が成り立つ.
\begin{align}
    \Hom_{\Domp^\mathrm{b}(A_X)}(F,\omega_X)
    &\cong
    \Hom_{\Domp^\mathrm{b}(\Mod(A))}(\RG_c(X;F),A),\\
    \RHom(F,\omega_X)
    &\cong
    \RHom(\RG_c(X;F),A).\label{eq:RPV-dual}
\end{align}
実際,\(\omega_X=\mathrm{a}_X^!A\)に対応する
\({\Rder\mathrm{a}_X}_!F\)を切断で表せばよい.

とくに,局所閉集合\(Z\)に対し\(F=A_Z\)を考えると
\begin{align*}
    \RG_Z(X;\omega_X)
    &\cong\RHom(A_Z,\omega_X)\\
    &\cong\RHom(\RG_c(Z;A_Z),A)
\end{align*}
となる.
\begin{comment}
\begin{CMT}
    \begin{align*}
        \RG_Z(X;\omega_X)
        &\cong \RG(X;\RG_Z(\omega_X))\\
        &\cong\RG(X;\RHOM(A_Z,\omega_X))\\
        &\cong\RG(X;{\Rder\mathrm{a}_X}_\ast\RHOM(A_Z,A))\\
        &\cong\RHom(\RG_c(Z;A_Z),A)
    \end{align*}
\end{CMT}
\end{comment}

\section{多様体における消滅定理{\cite[3.2]{KS90}}}

\(V\)を\(n\)次元実ベクトル空間とする.\footnote{ユークリッド空間の意味で言っている?}
\begin{LMM}
    \(F\in\Sh(V)\)とすると,
    \[H^j_c(V;F)=0.\quad(j>n)\]
\end{LMM}
\begin{proof}
    まず\(n=1\)の場合に示す.
    \(i\colon V\to [0,1]\)を\(V\)から\({]0,1[}\subset [0,1]\)
    への同相写像とする.\(\widetilde{F}\coloneqq i_!F\)とおくと,
    \[
        H^j_c(V;F)
        \overset{\sim}{\longrightarrow}
        H^j([0,1];\widetilde{F})
    \]
    が成り立つ.
    \begin{center}
        \begin{minipage}{.9\textwidth}
        \begin{redleftbar}
            \subparagraph*{同型のチェック}
            \begin{align*}
                H^j_c(V;F)
                &\cong H^j\RG_c(V;F)\\        
                &\cong H^j{\Rder\mathrm{a}_V}_!F        
            \end{align*}
            と
            \begin{align*}
                H^j([0,1];\widetilde{F})
                &\cong H^j\RG([0,1];\Rder{i}_!F)\\        
                &\cong H^j{\Rder\mathrm{a}_{[0,1]}}_\ast\Rder{i}_!F        
            \end{align*}
            より,
            \(
                {\Rder\mathrm{a}_V}_!
                \cong
                {\Rder\mathrm{a}_{[0,1]}}_\ast\Rder{i}_!
            \)
            であればよいが,
            これは\(\mathrm{a}_{[0,1]}\)が固有ならば成り立つ.
            \(\mathrm{a}_{[0,1]}\)は固有なので,同型.
        \end{redleftbar}
        \end{minipage}
    \end{center}
    よって,\cite[Proposition 2.7.3 (i)]{KS90}\footnote{
        \(I=[0,1]\subset\rr\)とする.\(F\in\Sh(I)\)に対し,
        \(H^j(I;F)=0.\quad(j>1)\)
    }より,
    \[
        0=H^j([0,1];\widetilde{F})
        \cong H^j_c(V;F)
    \]
    である.

    次に,一般の場合に示す.
    \(V'\)を\((n-1)\)次元ベクトル空間とし,
    \(f\colon V\to V'\)を全射線形写像とする.
    \textcolor{red}{先の結果から\(\Rder^jf_!F=0.\quad (j\ne0,1)\)}
    
    いま,
    \begin{align*}
        \Rder^0 f_!F&\cong\tau^{\leqq0}\Rder f_!F,\\
        \tau^{\geqq1}\Rder f_!F&\cong\Rder^1 f_!F[-1]
    \end{align*}
    である.
    \begin{center}
        \begin{minipage}{.9\textwidth}
        \begin{redleftbar}
            \subparagraph*{前半のチェック}
            \[
                \cdots\to0\to\left(\Rder f_!F\right)^0
                \to\left(\Rder f_!F\right)^1
                \to0\to\cdots
                \quad\text{in \(\Domp^+(V')\)}
            \]
            の切り落とし\(\tau^{\leqq0}\Rder f_!F\)は,
            \[
                \cdots\to0\to\Ker d^0
                \to0
                \to0\to\cdots
                \quad\text{in \(\Domp^+(V')\)}
            \]
            である.他方,\(\Rder^0 f_!F\)は
            \[
                \Rder^0 f_!F
                \cong H^0(\Rder f_!F)
                \cong \Ker d^0/0=\Ker d^0
            \]
            である.
            これを0次に集中した複体と見れば
            \(\tau^{\leqq0}\Rder f_!F\)
            と一致する.
        \end{redleftbar}
        \end{minipage}
    \end{center}
    \begin{center}
        \begin{minipage}{.9\textwidth}
        \begin{redleftbar}
            \subparagraph*{後半のチェック}
            \[
                \cdots\to0\to\left(\Rder f_!F\right)^0
                \to\left(\Rder f_!F\right)^1
                \to0\to\cdots
                \quad\text{in \(\Domp^+(V')\)}
            \]
            の切り落とし\(\tau^{\geqq1}\Rder f_!F\)は
            \[
                \cdots\to0\to0
                \to\Coker d^0
                \to0\to\cdots
                \quad\text{in \(\Domp^+(V')\)}
            \]
            である.
            他方,\(\Rder^1 f_!F\)は
            \[
                \Rder^1 f_!F
                \cong H^1(\Rder f_!F)
                \cong \Ker d^1/\Img d^0
                \cong(\Rder f_!F)^1/\Img d^0
                \cong\Coker d^0
            \]
            である.
            これを0次に集中した複体と見ると
            \[
                \cdots\to0^{(-1)}\to\Coker d^0
                \to0^{(1)}
                \to0^{(2)}\to\cdots
                \quad\text{in \(\Domp^+(V')\)}
            \]
            となる.この複体に\([-1]\)を適用すると
            \(\tau^{\geqq1}\Rder f_!F\)に一致する.
        \end{redleftbar}
        \end{minipage}
    \end{center}
    よって,特三角
    \[
        \Rder^0 f_!F
        \to\Rder f_!F
        \to\Rder^1 f_!F[-1]
        \overset{+1}{\to}
    \]
    を得る.(cf. \cite[(1.7.2)]{KS90})
    この特三角に\(\RG_c(V';\cdot)\)を適用すると,
    長完全列
    \[
        \cdots\to H^j(V';\Rder^0 f_!F)
        \to H^j(V;F)
        \to H^{j-1}(V';\Rder^1 f_!F)
        \to\cdots
    \]
    が得られる.
    よって,\(n\)に関する帰納法から,
    \(H_c^j(V;F)=0\)が\(j>n\)に対して成り立つ.
\end{proof}




\section{向きづけと双対性{\cite[3.3]{KS90}}}
\eqref{eq:VerdierDual}が同型になるための十分条件を調べる.
\begin{DFN}
    \(f\colon Y\to X\)を局所コンパクト空間の間の連続写像とする.
    \(f\)がファイバー次元\(l\)の
    \textbf{位相的沈めこみ} (topological submersion) であるとは,
    \(Y\)の各点\(y\)に対して,\(y\)の開近傍\(V\in I_y\)で,
    \(U=f(V)\)が\(X\)の開集合であり,次の図式が可換になることをいう.
    \[\begin{tikzcd}
        U\times \rr^l
        \arrow[rd,"\pr_1"']
        \arrow[r,"h", "\sim"']
        &
        V
        \arrow[d,"f\rvert_V"]
        \\
        {}&U. 
    \end{tikzcd}\]
\end{DFN}
\begin{EG*}
    \(X,Y\)が\(C^1\)級多様体で\(f\)を\(C^1\)沈めこみとすると,
    \(f\)は位相的沈めこみである.
\end{EG*}
\begin{PRP}
    \(f\colon Y\to X\)をファイバー次元\(l\)の位相的沈めこみとする.
    \begin{enumerate}[(i)]
        \item \(k\ne -l\)に対し\(H^k(f^!A_X)=0\)であり,局所的に\(H^{-l}(f^!A_X)\cong A_Y\)である.
        \item \(f^{-1}(\cdot)\otimes\omega_{Y/X}\to f^!(\cdot)\)は同型である.
    \end{enumerate}
\end{PRP}
\begin{proof}
    まず\(Y=\rr^l\), \(X=\{\pt\}\)の場合に(i)を示す.
    \eqref{eq:RPV-dual}より任意の開集合\(U\subset Y\)に対し,
    \[
        \RG(U;f^!A_X)\cong\RHom(\RG_c(U;A_Y),A)
    \]
    である.
    さらに,\(U\approx\rr^l\)なら
    \[
        \RG_c(U;A_Y)\cong A[-l]   
    \]
    であり,
    \begin{align*}
        H^j(U;f^!A_X)&\cong0 \quad (j\ne -l)\\
        \Gamma(U;H^{-l}(f^!A_X))
        &\cong
        \Hom(H^l_c(U;A_Y),A)        
    \end{align*}
    である.

    一般の場合,局所的に考えて
    \(Y=\rr^l\times X\)とし,\(f=\pr_2\)とする.
    \(p=\pr_1\colon Y\to \rr^l\)とおく.
    \[\begin{tikzcd}
        \rr^l\times X
        \arrow[rr,"f"]
        \arrow[dd,"p"']
        &&
        Y
        \arrow[dd,"\mathrm{a}_X"]
        \\
        {}&\square&{} 
        \\
        X\arrow[rr,"\mathrm{a}_{\rr^l}"']
        &&
        \{\pt\}.
    \end{tikzcd}\]
    固有基底変換から
    \(p^{-1}\mathrm{a}_{\rr^l}^!=p^{-1}\omega_{\rr^l}
    \to f^!A_X=f^!\mathrm{a}^{-1}_XA\)がある.
    任意の\(F\in\Domp^+(A_X)\)に対し次の射がある.
    \begin{equation}
        p^{-1}\omega_{\rr^l}A\otimes f^{-1}F
        \to f^!A_X\otimes f^{-1}F\to f^{-1}F.
    \end{equation}
    これが同型になることを示す.
    \(U\subset\rr^l\), \(V\subset X\)とする.
    \(h\colon U\approx \rr^l\)のとき,位相的沈めこみの図式は
    \[\begin{tikzcd}
        \rr^l\times V 
        \arrow[rd,"\pr_2"']
        \arrow[r,"h\times\id_X", "\sim"']
        &
        U\times V
        \arrow[d,"f\rvert_{U\times V}"]
        \\
        {}&V 
    \end{tikzcd}\]
    となっている.
    \begin{align*}
        &\RG(U\times V;f^!F)\\
        &\cong \RHom_{Y}(A_{U\times V},f^!F)&\text{\(\Gamma\)の定義}\\
        &\cong \RHom_X(\Rder f_!A_{U\times V},F)&\text{ポアンカレ・ヴェルディエ双対}\\
        &\cong \RHom_X(\RG_c(U;A_U)\otimes^\mathrm{L}A_V,F)\\
        &\cong \RHom(\RG_c(U;A_U),A)\otimes^\mathrm{L}\RHOM(A_V,F)\\
        &\cong \RG(U;\omega_{\rr^l})\otimes^\mathrm{L}\RG(V;F)&\text{上で示した}
    \end{align*}
    となる.
    よって同型.(ii)から(i)が従う.
\end{proof}
%===============================================
% 参考文献スペース
%===============================================
\begin{thebibliography}{20} 
    \bibitem[KS90]{KS90} Masaki Kashiwara, Pierre Schapira, 
    \textit{Sheaves on Manifolds}, 
    Grundlehren der Mathematischen Wissenschaften, 292, Springer, 1990.
\end{thebibliography}

%===============================================


\end{document}
