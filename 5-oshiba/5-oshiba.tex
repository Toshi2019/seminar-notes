%================================================
%    この tex ファイルは2022年度立命館大学数学研究会機関紙
%    『方程』の記事作成テンプレートです. 
%================================================

% -----------------------
% preamble
% -----------------------
% ここから本文 (\begin{document}) までの
% ソースコードに変更を加えた場合は
% 編集者まで連絡してください. 
% Don't change preamble code yourself. 
% If you add something
% (usepackage, newtheorem, newcommand, renewcommand),
% please tell it 
% to the editor of institutional paper of RUMS.

% ------------------------
% documentclass
% ------------------------
\documentclass[11pt, a4paper, dvipdfmx]{jsarticle}

% ------------------------
% usepackage
% ------------------------
\usepackage{algorithm}
\usepackage{algorithmic}
\usepackage{amscd}
\usepackage{amsfonts}
\usepackage{amsmath}
\usepackage[psamsfonts]{amssymb}
\usepackage{amsthm}
\usepackage{ascmac}
\usepackage{color}
\usepackage{enumerate}
\usepackage{fancybox}
\usepackage[stable]{footmisc}
\usepackage{graphicx}
\usepackage{listings}
\usepackage{mathrsfs}
\usepackage{mathtools}
\usepackage{otf}
\usepackage{pifont}
\usepackage{proof}
\usepackage{subfigure}
\usepackage{tikz}
\usepackage{verbatim}
\usepackage[all]{xy}
\usepackage{url}
\usetikzlibrary{cd}



% ================================
% パッケージを追加する場合のスペース 

%=================================


% --------------------------
% theoremstyle
% --------------------------
\theoremstyle{definition}

% --------------------------
% newtheoem
% --------------------------

% 日本語で定理, 命題, 証明などを番号付きで用いるためのコマンドです. 
% If you want to use theorem environment in Japanece, 
% you can use these code. 
% Attention!
% All theorem enivironment numbers depend on 
% only section numbers.
\newtheorem{Axiom}{公理}[section]
\newtheorem{Definition}[Axiom]{定義}
\newtheorem{Theorem}[Axiom]{定理}
\newtheorem{Proposition}[Axiom]{命題}
\newtheorem{Lemma}[Axiom]{補題}
\newtheorem{Corollary}[Axiom]{系}
\newtheorem{Example}[Axiom]{例}
\newtheorem{Claim}[Axiom]{主張}
\newtheorem{Property}[Axiom]{性質}
\newtheorem{Attention}[Axiom]{注意}
\newtheorem{Question}[Axiom]{問}
\newtheorem{Problem}[Axiom]{問題}
\newtheorem{Consideration}[Axiom]{考察}
\newtheorem{Alert}[Axiom]{警告}
\newtheorem{Fact}[Axiom]{事実}


% 日本語で定理, 命題, 証明などを番号なしで用いるためのコマンドです. 
% If you want to use theorem environment with no number in Japanese, You can use these code.
\newtheorem*{Axiom*}{公理}
\newtheorem*{Definition*}{定義}
\newtheorem*{Theorem*}{定理}
\newtheorem*{Proposition*}{命題}
\newtheorem*{Lemma*}{補題}
\newtheorem*{Example*}{例}
\newtheorem*{Corollary*}{系}
\newtheorem*{Claim*}{主張}
\newtheorem*{Property*}{性質}
\newtheorem*{Attention*}{注意}
\newtheorem*{Question*}{問}
\newtheorem*{Problem*}{問題}
\newtheorem*{Consideration*}{考察}
\newtheorem*{Alert*}{警告}
\newtheorem{Fact*}{事実}


% 英語で定理, 命題, 証明などを番号付きで用いるためのコマンドです. 
% If you want to use theorem environment in English, You can use these code.
%all theorem enivironment number depend on only section number.
\newtheorem{Axiom+}{Axiom}[section]
\newtheorem{Definition+}[Axiom+]{Definition}
\newtheorem{Theorem+}[Axiom+]{Theorem}
\newtheorem{Proposition+}[Axiom+]{Proposition}
\newtheorem{Lemma+}[Axiom+]{Lemma}
\newtheorem{Example+}[Axiom+]{Example}
\newtheorem{Corollary+}[Axiom+]{Corollary}
\newtheorem{Claim+}[Axiom+]{Claim}
\newtheorem{Property+}[Axiom+]{Property}
\newtheorem{Attention+}[Axiom+]{Attention}
\newtheorem{Question+}[Axiom+]{Question}
\newtheorem{Problem+}[Axiom+]{Problem}
\newtheorem{Consideration+}[Axiom+]{Consideration}
\newtheorem{Alert+}{Alert}
\newtheorem{Fact+}[Axiom+]{Fact}
\newtheorem{Remark+}[Axiom+]{Remark}

% ----------------------------
% commmand
% ----------------------------
% 執筆に便利なコマンド集です. 
% コマンドを追加する場合は下のスペースへ. 

% 集合の記号 (黒板文字)
\newcommand{\NN}{\mathbb{N}}
\newcommand{\ZZ}{\mathbb{Z}}
\newcommand{\QQ}{\mathbb{Q}}
\newcommand{\RR}{\mathbb{R}}
\newcommand{\CC}{\mathbb{C}}
\newcommand{\PP}{\mathbb{P}}
\newcommand{\KK}{\mathbb{K}}


% 集合の記号 (太文字)
\newcommand{\nn}{\mathbf{N}}
\newcommand{\zz}{\mathbf{Z}}
\newcommand{\qq}{\mathbf{Q}}
\newcommand{\rr}{\mathbf{R}}
\newcommand{\cc}{\mathbf{C}}
\newcommand{\pp}{\mathbf{P}}
\newcommand{\kk}{\mathbf{K}}

% 特殊な写像の記号
\newcommand{\ev}{\mathop{\mathrm{ev}}\nolimits} % 値写像
\newcommand{\pr}{\mathop{\mathrm{pr}}\nolimits} % 射影

% スクリプト体にするコマンド
%   例えば {\mcal C} のように用いる
\newcommand{\mcal}{\mathcal}

% 花文字にするコマンド 
%   例えば {\h C} のように用いる
\newcommand{\h}{\mathscr}

% ヒルベルト空間などの記号
\newcommand{\F}{\mcal{F}}
\newcommand{\X}{\mcal{X}}
\newcommand{\Y}{\mcal{Y}}
\newcommand{\Hil}{\mcal{H}}
\newcommand{\RKHS}{\Hil_{k}}
\newcommand{\Loss}{\mcal{L}_{D}}
\newcommand{\MLsp}{(\X, \Y, D, \Hil, \Loss)}

% 偏微分作用素の記号
\newcommand{\p}{\partial}

% 角カッコの記号 (内積は下にマクロがあります)
\newcommand{\lan}{\langle}
\newcommand{\ran}{\rangle}



% 圏の記号など
\newcommand{\Set}{{\bf Set}}
\newcommand{\Vect}{{\bf Vect}}
\newcommand{\FDVect}{{\bf FDVect}}
\newcommand{\Mod}{\mathop{\mathrm{Mod}}\nolimits}
\newcommand{\CGA}{{\bf CGA}}
\newcommand{\GVect}{{\bf GVect}}
\newcommand{\Lie}{{\bf Lie}}
\newcommand{\dLie}{{\bf Liec}}



% 射の集合など
\newcommand{\Map}{\mathop{\mathrm{Map}}\nolimits}
\newcommand{\Hom}{\mathop{\mathrm{Hom}}\nolimits}
\newcommand{\End}{\mathop{\mathrm{End}}\nolimits}
\newcommand{\Aut}{\mathop{\mathrm{Aut}}\nolimits}
\newcommand{\Mor}{\mathop{\mathrm{Mor}}\nolimits}

% その他便利なコマンド
\newcommand{\dip}{\displaystyle} % 本文中で数式モード
\newcommand{\e}{\varepsilon} % イプシロン
\newcommand{\dl}{\delta} % デルタ
\newcommand{\pphi}{\varphi} % ファイ
\newcommand{\ti}{\tilde} % チルダ
\newcommand{\pal}{\parallel} % 平行
\newcommand{\op}{{\rm op}} % 双対を取る記号
\newcommand{\lcm}{\mathop{\mathrm{lcm}}\nolimits} % 最小公倍数の記号
\newcommand{\Probsp}{(\Omega, \F, \P)} 
\newcommand{\argmax}{\mathop{\rm arg~max}\limits}
\newcommand{\argmin}{\mathop{\rm arg~min}\limits}





% ================================
% コマンドを追加する場合のスペース 
\numberwithin{equation}{section}
\newcommand{\cTop}{\textsf{Top}}
%\newcommand{\cOpen}{\textsf{Open}}
\newcommand{\Op}{\mathop{\textsf{Open}}\nolimits}
\newcommand{\Ob}{\mathop{\textrm{Ob}}\nolimits}
\newcommand{\id}{\mathop{\mathrm{id}}\nolimits}
\newcommand{\res}{\mathop{\rho}\nolimits}
\newcommand{\A}{\mcal{A}}
\newcommand{\B}{\mcal{B}}
\newcommand{\C}{\mcal{C}}
\newcommand{\D}{\mcal{D}}
\newcommand{\E}{\mcal{E}}
\newcommand{\G}{\mcal{G}}
%\newcommand{\H}{\mcal{H}}
\newcommand{\I}{\mcal{I}}
\newcommand{\J}{\mcal{J}}
\newcommand{\OO}{\mcal{O}}
\newcommand{\Ring}{\mathop{\textsf{Ring}}\nolimits}
\newcommand{\cAb}{\mathop{\textsf{Ab}}\nolimits}
% =================================





% ---------------------------
% new definition macro
% ---------------------------
% 便利なマクロ集です

% 内積のマクロ
%   例えば \inner<\pphi | \psi> のように用いる
\def\inner<#1>{\langle #1 \rangle}

% ================================
% マクロを追加する場合のスペース 

%=================================





% ----------------------------
% documenet 
% ----------------------------
% 以下, 本文の執筆スペースです. 
% Your main code must be written between 
% begin document and end document.
% ---------------------------

\title{超関数の理論}
\author{}
\date{}
\begin{document}
\maketitle

\section{はじめに}

\section{層}

層については,\cite{Sh16,KS90}にまとまった解説がある.

$X$を位相空間とし,$\Op(X)$で$X$の開集合全体のなす集合を表す.
$\Op(X)$は開集合を対象とし包含写像を射とする圏になる.

層を考える雛形として,ガウス平面上の関数環を考える.
$\cc$をガウス平面とする.$\cc$の開集合$U$に対し,
\begin{equation}
    \OO_{\cc}(U)\coloneqq\{U\text{上の正則関数}\}
\end{equation}
とおく.$\OO_{\cc}(U)$の加法と乗法を$(f+g)(z)=f(z)+g(z)$, 
$(fg)(z)=f(z)g(z)$
で定めることで,$\OO_{\cc}(U)$は環になる.

$U$と$V$を$V\subset U$をみたす$\cc$の開集合とする.
$f\in\OO_{\cc}(U)$に対し$f|_V\in\OO_{\cc}(V)$を対応させることで
環の射
\begin{equation}
    \res_{VU}\colon\OO_{\cc}(U)\to\OO_{\cc}(V);\quad \res_{VU}(f)=f|_V
\end{equation}
が定まる.
この射を包含写像の
ひきおこす\textbf{制限射} (restriction morphism) とよぶ.

今度は
$U$と$V$を$U\cap V\neq\varnothing$をみたす$\cc$の開集合とする.
複素関数論では,次の事実を学ぶ.
$f\in\OO_{\cc}(U)$, $g\in\OO_{\cc}(V)$に対し,
$U\cap V$で$f=g$となるとき,
$U\cup V$で定義された正則関数$h\in\OO_{\cc}(U\cup V)$で
\begin{align*}
    h|_U=f,\quad h|_V=g
\end{align*}
となるものがただ一つ(!)存在する.

以上の現象を完全列の言葉を用いて眺める.
$U$と$V$を$V\subset U$をみたす$\cc$の開集合とする.
次の列を考える.
\begin{equation}
    \begin{tikzcd}%[column sep=2.3cm]
        0 
        \arrow[r] 
        &[-0.03\linewidth]
        \OO_{\cc}(U\cup V)
        \arrow[r,"\res_{U(U\cup V)}\oplus\res_{V(U\cup V)}"]
        &[0.12\linewidth]
        \OO_{\cc}(U)\oplus\OO_{\cc}(V)
        \arrow[r, "\res_{(U\cap V)U}-\res_{(U\cap V)V}"] 
        &[0.12\linewidth]
        \OO_{\cc}(U\cap V).
    \end{tikzcd}
\end{equation}
ここで,
$\res_{U(U\cup V)}\oplus\res_{V(U\cup V)}\colon\OO_{\cc}(U\cup V)\to\OO_{\cc}(U)\oplus\OO_{\cc}(V)$
は$U\cup V$上の関数$f$に対し$f|_U$と$f|_V$の組$(f|_U,f|_V)$を
対応させる射である.
ただし環$A$と$B$に対し,
$A\oplus B$は単位元を持つ環と単位元を保つ射の圏$\Ring$における
有限積である.
\subparagraph{環の圏についてのコメント}
環の圏における積は一般には直積$A\times B$であり,
有限の積が直和$A\oplus B$である.
積の添字圏として有限圏を取れば$A\oplus B$と$A\times B$は一致する.
(直和は$\Ring$の余積ではない!)
$\Ring$における余積はテンソル積$A\otimes_{\zz} B$である.
一般に$A\times B$と$A\otimes B$は同形ではないため,
$\Ring$はアーベル圏ではないことにも注意.
(始対象は$\zz$で終対象は$0$.
したがって$\Ring$には零対象が存在しないので
アーベル圏ではないという議論もできる.)
\begin{Definition}\label{def-psh}
    $X$ を位相空間とする. 
    $X$上の \textbf{(アーベル群の) 前層} (presheaf) 
    $\F$は次のデータからなる. 
    \begin{itemize}
        \item $X$ の各開部分集合 $U$ に対するアーベル群 $\F(U)$
        \item 部分開集合の各組 $V \subset U$ に対する
        群準同型$\rho_{UV} \colon \F(U)\to \F(V)$
        で, 次の条件(\ref{def:psh1})--(\ref{def:psh3})を満たすもの.
        \begin{enumerate}
            \renewcommand{\labelenumi}{({\arabic{enumi}})}
            \item $\F(\varnothing) = 0,$\label{def:psh1} 
            \item $\rho_{UU} = \id$, \label{def:psh2}
            \item $W \subset V \subset U$ ならば, \label{def:psh3}
            $\rho_{UW} = \rho_{VW} \circ \rho_{UV}$.
        \end{enumerate}
    \end{itemize}
    元$s\in \F(U)$を$\F$の$U$上の\textbf{切断} (section) という. 
    $s|_V$ で $\rho_{UV}(s)\in\F(V)$ を表し, 
    $s$ の $V$ への\textbf{制限} (restriction) とよぶ. 
\end{Definition}
つまり,$\Op(X)$から$\cAb$への反変関手で
始対象$\varnothing$を終対象$0$にうつすものが前層である.

\section{超関数}
%===============================================
% 参考文献スペース
%===============================================
\begin{thebibliography}{20} 
    \bibitem[Sh16]{Sh16} 志甫淳, 層とホモロジー代数, 共立出版, 2016.
    \bibitem[KS90]{KS90} Masaki Kashiwara, Pierre Schapira, 
        \textit{Sheaves on Manifolds}, 
        Grundlehren der Mathematischen Wissenschaften, 292, Springer, 1990.

\end{thebibliography}

%===============================================


\end{document}
