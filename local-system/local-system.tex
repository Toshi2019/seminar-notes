%=====================================
%   local-system.tex
%   局所系と局所定数層(とモノドロミー)
%=====================================

% -----------------------
% preamble
% -----------------------
% ここから本文 (\begin{document}) までの
% ソースコードに変更を加えた場合は
% 編集者まで連絡してください. 
% Don't change preamble code yourself. 
% If you add something
% (usepackage, newtheorem, newcommand, renewcommand),
% please tell it 
% to the editor of institutional paper of RUMS.

% ------------------------
% documentclass
% ------------------------
\documentclass[11pt, a4paper, leqno, dvipdfmx]{jsarticle}

% ------------------------
% usepackage
% ------------------------
\usepackage{algorithm}
\usepackage{algorithmic}
\usepackage{amscd}
\usepackage{amsfonts}
\usepackage{amsmath}
\usepackage[psamsfonts]{amssymb}
\usepackage{amsthm}
\usepackage{ascmac}
\usepackage{bm}
\usepackage{color}
\usepackage{enumerate}
\usepackage{fancybox}
\usepackage[stable]{footmisc}
\usepackage{graphicx}
\usepackage{listings}
\usepackage{mathrsfs}
\usepackage{mathtools}
\usepackage{otf}
\usepackage{pifont}
\usepackage{proof}
\usepackage{subfigure}
\usepackage{tikz}
\usepackage{verbatim}
\usepackage[all]{xy}

\usetikzlibrary{cd}



% ================================
% パッケージを追加する場合のスペース 
\usepackage[dvipdfmx]{hyperref}
\usepackage{xcolor}
\definecolor{darkgreen}{rgb}{0,0.45,0} 
\definecolor{darkred}{rgb}{0.75,0,0}
\definecolor{darkblue}{rgb}{0,0,0.6} 
\hypersetup{
    colorlinks=true,
    citecolor=darkgreen,
    linkcolor=darkred,
    urlcolor=darkblue,
}
\usepackage{pxjahyper}
\usepackage[mathcal]{eucal}
%=================================


% --------------------------
% theoremstyle
% --------------------------
\theoremstyle{definition}

% --------------------------
% newtheoem
% --------------------------

% 日本語で定理, 命題, 証明などを番号付きで用いるためのコマンドです. 
% If you want to use theorem environment in Japanece, 
% you can use these code. 
% Attention!
% All theorem enivironment numbers depend on 
% only section numbers.
\newtheorem{Axiom}{公理}[section]
\newtheorem{Definition}[Axiom]{定義}
\newtheorem{Theorem}[Axiom]{定理}
\newtheorem{Proposition}[Axiom]{命題}
\newtheorem{Lemma}[Axiom]{補題}
\newtheorem{Corollary}[Axiom]{系}
\newtheorem{Example}[Axiom]{例}
\newtheorem{Claim}[Axiom]{主張}
\newtheorem{Property}[Axiom]{性質}
\newtheorem{Attention}[Axiom]{注意}
\newtheorem{Question}[Axiom]{問}
\newtheorem{Problem}[Axiom]{問題}
\newtheorem{Consideration}[Axiom]{考察}
\newtheorem{Alert}[Axiom]{警告}
\newtheorem{Fact}[Axiom]{事実}


% 日本語で定理, 命題, 証明などを番号なしで用いるためのコマンドです. 
% If you want to use theorem environment with no number in Japanese, You can use these code.
\newtheorem*{Axiom*}{公理}
\newtheorem*{Definition*}{定義}
\newtheorem*{Theorem*}{定理}
\newtheorem*{Proposition*}{命題}
\newtheorem*{Lemma*}{補題}
\newtheorem*{Example*}{例}
\newtheorem*{Corollary*}{系}
\newtheorem*{Claim*}{主張}
\newtheorem*{Property*}{性質}
\newtheorem*{Attention*}{注意}
\newtheorem*{Question*}{問}
\newtheorem*{Problem*}{問題}
\newtheorem*{Consideration*}{考察}
\newtheorem*{Alert*}{警告}
\newtheorem{Fact*}{事実}


% 英語で定理, 命題, 証明などを番号付きで用いるためのコマンドです. 
% If you want to use theorem environment in English, You can use these code.
%all theorem enivironment number depend on only section number.
\newtheorem{Axiom+}{Axiom}[section]
\newtheorem{Definition+}[Axiom+]{Definition}
\newtheorem{Theorem+}[Axiom+]{Theorem}
\newtheorem{Proposition+}[Axiom+]{Proposition}
\newtheorem{Lemma+}[Axiom+]{Lemma}
\newtheorem{Example+}[Axiom+]{Example}
\newtheorem{Corollary+}[Axiom+]{Corollary}
\newtheorem{Claim+}[Axiom+]{Claim}
\newtheorem{Property+}[Axiom+]{Property}
\newtheorem{Attention+}[Axiom+]{Attention}
\newtheorem{Question+}[Axiom+]{Question}
\newtheorem{Problem+}[Axiom+]{Problem}
\newtheorem{Consideration+}[Axiom+]{Consideration}
\newtheorem{Alert+}{Alert}
\newtheorem{Fact+}[Axiom+]{Fact}
\newtheorem{Remark+}[Axiom+]{Remark}

% ----------------------------
% commmand
% ----------------------------
% 執筆に便利なコマンド集です. 
% コマンドを追加する場合は下のスペースへ. 

% 集合の記号 (黒板文字)
\newcommand{\NN}{\mathbb{N}}
\newcommand{\ZZ}{\mathbb{Z}}
\newcommand{\QQ}{\mathbb{Q}}
\newcommand{\RR}{\mathbb{R}}
\newcommand{\CC}{\mathbb{C}}
\newcommand{\PP}{\mathbb{P}}
\newcommand{\KK}{\mathbb{K}}


% 集合の記号 (太文字)
\newcommand{\nn}{\mathbf{N}}
\newcommand{\zz}{\mathbf{Z}}
\newcommand{\qq}{\mathbf{Q}}
\newcommand{\rr}{\mathbf{R}}
\newcommand{\cc}{\mathbf{C}}
\newcommand{\pp}{\mathbf{P}}
\newcommand{\kk}{\mathbf{K}}

% 特殊な写像の記号
\newcommand{\ev}{\mathop{\mathrm{ev}}\nolimits} % 値写像
\newcommand{\pr}{\mathop{\mathrm{pr}}\nolimits} % 射影

% スクリプト体にするコマンド
%   例えば {\mcal C} のように用いる
\newcommand{\mcal}{\mathcal}

% 花文字にするコマンド 
%   例えば {\h C} のように用いる
\newcommand{\h}{\mathscr}

% ヒルベルト空間などの記号
\newcommand{\F}{\mcal{F}}
\newcommand{\X}{\mcal{X}}
\newcommand{\Y}{\mcal{Y}}
\newcommand{\Hil}{\mcal{H}}
\newcommand{\RKHS}{\Hil_{k}}
\newcommand{\Loss}{\mcal{L}_{D}}
\newcommand{\MLsp}{(\X, \Y, D, \Hil, \Loss)}

% 偏微分作用素の記号
\newcommand{\p}{\partial}

% 角カッコの記号 (内積は下にマクロがあります)
\newcommand{\lan}{\langle}
\newcommand{\ran}{\rangle}



% 圏の記号など
\newcommand{\Set}{{\bf Set}}
\newcommand{\Vect}{{\bf Vect}}
\newcommand{\FDVect}{{\bf FDVect}}
%\newcommand{\Ring}{{\bf Ring}}
\newcommand{\Ab}{{\bf Ab}}
\newcommand{\Mod}{\mathop{\mathrm{Mod}}\nolimits}
\newcommand{\CGA}{{\bf CGA}}
\newcommand{\GVect}{{\bf GVect}}
\newcommand{\Lie}{{\bf Lie}}
\newcommand{\dLie}{{\bf Liec}}



% 射の集合など
\newcommand{\Map}{\mathop{\mathrm{Map}}\nolimits} % 写像の集合
\newcommand{\Hom}{\mathop{\mathrm{Hom}}\nolimits} % 射集合
\newcommand{\End}{\mathop{\mathrm{End}}\nolimits} % 自己準同型の集合
\newcommand{\Aut}{\mathop{\mathrm{Aut}}\nolimits} % 自己同型の集合
\newcommand{\Mor}{\mathop{\mathrm{Mor}}\nolimits} % 射集合
\newcommand{\Ker}{\mathop{\mathrm{Ker}}\nolimits} % 核
\newcommand{\Img}{\mathop{\mathrm{Im}}\nolimits} % 像
\newcommand{\Cok}{\mathop{\mathrm{Coker}}\nolimits} % 余核
\newcommand{\Cim}{\mathop{\mathrm{Coim}}\nolimits} % 余像

% その他便利なコマンド
\newcommand{\dip}{\displaystyle} % 本文中で数式モード
\newcommand{\e}{\varepsilon} % イプシロン
\newcommand{\dl}{\delta} % デルタ
\newcommand{\pphi}{\varphi} % ファイ
\newcommand{\ti}{\tilde} % チルダ
\newcommand{\pal}{\parallel} % 平行
\newcommand{\op}{{\rm op}} % 双対を取る記号
\newcommand{\lcm}{\mathop{\mathrm{lcm}}\nolimits} % 最小公倍数の記号
\newcommand{\Probsp}{(\Omega, \F, \P)} 
\newcommand{\argmax}{\mathop{\rm arg~max}\limits}
\newcommand{\argmin}{\mathop{\rm arg~min}\limits}





% ================================
% コマンドを追加する場合のスペース 
%\newcommand{\OO}{\mcal{O}}



\renewcommand\proofname{\bf 証明} % 証明
\numberwithin{equation}{section}
\newcommand{\cTop}{\textsf{Top}}
%\newcommand{\cOpen}{\textsf{Open}}
\newcommand{\Op}{\mathop{\mathsf{Open}}\nolimits}
\newcommand{\Cl}{\mathop{\mathrm{Cl}}\nolimits}
\newcommand{\Ob}{\mathop{\mathrm{Ob}}\nolimits}
\newcommand{\id}{\mathord{\mathrm{id}}\nolimits}
\newcommand{\pt}{\mathord{\mathrm{pt}}\nolimits}
\newcommand{\res}{\mathop{\rho}\nolimits}
\newcommand{\A}{\mcal{A}}
\newcommand{\B}{\mcal{B}}
\newcommand{\C}{\mcal{C}}
\newcommand{\D}{\mcal{D}}
\newcommand{\E}{\mcal{E}}
\newcommand{\G}{\mcal{G}}
%\newcommand{\H}{\mcal{H}}
\newcommand{\I}{\mcal{I}}
\newcommand{\J}{\mcal{J}}
\newcommand{\OO}{\mcal{O}}
\newcommand{\Ring}{\mathop{\textsf{Ring}}\nolimits}
\newcommand{\cAb}{\mathop{\textsf{Ab}}\nolimits}
%\newcommand{\Ker}{\mathop{\mathrm{Ker}}\nolimits}
\newcommand{\im}{\mathop{\mathrm{Im}}\nolimits}
\newcommand{\Coker}{\mathop{\mathrm{Coker}}\nolimits}
\newcommand{\Coim}{\mathop{\mathrm{Coim}}\nolimits}
\newcommand{\rank}{\mathop{\mathrm{rank}}\nolimits}
\newcommand{\Ht}{\mathop{\mathrm{Ht}}\nolimits}
\newcommand{\supp}{\mathop{\mathrm{supp}}\nolimits}
\newcommand{\colim}{\mathop{\mathrm{colim}}}
\newcommand{\Tor}{\mathop{\mathrm{Tor}}\nolimits}

\newcommand{\cat}{\mathscr{C}}

\newcommand{\scA}{\mathscr{A}}
\newcommand{\scB}{\mathscr{B}}
\newcommand{\scC}{\mathscr{C}}
\newcommand{\scD}{\mathscr{D}}
\newcommand{\scE}{\mathscr{E}}
\newcommand{\scF}{\mathscr{F}}
\newcommand{\scN}{\mathscr{N}}
\newcommand{\scO}{\mathscr{O}}
\newcommand{\scV}{\mathscr{V}}


\newcommand{\ibA}{\mathop{\text{\textit{\textbf{A}}}}}
\newcommand{\ibB}{\mathop{\text{\textit{\textbf{B}}}}}
\newcommand{\ibC}{\mathop{\text{\textit{\textbf{C}}}}}
\newcommand{\ibD}{\mathop{\text{\textit{\textbf{D}}}}}
\newcommand{\ibE}{\mathop{\text{\textit{\textbf{E}}}}}
\newcommand{\ibF}{\mathop{\text{\textit{\textbf{F}}}}}
\newcommand{\ibG}{\mathop{\text{\textit{\textbf{G}}}}}
\newcommand{\ibH}{\mathop{\text{\textit{\textbf{H}}}}}
\newcommand{\ibI}{\mathop{\text{\textit{\textbf{I}}}}}
\newcommand{\ibJ}{\mathop{\text{\textit{\textbf{J}}}}}
\newcommand{\ibK}{\mathop{\text{\textit{\textbf{K}}}}}
\newcommand{\ibL}{\mathop{\text{\textit{\textbf{L}}}}}
\newcommand{\ibM}{\mathop{\text{\textit{\textbf{M}}}}}
\newcommand{\ibN}{\mathop{\text{\textit{\textbf{N}}}}}
\newcommand{\ibO}{\mathop{\text{\textit{\textbf{O}}}}}
\newcommand{\ibP}{\mathop{\text{\textit{\textbf{P}}}}}
\newcommand{\ibQ}{\mathop{\text{\textit{\textbf{Q}}}}}
\newcommand{\ibR}{\mathop{\text{\textit{\textbf{R}}}}}
\newcommand{\ibS}{\mathop{\text{\textit{\textbf{S}}}}}
\newcommand{\ibT}{\mathop{\text{\textit{\textbf{T}}}}}
\newcommand{\ibU}{\mathop{\text{\textit{\textbf{U}}}}}
\newcommand{\ibV}{\mathop{\text{\textit{\textbf{V}}}}}
\newcommand{\ibW}{\mathop{\text{\textit{\textbf{W}}}}}
\newcommand{\ibX}{\mathop{\text{\textit{\textbf{X}}}}}
\newcommand{\ibY}{\mathop{\text{\textit{\textbf{Y}}}}}
\newcommand{\ibZ}{\mathop{\text{\textit{\textbf{Z}}}}}

\newcommand{\ibx}{\mathop{\text{\textit{\textbf{x}}}}}

%\newcommand{\Comp}{\mathop{\mathrm{C}}\nolimits}
%\newcommand{\Komp}{\mathop{\mathrm{K}}\nolimits}
%\newcommand{\Domp}{\mathop{\mathsf{D}}\nolimits}%複体のホモトピー圏
%\newcommand{\Comp}{\mathrm{C}}
%\newcommand{\Komp}{\mathrm{K}}
%\newcommand{\Domp}{\mathsf{D}}%複体のホモトピー圏

\newcommand{\Comp}{\mathop{\mathsf{C}}\nolimits}
\newcommand{\Komp}{\mathop{\mathsf{K}}\nolimits}
\newcommand{\Domp}{\mathop{\mathsf{D}}\nolimits}
\newcommand{\Kompl}{\mathop{\mathsf{K}^\mathrm{+}}\nolimits}
\newcommand{\Kompu}{\mathop{\mathsf{K}^\mathrm{-}}\nolimits}
\newcommand{\Kompb}{\mathop{\mathsf{K}^\mathrm{b}}\nolimits}
\newcommand{\Dompl}{\mathop{\mathsf{D}^\mathrm{+}}\nolimits}
\newcommand{\Dompu}{\mathop{\mathsf{D}^\mathrm{-}}\nolimits}
\newcommand{\Dompb}{\mathop{\mathsf{D}^\mathrm{b}}\nolimits}




\newcommand{\CCat}{\Comp(\cat)}
\newcommand{\KCat}{\Komp(\cat)}
\newcommand{\DCat}{\Domp(\cat)}%圏Cの複体のホモトピー圏
\newcommand{\HOM}{\mathop{\mathscr{H}\hspace{-2pt}om}\nolimits}%内部Hom
\newcommand{\RHOM}{\mathop{\mathrm{R}\hspace{-1.5pt}\HOM}\nolimits}

\newcommand{\muS}{\mathop{\mathrm{SS}}\nolimits}
\newcommand{\RG}{\mathop{\mathrm{R}\hspace{-0pt}\Gamma}\nolimits}
\newcommand{\RHom}{\mathop{\mathrm{R}\hspace{-1.5pt}\Hom}\nolimits}
\newcommand{\Rder}{\mathrm{R}}

\newcommand{\simar}{\mathrel{\overset{\sim}{\rightarrow}}}%同型右矢印
\newcommand{\simarr}{\mathrel{\overset{\sim}{\longrightarrow}}}%同型右矢印
\newcommand{\simra}{\mathrel{\overset{\sim}{\leftarrow}}}%同型左矢印
\newcommand{\simrra}{\mathrel{\overset{\sim}{\longleftarrow}}}%同型左矢印

\newcommand{\hocolim}{{\mathrm{hocolim}}}
\newcommand{\indlim}[1][]{\mathop{\varinjlim}\limits_{#1}}
\newcommand{\sindlim}[1][]{\smash{\mathop{\varinjlim}\limits_{#1}}\,}
\newcommand{\Pro}{\mathrm{Pro}}
\newcommand{\Ind}{\mathrm{Ind}}
\newcommand{\prolim}[1][]{\mathop{\varprojlim}\limits_{#1}}
\newcommand{\sprolim}[1][]{\smash{\mathop{\varprojlim}\limits_{#1}}\,}

\newcommand{\Sh}{\mathrm{Sh}}
\newcommand{\PSh}{\mathrm{PSh}}

\newcommand{\rmD}{\mathrm{D}}

\newcommand{\Lloc}[1][]{\mathord{\mathcal{L}^1_{\mathrm{loc},{#1}}}}
\newcommand{\ori}{\mathord{\mathrm{or}}}
\newcommand{\Db}{\mathord{\mathscr{D}b}}

\newcommand{\codim}{\mathop{\mathrm{codim}}\nolimits}



\newcommand{\gld}{\mathop{\mathrm{gld}}\nolimits}
\newcommand{\wgld}{\mathop{\mathrm{wgld}}\nolimits}

\newcommand{\Cov}{\mathop{\mathsf{Cov}}\nolimits}
\newcommand{\Inv}{\mathop{\mathrm{Inv}}\nolimits}










%================================================
% 自前の定理環境
%   https://mathlandscape.com/latex-amsthm/
% を参考にした
\newtheoremstyle{mystyle}%   % スタイル名
    {5pt}%                   % 上部スペース
    {5pt}%                   % 下部スペース
    {}%              % 本文フォント
    {}%                  % 1行目のインデント量
    {\bfseries}%                      % 見出しフォント
    {.}%                     % 見出し後の句読点
    {12pt}%                     % 見出し後のスペース
    {\thmname{#1}\thmnumber{ #2}\thmnote{{\hspace{2pt}\normalfont (#3)}}}% % 見出しの書式

\theoremstyle{mystyle}
\newtheorem{AXM}{公理}[section]
\newtheorem{DFN}[AXM]{定義}
\newtheorem{THM}[AXM]{定理}
\newtheorem*{THM*}{定理}
\newtheorem{PRP}[AXM]{命題}
\newtheorem{LMM}[AXM]{補題}
\newtheorem{CRL}[AXM]{系}
\newtheorem{EG}[AXM]{例}
\newtheorem*{EG*}{例}
\newtheorem{RMK}[AXM]{注意}
\newtheorem{CNV}[AXM]{約束}
\newtheorem{CMT}[AXM]{コメント}
\newtheorem{NTN}[AXM]{記号}

% 定理環境ここまで
%====================================================

\usepackage{framed}
\definecolor{lightgray}{rgb}{0.75,0.75,0.75}
\renewenvironment{leftbar}{%
  \def\FrameCommand{\textcolor{lightgray}{\vrule width 4pt} \hspace{10pt}}% 
  \MakeFramed {\advance\hsize-\width \FrameRestore}}%
{\endMakeFramed}
\newenvironment{redleftbar}{%
  \def\FrameCommand{\textcolor{lightgray}{\vrule width 1pt} \hspace{10pt}}% 
  \MakeFramed {\advance\hsize-\width \FrameRestore}}%
 {\endMakeFramed}


% =================================





% ---------------------------
% new definition macro
% ---------------------------
% 便利なマクロ集です

% 内積のマクロ
%   例えば \inner<\pphi | \psi> のように用いる
\def\inner<#1>{\langle #1 \rangle}

% ================================
% マクロを追加する場合のスペース 

%=================================





% ----------------------------
% documenet 
% ----------------------------
% 以下, 本文の執筆スペースです. 
% Your main code must be written between 
% begin document and end document.
% ---------------------------

\title{局所系と局所定数層}
\author{toshi2019}
\date{}
\begin{document}
\maketitle

\begin{abstract}
    局所系と局所定数層が同じことを説明する.
    余裕があればモノドロミーについて述べる.
\end{abstract}
\section*{記号と用語}
\begin{itemize}
    \item 単位閉区間\([0,1]\)を\(I\)とかく.
    \item 位相空間\(X\)の部分集合\(A\)に対し,閉包を\(\Cl{A}\)とかく.
    \item 位相空間\(X\)の点\(x\), \(y\)に対し,
    \(x\)を始点,\(y\)を終点とする\(X\)内の道のホモトピー類の集合を
    \(\varOmega(X;x,y)\)とかく.
    \(X\)が明らかな場合は\(\varOmega(x,y)\)ともかく.
    \item 開集合\(U\)の開被覆の族を\(\Cov(U)\)とかく.
\end{itemize}



\section{局所系}

\subsection{亜群}

\(X\)を弧状連結な位相空間とする.
このとき,\(X\)上の\textbf{基本亜群} (fundamental groupoid) 
\(\varPi(X)\)を次で定める.
\begin{description}\setlength{\leftskip}{2zw}
    \item[対象] \(x\)の各点.
    \item[射] \(x, y\in X\)に対し,\(x\)から\(y\)への道\(l\colon I\to X\)たちのホモトピー類.
    \item[合成] 道の結合.
\end{description}
次のように書いてもよい.
\begin{align*}
    \Ob(\varPi(X))&\coloneqq X,\\
    \Hom_{\varPi(X)}(x,y)&\coloneqq \varOmega(x,y),\\
    [l']\circ [l]&\coloneqq [l\cdot l'] \quad (l\in \varOmega(x,y), l'\in \varOmega(y,z)). 
\end{align*}
\(\varPi(X)\)の射はすべて同型なので\(\varPi(X)\)は亜群である.

\subsection{局所系}
\(X\)を弧状連結な位相空間とする.
亜群\(\varPi(X)\)上の共変換手\(
    F\colon \varPi(X)\to \cat
\)を\(X\)上の\(\cat\)に値をとる局所系という.

\section{局所定数層}

\(X\)上の\(\cat\)に値をとる層\(F\)
が\textbf{局所定数層} (locally constant sheaf) であるとは,
\(X\)の開被覆\((U_i)_i\)で,各開集合\(U_i\)ごとに,
\(F\rvert_{U_i}\)が定数層と同型となるものが存在することをいう.





















\section{{\cite{Sp66}}から}





\subsection{{\cite[Chap 1]{Sp66}}}
\cite[Chap 1]{Sp66}のExercise Fが局所系に関するものである.







\subsubsection*{F 局所系\footnote{
    Steenrod, Homology with local coefficients, 
    Ann Math., 44, pp. 610--627, 1943 参照.
}}

\subparagraph{1} 空間\(X\)上の局所系とは,\(X\)の基本亜群から
別の圏への共変関手である.任意の圏\(\mathcal{C}\)に対し,
\(\mathcal{C}\)に値をとる\(X\)上の局所系の圏が定まることを示せ.
(2つの局所系がこの圏の対象として同値であるとき,
それらが同値であるという.)

\subparagraph{2}
\(f\colon X\to Y\)を写像とする.
\(f\)は\(\mathcal{C}\)に値をとる\(Y\)上の局所系の圏から
\(\mathcal{C}\)に値をとる\(X\)上の局所系の圏への関手
をひきおこすことを示せ.

\subparagraph{3}
\(A\)が圏\(\mathcal{C}\)の対象であるとき,
\(\Aut{A}\)を\(\mathcal{C}\)における\(A\)の自己同値のなす群とする.
\(\varphi\colon A\approx B\)が\(\mathcal{C}\)における同値とであるとき,
\(
    \mathop{\overline{\varphi}}(\alpha)
    =\varphi\circ\alpha\varphi^{-1}
\)で定義される射\(
    \mathop{\overline{\varphi}}\colon\Aut{A}\to\Aut{B}
\)は群の同型であることを示せ.

\subparagraph{4}
\(\Gamma\)を\(X\)上の局所系とし\(x_0\in X\)とする.
このとき,\(\Gamma\)は準同型
\[
    \mathop{\bar{\Gamma}_{x_0}}\colon \pi(X,x_0)\to\Aut{\Gamma(x_0)}
\]
をひきおこすことを示せ.

\subparagraph{5}
\(X\)が弧状連結であるとき,
\(X\)上の\(\mathcal{C}\)に値をとる局所系\(\Gamma\), \(\Gamma'\)が同値であるためには,
同値\(\varphi\colon \Gamma(x_0)\to\Gamma'(x_0)\)で,
\(\Aut{\Gamma'(x_0)}\)において
\(\bar{\varphi}\circ\bar{\Gamma}_{x_0}\)が
\(\bar{\Gamma'}_{x_0}\)と共役となるものが存在する
ことが必要十分であることを示せ.

\subparagraph{6}
\(X\)が弧状連結であるとき,
\(A\in\mathcal{C}\)と準同型\(\alpha\colon\pi(X,x_0)\to\Aut{A}\)に対して,
\(X\)上の\(\mathcal{C}\)に値をとる局所系\(\Gamma\)で,
\(\Gamma(x_0)=A\)と\(\bar{\Gamma}_{x_0}=\alpha\)を
みたすものが存在することを示せ.


\subsection{{\cite[Chap 6]{Sp66}}}

局所系と層の関係については\cite[Chap 6]{Sp66}のExercise Fに書いてある.

\subsubsection*{F 局所系と層}

以下の一連の演習問題では,\(X\)はパラコンパクトハウスドルフであるとする.

\subparagraph{1}
\(\Gamma\)が\(X\)上の局所系であるとき,
\(\bar{\Gamma}\)を,開集合\(V\subset X\)に対し,
\[\mathop{\bar{\Gamma}}(V)\coloneqq\left\{
    f\colon X\to \Gamma(X);\text{
        \(V\)内の任意の道\(\omega\)に対して,
        \(f(\omega(1))
        =\Gamma(\omega)\left(f\left(\omega(0)\right)\right)\)
    }
\right\}\]で定まる\(X\)上の前層とする.
\(\bar{\Gamma}\)は\(X\)上の層であり,
\(\Gamma\)に\(\bar{\Gamma}\)を対応させる操作は,
局所系から層への自然変換であることを示せ.


\subparagraph{2}
\(X\)上の前層\(\Gamma\)が局所定数層であるとは,
\(X\)の開被覆\(\mathscr{U}=\{U\}\)で
\(U\in\mathscr{U}\)かつ\(x\in U\)のとき,
\(\Gamma(U)\approx\indlim[V\ni x]\left\{\Gamma(V)\right\}\)が
成りたつことをいう.
\(U\in\mathscr{U}\)で\(U'\)が\(U\)の連結開集合のとき,
合成
\[
    \Gamma(U)\to\Gamma(U')\to\hat{\Gamma}(U')
\]
が同型射になることを示せ.\footnote{
    \(\hat{\Gamma}\)は前層\(\Gamma\)の完備化と呼ばれ,
    次のように定義される.
    \[
        \Gamma(\mathscr{U})\coloneqq\left\{(s_U)_U\in\prod_{U\in\mathscr{U}}^{}\Gamma(U);
            \text{各\(U, V\in\mathscr{U}\)に対し,
            \(s_U\rvert_{U\cap V}
            =s_V\rvert_{U\cap V}\)}\right\}
    \]とおくとき,
    \[
        \hat{\Gamma}(W)\coloneqq 
        \indlim[\mathscr{U}\in\Cov(W)]\Gamma(\mathscr{U}).
    \]要するに層化.
    \cite[p.78]{Sp66}
    }
\(\Gamma\)が局所定数層で,
\(U'\)が\(U\in\mathscr{U}\)の連結開集合であるとき,
\(\Gamma(U)\approx\Gamma(U')\)となることを示せ.


\subparagraph{3}
\(X\)が局所弧状連結で\(\Gamma'\)が\(X\)上の局所定数層であるとき,
\(X\)上の局所系\(\Gamma\)で\(\bar{\Gamma}\approx\Gamma'\)と
なるものが存在することを示せ.

\subparagraph{4}
\(X\)が局所弧状連結かつ半局所1連結\footnote{
    空間\(X\)が\textbf{半局所1連結} (semilocally 1-connected) で
    あるとは,
    各点\(x_0\in X\)に対し近傍\(N\)で\(
        \pi(N,x_0)\to\pi(X,x_0)
    \)が自明となることをいう.
    \cite[p.78]{Sp66}
}であるとき,
\(X\)上の局所系の同値類と
\(X\)上の局所定数層の同値類と
の間に一対一対応が存在することを示せ.

\clearpage
\section{局所定数層の分類{\cite[IV.9]{Iv}}}

同変\(k\)層を導入する.
\(k\)を可換環とし,\(X\)を位相空間で群\(G\)が右から作用するもの
とするとき,層\(E\)に対する\(G\)作用とは,\(k\)層の射
\(a_\sigma\colon E\to \sigma_{\ast}E\)で次をみたすものの族
\((a_\sigma)_{\sigma\in G}\)のことである.
\begin{equation}
    a_{\sigma\tau}=\tau_{\ast}a_\sigma\circ a_{\tau}. \quad(\sigma,\tau\in G)\label{Iv91}
\end{equation}
\(G\)作用をもつ層を\(G\)同変層\(E\)という.
\(k\)層の射で任意の\(\sigma\in G\)に対し
次の図式が可換になるものを
\(G\)同変層の射\(h\colon E\to F\)という.
\[
    \vcenter{\xymatrix@C=26pt@R=26pt{
        E
        \ar[r]^-{}
        \ar[d]^-{h}
        &
        \sigma_\ast E
        \ar[d]^-{\sigma_\ast h} 
        \\
        F \ar[r]^-{}
        &
        \sigma_\ast F.
    }}
\]
\(X\)上の\(G\)同変層の圏を\(\Sh^G(X,k)\)とかく.
以下で\(B=X/G\)とおき,\(f\colon X\to B\)で射影を表す.
ここで,関手を2つ導入する.
\begin{equation}
    \vcenter{\xymatrix@C=26pt@R=26pt{
        \Sh^G(X,k)
        \ar@<-0.5ex>[r]_-{\Inv^G}
        &
        \Sh(B,k)
        \ar@<-0.5ex>[l]_-{f^\ast}.
    }}\label{Iv92}
\end{equation}
まず,\(X\)上の\(k\)層\(E\)への\(G\)作用は,随伴を考えることで,
\(a^{\sigma}\colon\sigma^\ast E\to E\)という同型によって表すことも
できることに注意する.
したがって,\(B\)上の層\(F\)に対し,
標準的な同型\[
    \sigma^\ast f^\ast F\to f^\ast F\quad (\sigma\in G)
\]によって\(f^\ast F\)への\(G\)作用が定まる.
いま,\(E\)を\(X\)上の\(G\)同変層とする.
\(B\)の開集合\(V\)での\(\Inv^G E\)の切断について述べるために,
式\eqref{Iv91}から,\(\Gamma(f^{-1}(V),E)\)への左\(G\)作用が
定まることに注意する.
このとき,
\[
    \Gamma(V;\Inv^G E)\coloneqq 
    \Gamma(f^{-1}(V),E)^G
\]
とおく.ただし,\(\Gamma(f^{-1}(V),E)^G\)は
\(E\)の\(f^{-1}(V)\)における\(G\)不変な切断のなす集合である.
\(f^\ast\)は\(\Inv^G\)の左随伴関手である.
すなわち,任意の\(F\in\Sh(B,k)\), \(E\in\Sh^G(X,k)\)に対し
\begin{equation}
    \Hom_F(f^\ast{F},E)\cong \Hom(F,\Inv^G{E})
\end{equation}
である.
証明は読者に任せる.
\(G\)作用をもつ\(k\)加群\(N\)に対し,
\(N\)上の\(G\)作用を対応させることができる.
\(\sigma\in G\)と\(X\)の開集合\(U\)に対し,
次の写像を対応させる.
\begin{equation}
    \Hom(U,N)\to \Hom(\sigma(U),N);
    \quad 
    \varphi\to \sigma_N\circ \varphi\circ\sigma.\label{94}
\end{equation}

\begin{LMM}\(X\)が連結なら,
    関手\(E\to \Gamma(X,E)\)は\(X\)上の\(G\)同変層で
    下部\(k\)層が定数層であるものの圏と,
    左\(G\)作用をもつ\(k\)加群の圏の同値を与える.
\end{LMM}

\begin{proof}
    逆関手は\eqref{94}で与えられる.
\end{proof}

\begin{LMM}
    各点\(b\in B\)に対し,
    開近傍\(V\)で\(f^{-1}(V)\to V\)が
    \(\pr\colon V\times G\to V\)と
    \(G\)同変に同型であるとする.
    このとき,\(f^\ast\)と\(\Inv^G\)は\(B\)上の
    局所定数層の圏と\(G\)同変層で
    下部\(k\)層が定数層であるものの圏の間の圏同値をひきおこす.
\end{LMM}




%===============================================
% 参考文献スペース
%===============================================
\begin{thebibliography}{20} 
    \bibitem[B+84]{B+84} Borel, 
    \textit{Intersection Cohomology}, 
    Progress in Mathematics, 50, Birkh\"auser, 1984.
    \bibitem[Iv]{Iv} Iversen, Cohomology of Sheaves.
\bibitem[KS90]{KS90} Masaki Kashiwara, Pierre Schapira, 
    \textit{Sheaves on Manifolds}, 
    Grundlehren der Mathematischen Wissenschaften, 292, Springer, 1990.
\bibitem[KS06]{KS06} Masaki Kashiwara, Pierre Schapira, 
    \textit{Categories and Sheaves}, 
    Grundlehren der Mathematischen Wissenschaften, 332, Springer, 2006.
\bibitem[Sh16]{Sh16} 志甫淳, 層とホモロジー代数, 共立出版, 2016.
\bibitem[Sp66]{Sp66} Spanier: Algebraic Topology. McGraw-Hill, 1966.
\end{thebibliography}

%===============================================


\end{document}
