
%Don't forget to delete
%showkeys
%overfullrule
%\date \ber \er \cmt

%------------------------
\documentclass[9pt, a4paper, dvipdfmx]{jlreq}


%usepackage
%------------------------
\usepackage{amsmath}
\usepackage{amsthm}
%\usepackage[psamsfonts]{amssymb}
\usepackage{color}
\usepackage{ascmac}
\usepackage{amsfonts}
\usepackage{mathrsfs}
\usepackage{mathtools}
\usepackage{amssymb}
\usepackage{graphicx}
\usepackage{fancybox}
%\usepackage{enumerate}
\usepackage{enumitem}
\usepackage{verbatim}
\usepackage{subfigure}
\usepackage{proof}
\usepackage{listings}
\usepackage{otf}
\usepackage{algorithm}
\usepackage{algorithmic}
\usepackage{tikz}
\usetikzlibrary{cd}
\usepackage[all]{xy}
\usepackage{amscd}

\usepackage{pb-diagram}

\usepackage[dvipdfmx]{hyperref}
\usepackage{xcolor}
\definecolor{darkgreen}{rgb}{0,0.45,0} 
\definecolor{darkred}{rgb}{0.75,0,0}
\definecolor{darkblue}{rgb}{0,0,0.6} 
\hypersetup{
    colorlinks=true,
    citecolor=darkgreen,
    linkcolor=darkred,
    urlcolor=darkblue,
}
\usepackage{pxjahyper}

\usepackage{enumitem}

\usepackage{bbm}

% ================================
% パッケージを追加する場合のスペース 
\usepackage{latexsym}
\usepackage{wrapfig}
\usepackage{layout}
\usepackage{url}

\usepackage{okumacro}
%\usepackage{endnotes}
\usepackage[french]{babel}

%=================================


% --------------------------
% theoremstyle
% --------------------------
\theoremstyle{definition}


% --------------------------
% newtheoem
% --------------------------

% 日本語で定理, 命題, 証明などを番号付きで用いるためのコマンドです. 
% If you want to use theorem environment in Japanece, 
% you can use these code. 
% Attention!
% All theorem enivironment numbers depend on 
% only section numbers.
\newtheorem{Axiom}{公理}[section]
\newtheorem{Definition}[Axiom]{定義}
\newtheorem{Theorem}[Axiom]{定理}
\newtheorem{Proposition}[Axiom]{命題}
\newtheorem{Lemma}[Axiom]{補題}
\newtheorem{Corollary}[Axiom]{系}
\newtheorem{Example}[Axiom]{例}
\newtheorem{Claim}[Axiom]{主張}
\newtheorem{Property}[Axiom]{性質}
\newtheorem{Attention}[Axiom]{注意}
\newtheorem{Question}[Axiom]{問}
\newtheorem{Problem}[Axiom]{問題}
\newtheorem{Consideration}[Axiom]{考察}
\newtheorem{Alert}[Axiom]{警告}
\newtheorem{Fact}[Axiom]{事実}


% 日本語で定理, 命題, 証明などを番号なしで用いるためのコマンドです. 
% If you want to use theorem environment with no number in Japanese, You can use these code.
\newtheorem*{Axiom*}{公理}
\newtheorem*{Definition*}{定義}
\newtheorem*{Theorem*}{定理}
\newtheorem*{Proposition*}{命題}
\newtheorem*{Lemma*}{補題}
\newtheorem*{Example*}{例}
\newtheorem*{Corollary*}{系}
\newtheorem*{Claim*}{主張}
\newtheorem*{Property*}{性質}
\newtheorem*{Attention*}{注意}
\newtheorem*{Question*}{問}
\newtheorem*{Problem*}{問題}
\newtheorem*{Consideration*}{考察}
\newtheorem*{Alert*}{警告}
\newtheorem{Fact*}{事実}


% 英語で定理, 命題, 証明などを番号付きで用いるためのコマンドです. 
% If you want to use theorem environment in English, You can use these code.
%all theorem enivironment number depend on only section number.
\newtheorem{Axiom+}{Axiom}[section]
\newtheorem{Definition+}[Axiom+]{Definition}
\newtheorem{Theorem+}[Axiom+]{Theorem}
\newtheorem{Proposition+}[Axiom+]{Proposition}
\newtheorem{Lemma+}[Axiom+]{Lemma}
\newtheorem{Example+}[Axiom+]{Example}
\newtheorem{Corollary+}[Axiom+]{Corollary}
\newtheorem{Claim+}[Axiom+]{Claim}
\newtheorem{Property+}[Axiom+]{Property}
\newtheorem{Attention+}[Axiom+]{Attention}
\newtheorem{Question+}[Axiom+]{Question}
\newtheorem{Problem+}[Axiom+]{Problem}
\newtheorem{Consideration+}[Axiom+]{Consideration}
\newtheorem{Alert+}{Alert}
\newtheorem{Fact+}[Axiom+]{Fact}
\newtheorem{Remark+}[Axiom+]{Remark}

% ----------------------------
% commmand
% ----------------------------
% 執筆に便利なコマンド集です. 
% コマンドを追加する場合は下のスペースへ. 

% 集合の記号 (黒板文字)
\newcommand{\NN}{\mathbb{N}}
\newcommand{\ZZ}{\mathbb{Z}}
\newcommand{\QQ}{\mathbb{Q}}
\newcommand{\RR}{\mathbb{R}}
\newcommand{\CC}{\mathbb{C}}
\newcommand{\PP}{\mathbb{P}}
\newcommand{\KK}{\mathbb{K}}


% 集合の記号 (太文字)
\newcommand{\nn}{\mathbf{N}}
\newcommand{\zz}{\mathbf{Z}}
\newcommand{\qq}{\mathbf{Q}}
\newcommand{\rr}{\mathbf{R}}
\newcommand{\cc}{\mathbf{C}}
\newcommand{\pp}{\mathbf{P}}
\newcommand{\kk}{\mathbf{K}}

% 特殊な写像の記号
\newcommand{\ev}{\mathop{\mathrm{ev}}\nolimits} % 値写像
\newcommand{\pr}{\mathop{\mathrm{pr}}\nolimits} % 射影

% スクリプト体にするコマンド
%   例えば {\mcal C} のように用いる
\newcommand{\mcal}{\mathcal}

% 花文字にするコマンド 
%   例えば {\h C} のように用いる
\newcommand{\h}{\mathscr}

% ヒルベルト空間などの記号
\newcommand{\F}{\mcal{F}}
\newcommand{\X}{\mcal{X}}
\newcommand{\Y}{\mcal{Y}}
\newcommand{\HH}{\mcal{H}}
\newcommand{\RKHS}{\Hil_{k}}
\newcommand{\Loss}{\mcal{L}_{D}}
\newcommand{\MLsp}{(\X, \Y, D, \Hil, \Loss)}

% 偏微分作用素の記号
\newcommand{\p}{\partial}

% 角カッコの記号 (内積は下にマクロがあります)
\newcommand{\lan}{\langle}
\newcommand{\ran}{\rangle}



% 圏の記号など
\newcommand{\Set}{{\bf Set}}
\newcommand{\Vect}{{\bf Vect}}
\newcommand{\FDVect}{{\bf FDVect}}
\newcommand{\Ring}{{\bf Ring}}
\newcommand{\Ab}{{\bf Ab}}
\newcommand{\Mod}{\mathop{\mathrm{Mod}}\nolimits}
\newcommand{\CGA}{{\bf CGA}}
\newcommand{\GVect}{{\bf GVect}}
\newcommand{\Lie}{{\bf Lie}}
\newcommand{\dLie}{{\bf Liec}}



% 射の集合など
\newcommand{\Map}{\mathop{\mathrm{Map}}\nolimits}
\newcommand{\Hom}{\mathop{\mathrm{Hom}}\nolimits}
\newcommand{\End}{\mathop{\mathrm{End}}\nolimits}
\newcommand{\Aut}{\mathop{\mathrm{Aut}}\nolimits}
\newcommand{\Mor}{\mathop{\mathrm{Mor}}\nolimits}

% その他便利なコマンド
\newcommand{\dip}{\displaystyle} % 本文中で数式モード
\newcommand{\e}{\varepsilon} % イプシロン
\newcommand{\dl}{\delta} % デルタ
\newcommand{\pphi}{\varphi} % ファイ
\newcommand{\ti}{\tilde} % チルダ
\newcommand{\pal}{\parallel} % 平行
\newcommand{\op}{{\rm op}} % 双対を取る記号
\newcommand{\lcm}{\mathop{\mathrm{lcm}}\nolimits} % 最小公倍数の記号
\newcommand{\Probsp}{(\Omega, \F, \P)} 
\newcommand{\argmax}{\mathop{\rm arg~max}\limits}
\newcommand{\argmin}{\mathop{\rm arg~min}\limits}





% ================================
% コマンドを追加する場合のスペース 
\newcommand{\UU}{\mcal{U}}
\newcommand{\OO}{\mcal{O}}
\newcommand{\emp}{\varnothing}
\newcommand{\ceq}{\coloneqq}
\newcommand{\sbs}{\subset}
\newcommand{\mapres}[2]{\left. #1 \right|_{#2}}
\newcommand{\ded}{\hfill $\blacksquare$}
\newcommand{\id}{\mathrm{id}}
\newcommand{\isom}{\overset{\sim}{\longrightarrow}}
\newcommand{\tTop}{\textsf{Top}}
\newcommand{\pfb}{\textbf{証明}}
\newcommand{\Int}{\mathop{\mathrm{Int}}\nolimits} % 内部

\newcommand{\HOM}{\mathop{\mathscr{H}\hspace{-2pt}om}\nolimits}%内部Hom

\newcommand{\RHOM}{\mathop{\mathrm{R}\hspace{-1.5pt}\HOM}\nolimits}


% 自前の定理環境
%   https://mathlandscape.com/latex-amsthm/
% を参考にした
\newtheoremstyle{mystyle}%   % スタイル名
    {5pt}%                   % 上部スペース
    {5pt}%                   % 下部スペース
    {}%              % 本文フォント
    {}%                  % 1行目のインデント量
    {\bfseries}%                      % 見出しフォント
    {.}%                     % 見出し後の句読点
    {12pt}%                     % 見出し後のスペース
    {\thmname{#1}\thmnumber{ #2 }\thmnote{{\normalfont (#3)}}}% % 見出しの書式

\theoremstyle{mystyle}
\newtheorem{AXM}{公理}[section]
\newtheorem{DFN}[Axiom]{定義}
\newtheorem{THM}[Axiom]{定理}
\newtheorem*{THM*}{定理}
\newtheorem{PRP}[Axiom]{命題}
\newtheorem{LMM}[Axiom]{補題}
\newtheorem{CRL}[Axiom]{系}
\newtheorem{EG}[Axiom]{例}

%\newtheorem{}{Axiom}[]
\numberwithin{equation}{section} % 式番号を「(3.5)」のように印刷

\newcommand{\MM}{\mcal{M}}

% =================================


% ---------------------------
% new definition macro
% ---------------------------
% 便利なマクロ集です

% 内積のマクロ
%   例えば \inner<\pphi | \psi> のように用いる
\def\inner<#1>{\langle #1 \rangle}

% ================================
% マクロを追加する場合のスペース 

%=================================





% ----------------------------
% documenet 
% ----------------------------
% 以下, 本文の執筆スペースです. 
% Your main code must be written between 
% begin document and end document.
% ---------------------------




\begin{document}

\title{収録されなかった原稿}
\author{Pierre Schapira}
\date{}
\maketitle
\begin{abstract}
    Inference Vol.\,7, No.\,3
    に掲載された
    Pierre Schapira による書評 ``A Truncated Manuscript'' の翻訳.
\end{abstract}


%\section*{2023/01/09}
厳密に言うと,この小文は,Alexander Grothendieck による
R\'ecolts et Semailles(収穫と蒔いた種と)の
書評のためだけのものではない\footnote[1]{
    Leila Schnepsの批判的かつ建設的な忠告に感謝する.
}.
全体として,本については,Grothendieckの仕事と人生とともに,
ここで論じられはするけれど,本稿(小文)の大部分は,Grothendieckが
原文全体を通して重ねた主張を論駁するのに充てられている.
この点については重要な支持者がいる.著者本人である.
今度の新版には収録されていない重要な追記のなかで,
Grothendieckは自身の言いがかりのうち,そのいくつかは撤回している.

Grothendieckという人物は,20世紀後半の間,数学の多くの部分を支配した.
彼の仕事が本質的には代数幾何学に関連するとしても,
彼のものの見方と手法は代数幾何学を超えて大きく広まった.
それは,代数的位相幾何学,表現論,複素幾何学,シンプレクティック幾何学,
代数解析学,そして近年においては,計算幾何学にさえも及ぶ.
端的にいって,力学系や確率論,そしてリーマン幾何学やハミルトン幾何学のような
純粋に幾何学的な幾何学を除く,あらゆる線形数学である.
導来圏や層の理論がこれらの分野において建設されたのは,
彼の影響下においてである.
導来圏の理論の根幹を直感的に理解し,その大筋の部分を
定式化したのもGrothendieckである.
彼は自身の学生である Jean-Louis Verdier に,学位論文として,
理論の詳細を全て書き出させた.
その学位論文において,Verdierは三角圏という重要な概念を明確にした
\footnote[2]{
    Alexander Grothendieck,
    \textit{R\'ecoltes\ et\ Semailles:\ I,\ II.\ R\'eflexions\ et\ t\'emoignage\ sur\ un\ pass\'e\ de\ math\'ematicien} 
    (Paris: Gallimard, 2022), 439.
}.
しかし,後述するように,
導来圏の理論の枠組みで層の理論と六則演算を作ったのはGrothendieckである.

1950年から1970年まで,
関数あるいは一般化関数は,フーリエ変換を用いて,
実または複素多様体の,とりわけユークリッド空間の上で研究されていた.
ところが,複素多様体の上では,
我々が関数というとき実際には正則関数を意味するが,
この関数が重大な困難をもたらす.
どういうことかというと,存在しないのである.
射影直線のようなコンパクト多様体上の正則関数は,
少なくとも大域的には,もちろん定数関数は除いて,存在しない.
したがって,実可微分多様体とは異なり,
大域的な知識は何の情報ももたらさず,局所的に調べるしかない.
この目的のための素晴らしい道具がある.層の理論である.
これはJean Lerayが1941年から1945年までの間
ドイツで戦争捕虜になっていた時に発明された
\footnote[3]{
    Christian Houzelによる歴史に関する次の記事を参照せよ.
    ``Les d\'ebut de la th\'eorie des faisceaux,'' in Masaki Kashiwara and Pierre Schapira, \textit{Sheaves on Manifolds, Grundlehren der Mathematischen Wissenschaften, vol.\,292} (Berlin: Springer-Verlag, 1990), doi:10.1007/978-3-662-02661-8.
}.
Lerayの原論文はどこか読みづらいところがあったが,
のちにHenri CartanとJean-Pierre Serreによって整理され,
Th\'eorie des Faisceaux(層の理論)\footnote[4]{
    Roger Godement, \textit{Th\'eorie des faisceaux} (Pairs: Hermann, 1958).
}というRoger Godementの
有名な本にまとめられた.
CartanとSerreはこの道具を次元$\geqq1$における正則関数を調べるという,
岡潔による金字塔以来の研究にそれぞれ用いて,
Cartanの定理A, BとSerreによる
重要な論文``Faisceaux alg\'ebrique coh\'erents''(代数的連接層)
\footnote[5]{
    Jean-Pierre Serre, ``Faisceaux alg\'ebrique coh\'erents,''\textit{Annals of Mathematics, 2nd Series} 61, no.\,2. (1955): 197--278, doi:10.2307/1969915.
}が生まれた.

1955年頃にGrothendieckが代数幾何学に着手し,
東北数学雑誌から出版された基礎的な論文\footnote[6]{
    Alexander Grothendieck, 
    ``Sur quelques points d'alg\'ebre homologique,'' 
    \textit{T\=ohoku Mathematical journal} 9, no.\,3 (1957): 
    119--21, doi:10.2748/tmj/1178244774. 
    この記事はRick Jardineが詳細に分析している.
    ``T\=ohoku,'' \textit{Inference} 1, no.\,3 (2015), 
    doi:10.37282/991819.15.13.
}において
層コホモロジーの基礎づけを与えたのは,この文脈においてである.
この論文において,圏論の重大な困難,
すなわち宇宙の問題というGrothendieckによりSGA4で解決される問題
が暗に現れている.
\renewcommand{\thefootnote}{[訳注]}
その解決は快刀乱麻を断つ方法でなされた\footnote{
    原文では「もう一人のアレクサンダー(大王)が
    ゴルディアスの結び目を断つ方法でなされた」という,
    Grothendieckのファーストネームを
    もじった表現になっている
}
\renewcommand{\thefootnote}{\arabic{footnote})}
\footnote[7]{
    Mike Artin, Alexandre Grothendieck, and Jean-Louis Verdier, 
    \textit{The\'eorie des topos et 
    cohomologie \'etale des sch\'emas, 
    Lecture Notes in Mathematics}, 
    vols.\,269, 270, 305 (Berlin: Springer-Verlag, 1972--73).
}.Grothendieckは,
どの集合もある宇宙に含まれるという公理を付け加えたのである.
宇宙は到達不能基数という,
圏論を運用することができる範囲を超えたものとしても知られる.
この問題が原因で,
ブルバキは圏論を扱うことを諦め,
Grothendieckはブルバキを離れたのだと思われる.
この問題についてはRalf Kr\"omerが素晴らしい記事\footnote[8]{
    Ralf Kr\"omer, 
    ``La {\og{machine de Grothendieck}\fg} 
    se fonde-t-elle seulement sur des vocables 
    m\'etamath\'ematiques? 
    Bourbaki et les cat\'egories au 
    cours des ann\'ees cinquante,'' 
    \textit{Revue d'histoire des math\'ematiques} 12 (2006):119--62.
}を書いている.

%\section*{2023/01/10}

1940年代から1950年代の間に起きたものの,
その重要性がすぐには理解されなかった2つの概念の革命がある.
上で述べた層の理論と圏論である.
後者の革命を起こしたのはSamuel EilenbergとSaunders Mac Laneである\footnote[9]{
    Samuel Eilenberg and Saunders Mac Lane, 
    ``Natural Isomorphisms in Group Theory,'' 
    \textit{Proceedings of the National Academy of Sciences} 
    28 (1942): 537--43, doi:10.1073/pnas.28.12.537;
    Samuel Eilenberg and Saunders Mac Lane, ``General Theory of Natural Equivalences,'' 
    \textit{Transactions of the American Mathematical Society} 
    58 (1945): 231--94, doi:10.1090/S0002-9947-1945-0013131-6.
}.もっというと,圏論的な視点は,
Claude L\'evi-Straussの構造主義思想やNoam Chomskyの言語学に代表される
大きな思想運動の一部である.
圏論は,何らかの構造を備えた集合を考えるのではなく,
対象間に存在しうる関係に焦点を当てる.
どういうことかというと,
圏$C$は,集合が元の族であるのと同じように対象の族であるが,
2つの対象$X$と$Y$が与えられると,
そこには$\Hom_C(X,Y)$と呼ばれる,
$X$から$Y$への射を表す集合がアプリオリに存在する.
これらのデータは,もちろん射の合成や恒等射などいくつかの公理に従う.
新しい一歩は,関手と呼ばれる,圏の間の射に注目するというところにある.
そうすることで,随伴関手や終対象と始対象,極限と余極限といった
重要な概念が現れ,数学の根底にある多くのアイデアに対し,
正確かつ統一的な意味づけを与えるのである.

圏の族のうち,中心的な役割を担うものがある.
それは,加法圏とその中のアーベル圏という,
環上の加群の圏をモデルとする圏の族である.
ところが,体上のベクトル空間を環上の加群に取り替えると,
テンソル積をとる関手や内部$\Hom$をとる関手が存在しなくなってしまう.
つまり,それらの操作によって,完全列は完全列にうつらなくなる.
部分空間が補空間をもつとも限らなくなる.
そこで,線形代数の自然な一般化であるホモロジー代数の領域に入る.
ここで,Grothendieckの東北論文にとって代わられる前の初期の参考文献は
CartanとEilenbergによるものであった\footnote[10]{
    Henri Cartan and Samuel Eilenberg, 
    \textit{Homological Algebra} 
    (Princeton, NJ: Princeton University Press, 1956).
}.
しかし,2つの関手の合成に対する導来関手を求めるためには,
Lerayのスペクトル系列という,
しばしば煩雑な計算を要するものを用いなければならなかった.
ここで導来圏がその力を発揮する.
この言葉の下では,あらゆることが驚くほど単純になる.

六則演算とは何か.通常の関数ついては,和の他に3つの自然な演算がある.
積と,実多様体の間の写像$f\colon X\to Y$から定まる,
積分という(技術的な細部を省くと)$X$上の関数を$Y$上の関数にうつす演算,
そして,$f$の合成\footnote{
    [訳注]ふつう,引き戻しと呼ばれる.
}という,$Y$上の関数を$X$上の関数にうつす演算である.
層の理論においては,テンソル積$\overset{\mathrm{L}}{\otimes}$が積の類似,
固有順像$Rf_!$が積分の類似,そして,逆像$f^{-1}$が$f$による合成の類似である.
ところが,テンソル積には$\RHOM$という右随伴が,
関手$f^{-1}$には順像$Rf_{\ast}$という右随伴が,
そして,関手$Rf_{!}$には右随伴$f^!$が存在する.

関手$f^!$は他の5つとは異なり,導来圏の枠組みの中でしか存在しない.
これはエタールコホモロジーの文脈でGrothendieckが発見し,
その後局所コンパクト空間の場合にVerdierが定式化した.
Grothendieckが見通したように,
$f^!$はPoincar\'e双対の広い一般化になっており,
いまや中心的な役割を担っている.
しかし,エタール位相よりも局所コンパクト空間の方が頻出するため,
双対性に残っているのは,Verdier一人の名前ではないものの,
Poincar\'e-Verdierという名前である.
この名前の帰属のさせ方は大いに不公平であり,Grothendieck
はいくらか苦い思いをしたであろうが無理もないことである\footnote[11]{
    Grothendieck,
    \textit{R\'ecoltes\ et\ Semailles}, 158.
}.

このような抽象的な枠組みによって,
具体的な計算が必要なくなると考える向きもあるかもしれないが,
それは誤解である.
簡単にいうと,計算の仕方がまるっきり異なるのである.
順像関手を用いても積文を明示的に求められはしないとしても,
六則演算の定式化によって得られるのは,
やはりホモロジー空間の次元の計算のような洗練された数値結果である.
Riemann-Roch-Hilzebruch-Grothendieckの定理は美しい例である.

同じくGrothendieckの基本的な発見の1つに,層の理論を圏の上に,
したがって特に,
点を一切もたない空間の上に構築するというものがある.
層が存在するためには何が必要であろうか.
それは,開集合とその包含関係のデータと,被覆の概念である.
圏の対象に開集合の役割を担わせてはならない理由など何処にもなく,
このとき,この圏は前景(準景)と呼ばれる.
あとは被覆が何かを公理的に定めれば,景,
すなわち,Grothendieck位相の入った圏が得られる.
このように普通の位相空間を自然に一般化しておくことは,
非常に役立つことが示される.
そして,解析学者もそこから発想を得ることができるであろう.
実多様体上には,開集合の淵で何が起こるかを調べようとすると,
あまりにも多くの病的な開集合とあまりにも多くの被覆が現れる.

そしてトポスの理論に至る.--- 学問的にはトポイという.
その根底にあるのは,
空間,今の場合の景だが,
これは景上の層の圏から復元できるというアイデアである.
これは特別な場合についてIsrael Gelfandまで遡る.
このときトポスは層のなす圏と同値な圏である.
例えば,集合の圏は,1点をトポスとみなしたものに他ならない.
しかし,連続体仮説の独立性に関するPaul Cohen による
新しい証明にトポス理論が用いられたとしても,
数学における応用は不確かなままである.

ここで紹介したGrothendieckの代表的な仕事の選択は完全とはほど遠く,
評者の興味しか反映していない.
R\&SにおいてGrothendieckは,自身の仕事において鍵となる,
自分の思う12の考え方を列挙し,自分の一覧も載せている\footnote[12]{
    Grothendieck,
    \textit{R\'ecoltes\ et\ Semailles}, 42, 377.
}.

1955年ごろからの関数解析についての彼の初仕事や,
代数幾何学に革命をもたらした(一新した)概形(スキーム)理論,
部分的に予想し,
のちにPierre DeligneやVladimir Voevodsky, Joseph Ayoub, 
他にも多くの人々の手によって発展した理論である,
モチーフについての直観についてももちろん言及しなければならない.
Grothendieckが$\infty$圏とホモトピー代数についての礎を築いた
重要な論文 ``\textit{A la poursuit des champs}'' についても
述べるべきである\footnote[13]{
    Alexander Grothendieck, ``A la poursuit des champs,'' (1987).
}.
実は,三角圏は驚くほど単純で効果的な道具であるとしても,
例えば,貼り合わせの問題など適用範囲が限られるという欠点がある.
この欠点に関係しているのは次の事実である.
ある射は同型なものを除いてただ一つしか存在しないのだが,
その同型は一意的ではないのである!この$\infty$圏という新理論には,
Jacob Lurie, Graeme Segel, Bertrand T\"oen, 
他にも多くの人が名を連ねるが,それは,導来圏の古典的な理論に
完全に取って代わられる最中にある.
それでも,当面は,それは控えめに言ってとっつきづらいものである.


%\section*{2023/01/11(2023/08/30に入力.)}

Grothendieckの生涯についてはPierre Cartierが
素晴らしい記事を書いており,ここで言いかえても無意味である.
\footnote[14]{
    Pierre Cartier, ``A Country Known Only by Name,'' 
    Inference 1, no. 1 (2014), \url{doi:10.37282/991819.14.7.} 
    追加の詳細については “Sascha Shapiro,” Wikipedia をご覧いただきたい.
}
Allyn JacksonとWinfried Scharlauによる,これまた素晴らしい記事もある.
全てLeila Schnepsの運営しているサイトThe Grothendieck Circleに
リンクがある.
\footnote[15]{
    Allyn Jackson, 
    ``Comme Appel\'e du N\'eant\text{---}As if 
    Summoned from the Void: 
    The Life of Alexandre Grothendieck,'' 
    Notices of the AMS 51, no.\ 4 and 51, no.\ 10 (2004): 
    1,038–-54 and 1,196–-212; 
    Winfried Scharlau, ``Who is Alexander Grothendiek?,'' 
    Notices of the AMS 55, no. 8 (2008): 930–-41; 
    Leila Schneps, ``The Grothendieck Circle'' 
    (grothendieckcircle.org).
}

それでも,この話題について少しでも述べておくのは無駄なことではない.
Grothendieckの父親Sacha Schapiroはロシアの無政府主義者で,
1905年の中止された革命に参加していた.
そこから,1917年の革命によって解放されるまで,
ニコライ2世刑務所に10年間拘留されていた.
はじめは英雄として尊敬されていたにも関わらず,
Schapiroはまもなく人民の敵とされてしまったのである.
%\footnote{以下2023/01/12(2023/08/30に入力.)}
彼はその後フランスで旅する写真家になるまでは,
スペイン人民戦争の間,連邦の近くで戦った.
1939年にフランス領のル・ヴェルネ収容所に送られた.
1942年にヴィシー警察によりナチスに引き渡され,
アウシュヴィッツに送られた.

Grothendieckの母親Hankaは1920年代の間,ドイツの極左過激派であったが,
Adolf Hitlerが台頭してくるとフランスへ亡命した.\footnote[16]{
    一世紀の時代の変遷から言って
    「1920年代のドイツにおける極左過激派」という言葉遣いが
    当時と全く同じ意味を持つわけではない.
}息子は1938年に10歳になるまで母親と行動を共にせず,
ドイツの牧場で隠遁していた.
Grothendieckはルシャンボン・シュル・リニョンでの戦争の一部を,
非常に多くのユダヤ人の子供を救ったことで有名なコレージュセブノールで過ごした.

Grothendieckの数学研究は1950年代にナンシーで始まった.
そこでは,Jean Dieudonn\'eとLaurent Schwartzが彼の面倒を見た.
いまでも関数解析ににおいて根幹とされる初めての仕事の後,
代数幾何に分野を変え大成功した.今ではよく知られている話である.

Grothendieckは1959年にフランス高等科学研究院 (IHES) に招致された
最初の二人の教授の一人である.
彼の結果のほとんどをそこで得て,IHESの
もう一人の教授Jean Dieudonn\'eの助力により,
かの有名な『代数幾何学原論』(EGA) を出版した.
彼は代数幾何のセミナーを主導し,その結果を自身の学生数名との共著により
5000ページを超える出版物としてまとめた.
これはSGA (S\'eminaire de G\'eom\'etrie Alg\'ebrique 
du Bois Marie\footnote{[訳注]マリーの森の数学セミナー}) として
知られている.
Grothendieckは1966年の国際数学者会議でフィールズメダルを受賞したが,
それを受け取るためにモスクワまで赴くことはなかった.
1988年には,権威のある,基金も多いGrafoord賞を受賞したが,
彼はそれを辞退した.

IHESが軍から献金を受け取っていたことが判明すると,
Grothendieckは機関を離れ,自ら環境保護活動を始めた.
はじめはSurvive,その後Survive et vivreという雑誌
を通じたものであった.
しかし,GrothendieckはIHESを去ったのみならず,
数学の世界から,とりわけ自身の学生からも離れてしまったのである.
%\footnote{以降2023/01/16}
1983年に数学に戻ってきたが,
そのやり方はまるっきり異なるものであった.
それは``Esquise d'un programme''(プログラムの概要)
と``A la pousuite des champs''(スタック
\footnote{[訳注]
    辻雄一『収穫と巻いた種と』による訳語「園」(シャン)は
    定着しているとは言い難い.
    ``champ''は社会学における用語としても用いられており,「界」という
    訳語が定着している.}
の探究)という出版物である.\footnote[17]{
    Grothendieck, ``A la poursuite des champs.''
}
コレージュ・ド・フランスで一年勤務したのち,
モンペリエールに任命され,1988年に退官するまでそこで働いた.
彼は最後の時をほとんど一人きり,郊外で過ごし,2014年に86歳で亡くなった.

%\section*{2023/01/16(2023/08/30に入力.)}

ここまで見てきたように,
Grothendieckは尋常ならざる量の数学著作の筆者である.
しかし,彼は重要な文学作品の著者でもある.
R \& Sもそれに含まれる.これは,Grothendieckが1986年に原文を書いてから,
インターネットを通じて広く配布し,2022年1月にガリマール (Gallimard) から
出版された.1900ページ以上にもわたるこの本は数多くの話題を扱っている.
著者の数学者としての生涯,情熱,誤解と幻滅,創造の過程,
その他多くの事項などが扱われている.
陰陽や男女の数学の仕方,母と父と子,夢などに関する長い文章も含まれる.
原著の大部分を占めるのは,1976年に彼が得たと言われる天啓と,
それに続く長い瞑想の期間についての文章である.
それはかなり妄想めいたと言わざるを得ない自己分析のようなものである.
繰り返し登場する話題は,かつての学生に対する,裏切られたという感覚であり,
それは彼の仕事が蔑ろにされ,忘れられていくことに対して感じたものである.
「埋葬」,「亡き」,「大量虐殺」,「墓荒らし」といった単語が
目次で最初に登場してから,すぐに至る所で現れるようになる.
より一般に,数学界全体で倫理観が欠如していることを本書は訴えている.

Grothendieckは読者に,数学は「かつて(すなわち,1960年以前)は
もっと良かった」と,まるで昔の世代に非の打ち所がないかのように
述べている!
実際はその逆で,1990年代以降,数学ははるかに公正になっている
といっていい.この奇跡の根源には名前がある.arXivである.
他人のアイデアを横取りするのは,もちろんできないことはないが,
いまだかつてないほどに難しくなっているのが現状である.
数学の研究機関それ自体も大いに改善しているか,少なくとも変化している.
Grothendieckが一切煩わされず,一言も言及しなかったフランス数学界に
1970年代まで献金していた官僚システムはほとんど無くなったのである.

原著全体を通して非常に自己批判的であるGrothendieckだが
1960年代と1970年代の全盛期の頃は傲慢であった,
あるいは周りを軽蔑していたかもしれないと考え込むことがある.
こうした思慮にも関わらず,
彼が読者の機嫌を取ることについてあまり考えていなかったのは明らかである.
それどころか,彼は1900ページ以上の本を出し,他方では,IHESができたばかり
の頃,図書館についての質問に対する回答の中で次のように述べている.
「我々は本を読むのではない.書くのである!」\footnote[18]{
    Jackson, ``Comme Appel\'e du N\'eant,'' 1,050.
}R\&Sには多くの矛盾が含まれており,(一連の付記によって)訂正されているのは
その一部のみである.その訂正についても,それなりの重要性があるのにも関わらず
この新版に収録されていないものがある.これらの矛盾を適切に改定すると,
原文を完全に書き直すことを余儀なくされたであろうことに疑いはない.

Grothendieckは謙虚さを欠いていたために麻痺していたのでは決してない.
\begin{quotation}
    \footnote{2023/08/31}私から離れた数学者で,
    このような多種多様の革新的なアイデア,
    しかもそれぞれが多少でも離れてもおらず,
    壮大で統一的な構想の一部としてのアイデアになっているもの
    を提唱した人間(物理学や
    宇宙論におけるニュートンやアインシュタイン,
    生物学におけるダーウィンやパスツールのように)は,
    私よりも歴史に精通している友人や同僚の噂からでさえも
    耳にしていないことに私は衝撃を受けた.\footnote[19]{
        Grothendieck, R\'ecoltes et Semailles, 935.
    }
\end{quotation}

別のところでは次のように書いている.
「私の中に生まれた壮大で統一的な構想に身を捧げるべき人間としては
その起源から現在までの数学の歴史において,
自分こそ「その種の一人」であるようには思う.」\footnote[20]{
    Grothendieck, R\'ecoltes et Semailles, 94.
}\footnote{
    [訳注]
    辻雄一訳では「私の中に生まれた広大な統一的ビジョンの奉仕者として、
    起源から今日に至るまでの数学の歴史において、
    私は「この種のものとしては唯一」であるように思えます。」(92ページ).
}

%\section*{2023/08/31}

書き口は発想に乏しいわけではないが,それでも
不公平であり,時には(故意に)馴れ馴れしい感じである.
Grothendieckはサン・シモン公\footnote{
    [訳注]『回想録』との類比
}ではない.

\bigskip\begin{center}
    \textbf{以下の考察では,数学と数学界に関する内容のみに焦点を絞る.}
\end{center}\bigskip

原著において,Grothendieckは長い文章を費やして
彼のアイデアがかつての学生により,言及もないままに盗まれたことや
そのアイデアが単純に抹消されたり忘れられたりしていること
に対する意義を唱えている.
%\section*{2024/09/06}
これらの主張はいつでもしっかりした論拠や正確な参考文献に
基づくものであるわけではない.
しかし,何にもまして,矮小化されその著者らが忘れられるのは発見の常であり,
その根底にあるアイデアが後から見ると当たり前であればあるほど,
その傾向は増すものである.

Grothendieckの非難は自身の学生全員に,
とりわけDeligneとVerdierに向けられている.
Deligneの名前はほぼ毎回「かつての友」であることを
仄めかしつつ「我が友」という語から始まっている.
DeligneがGrothendieckのモチーフに関する先取権にすこししか
関わっていないことや,
上述の「Verdier双対性」を「Grothendieck双対性」とも
呼びうることは想像に難くない.
しかし,他にも,スキームやモチーフ,Grothendieck位相,トポスを
発明したのがGrothendieckであること,そして何より,
六則演算と導来圏を通じた関手的視点を前面に押し出したのも彼であることは
皆知っている.
DeligneがAndr\'e Weilの最後の予想を証明することができたのも,
Grothendieckの発明した装置のおかげであることも皆知っている.
1970年代以降の数学界では倫理観が完全に欠如してしまったという
自身の主張を裏付けるためのGrothendieckの主張は全て,
郊外にある自身の家に複数回来た一人の,
そしてたった一人の数学者による唯一の証言に基づいている.

調査対象のグループ出身でその言語を話す情報提供者に頼るというのは
民俗学でよくあることである.問題なのは,
その情報提供者がいつでも信頼できるとは限らないことと,
実際には何でも言えてしまうことである.
ここではもっと酷いことが起こっている.
というのも,この情報提供者は自分こそが,
彼の話そうとしていること,すなわちRiemann-Hilbert (RH) 対応
の影響を受けた最初の人間であると宣言しているのである.

情報提供者はGrothendieckに,自分はある意味で彼の精神的な
息子 ---「私の仕事の後継者」\footnote[21]{
    Grothendieck, R\'ecoltes et Semailles, 1,664.
}であると納得させることに成功した.
そのうえ,
すこしのアドバイスも受けず,周囲の(あからさまな
敵対心とまではいかないまでも)無関心の中で
自分はRH対応の証明に成功した,
それもその分野で初めて導来圏の言葉を用いたと
情報提供者はGrothendieckを説得したのである\footnote[22]{
    Grothendieck, R\'ecoltes et Semailles, 413.
}.情報提供者についてGrothendieckは「1972年以降の彼の先駆的な仕事は
完全な孤独の中で行われた」\footnote[23]{
    Grothendieck, R\'ecoltes et Semailles, 1,663.
}とも書いている.

\emph{これらはすべて真っ赤な嘘である.}

情報提供者は自身の1974年の大学院論文を,私の指導の下,
私が彼に与えたテーマで書いている.
彼は自身の最初の論文を準備をしていた1975年当時,
柏原正樹から受けた私的な講義から大きな恩恵を受けている.
ところがその論文ではその決定的な講義についての言及は何もしていない.
彼は柏原の1970年の修士論文\footnote[24]{
    Masaki Kashiwara, 
    ``Algebraic Study of Pertial Differential Equations,'' 
    Master's thesis (Tokyo University, 1970), 
    M\'emoires de la Soci\'et\'es Math\'ematique de 
    France 63 (1995).
    [訳注] 柏原正樹『偏微分方程式の代数的研究』.
}の恩恵も受けている.この修士論文は日本語で書かれているが,
翻訳者には困らなかった.
また,Cristian Houzelから論文執筆中に繰り返しアドバイスを受けており,
その恩恵も受けている.
導来圏については,佐藤幹夫,河合隆裕,柏原正樹による
1973年出版の基礎的な論文\footnote[25]{
    Mikio Sato,
    Takahiro Kawai,
    Masaki Kashiwara, 
    ``Microfunctions and Pseudo-Differential Equations,'' 
    in \textit{Proceedings of a Conference at Katata, 
    1971; Dedicated to the Memory of Andr\'e Martineau, 
    Lecture Notes in Mathematics}, 
    vol.~287 (Berlin: Springer-Verlag, 1973), 265--529.
}の最初のページに現れている.

RH対応は柏原正樹が1975年に定式化した,ある種の「圏の同値」であり,
1980年にこれも柏原によって証明された\footnote[26]{
    以下,RH対応について少し述べる.
    現代的な定式化は\(D\)加群の理論を用いる.
    これは1960年代に佐藤によって創始され,
    柏原の修論で完全な形で展開された.(関連の理論は
    Joseph Bernsteinによって独立に展開された.)
    日常的な言葉では,連接\(D\)加群は正則関数係数の
    偏微分方程式系のことである.
    ホロノミー加群は次元\(>1\)の場合の常微分方程式で,
    その中にある確定特異点型ホロノミー加群は
    フックス型方程式という古典的な概念の一般化である.
    1975年に柏原は,ホロノミー加群に対し
    正則関数解の複体を対応させる関手\(\mathrm{Sol}\)が
    構成可能層という,局所的には,
    滑層分割に沿った定数層の直和として振る舞う層に値をとることを証明した.
    同年,ホロノミー加群の部分三角圏で,
    関手\(\mathrm{Sol}\)が「圏の同値」をひきおこすもの,
    すなわち確定特異点型ホロノミー加群の圏が存在することも柏原が予想した.
    柏原は自身の予想を1980年に証明し,
    その主要なステップについての詳細な説明を
    エコールポリテクニークでのセミナーで行い,出版もされた.
    その証明には,広中平祐の特異点解消定理と,「有理型接続」という
    特殊な場合を扱った,Deligneによる先行研究が用いられる.
}.ついでに,面白い圏同値は,最初は無関係に思える異分野間の橋渡しをする
ということに言及しておいて損はない.
ここでは解析学における偏微分方程式と代数的位相幾何学における構成可能層である.
もうひとつ,
もっと最近の非常に重要な圏同値はMaxim Kontsevichの「ミラー対称性」に
よって与えられるもので,こちらは
複素幾何学とシンプレクティック幾何学とを結びつけるものである.

RH対応をめぐる自身の弟子について,
1,900ページにわたる原文全体に散乱し繰り返される
Grothendieckの主張全体は,したがって誤った証言に基づいている.
しかし,






%\date{December 13, 2021}

%\maketitle
%\setcounter{page}{1}
%\providecommand{\bysame}{\leavevmode\hbox to3em{\hrulefill}\thinspace}
%\begin{thebibliography}{15}
%\end{thebibliography}
%\bibliographystyle{junsrt}
%\bibliography{ref}

\end{document}


