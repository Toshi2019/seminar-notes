%=====================================
%   geometry.tex
%   幾何学まとめノート toshi2019
%=====================================

% -----------------------
% preamble
% -----------------------
% ここから本文 (\begin{document}) までの
% ソースコードに変更を加えた場合は
% 編集者まで連絡してください. 
% Don't change preamble code yourself. 
% If you add something
% (usepackage, newtheorem, newcommand, renewcommand),
% please tell it 
% to the editor of institutional paper of RUMS.

% ------------------------
% documentclass
% ------------------------
\documentclass[10pt, a4paper, dvipdfmx]{jsarticle}

% ------------------------
% usepackage
% ------------------------
\usepackage{algorithm}
\usepackage{algorithmic}
\usepackage{amscd}
\usepackage{amsfonts}
\usepackage{amsmath}
\usepackage[psamsfonts]{amssymb}
\usepackage{amsthm}
\usepackage{ascmac}
\usepackage{bm}
\usepackage{color}
\usepackage{enumerate}
\usepackage{fancybox}
\usepackage[stable]{footmisc}
\usepackage{graphicx}
\usepackage{listings}
\usepackage{mathrsfs}
\usepackage{mathtools}
\usepackage{otf}
\usepackage{pifont}
\usepackage{proof}
\usepackage{subfigure}
\usepackage{tikz}
\usepackage{verbatim}
\usepackage[all]{xy}

\usetikzlibrary{cd}
\usetikzlibrary{arrows.meta}


% ================================
% パッケージを追加する場合のスペース 
\usepackage[dvipdfmx]{hyperref}
\usepackage{xcolor}
\definecolor{darkgreen}{rgb}{0,0.45,0} 
\definecolor{darkred}{rgb}{0.75,0,0}
\definecolor{darkblue}{rgb}{0,0,0.6} 
\hypersetup{
    colorlinks=true,
    citecolor=darkgreen,
    linkcolor=darkred,
    urlcolor=darkblue,
}
\usepackage{pxjahyper}

\usepackage[mathscr]{euscript} % mathscr のフォントを変更


%=================================


% --------------------------
% theoremstyle
% --------------------------
\theoremstyle{definition}

% --------------------------
% newtheoem
% --------------------------

% 日本語で定理, 命題, 証明などを番号付きで用いるためのコマンドです. 
% If you want to use theorem environment in Japanece, 
% you can use these code. 
% Attention!
% All theorem enivironment numbers depend on 
% only section numbers.
\newtheorem{Axiom}{公理}[section]
\newtheorem{Definition}[Axiom]{定義}
\newtheorem{Theorem}[Axiom]{定理}
\newtheorem{Proposition}[Axiom]{命題}
\newtheorem{Lemma}[Axiom]{補題}
\newtheorem{Corollary}[Axiom]{系}
\newtheorem{Example}[Axiom]{例}
\newtheorem{Claim}[Axiom]{主張}
\newtheorem{Property}[Axiom]{性質}
\newtheorem{Attention}[Axiom]{注意}
\newtheorem{Question}[Axiom]{問}
\newtheorem{Problem}[Axiom]{問題}
\newtheorem{Consideration}[Axiom]{考察}
\newtheorem{Alert}[Axiom]{警告}
\newtheorem{Fact}[Axiom]{事実}


% 日本語で定理, 命題, 証明などを番号なしで用いるためのコマンドです. 
% If you want to use theorem environment with no number in Japanese, You can use these code.
\newtheorem*{Axiom*}{公理}
\newtheorem*{Definition*}{定義}
\newtheorem*{Theorem*}{定理}
\newtheorem*{Proposition*}{命題}
\newtheorem*{Lemma*}{補題}
\newtheorem*{Example*}{例}
\newtheorem*{Corollary*}{系}
\newtheorem*{Claim*}{主張}
\newtheorem*{Property*}{性質}
\newtheorem*{Attention*}{注意}
\newtheorem*{Question*}{問}
\newtheorem*{Problem*}{問題}
\newtheorem*{Consideration*}{考察}
\newtheorem*{Alert*}{警告}
\newtheorem{Fact*}{事実}


% 英語で定理, 命題, 証明などを番号付きで用いるためのコマンドです. 
% If you want to use theorem environment in English, You can use these code.
%all theorem enivironment number depend on only section number.
\newtheorem{Axiom+}{Axiom}[section]
\newtheorem{Definition+}[Axiom+]{Definition}
\newtheorem{Theorem+}[Axiom+]{Theorem}
\newtheorem{Proposition+}[Axiom+]{Proposition}
\newtheorem{Lemma+}[Axiom+]{Lemma}
\newtheorem{Example+}[Axiom+]{Example}
\newtheorem{Corollary+}[Axiom+]{Corollary}
\newtheorem{Claim+}[Axiom+]{Claim}
\newtheorem{Property+}[Axiom+]{Property}
\newtheorem{Attention+}[Axiom+]{Attention}
\newtheorem{Question+}[Axiom+]{Question}
\newtheorem{Problem+}[Axiom+]{Problem}
\newtheorem{Consideration+}[Axiom+]{Consideration}
\newtheorem{Alert+}{Alert}
\newtheorem{Fact+}[Axiom+]{Fact}
\newtheorem{Remark+}[Axiom+]{Remark}

% ----------------------------
% commmand
% ----------------------------
% 執筆に便利なコマンド集です. 
% コマンドを追加する場合は下のスペースへ. 

% 集合の記号 (黒板文字)
\newcommand{\NN}{\mathbb{N}}
\newcommand{\ZZ}{\mathbb{Z}}
\newcommand{\QQ}{\mathbb{Q}}
\newcommand{\RR}{\mathbb{R}}
\newcommand{\CC}{\mathbb{C}}
\newcommand{\PP}{\mathbb{P}}
\newcommand{\KK}{\mathbb{K}}


% 集合の記号 (太文字)
\newcommand{\nn}{\mathbf{N}}
\newcommand{\zz}{\mathbf{Z}}
\newcommand{\qq}{\mathbf{Q}}
\newcommand{\rr}{\mathbf{R}}
\newcommand{\cc}{\mathbf{C}}
\newcommand{\pp}{\mathbf{P}}
\newcommand{\kk}{\mathbf{K}}

% 特殊な写像の記号
\newcommand{\ev}{\mathop{\mathrm{ev}}\nolimits} % 値写像
\newcommand{\pr}{\mathop{\mathrm{pr}}\nolimits} % 射影

% スクリプト体にするコマンド
%   例えば {\mcal C} のように用いる
\newcommand{\mcal}{\mathcal}

% 花文字にするコマンド 
%   例えば {\h C} のように用いる
\newcommand{\h}{\mathscr}

% ヒルベルト空間などの記号
\newcommand{\F}{\mcal{F}}
\newcommand{\X}{\mcal{X}}
\newcommand{\Y}{\mcal{Y}}
\newcommand{\Hil}{\mcal{H}}
\newcommand{\RKHS}{\Hil_{k}}
\newcommand{\Loss}{\mcal{L}_{D}}
\newcommand{\MLsp}{(\X, \Y, D, \Hil, \Loss)}

% 偏微分作用素の記号
\newcommand{\p}{\partial}

% 角カッコの記号 (内積は下にマクロがあります)
\newcommand{\lan}{\langle}
\newcommand{\ran}{\rangle}



% 圏の記号など
\newcommand{\Set}{{\bf Set}}
\newcommand{\Vect}{{\bf Vect}}
\newcommand{\FDVect}{{\bf FDVect}}
%\newcommand{\Ring}{{\bf Ring}}
\newcommand{\Ab}{{\bf Ab}}
\newcommand{\Mod}{\mathop{\mathrm{Mod}}\nolimits}
\newcommand{\Modf}{\mathop{\mathrm{Mod}^\mathrm{f}}\nolimits}
\newcommand{\CGA}{{\bf CGA}}
\newcommand{\GVect}{{\bf GVect}}
\newcommand{\Lie}{{\bf Lie}}
\newcommand{\dLie}{{\bf Liec}}



% 射の集合など
\newcommand{\Map}{\mathop{\mathrm{Map}}\nolimits} % 写像の集合
\newcommand{\Hom}{\mathop{\mathrm{Hom}}\nolimits} % 射集合
\newcommand{\End}{\mathop{\mathrm{End}}\nolimits} % 自己準同型の集合
\newcommand{\Aut}{\mathop{\mathrm{Aut}}\nolimits} % 自己同型の集合
\newcommand{\Mor}{\mathop{\mathrm{Mor}}\nolimits} % 射集合
\newcommand{\Ker}{\mathop{\mathrm{Ker}}\nolimits} % 核
\newcommand{\Img}{\mathop{\mathrm{Im}}\nolimits} % 像
\newcommand{\Cok}{\mathop{\mathrm{Coker}}\nolimits} % 余核
\newcommand{\Cim}{\mathop{\mathrm{Coim}}\nolimits} % 余像

% その他便利なコマンド
\newcommand{\dip}{\displaystyle} % 本文中で数式モード
\newcommand{\e}{\varepsilon} % イプシロン
\newcommand{\dl}{\delta} % デルタ
\newcommand{\pphi}{\varphi} % ファイ
\newcommand{\ti}{\tilde} % チルダ
\newcommand{\pal}{\parallel} % 平行
\newcommand{\op}{{\rm op}} % 双対を取る記号
\newcommand{\lcm}{\mathop{\mathrm{lcm}}\nolimits} % 最小公倍数の記号
\newcommand{\Probsp}{(\Omega, \F, \P)} 
\newcommand{\argmax}{\mathop{\rm arg~max}\limits}
\newcommand{\argmin}{\mathop{\rm arg~min}\limits}





% ================================
% コマンドを追加する場合のスペース 
%\newcommand{\OO}{\mcal{O}}



\makeatletter
\renewenvironment{proof}[1][\proofname]{\par
  \pushQED{\qed}%
  \normalfont \topsep6\p@\@plus6\p@\relax
  \trivlist
  \item[\hskip\labelsep
%        \itshape
         \bfseries
%    #1\@addpunct{.}]\ignorespaces
    {#1}]\ignorespaces
}{%
  \popQED\endtrivlist\@endpefalse
}
\makeatother

\renewcommand{\proofname}{証明.}



%\renewcommand\proofname{\bf 証明} % 証明
\numberwithin{equation}{section}
\newcommand{\cTop}{\textsf{Top}}
%\newcommand{\cOpen}{\textsf{Open}}
\newcommand{\Op}{\mathop{\textsf{Op}}\nolimits}
\newcommand{\Ob}{\mathop{\textrm{Ob}}\nolimits}
\newcommand{\id}{\mathop{\mathrm{id}}\nolimits}
\newcommand{\pt}{\mathop{\mathrm{pt}}\nolimits}
\newcommand{\res}{\mathop{\rho}\nolimits}
\newcommand{\A}{\mcal{A}}
\newcommand{\B}{\mcal{B}}
\newcommand{\C}{\mcal{C}}
\newcommand{\D}{\mcal{D}}
\newcommand{\E}{\mcal{E}}
\newcommand{\G}{\mcal{G}}
%\newcommand{\H}{\mcal{H}}
\newcommand{\I}{\mcal{I}}
\newcommand{\J}{\mcal{J}}
\newcommand{\OO}{\mcal{O}}
\newcommand{\Ring}{\mathop{\textsf{Ring}}\nolimits}
\newcommand{\cAb}{\mathop{\textsf{Ab}}\nolimits}
%\newcommand{\Ker}{\mathop{\mathrm{Ker}}\nolimits}
\newcommand{\im}{\mathop{\mathrm{Im}}\nolimits}
\newcommand{\Coker}{\mathop{\mathrm{Coker}}\nolimits}
\newcommand{\Coim}{\mathop{\mathrm{Coim}}\nolimits}
\newcommand{\rank}{\mathop{\mathrm{rank}}\nolimits}
\newcommand{\Ht}{\mathop{\mathrm{Ht}}\nolimits}
\newcommand{\supp}{\mathop{\mathrm{supp}}\nolimits}
\newcommand{\colim}{\mathop{\mathrm{colim}}}
\newcommand{\Tor}{\mathop{\mathrm{Tor}}\nolimits}

\newcommand{\cat}{\mathscr{C}}

\newcommand{\scA}{\mathscr{A}}
\newcommand{\scB}{\mathscr{B}}
\newcommand{\scC}{\mathscr{C}}
\newcommand{\scD}{\mathscr{D}}
\newcommand{\scE}{\mathscr{E}}
\newcommand{\scF}{\mathscr{F}}
\newcommand{\scN}{\mathscr{N}}
\newcommand{\scO}{\mathscr{O}}
\newcommand{\scV}{\mathscr{V}}
\newcommand{\scU}{\mathscr{U}}

\newcommand{\cU}{\mcal{U}}


\newcommand{\ibA}{\mathop{\text{\textit{\textbf{A}}}}}
\newcommand{\ibB}{\mathop{\text{\textit{\textbf{B}}}}}
\newcommand{\ibC}{\mathop{\text{\textit{\textbf{C}}}}}
\newcommand{\ibD}{\mathop{\text{\textit{\textbf{D}}}}}
\newcommand{\ibE}{\mathop{\text{\textit{\textbf{E}}}}}
\newcommand{\ibF}{\mathop{\text{\textit{\textbf{F}}}}}
\newcommand{\ibG}{\mathop{\text{\textit{\textbf{G}}}}}
\newcommand{\ibH}{\mathop{\text{\textit{\textbf{H}}}}}
\newcommand{\ibI}{\mathop{\text{\textit{\textbf{I}}}}}
\newcommand{\ibJ}{\mathop{\text{\textit{\textbf{J}}}}}
\newcommand{\ibK}{\mathop{\text{\textit{\textbf{K}}}}}
\newcommand{\ibL}{\mathop{\text{\textit{\textbf{L}}}}}
\newcommand{\ibM}{\mathop{\text{\textit{\textbf{M}}}}}
\newcommand{\ibN}{\mathop{\text{\textit{\textbf{N}}}}}
\newcommand{\ibO}{\mathop{\text{\textit{\textbf{O}}}}}
\newcommand{\ibP}{\mathop{\text{\textit{\textbf{P}}}}}
\newcommand{\ibQ}{\mathop{\text{\textit{\textbf{Q}}}}}
\newcommand{\ibR}{\mathop{\text{\textit{\textbf{R}}}}}
\newcommand{\ibS}{\mathop{\text{\textit{\textbf{S}}}}}
\newcommand{\ibT}{\mathop{\text{\textit{\textbf{T}}}}}
\newcommand{\ibU}{\mathop{\text{\textit{\textbf{U}}}}}
\newcommand{\ibV}{\mathop{\text{\textit{\textbf{V}}}}}
\newcommand{\ibW}{\mathop{\text{\textit{\textbf{W}}}}}
\newcommand{\ibX}{\mathop{\text{\textit{\textbf{X}}}}}
\newcommand{\ibY}{\mathop{\text{\textit{\textbf{Y}}}}}
\newcommand{\ibZ}{\mathop{\text{\textit{\textbf{Z}}}}}

\newcommand{\ibx}{\mathop{\text{\textit{\textbf{x}}}}}

%\newcommand{\Comp}{\mathop{\mathrm{C}}\nolimits}
%\newcommand{\Komp}{\mathop{\mathrm{K}}\nolimits}
%\newcommand{\Domp}{\mathop{\mathsf{D}}\nolimits}%複体のホモトピー圏
%\newcommand{\Comp}{\mathrm{C}}
%\newcommand{\Komp}{\mathrm{K}}
%\newcommand{\Domp}{\mathsf{D}}%複体のホモトピー圏

\newcommand{\Comp}{\mathop{\mathrm{C}}\nolimits}
\newcommand{\Komp}{\mathop{\mathsf{K}}\nolimits}
\newcommand{\Domp}{\mathop{\mathsf{D}}\nolimits}
\newcommand{\Kompl}{\mathop{\mathsf{K}^\mathrm{+}}\nolimits}
\newcommand{\Kompu}{\mathop{\mathsf{K}^\mathrm{-}}\nolimits}
\newcommand{\Kompb}{\mathop{\mathsf{K}^\mathrm{b}}\nolimits}
\newcommand{\Dompl}{\mathop{\mathsf{D}^\mathrm{+}}\nolimits}
\newcommand{\Dompu}{\mathop{\mathsf{D}^\mathrm{-}}\nolimits}
\newcommand{\Dompb}{\mathop{\mathsf{D}^\mathrm{b}}\nolimits}
\newcommand{\Dompbf}{\mathop{\mathsf{D}_\mathrm{f}^\mathrm{b}}\nolimits}




\newcommand{\CCat}{\Comp(\cat)}
\newcommand{\KCat}{\Komp(\cat)}
\newcommand{\DCat}{\Domp(\cat)}%圏Cの複体のホモトピー圏
\newcommand{\HOM}{\mathop{\mathscr{H}\hspace{-2pt}om}\nolimits}%内部Hom
\newcommand{\RHOM}{\mathop{\mathrm{R}\hspace{-1.5pt}\HOM}\nolimits}

\newcommand{\muS}{\mathop{\mathrm{SS}}\nolimits}
\newcommand{\RG}{\mathop{\mathrm{R}\hspace{-0pt}\Gamma}\nolimits}
\newcommand{\RHom}{\mathop{\mathrm{R}\hspace{-1.5pt}\Hom}\nolimits}
\newcommand{\Rder}{\mathrm{R}}

\newcommand{\simar}{\mathrel{\overset{\sim}{\rightarrow}}}%同型右矢印
\newcommand{\simarr}{\mathrel{\overset{\sim}{\longrightarrow}}}%同型右矢印
\newcommand{\simra}{\mathrel{\overset{\sim}{\leftarrow}}}%同型左矢印
\newcommand{\simrra}{\mathrel{\overset{\sim}{\longleftarrow}}}%同型左矢印

\newcommand{\hocolim}{{\mathrm{hocolim}}}
\newcommand{\indlim}[1][]{\mathop{\varinjlim}\limits_{#1}}
\newcommand{\sindlim}[1][]{\smash{\mathop{\varinjlim}\limits_{#1}}\,}
\newcommand{\Pro}{\mathrm{Pro}}
\newcommand{\Ind}{\mathrm{Ind}}
\newcommand{\prolim}[1][]{\mathop{\varprojlim}\limits_{#1}}
\newcommand{\sprolim}[1][]{\smash{\mathop{\varprojlim}\limits_{#1}}\,}

\newcommand{\Sh}{\mathrm{Sh}}
\newcommand{\PSh}{\mathrm{PSh}}

\newcommand{\rmD}{\mathrm{D}}

\newcommand{\Lloc}[1][]{\mathord{\mathcal{L}^1_{\mathrm{loc},{#1}}}}
\newcommand{\ori}{\mathord{\mathrm{or}}}
\newcommand{\Db}{\mathord{\mathscr{D}b}}

\newcommand{\codim}{\mathop{\mathrm{codim}}\nolimits}



\newcommand{\gld}{\mathop{\mathrm{gld}}\nolimits}
\newcommand{\wgld}{\mathop{\mathrm{wgld}}\nolimits}


\newcommand{\tens}[1][]{\mathbin{\otimes_{\raise1.5ex\hbox to-.1em{}{#1}}}}
\newcommand{\etens}{\mathbin{\boxtimes}}
\newcommand{\ltens}[1][]{\mathbin{\overset{\mathrm{L}}\tens}_{#1}}
\newcommand{\mtens}[1][]{\mathbin{\overset{\mathrm{\mu}}\tens}_{#1}}
\newcommand{\lltens}[1][]{{\mathop{\tens}\limits^{\mathrm{L}}_{#1}}}
\newcommand{\letens}{\overset{\mathrm{L}}{\etens}}
\newcommand{\detens}{\underline{\etens}}
\newcommand{\ldetens}{\overset{\mathrm{L}}{\underline{\etens}}}
\newcommand{\dtens}[1][]{{\overset{\mathrm{L}}{\underline{\otimes}}}_{#1}}

\newcommand{\blk}{\mathord{\ \cdot\ }}


%\newcommand{\hocolim}{{\mathrm{hocolim}}}
%\newcommand{\indlim}[1][]{\mathop{\varinjlim}\limits_{#1}}
%\newcommand{\sindlim}[1][]{\smash{\mathop{\varinjlim}\limits_{#1}}\,}
%\newcommand{\Pro}{\mathrm{Pro}}
%\newcommand{\Ind}{\mathrm{Ind}}
%\newcommand{\prolim}[1][]{\mathop{\varprojlim}\limits_{#1}}
%\newcommand{\sprolim}[1][]{\smash{\mathop{\varprojlim}\limits_{#1}}\,}
\newcommand{\proolim}[1][]{\mathop{\text{\rm``{$\varprojlim$}''}}\limits_{#1}}
\newcommand{\sproolim}[1][]{\smash{\mathop{\rm``{\varprojlim}''}\limits_{#1}}}
\newcommand{\inddlim}[1][]{\mathop{\text{\rm``{$\varinjlim$}''}}\limits_{#1}}
\newcommand{\sinddlim}[1][]{\smash{\mathop{\text{\rm``{$\varinjlim$}''}}\limits_{#1}}\,}
\newcommand{\ooplus}{\mathop{\text{\rm``{$\oplus$}''}}\limits}
\newcommand{\bbigsqcup}{\mathop{``\bigsqcup''}\limits}
\newcommand{\bsqcup}{\mathop{``\sqcup''}\limits}
\newcommand{\dsum}[1][]{\mathbin{\oplus_{#1}}}

\newcommand{\Fct}{\mathop{\mathsf{Fct}}\nolimits}





%================================================
% 自前の定理環境
%   https://mathlandscape.com/latex-amsthm/
% を参考にした
\newtheoremstyle{mystyle}%   % スタイル名
    {5pt}%                   % 上部スペース
    {5pt}%                   % 下部スペース
    {}%              % 本文フォント
    {}%                  % 1行目のインデント量
    {\bfseries}%                      % 見出しフォント
    {.}%                     % 見出し後の句読点
    {12pt}%                     % 見出し後のスペース
    {\thmname{#1}\thmnumber{ #2}\thmnote{{\hspace{2pt}\normalfont (#3)}}}% % 見出しの書式

\theoremstyle{mystyle}
\newtheorem{AXM}{公理}%[section]
\newtheorem{DFN}[AXM]{定義}
\newtheorem{THM}[AXM]{定理}
\newtheorem*{THM*}{定理}
\newtheorem{PRP}[AXM]{命題}
\newtheorem{LMM}[AXM]{補題}
\newtheorem{CRL}[AXM]{系}
\newtheorem{EG}[AXM]{例}
\newtheorem*{EG*}{例}
\newtheorem{RMK}[AXM]{注意}
\newtheorem{CNV}[AXM]{約束}
\newtheorem{CMT}[AXM]{コメント}
\newtheorem*{CMT*}{コメント}
\newtheorem{NTN}[AXM]{記号}

% 定理環境ここまで
%====================================================

\usepackage{framed}
\definecolor{lightgray}{rgb}{0.75,0.75,0.75}
\renewenvironment{leftbar}{%
  \def\FrameCommand{\textcolor{lightgray}{\vrule width 4pt} \hspace{10pt}}% 
  \MakeFramed {\advance\hsize-\width \FrameRestore}}%
{\endMakeFramed}
\newenvironment{redleftbar}{%
  \def\FrameCommand{\textcolor{lightgray}{\vrule width 1pt} \hspace{10pt}}% 
  \MakeFramed {\advance\hsize-\width \FrameRestore}}%
 {\endMakeFramed}



\renewcommand{\Re}{\mathop{\mathrm{Re}}\nolimits}

% =================================





% ---------------------------
% new definition macro
% ---------------------------
% 便利なマクロ集です

% 内積のマクロ
%   例えば \inner<\pphi | \psi> のように用いる
\def\inner<#1>{\langle #1 \rangle}

% ================================
% マクロを追加する場合のスペース 

%=================================





% ----------------------------
% documenet 
% ----------------------------
% 以下, 本文の執筆スペースです. 
% Your main code must be written between 
% begin document and end document.
% ---------------------------

\title{幾何学まとめノート}
\author{toshi2019}
\date{\today\footnote{2024/03/29作成開始}}
\begin{document}
\maketitle
\tableofcontents

\section{多様体}

\end{document}
