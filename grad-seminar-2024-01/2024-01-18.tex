%=====================================
%   2024-01-18.tex
%   院生セミナー資料 大柴寿浩
%=====================================

% -----------------------
% preamble
% -----------------------
% ここから本文 (\begin{document}) までの
% ソースコードに変更を加えた場合は
% 編集者まで連絡してください. 
% Don't change preamble code yourself. 
% If you add something
% (usepackage, newtheorem, newcommand, renewcommand),
% please tell it 
% to the editor of institutional paper of RUMS.

% ------------------------
% documentclass
% ------------------------
\documentclass[11pt, a4paper, dvipdfmx]{jsarticle}

% ------------------------
% usepackage
% ------------------------
\usepackage{algorithm}
\usepackage{algorithmic}
\usepackage{amscd}
\usepackage{amsfonts}
\usepackage{amsmath}
\usepackage[psamsfonts]{amssymb}
\usepackage{amsthm}
\usepackage{ascmac}
\usepackage{color}
\usepackage{enumerate}
\usepackage{fancybox}
\usepackage[stable]{footmisc}
\usepackage{graphicx}
\usepackage{listings}
\usepackage{mathrsfs}
\usepackage{mathtools}
\usepackage{otf}
\usepackage{pifont}
\usepackage{proof}
\usepackage{subfigure}
\usepackage{tikz}
\usepackage{verbatim}
\usepackage[all]{xy}

\usetikzlibrary{cd}



% ================================
% パッケージを追加する場合のスペース 
\usepackage[dvipdfmx]{hyperref}
\usepackage{xcolor}
\definecolor{darkgreen}{rgb}{0,0.45,0} 
\definecolor{darkred}{rgb}{0.75,0,0}
\definecolor{darkblue}{rgb}{0,0,0.6} 
\hypersetup{
    colorlinks=true,
    citecolor=darkgreen,
    linkcolor=darkred,
    urlcolor=darkblue,
}
\usepackage{pxjahyper}

%=================================


% --------------------------
% theoremstyle
% --------------------------
\theoremstyle{definition}

% --------------------------
% newtheoem
% --------------------------

% 日本語で定理, 命題, 証明などを番号付きで用いるためのコマンドです. 
% If you want to use theorem environment in Japanece, 
% you can use these code. 
% Attention!
% All theorem enivironment numbers depend on 
% only section numbers.
\newtheorem{Axiom}{公理}[section]
\newtheorem{Definition}[Axiom]{定義}
\newtheorem{Theorem}[Axiom]{定理}
\newtheorem{Proposition}[Axiom]{命題}
\newtheorem{Lemma}[Axiom]{補題}
\newtheorem{Corollary}[Axiom]{系}
\newtheorem{Example}[Axiom]{例}
\newtheorem{Claim}[Axiom]{主張}
\newtheorem{Property}[Axiom]{性質}
\newtheorem{Attention}[Axiom]{注意}
\newtheorem{Question}[Axiom]{問}
\newtheorem{Problem}[Axiom]{問題}
\newtheorem{Consideration}[Axiom]{考察}
\newtheorem{Alert}[Axiom]{警告}
\newtheorem{Fact}[Axiom]{事実}


% 日本語で定理, 命題, 証明などを番号なしで用いるためのコマンドです. 
% If you want to use theorem environment with no number in Japanese, You can use these code.
\newtheorem*{Axiom*}{公理}
\newtheorem*{Definition*}{定義}
\newtheorem*{Theorem*}{定理}
\newtheorem*{Proposition*}{命題}
\newtheorem*{Lemma*}{補題}
\newtheorem*{Example*}{例}
\newtheorem*{Corollary*}{系}
\newtheorem*{Claim*}{主張}
\newtheorem*{Property*}{性質}
\newtheorem*{Attention*}{注意}
\newtheorem*{Question*}{問}
\newtheorem*{Problem*}{問題}
\newtheorem*{Consideration*}{考察}
\newtheorem*{Alert*}{警告}
\newtheorem{Fact*}{事実}


% 英語で定理, 命題, 証明などを番号付きで用いるためのコマンドです. 
% If you want to use theorem environment in English, You can use these code.
%all theorem enivironment number depend on only section number.
\newtheorem{Axiom+}{Axiom}[section]
\newtheorem{Definition+}[Axiom+]{Definition}
\newtheorem{Theorem+}[Axiom+]{Theorem}
\newtheorem{Proposition+}[Axiom+]{Proposition}
\newtheorem{Lemma+}[Axiom+]{Lemma}
\newtheorem{Example+}[Axiom+]{Example}
\newtheorem{Corollary+}[Axiom+]{Corollary}
\newtheorem{Claim+}[Axiom+]{Claim}
\newtheorem{Property+}[Axiom+]{Property}
\newtheorem{Attention+}[Axiom+]{Attention}
\newtheorem{Question+}[Axiom+]{Question}
\newtheorem{Problem+}[Axiom+]{Problem}
\newtheorem{Consideration+}[Axiom+]{Consideration}
\newtheorem{Alert+}{Alert}
\newtheorem{Fact+}[Axiom+]{Fact}
\newtheorem{Remark+}[Axiom+]{Remark}

% ----------------------------
% commmand
% ----------------------------
% 執筆に便利なコマンド集です. 
% コマンドを追加する場合は下のスペースへ. 

% 集合の記号 (黒板文字)
\newcommand{\NN}{\mathbb{N}}
\newcommand{\ZZ}{\mathbb{Z}}
\newcommand{\QQ}{\mathbb{Q}}
\newcommand{\RR}{\mathbb{R}}
\newcommand{\CC}{\mathbb{C}}
\newcommand{\PP}{\mathbb{P}}
\newcommand{\KK}{\mathbb{K}}


% 集合の記号 (太文字)
\newcommand{\nn}{\mathbf{N}}
\newcommand{\zz}{\mathbf{Z}}
\newcommand{\qq}{\mathbf{Q}}
\newcommand{\rr}{\mathbf{R}}
\newcommand{\cc}{\mathbf{C}}
\newcommand{\pp}{\mathbf{P}}
\newcommand{\kk}{\mathbf{K}}

% 特殊な写像の記号
\newcommand{\ev}{\mathop{\mathrm{ev}}\nolimits} % 値写像
\newcommand{\pr}{\mathop{\mathrm{pr}}\nolimits} % 射影

% スクリプト体にするコマンド
%   例えば {\mcal C} のように用いる
\newcommand{\mcal}{\mathcal}

% 花文字にするコマンド 
%   例えば {\h C} のように用いる
\newcommand{\h}{\mathscr}

% ヒルベルト空間などの記号
\newcommand{\F}{\mcal{F}}
\newcommand{\X}{\mcal{X}}
\newcommand{\Y}{\mcal{Y}}
\newcommand{\Hil}{\mcal{H}}
\newcommand{\RKHS}{\Hil_{k}}
\newcommand{\Loss}{\mcal{L}_{D}}
\newcommand{\MLsp}{(\X, \Y, D, \Hil, \Loss)}

% 偏微分作用素の記号
\newcommand{\p}{\partial}

% 角カッコの記号 (内積は下にマクロがあります)
\newcommand{\lan}{\langle}
\newcommand{\ran}{\rangle}



% 圏の記号など
\newcommand{\Set}{{\bf Set}}
\newcommand{\Vect}{{\bf Vect}}
\newcommand{\FDVect}{{\bf FDVect}}
%\newcommand{\Ring}{{\bf Ring}}
\newcommand{\Ab}{{\bf Ab}}
\newcommand{\Mod}{\mathop{\mathrm{Mod}}\nolimits}
\newcommand{\CGA}{{\bf CGA}}
\newcommand{\GVect}{{\bf GVect}}
\newcommand{\Lie}{{\bf Lie}}
\newcommand{\dLie}{{\bf Liec}}



% 射の集合など
\newcommand{\Map}{\mathop{\mathrm{Map}}\nolimits} % 写像の集合
\newcommand{\Hom}{\mathop{\mathrm{Hom}}\nolimits} % 射集合
\newcommand{\End}{\mathop{\mathrm{End}}\nolimits} % 自己準同型の集合
\newcommand{\Aut}{\mathop{\mathrm{Aut}}\nolimits} % 自己同型の集合
\newcommand{\Mor}{\mathop{\mathrm{Mor}}\nolimits} % 射集合
\newcommand{\Ker}{\mathop{\mathrm{Ker}}\nolimits} % 核
\newcommand{\Img}{\mathop{\mathrm{Im}}\nolimits} % 像
\newcommand{\Cok}{\mathop{\mathrm{Coker}}\nolimits} % 余核
\newcommand{\Cim}{\mathop{\mathrm{Coim}}\nolimits} % 余像

% その他便利なコマンド
\newcommand{\dip}{\displaystyle} % 本文中で数式モード
\newcommand{\e}{\varepsilon} % イプシロン
\newcommand{\dl}{\delta} % デルタ
\newcommand{\pphi}{\varphi} % ファイ
\newcommand{\ti}{\tilde} % チルダ
\newcommand{\pal}{\parallel} % 平行
\newcommand{\op}{{\rm op}} % 双対を取る記号
\newcommand{\lcm}{\mathop{\mathrm{lcm}}\nolimits} % 最小公倍数の記号
\newcommand{\Probsp}{(\Omega, \F, \P)} 
\newcommand{\argmax}{\mathop{\rm arg~max}\limits}
\newcommand{\argmin}{\mathop{\rm arg~min}\limits}





% ================================
% コマンドを追加する場合のスペース 
%\newcommand{\OO}{\mcal{O}}



\renewcommand\proofname{\bf 証明} % 証明
\numberwithin{equation}{section}
\newcommand{\cTop}{\textsf{Top}}
%\newcommand{\cOpen}{\textsf{Open}}
\newcommand{\Op}{\mathop{\textsf{Open}}\nolimits}
\newcommand{\Ob}{\mathop{\textrm{Ob}}\nolimits}
\newcommand{\id}{\mathop{\mathrm{id}}\nolimits}
\newcommand{\pt}{\mathop{\mathrm{pt}}\nolimits}
\newcommand{\res}{\mathop{\rho}\nolimits}
\newcommand{\A}{\mcal{A}}
\newcommand{\B}{\mcal{B}}
\newcommand{\C}{\mcal{C}}
\newcommand{\D}{\mcal{D}}
\newcommand{\E}{\mcal{E}}
\newcommand{\G}{\mcal{G}}
%\newcommand{\H}{\mcal{H}}
\newcommand{\I}{\mcal{I}}
\newcommand{\J}{\mcal{J}}
\newcommand{\OO}{\mcal{O}}
\newcommand{\Ring}{\mathop{\textsf{Ring}}\nolimits}
\newcommand{\cAb}{\mathop{\textsf{Ab}}\nolimits}
%\newcommand{\Ker}{\mathop{\mathrm{Ker}}\nolimits}
\newcommand{\im}{\mathop{\mathrm{Im}}\nolimits}
\newcommand{\Coker}{\mathop{\mathrm{Coker}}\nolimits}
\newcommand{\Coim}{\mathop{\mathrm{Coim}}\nolimits}
\newcommand{\rank}{\mathop{\mathrm{rank}}\nolimits}
\newcommand{\Ht}{\mathop{\mathrm{Ht}}\nolimits}
\newcommand{\supp}{\mathop{\mathrm{supp}}\nolimits}
\newcommand{\colim}{\mathop{\mathrm{colim}}}
\newcommand{\Tor}{\mathop{\mathrm{Tor}}\nolimits}

\newcommand{\cat}{\mathscr{C}}

\newcommand{\scA}{\mathscr{A}}
\newcommand{\scB}{\mathscr{B}}
\newcommand{\scC}{\mathscr{C}}
\newcommand{\scD}{\mathscr{D}}
\newcommand{\scE}{\mathscr{E}}
\newcommand{\scF}{\mathscr{F}}

\newcommand{\ibA}{\mathop{\text{\textit{\textbf{A}}}}}
\newcommand{\ibB}{\mathop{\text{\textit{\textbf{B}}}}}
\newcommand{\ibC}{\mathop{\text{\textit{\textbf{C}}}}}
\newcommand{\ibD}{\mathop{\text{\textit{\textbf{D}}}}}
\newcommand{\ibE}{\mathop{\text{\textit{\textbf{E}}}}}
\newcommand{\ibF}{\mathop{\text{\textit{\textbf{F}}}}}
\newcommand{\ibG}{\mathop{\text{\textit{\textbf{G}}}}}
\newcommand{\ibH}{\mathop{\text{\textit{\textbf{H}}}}}
\newcommand{\ibI}{\mathop{\text{\textit{\textbf{I}}}}}
\newcommand{\ibJ}{\mathop{\text{\textit{\textbf{J}}}}}
\newcommand{\ibK}{\mathop{\text{\textit{\textbf{K}}}}}
\newcommand{\ibL}{\mathop{\text{\textit{\textbf{L}}}}}
\newcommand{\ibM}{\mathop{\text{\textit{\textbf{M}}}}}
\newcommand{\ibN}{\mathop{\text{\textit{\textbf{N}}}}}
\newcommand{\ibO}{\mathop{\text{\textit{\textbf{O}}}}}
\newcommand{\ibP}{\mathop{\text{\textit{\textbf{P}}}}}
\newcommand{\ibQ}{\mathop{\text{\textit{\textbf{Q}}}}}
\newcommand{\ibR}{\mathop{\text{\textit{\textbf{R}}}}}
\newcommand{\ibS}{\mathop{\text{\textit{\textbf{S}}}}}
\newcommand{\ibT}{\mathop{\text{\textit{\textbf{T}}}}}
\newcommand{\ibU}{\mathop{\text{\textit{\textbf{U}}}}}
\newcommand{\ibV}{\mathop{\text{\textit{\textbf{V}}}}}
\newcommand{\ibW}{\mathop{\text{\textit{\textbf{W}}}}}
\newcommand{\ibX}{\mathop{\text{\textit{\textbf{X}}}}}
\newcommand{\ibY}{\mathop{\text{\textit{\textbf{Y}}}}}
\newcommand{\ibZ}{\mathop{\text{\textit{\textbf{Z}}}}}

\newcommand{\ibx}{\mathop{\text{\textit{\textbf{x}}}}}

%\newcommand{\Comp}{\mathop{\mathrm{C}}\nolimits}
%\newcommand{\Komp}{\mathop{\mathrm{K}}\nolimits}
%\newcommand{\Domp}{\mathop{\mathsf{D}}\nolimits}%複体のホモトピー圏
%\newcommand{\Comp}{\mathrm{C}}
%\newcommand{\Komp}{\mathrm{K}}
%\newcommand{\Domp}{\mathsf{D}}%複体のホモトピー圏

\newcommand{\Comp}{\mathop{\mathrm{C}}\nolimits}
\newcommand{\Komp}{\mathop{\mathsf{K}}\nolimits}
\newcommand{\Domp}{\mathop{\mathsf{D}}\nolimits}
\newcommand{\Kompl}{\mathop{\mathsf{K}^\mathrm{+}}\nolimits}
\newcommand{\Kompu}{\mathop{\mathsf{K}^\mathrm{-}}\nolimits}
\newcommand{\Kompb}{\mathop{\mathsf{K}^\mathrm{b}}\nolimits}
\newcommand{\Dompl}{\mathop{\mathsf{D}^\mathrm{+}}\nolimits}
\newcommand{\Dompu}{\mathop{\mathsf{D}^\mathrm{-}}\nolimits}
\newcommand{\Dompb}{\mathop{\mathsf{D}^\mathrm{b}}\nolimits}




\newcommand{\CCat}{\Comp(\cat)}
\newcommand{\KCat}{\Komp(\cat)}
\newcommand{\DCat}{\Domp(\cat)}%圏Cの複体のホモトピー圏
\newcommand{\HOM}{\mathop{\mathscr{H}\hspace{-2pt}om}\nolimits}%内部Hom
\newcommand{\RHOM}{\mathop{\mathrm{R}\hspace{-1.5pt}\HOM}\nolimits}

\newcommand{\muS}{\mathop{\mathrm{SS}}\nolimits}
\newcommand{\RG}{\mathop{\mathrm{R}\hspace{-0pt}\Gamma}\nolimits}
\newcommand{\RHom}{\mathop{\mathrm{R}\hspace{-1.5pt}\Hom}\nolimits}
\newcommand{\Rder}{\mathrm{R}}

\newcommand{\simar}{\mathrel{\overset{\sim}{\rightarrow}}}%同型右矢印
\newcommand{\simarr}{\mathrel{\overset{\sim}{\longrightarrow}}}%同型右矢印
\newcommand{\simra}{\mathrel{\overset{\sim}{\leftarrow}}}%同型左矢印
\newcommand{\simrra}{\mathrel{\overset{\sim}{\longleftarrow}}}%同型左矢印

\newcommand{\hocolim}{{\mathrm{hocolim}}}
\newcommand{\indlim}[1][]{\mathop{\varinjlim}\limits_{#1}}
\newcommand{\sindlim}[1][]{\smash{\mathop{\varinjlim}\limits_{#1}}\,}
\newcommand{\Pro}{\mathrm{Pro}}
\newcommand{\Ind}{\mathrm{Ind}}
\newcommand{\prolim}[1][]{\mathop{\varprojlim}\limits_{#1}}
\newcommand{\sprolim}[1][]{\smash{\mathop{\varprojlim}\limits_{#1}}\,}

\newcommand{\Sh}{\mathrm{Sh}}
\newcommand{\PSh}{\mathrm{PSh}}

\newcommand{\rmD}{\mathrm{D}}

\newcommand{\ori}{\mathord{\mathrm{or}}}












%================================================
% 自前の定理環境
%   https://mathlandscape.com/latex-amsthm/
% を参考にした
\newtheoremstyle{mystyle}%   % スタイル名
    {5pt}%                   % 上部スペース
    {5pt}%                   % 下部スペース
    {}%              % 本文フォント
    {}%                  % 1行目のインデント量
    {\bfseries}%                      % 見出しフォント
    {.}%                     % 見出し後の句読点
    {12pt}%                     % 見出し後のスペース
    {\thmname{#1}\thmnumber{ #2}\thmnote{{\hspace{2pt}\normalfont (#3)}}}% % 見出しの書式

\theoremstyle{mystyle}
\newtheorem{AXM}{公理}[section]
\newtheorem{DFN}[AXM]{定義}
\newtheorem{THM}[AXM]{定理}
\newtheorem*{THM*}{定理}
\newtheorem{PRP}[AXM]{命題}
\newtheorem{LMM}[AXM]{補題}
\newtheorem{CRL}[AXM]{系}
\newtheorem{EG}[AXM]{例}
\newtheorem*{EG*}{例}
\newtheorem{CNV}[AXM]{規約}
\newtheorem{CMT}[AXM]{コメント}

% 定理環境ここまで
%====================================================

\usepackage{framed}
\definecolor{lightgray}{rgb}{0.75,0.75,0.75}
\renewenvironment{leftbar}{%
  \def\FrameCommand{\textcolor{lightgray}{\vrule width 0.7zw} \hspace{10pt}}% 
  \MakeFramed {\advance\hsize-\width \FrameRestore}}%
{\endMakeFramed}
\newenvironment{redleftbar}{%
  \def\FrameCommand{\textcolor{lightgray}{\vrule width 1pt} \hspace{10pt}}% 
  \MakeFramed {\advance\hsize-\width \FrameRestore}}%
 {\endMakeFramed}


% =================================





% ---------------------------
% new definition macro
% ---------------------------
% 便利なマクロ集です

% 内積のマクロ
%   例えば \inner<\pphi | \psi> のように用いる
\def\inner<#1>{\langle #1 \rangle}

% ================================
% マクロを追加する場合のスペース 

%=================================





% ----------------------------
% documenet 
% ----------------------------
% 以下, 本文の執筆スペースです. 
% Your main code must be written between 
% begin document and end document.
% ---------------------------

\title{2024/01/18 セミナー資料}
\author{大柴寿浩}
\date{2024/01/18}
\begin{document}
\maketitle


\section{向きづけと双対性{\cite[3.3]{KS90}}}

\subsection{前回示したもの}
次がある.
\begin{equation}
    f^{-1}(\cdot)\otimes\omega_{Y/X}\to f^!(\cdot).\label{eq:VerdierDual}
\end{equation}


\(F\in\Domp^\mathrm{b}(A_X)\)に対して次が成り立つ.
\begin{align}
    \Hom_{\Domp^\mathrm{b}(A_X)}(F,\omega_X)
    &\cong
    \Hom_{\Domp^\mathrm{b}(\Mod(A))}(\RG_c(X;F),A),\\
    \RHom(F,\omega_X)
    &\cong
    \RHom(\RG_c(X;F),A).\label{eq:RPV-dual}
\end{align}

\begin{DFN}
    \(f\colon Y\to X\)を局所コンパクト空間の間の連続写像とする.
    \(f\)がファイバー次元\(l\)の
    \textbf{位相的沈めこみ} (topological submersion) であるとは,
    \(Y\)の各点\(y\)に対して,\(y\)の開近傍\(V\in I_y\)で,
    \(U=f(V)\)が\(X\)の開集合であり,次の図式が可換になることをいう.
    \[\begin{tikzcd}
        U\times \rr^l
        \arrow[rd,"\pr_1"']
        \arrow[r,"h", "\sim"']
        &
        V
        \arrow[d,"f\rvert_V"]
        \\
        {}&U. 
    \end{tikzcd}\]
\end{DFN}
\begin{EG*}
    \(X,Y\)が\(C^1\)級多様体で\(f\)を\(C^1\)沈めこみとすると,
    \(f\)は位相的沈めこみである.
\end{EG*}
\subsection{今回}

\begin{PRP}\label{PRP:subm}
    \(f\colon Y\to X\)をファイバー次元\(l\)の位相的沈めこみとする.
    \begin{enumerate}[(i)]
        \item \(k\ne -l\)に対し\(H^k(f^!A_X)=0\)であり,局所的に\(H^{-l}(f^!A_X)\cong A_Y\)である.
        \item \(f^{-1}(\cdot)\otimes\omega_{Y/X}\to f^!(\cdot)\)は同型である.
    \end{enumerate}
\end{PRP}
\begin{proof}
    まず\(Y=\rr^l\), \(X=\{\pt\}\)の場合に(i)を示す.
    \eqref{eq:RPV-dual}より任意の開集合\(U\subset Y\)に対し,
    \[
        \RG(U;f^!A_X)\cong\RHom(\RG_c(U;A_Y),A)
    \]
    である.
    さらに,\(U\approx\rr^l\)なら
    \[
        \RG_c(U;A_Y)\cong A[-l]   
    \]
    であり,
    \begin{align*}
        H^j(U;f^!A_X)&\cong0 \quad (j\ne -l)\\
        \Gamma(U;H^{-l}(f^!A_X))
        &\cong
        \Hom(H^l_c(U;A_Y),A)        
    \end{align*}
    である.

    一般の場合,局所的に考えて
    \(Y=\rr^l\times X\)とし,\(f=\pr_2\)とする.
    \(p=\pr_1\colon Y\to \rr^l\)とおく.
    \[\begin{tikzcd}
        \rr^l\times X
        \arrow[rr,"f"]
        \arrow[dd,"p"']
        &&
        X
        \arrow[dd,"\mathrm{a}_X"]
        \\
        {}&\square&{} 
        \\
        \rr^l\arrow[rr,"\mathrm{a}_{\rr^l}"']
        &&
        \{\pt\}.
    \end{tikzcd}\]
    固有基底変換から射
    \(p^{-1}\omega_{\rr^l}
    =p^{-1}\mathrm{a}_{\rr^l}^! A
    \to 
    f^!\mathrm{a}^{-1}_XA
    =f^!A_X\)が定まる.
    任意の\(F\in\Domp^+(A_X)\)に対し,
    この射と\eqref{eq:VerdierDual}から
    次の射が定まる.
    \begin{equation}
        p^{-1}\omega_{\rr^l}\otimes f^{-1}F
        \to f^!A_X\otimes f^{-1}F\to f^{-1}F.\label{eq:subm-mor}
    \end{equation}
    これが同型になることを示す.
    \(U\subset\rr^l\), \(V\subset X\)とする.
    \(h\colon U\approx \rr^l\)のとき,位相的沈めこみの図式は
    \[\begin{tikzcd}
        \rr^l\times V 
        \arrow[rd,"\pr_2"']
        \arrow[r,"h\times\id_X", "\sim"']
        &
        U\times V
        \arrow[d,"f\rvert_{U\times V}"]
        \\
        {}&V 
    \end{tikzcd}\]
    となる.
    \begin{align*}
        &\RG(U\times V;f^!F)\\
        &\cong \RHom_{Y}(A_{U\times V},f^!F)&\text{\(\Gamma\)の定義}\\
        &\cong \RHom_X(\Rder f_!A_{U\times V},F)&\text{ポアンカレ・ヴェルディエ双対}\\
        &\cong \RHom_X(\RG_c(U;A_U)\otimes^\mathrm{L}A_V,F)\tag{\(\natural\)}\label{eq:natural}\\
        &\simra \RHom(\RG_c(U;A_U),A)\otimes^\mathrm{L}\RHom(A_V,F)\tag{\(\flat\)}\label{eq:flat}\\
        &\simra \RG(U;\omega_{\rr^l})\otimes^\mathrm{L}\RG(V;F)&\text{上で示した}
    \end{align*}
    となる.ただし\eqref{eq:natural}の部分は
    \begin{align*}
        \Rder f_!A_{U\times V}
        &\cong \Rder f_!\left(
            A_{U\times V}
            \otimes 
            A_{U\times V}
        \right)\\
        &\cong \Rder f_!\left(
            \mathrm{a}^{-1}_{{U\times V}}A
            \otimes
            \mathrm{a}^{-1}_{{U\times V}}A
        \right)\\
        &\cong \Rder f_!\left(
            p^{-1}\mathrm{a}^{-1}_{U}A
            \otimes
            f^{-1}\mathrm{a}^{-1}_{V}A
        \right)\\
        &\cong \Rder f_!\left(
            p^{-1}A_{U}
            \otimes
            f^{-1}A_{V}
        \right)
    \end{align*}
    であり,射影公式から,
    \begin{align*}
        \Rder f_!\left(
            p^{-1}A_{U}
            \otimes
            f^{-1}A_{V}
        \right)
        &\cong \Rder f_!p^{-1}A_{U}\otimes A_{V}
    \end{align*}
    であり,固有基底変換から,
    \begin{align*}
        \Rder f_!p^{-1}A_{U}\otimes f^{-1}A_{V}
        &\cong
        \mathrm{a}^{-1}_{V} \Rder{\mathrm{a}_{U}}_! A_U\otimes A_V\\
        &\cong
        \mathrm{a}^{-1}_{V}\RG_c(U;A_U)\otimes A_V
    \end{align*}
    となることから従う.\eqref{eq:flat}の部分は分かってない.
    
    以上より\eqref{eq:subm-mor}は同型である.(ii)から(i)が従う.
\end{proof}

\begin{DFN}
    \(f\colon Y\to X\)をファイバー次元が\(l\)の位相的しずめ込みとする.
    \[
        \ori_{Y/X}\coloneqq H^{-l}(\omega_{Y/X})\in\Mod(A_Y)
    \]
    とおき,\textbf{相対向きづけ層} (relative orientation sheaf) 
    と呼ぶ.
    \(X=\{\pt\}\)のとき,\(\ori_Y\)とかき,向きづけ層という.
\end{DFN}

命題\ref{PRP:subm}より,
\begin{equation}
    \omega_{Y/X}\cong \ori_{Y/X}[l]
\end{equation}
である.

\begin{proof}[\textbf{チェック}]
    \(\ori_{Y/X}\)を0時に集中した複体とみなす.
    このとき,\(
        \left(\ori_{Y/X}[l]\right)^n=H^{-l}(\omega_{Y_X})
    \)となるのは\(n=-l\)のときである.
    したがって,\[
        H^{-l}(\ori_{Y/X}[l])=H^{-l}(\omega_{Y/X})
    \]
    である.
\end{proof}

\begin{PRP}
    \(f\colon Y\to X\)をファイバー次元が\(l\)の位相的しずめ込みとする.
    \begin{enumerate}[(i)]
        \item \(\omega_{Y/X}^\zz\in\Dompl(\zz_Y)\)と
        \(\ori_{Y/X}^\zz\in\Mod(\zz_Y)\)を\(\zz\)上の
        双対化複体と向きづけ層とするとき,次の同型が成り立つ.\[
            \omega_{Y/X}\cong A_Y
            \mathop{\otimes}\limits_{\zz_Y}
            \omega_{Y/X}^\zz, 
            \quad
            \ori_{Y/X}\cong A_Y
            \mathop{\otimes}\limits_{\zz_Y}
            \ori_{Y/X}^\zz. 
        \]
        \item 次の自然な同型が成り立つ.
        \begin{align*}
                \ori_{Y/X}\otimes \ori_{Y/X}\cong A_Y,\\
                \HOM(\ori_{Y/X},A_Y)\cong A_Y.
        \end{align*}
        \item \(g\colon Z\to Y\)を連続写像で
        \(f\circ g\)がファイバー次元\(m\)の位相的しずめ込み
        になるものとする.\(F\in\Dompl(A_X)\)に対して,
        \[
            g^{!}\circ f^{-1}F
            \cong
            \left(f\circ g\right)^{-1}F
            \otimes \ori_{Z/X}
            \otimes g^{-1}\ori_{Y/X}[m-l]
        \]が成り立つ.
    \end{enumerate}
\end{PRP}
\begin{proof}
    (i) 
    命題\ref{PRP:subm} (ii) より,
    \begin{align*}
        A_Y\otimes_{\zz_Y}\omega_{Y/X}^\zz 
        \cong
        f^{-1}A_X\otimes_{\zz_Y}f^!\zz_X
        \cong
        f^!(A_X)=\omega_{Y/X}
    \end{align*}
    と
    \begin{align*}
        A_Y\otimes_{\zz_Y}\ori_{Y/X}^\zz 
        \cong
        f^{-1}A_X\otimes_{\zz_Y}f^!\zz_X[-l]
        \cong
        f^!(A_X[-l])
        \cong
        \omega_{Y/X}[-l]
        =\ori_{Y/X}
    \end{align*}

    (ii) 
    \cite[Exercise III.3]{KS90}\begin{quote}
        \(X\in\cTop\)とし,\(F\in\Sh(X)\)を\(\zz_X\)に
        局所的に同型な層とする.
        このとき,\[
            F\otimes F\cong \zz_Z,\quad \rmD'F\cong F
        \]が成り立つことを示せ.
    \end{quote}
    の結果を用いると,\[
        \ori_{Y/X}^\zz\otimes \ori_{Y/X}^\zz\cong Z_Y
    \]であり,これの両辺に\(A_Y\)をかければ,
    \[
        \ori_{Y/X}\otimes \ori_{Y/X}
        \cong
        \left(\ori_{Y/X}^\zz\otimes \ori_{Y/X}^\zz\right)\otimes A_Y
        \cong Z_Y\otimes A_Y
        \cong A_Y
    \]
    である.

    (iii) 
    命題\ref{PRP:subm}と前の結果から
    \begin{align*}
        \left(g^!f^!F=\right) 
        \left(f\circ g\right)^!F
        &\cong
        \left(f\circ g\right)^!A_X
        \otimes
        \left(f\circ g\right)^{-1}F\\
        &\cong
        \omega_{Z/X}
        \otimes
        \left(f\circ g\right)^{-1}F\\
        &\cong
        \ori_{Z/X}[m]\otimes
        \left(f\circ g\right)^{-1}F
    \end{align*}
    と
    \begin{align*}
        f^!F
        \cong f^{-1}F\otimes \omega_{Y/X}
        \cong f^{-1}F\otimes \ori_{Y/X}[l]
    \end{align*}
    が成り立つ.いま
    \begin{align*}
        g^!\left(f^{-1}F\otimes\ori_{Y/X}[l]\right)
        &\cong
        g^!\left(f^!F\right)\\
        &\cong
        (f\circ g)^!F\\
        &\cong
        \ori_{Z/X}[m]\otimes
        \left(f\circ g\right)^{-1}F
    \end{align*}
    に\(g^{-1}\ori_{Y/X}[-l]\)をかけると
    \[
        g^!\left(f^{-1}F\otimes\ori_{Y/X}[l]\right)
        \otimes g^{-1}\ori_{Y/X}[-l]
        \cong
        \ori_{Z/X}[m]\otimes\left(f\circ g\right)^{-1}F
        \otimes g^{-1}\ori_{Y/X}[-l]
    \]
    が成り立つ.左辺は
    \begin{align*}
        g^!\left(f^{-1}F\otimes\ori_{Y/X}[l]\right)
        \otimes g^{-1}\ori_{Y/X}[-l]
        &\cong
        g^!\left(
            f^{-1}F\otimes\ori_{Y/X}[l]\otimes\ori_{Y/X}[-l]
        \right)\\
        &\cong
        g^!\circ f^{-1}F
    \end{align*}
    であり,右辺は\begin{align*}
        &\ori_{Z/X}[m]\otimes\left(f\circ g\right)^{-1}F
        \otimes g^{-1}\ori_{Y/X}[-l]\\
        \cong&
        \ori_{Z/X}\otimes\left(f\circ g\right)^{-1}F
        \otimes g^{-1}\ori_{Y/X}[m-l]
    \end{align*}
    となることから主張が従う.
\end{proof}


















\clearpage
\appendix

\section{前回の補足}
\subsection{はめ込み・うめ込み・しずめ込み}
参考文献は\cite{GP74}.
\(X\), \(Y\)を多様体とする.
\(d_X\coloneqq\dim X\), \(d_Y\coloneqq \dim Y\)とかく.

\paragraph{局所微分同相}
\(d_X=d_Y\)とする.
\(f\colon X\to Y\)が\(x\in X\)において
\textbf{局所微分同相写像} (local diffeomorphism) 
であるとは,
\(x\)の近傍\(U\)と\(y=f(x)\)の近傍\(V\)で
\(U\)と\(V\)が微分同相となるものが存在することをいう.
逆写像定理から,\(f\)が\(x\)において局所微分同相であるためには,
接写像\(df_x\colon T_xX\to T_{f(y)}Y\)が同形である
ことが必要十分である.
\begin{EG}
    1次元多様体\(\rr\)から\(S^1\)への写像\(f\colon\rr\to S^1\)を
    \[
        f(t)\coloneqq (\cos{t},\sin{t})
    \]で定める.\(\rank df_t=\rank(-\sin{t},\cos{t})\equiv 1\)である.
    \(f\)は局所微分同相写像であるが,
    微分同相写像ではない.
\end{EG}

\paragraph{はめ込み}
\(d_X<d_Y\)のとき,
\(\rank{df_x}\leqq d_X\)であり,\(df_x\)は全射にはなりえない.
\(df_x\)が単射となるのは\(\rank{df_x}=d_X\)のときである.
\begin{DFN}
    \(X\), \(Y\)を\(C^1\)級多様体とする.
    \(f\colon X\to Y\)が\(x\)において
    \textbf{はめ込み} (immersion) 
    であるとは,
    \(df_x\colon T_xX\to T_{f(y)}Y\)
    が単射であることをいう.
    各点\(x\in X\)で\(df_x\colon T_xX\to T_{f(y)}Y\)
    が単射であるとき,\(f\)をはめ込みという.
\end{DFN}

\paragraph{うめ込み}
はめ込みは局所的な条件である.
大域的な性質を得るには位相的な条件が必要である.
\begin{DFN}
    \(X\), \(Y\)を\(C^1\)級多様体とする.
    \(f\colon X\to Y\)が
    \textbf{うめ込み} (embedding, imbedding) 
    であるとは,
    各点\(x\in X\)で\(df_x\colon T_xX\to T_{f(y)}Y\)
    が単射であり,\(f\)が\(X\)から\(f(X)\)の上への
    微分同相であることをいう.
\end{DFN}

\(f\)がうめ込みであることは,
\(f\)が単射かつ固有なはめ込みであることと同値である.

\paragraph{しずめ込み}

\(d_X> d_Y\)のとき,
\(df_x\)は単射になりえない.
\begin{DFN}
    \(X\), \(Y\)を\(C^1\)級多様体とする.
    \(f\colon X\to Y\)が\(x\)において
    \textbf{しずめ込み} (submersion) 
    であるとは,
    \(df_x\colon T_xX\to T_{f(y)}Y\)
    が全射であることをいう.
    各点\(x\in X\)で\(df_x\colon T_xX\to T_{f(y)}Y\)
    が全射であるとき,\(f\)をたんにしずめ込みという.
\end{DFN}
\(f\)が全射であることを課すこともある.
\(X\)がコンパクト,\(Y\)が連結であるとき,
しずめ込み\(f\colon X\to Y\)は全射である.

\begin{EG}
    接束\(TX\)から\(X\)への射影\(\pi\colon TX\to X\)はしずめ込みである.
\end{EG}

\subsection{位相的はめ込み・うめ込み・しずめ込み}
\cite{Le13}に滑らかでない場合のうめ込みとしずめ込み
に関する記述があったのでまとめておく.

\(X\), \(Y\)が位相空間であるとき,
\textbf{位相的うめ込み} (topological embedding, -- imbedding) 
を\(f\colon X\to Y\)が中への同相であることとして定義する\footnote{
    他の文脈だとこちらを immersion と呼ぶ本もあるらしい.
    例えば代数幾何関係だと,
    Hartshorne, Liu, 飯高, 上野の本は全て immersion と呼んでいる.
    訳語について,飯高は埋入,上野は移入と訳している.
}.

\begin{DFN}
    \(X\), \(Y\)を位相空間とする.
    \(f\colon X\to Y\)が,
    \textbf{位相的はめ込み} (topological immersion) 
    であるとは,各点\(x\in X\)に対し,
    その近傍\(U\)で\(f\rvert_U\)が位相的うめ込みであることをいう.
\end{DFN}
\begin{DFN}
    \(X\), \(Y\)を位相空間とする.
    \(f\colon X\to Y\)が,
    \textbf{位相的しずめ込み} (topological submersion) 
    であるとは,各点\(x\in X\)が,
    局所切断\(\pi\colon Y\to X\)の像に含まれることをいう.
\end{DFN}
これと\cite[Definition 3.3.1]{KS90}の関係については
まだ調べている途中.

%===============================================
% 参考文献スペース
%===============================================
\begin{thebibliography}{20} 
    \bibitem[GP74]{GP74} Victor Guillemin, Alan Pollack, 
    \textit{Differential Topology}, 
    Prentice-Hall, 1974.
    \bibitem[KS90]{KS90} Masaki Kashiwara, Pierre Schapira, 
    \textit{Sheaves on Manifolds}, 
    Grundlehren der Mathematischen Wissenschaften, 292, Springer, 1990.
    \bibitem[Le13]{Le13} John M. Lee, 
    \textit{Introduction to Smooth Manifolds}, Second Edition,
    Graduate Texts in Mathematics, \textbf{218}, Springer, 2013.
    \bibitem[Sp65]{Sp65} Michael Spivak, 
    \textit{Calculus on Manifolds}, 
    Benjamin, 1965.
\end{thebibliography}

%===============================================


\end{document}
