%=====================================
%   2024-01-25.tex
%   院生セミナー資料 大柴寿浩
%=====================================

% -----------------------
% preamble
% -----------------------
% ここから本文 (\begin{document}) までの
% ソースコードに変更を加えた場合は
% 編集者まで連絡してください. 
% Don't change preamble code yourself. 
% If you add something
% (usepackage, newtheorem, newcommand, renewcommand),
% please tell it 
% to the editor of institutional paper of RUMS.

% ------------------------
% documentclass
% ------------------------
\documentclass[11pt, a4paper, leqno, dvipdfmx]{jsarticle}

% ------------------------
% usepackage
% ------------------------
\usepackage{algorithm}
\usepackage{algorithmic}
\usepackage{amscd}
\usepackage{amsfonts}
\usepackage{amsmath}
\usepackage[psamsfonts]{amssymb}
\usepackage{amsthm}
\usepackage{ascmac}
\usepackage{bm}
\usepackage{color}
\usepackage{enumerate}
\usepackage{fancybox}
\usepackage[stable]{footmisc}
\usepackage{graphicx}
\usepackage{listings}
\usepackage{mathrsfs}
\usepackage{mathtools}
\usepackage{otf}
\usepackage{pifont}
\usepackage{proof}
\usepackage{subfigure}
\usepackage{tikz}
\usepackage{verbatim}
\usepackage[all]{xy}

\usetikzlibrary{cd}



% ================================
% パッケージを追加する場合のスペース 
\usepackage[dvipdfmx]{hyperref}
\usepackage{xcolor}
\definecolor{darkgreen}{rgb}{0,0.45,0} 
\definecolor{darkred}{rgb}{0.75,0,0}
\definecolor{darkblue}{rgb}{0,0,0.6} 
\hypersetup{
    colorlinks=true,
    citecolor=darkgreen,
    linkcolor=darkred,
    urlcolor=darkblue,
}
\usepackage{pxjahyper}

%=================================


% --------------------------
% theoremstyle
% --------------------------
\theoremstyle{definition}

% --------------------------
% newtheoem
% --------------------------

% 日本語で定理, 命題, 証明などを番号付きで用いるためのコマンドです. 
% If you want to use theorem environment in Japanece, 
% you can use these code. 
% Attention!
% All theorem enivironment numbers depend on 
% only section numbers.
\newtheorem{Axiom}{公理}[section]
\newtheorem{Definition}[Axiom]{定義}
\newtheorem{Theorem}[Axiom]{定理}
\newtheorem{Proposition}[Axiom]{命題}
\newtheorem{Lemma}[Axiom]{補題}
\newtheorem{Corollary}[Axiom]{系}
\newtheorem{Example}[Axiom]{例}
\newtheorem{Claim}[Axiom]{主張}
\newtheorem{Property}[Axiom]{性質}
\newtheorem{Attention}[Axiom]{注意}
\newtheorem{Question}[Axiom]{問}
\newtheorem{Problem}[Axiom]{問題}
\newtheorem{Consideration}[Axiom]{考察}
\newtheorem{Alert}[Axiom]{警告}
\newtheorem{Fact}[Axiom]{事実}


% 日本語で定理, 命題, 証明などを番号なしで用いるためのコマンドです. 
% If you want to use theorem environment with no number in Japanese, You can use these code.
\newtheorem*{Axiom*}{公理}
\newtheorem*{Definition*}{定義}
\newtheorem*{Theorem*}{定理}
\newtheorem*{Proposition*}{命題}
\newtheorem*{Lemma*}{補題}
\newtheorem*{Example*}{例}
\newtheorem*{Corollary*}{系}
\newtheorem*{Claim*}{主張}
\newtheorem*{Property*}{性質}
\newtheorem*{Attention*}{注意}
\newtheorem*{Question*}{問}
\newtheorem*{Problem*}{問題}
\newtheorem*{Consideration*}{考察}
\newtheorem*{Alert*}{警告}
\newtheorem{Fact*}{事実}


% 英語で定理, 命題, 証明などを番号付きで用いるためのコマンドです. 
% If you want to use theorem environment in English, You can use these code.
%all theorem enivironment number depend on only section number.
\newtheorem{Axiom+}{Axiom}[section]
\newtheorem{Definition+}[Axiom+]{Definition}
\newtheorem{Theorem+}[Axiom+]{Theorem}
\newtheorem{Proposition+}[Axiom+]{Proposition}
\newtheorem{Lemma+}[Axiom+]{Lemma}
\newtheorem{Example+}[Axiom+]{Example}
\newtheorem{Corollary+}[Axiom+]{Corollary}
\newtheorem{Claim+}[Axiom+]{Claim}
\newtheorem{Property+}[Axiom+]{Property}
\newtheorem{Attention+}[Axiom+]{Attention}
\newtheorem{Question+}[Axiom+]{Question}
\newtheorem{Problem+}[Axiom+]{Problem}
\newtheorem{Consideration+}[Axiom+]{Consideration}
\newtheorem{Alert+}{Alert}
\newtheorem{Fact+}[Axiom+]{Fact}
\newtheorem{Remark+}[Axiom+]{Remark}

% ----------------------------
% commmand
% ----------------------------
% 執筆に便利なコマンド集です. 
% コマンドを追加する場合は下のスペースへ. 

% 集合の記号 (黒板文字)
\newcommand{\NN}{\mathbb{N}}
\newcommand{\ZZ}{\mathbb{Z}}
\newcommand{\QQ}{\mathbb{Q}}
\newcommand{\RR}{\mathbb{R}}
\newcommand{\CC}{\mathbb{C}}
\newcommand{\PP}{\mathbb{P}}
\newcommand{\KK}{\mathbb{K}}


% 集合の記号 (太文字)
\newcommand{\nn}{\mathbf{N}}
\newcommand{\zz}{\mathbf{Z}}
\newcommand{\qq}{\mathbf{Q}}
\newcommand{\rr}{\mathbf{R}}
\newcommand{\cc}{\mathbf{C}}
\newcommand{\pp}{\mathbf{P}}
\newcommand{\kk}{\mathbf{K}}

% 特殊な写像の記号
\newcommand{\ev}{\mathop{\mathrm{ev}}\nolimits} % 値写像
\newcommand{\pr}{\mathop{\mathrm{pr}}\nolimits} % 射影

% スクリプト体にするコマンド
%   例えば {\mcal C} のように用いる
\newcommand{\mcal}{\mathcal}

% 花文字にするコマンド 
%   例えば {\h C} のように用いる
\newcommand{\h}{\mathscr}

% ヒルベルト空間などの記号
\newcommand{\F}{\mcal{F}}
\newcommand{\X}{\mcal{X}}
\newcommand{\Y}{\mcal{Y}}
\newcommand{\Hil}{\mcal{H}}
\newcommand{\RKHS}{\Hil_{k}}
\newcommand{\Loss}{\mcal{L}_{D}}
\newcommand{\MLsp}{(\X, \Y, D, \Hil, \Loss)}

% 偏微分作用素の記号
\newcommand{\p}{\partial}

% 角カッコの記号 (内積は下にマクロがあります)
\newcommand{\lan}{\langle}
\newcommand{\ran}{\rangle}



% 圏の記号など
\newcommand{\Set}{{\bf Set}}
\newcommand{\Vect}{{\bf Vect}}
\newcommand{\FDVect}{{\bf FDVect}}
%\newcommand{\Ring}{{\bf Ring}}
\newcommand{\Ab}{{\bf Ab}}
\newcommand{\Mod}{\mathop{\mathrm{Mod}}\nolimits}
\newcommand{\CGA}{{\bf CGA}}
\newcommand{\GVect}{{\bf GVect}}
\newcommand{\Lie}{{\bf Lie}}
\newcommand{\dLie}{{\bf Liec}}



% 射の集合など
\newcommand{\Map}{\mathop{\mathrm{Map}}\nolimits} % 写像の集合
\newcommand{\Hom}{\mathop{\mathrm{Hom}}\nolimits} % 射集合
\newcommand{\End}{\mathop{\mathrm{End}}\nolimits} % 自己準同型の集合
\newcommand{\Aut}{\mathop{\mathrm{Aut}}\nolimits} % 自己同型の集合
\newcommand{\Mor}{\mathop{\mathrm{Mor}}\nolimits} % 射集合
\newcommand{\Ker}{\mathop{\mathrm{Ker}}\nolimits} % 核
\newcommand{\Img}{\mathop{\mathrm{Im}}\nolimits} % 像
\newcommand{\Cok}{\mathop{\mathrm{Coker}}\nolimits} % 余核
\newcommand{\Cim}{\mathop{\mathrm{Coim}}\nolimits} % 余像

% その他便利なコマンド
\newcommand{\dip}{\displaystyle} % 本文中で数式モード
\newcommand{\e}{\varepsilon} % イプシロン
\newcommand{\dl}{\delta} % デルタ
\newcommand{\pphi}{\varphi} % ファイ
\newcommand{\ti}{\tilde} % チルダ
\newcommand{\pal}{\parallel} % 平行
\newcommand{\op}{{\rm op}} % 双対を取る記号
\newcommand{\lcm}{\mathop{\mathrm{lcm}}\nolimits} % 最小公倍数の記号
\newcommand{\Probsp}{(\Omega, \F, \P)} 
\newcommand{\argmax}{\mathop{\rm arg~max}\limits}
\newcommand{\argmin}{\mathop{\rm arg~min}\limits}





% ================================
% コマンドを追加する場合のスペース 
%\newcommand{\OO}{\mcal{O}}



\renewcommand\proofname{\bf 証明} % 証明
\numberwithin{equation}{section}
\newcommand{\cTop}{\textsf{Top}}
%\newcommand{\cOpen}{\textsf{Open}}
\newcommand{\Op}{\mathop{\textsf{Open}}\nolimits}
\newcommand{\Ob}{\mathop{\textrm{Ob}}\nolimits}
\newcommand{\id}{\mathop{\mathrm{id}}\nolimits}
\newcommand{\pt}{\mathop{\mathrm{pt}}\nolimits}
\newcommand{\res}{\mathop{\rho}\nolimits}
\newcommand{\A}{\mcal{A}}
\newcommand{\B}{\mcal{B}}
\newcommand{\C}{\mcal{C}}
\newcommand{\D}{\mcal{D}}
\newcommand{\E}{\mcal{E}}
\newcommand{\G}{\mcal{G}}
%\newcommand{\H}{\mcal{H}}
\newcommand{\I}{\mcal{I}}
\newcommand{\J}{\mcal{J}}
\newcommand{\OO}{\mcal{O}}
\newcommand{\Ring}{\mathop{\textsf{Ring}}\nolimits}
\newcommand{\cAb}{\mathop{\textsf{Ab}}\nolimits}
%\newcommand{\Ker}{\mathop{\mathrm{Ker}}\nolimits}
\newcommand{\im}{\mathop{\mathrm{Im}}\nolimits}
\newcommand{\Coker}{\mathop{\mathrm{Coker}}\nolimits}
\newcommand{\Coim}{\mathop{\mathrm{Coim}}\nolimits}
\newcommand{\rank}{\mathop{\mathrm{rank}}\nolimits}
\newcommand{\Ht}{\mathop{\mathrm{Ht}}\nolimits}
\newcommand{\supp}{\mathop{\mathrm{supp}}\nolimits}
\newcommand{\colim}{\mathop{\mathrm{colim}}}
\newcommand{\Tor}{\mathop{\mathrm{Tor}}\nolimits}

\newcommand{\cat}{\mathscr{C}}

\newcommand{\scA}{\mathscr{A}}
\newcommand{\scB}{\mathscr{B}}
\newcommand{\scC}{\mathscr{C}}
\newcommand{\scD}{\mathscr{D}}
\newcommand{\scE}{\mathscr{E}}
\newcommand{\scF}{\mathscr{F}}
\newcommand{\scN}{\mathscr{N}}
\newcommand{\scO}{\mathscr{O}}
\newcommand{\scV}{\mathscr{V}}


\newcommand{\ibA}{\mathop{\text{\textit{\textbf{A}}}}}
\newcommand{\ibB}{\mathop{\text{\textit{\textbf{B}}}}}
\newcommand{\ibC}{\mathop{\text{\textit{\textbf{C}}}}}
\newcommand{\ibD}{\mathop{\text{\textit{\textbf{D}}}}}
\newcommand{\ibE}{\mathop{\text{\textit{\textbf{E}}}}}
\newcommand{\ibF}{\mathop{\text{\textit{\textbf{F}}}}}
\newcommand{\ibG}{\mathop{\text{\textit{\textbf{G}}}}}
\newcommand{\ibH}{\mathop{\text{\textit{\textbf{H}}}}}
\newcommand{\ibI}{\mathop{\text{\textit{\textbf{I}}}}}
\newcommand{\ibJ}{\mathop{\text{\textit{\textbf{J}}}}}
\newcommand{\ibK}{\mathop{\text{\textit{\textbf{K}}}}}
\newcommand{\ibL}{\mathop{\text{\textit{\textbf{L}}}}}
\newcommand{\ibM}{\mathop{\text{\textit{\textbf{M}}}}}
\newcommand{\ibN}{\mathop{\text{\textit{\textbf{N}}}}}
\newcommand{\ibO}{\mathop{\text{\textit{\textbf{O}}}}}
\newcommand{\ibP}{\mathop{\text{\textit{\textbf{P}}}}}
\newcommand{\ibQ}{\mathop{\text{\textit{\textbf{Q}}}}}
\newcommand{\ibR}{\mathop{\text{\textit{\textbf{R}}}}}
\newcommand{\ibS}{\mathop{\text{\textit{\textbf{S}}}}}
\newcommand{\ibT}{\mathop{\text{\textit{\textbf{T}}}}}
\newcommand{\ibU}{\mathop{\text{\textit{\textbf{U}}}}}
\newcommand{\ibV}{\mathop{\text{\textit{\textbf{V}}}}}
\newcommand{\ibW}{\mathop{\text{\textit{\textbf{W}}}}}
\newcommand{\ibX}{\mathop{\text{\textit{\textbf{X}}}}}
\newcommand{\ibY}{\mathop{\text{\textit{\textbf{Y}}}}}
\newcommand{\ibZ}{\mathop{\text{\textit{\textbf{Z}}}}}

\newcommand{\ibx}{\mathop{\text{\textit{\textbf{x}}}}}

%\newcommand{\Comp}{\mathop{\mathrm{C}}\nolimits}
%\newcommand{\Komp}{\mathop{\mathrm{K}}\nolimits}
%\newcommand{\Domp}{\mathop{\mathsf{D}}\nolimits}%複体のホモトピー圏
%\newcommand{\Comp}{\mathrm{C}}
%\newcommand{\Komp}{\mathrm{K}}
%\newcommand{\Domp}{\mathsf{D}}%複体のホモトピー圏

\newcommand{\Comp}{\mathop{\mathrm{C}}\nolimits}
\newcommand{\Komp}{\mathop{\mathsf{K}}\nolimits}
\newcommand{\Domp}{\mathop{\mathsf{D}}\nolimits}
\newcommand{\Kompl}{\mathop{\mathsf{K}^\mathrm{+}}\nolimits}
\newcommand{\Kompu}{\mathop{\mathsf{K}^\mathrm{-}}\nolimits}
\newcommand{\Kompb}{\mathop{\mathsf{K}^\mathrm{b}}\nolimits}
\newcommand{\Dompl}{\mathop{\mathsf{D}^\mathrm{+}}\nolimits}
\newcommand{\Dompu}{\mathop{\mathsf{D}^\mathrm{-}}\nolimits}
\newcommand{\Dompb}{\mathop{\mathsf{D}^\mathrm{b}}\nolimits}




\newcommand{\CCat}{\Comp(\cat)}
\newcommand{\KCat}{\Komp(\cat)}
\newcommand{\DCat}{\Domp(\cat)}%圏Cの複体のホモトピー圏
\newcommand{\HOM}{\mathop{\mathscr{H}\hspace{-2pt}om}\nolimits}%内部Hom
\newcommand{\RHOM}{\mathop{\mathrm{R}\hspace{-1.5pt}\HOM}\nolimits}

\newcommand{\muS}{\mathop{\mathrm{SS}}\nolimits}
\newcommand{\RG}{\mathop{\mathrm{R}\hspace{-0pt}\Gamma}\nolimits}
\newcommand{\RHom}{\mathop{\mathrm{R}\hspace{-1.5pt}\Hom}\nolimits}
\newcommand{\Rder}{\mathrm{R}}

\newcommand{\simar}{\mathrel{\overset{\sim}{\rightarrow}}}%同型右矢印
\newcommand{\simarr}{\mathrel{\overset{\sim}{\longrightarrow}}}%同型右矢印
\newcommand{\simra}{\mathrel{\overset{\sim}{\leftarrow}}}%同型左矢印
\newcommand{\simrra}{\mathrel{\overset{\sim}{\longleftarrow}}}%同型左矢印

\newcommand{\hocolim}{{\mathrm{hocolim}}}
\newcommand{\indlim}[1][]{\mathop{\varinjlim}\limits_{#1}}
\newcommand{\sindlim}[1][]{\smash{\mathop{\varinjlim}\limits_{#1}}\,}
\newcommand{\Pro}{\mathrm{Pro}}
\newcommand{\Ind}{\mathrm{Ind}}
\newcommand{\prolim}[1][]{\mathop{\varprojlim}\limits_{#1}}
\newcommand{\sprolim}[1][]{\smash{\mathop{\varprojlim}\limits_{#1}}\,}

\newcommand{\Sh}{\mathrm{Sh}}
\newcommand{\PSh}{\mathrm{PSh}}

\newcommand{\rmD}{\mathrm{D}}

\newcommand{\Lloc}[1][]{\mathord{\mathcal{L}^1_{\mathrm{loc},{#1}}}}
\newcommand{\ori}{\mathord{\mathrm{or}}}
\newcommand{\Db}{\mathord{\mathscr{D}b}}

\newcommand{\codim}{\mathop{\mathrm{codim}}\nolimits}











%================================================
% 自前の定理環境
%   https://mathlandscape.com/latex-amsthm/
% を参考にした
\newtheoremstyle{mystyle}%   % スタイル名
    {5pt}%                   % 上部スペース
    {5pt}%                   % 下部スペース
    {}%              % 本文フォント
    {}%                  % 1行目のインデント量
    {\bfseries}%                      % 見出しフォント
    {.}%                     % 見出し後の句読点
    {12pt}%                     % 見出し後のスペース
    {\thmname{#1}\thmnumber{ #2}\thmnote{{\hspace{2pt}\normalfont (#3)}}}% % 見出しの書式

\theoremstyle{mystyle}
\newtheorem{AXM}{公理}[section]
\newtheorem{DFN}[AXM]{定義}
\newtheorem{THM}[AXM]{定理}
\newtheorem*{THM*}{定理}
\newtheorem{PRP}[AXM]{命題}
\newtheorem{LMM}[AXM]{補題}
\newtheorem{CRL}[AXM]{系}
\newtheorem{EG}[AXM]{例}
\newtheorem*{EG*}{例}
\newtheorem{RMK}[AXM]{注意}
\newtheorem{CNV}[AXM]{約束}
\newtheorem{CMT}[AXM]{コメント}
\newtheorem{NTN}[AXM]{記号}

% 定理環境ここまで
%====================================================

\usepackage{framed}
\definecolor{lightgray}{rgb}{0.75,0.75,0.75}
\renewenvironment{leftbar}{%
  \def\FrameCommand{\textcolor{lightgray}{\vrule width 4pt} \hspace{10pt}}% 
  \MakeFramed {\advance\hsize-\width \FrameRestore}}%
{\endMakeFramed}
\newenvironment{redleftbar}{%
  \def\FrameCommand{\textcolor{lightgray}{\vrule width 1pt} \hspace{10pt}}% 
  \MakeFramed {\advance\hsize-\width \FrameRestore}}%
 {\endMakeFramed}


% =================================





% ---------------------------
% new definition macro
% ---------------------------
% 便利なマクロ集です

% 内積のマクロ
%   例えば \inner<\pphi | \psi> のように用いる
\def\inner<#1>{\langle #1 \rangle}

% ================================
% マクロを追加する場合のスペース 

%=================================





% ----------------------------
% documenet 
% ----------------------------
% 以下, 本文の執筆スペースです. 
% Your main code must be written between 
% begin document and end document.
% ---------------------------

\title{2024/01/25 セミナー資料}
\author{大柴寿浩}
\date{2024/01/25}
\begin{document}
\maketitle

\section*{あとで使う定理(復習)}
\begin{leftbar}
\begin{PRP}[{\cite[Proposition 3.1.8]{KS90}}]\label{PRP318}
    \(f\colon Y\to X\),\(g\colon Z\to Y\)を
    局所コンパクト空間の間の連続写像とする.
    \(f_!\)と\(g_!\)のコホモロジー次元が有限であるとする.
    このとき,\((f\circ g)_!\)のコホモロジー次元は有限で
    \[
        (f\circ g)^!\cong g^!\circ f^!
    \]である.
\end{PRP}
\end{leftbar}
\begin{leftbar}
    \begin{PRP}[{\cite[Proposition 3.1.12]{KS90}}]\label{PRP3112}
        \(f\colon Y\to X\)を
        \(Y\)から\(X\)の局所閉集合\(Z\)の上への同相写像とする.
        このとき,
        \[
            f^!\cong f^{-1}\circ \RG_{f(Y)}(\cdot)
        \]である.
    \end{PRP}
\end{leftbar}

\begin{leftbar}
    \begin{PRP}[{\cite[Proposition 3.3.2]{KS90}}]\label{PRP332}
        \(f\colon Y\to X\)をファイバー次元\(l\)の位相的沈めこみとする.
        \begin{enumerate}[(i)]
            \item \(k\ne -l\)に対し\(H^k(f^!A_X)=0\)であり,局所的に\(H^{-l}(f^!A_X)\cong A_Y\)である.
            \item \(f^{-1}(\cdot)\otimes\omega_{Y/X}\to f^!(\cdot)\)は同型である.\label{332-2}
        \end{enumerate}
    \end{PRP}    
\end{leftbar}
\section*{層の例({\cite[\S2.9]{KS90}}から)}

\subsection{向きづけ,微分形式,密度}
\(C^0\)多様体\(M\)上の層として,向きづけ層\(\ori _M\)を
考えることも必要になってくる.
\(\ori_M\)は\(\zz_M\)と局所的に同型な層であり,
\(M\)の向きが存在する場合,
その向きを選ぶことと同型\(\ori_M\cong\zz_M\)を選ぶことが
同義となるようなものである.
\(\ori_M\)については次章で詳しくしらべる.

いま,\(\alpha=\infty\)または\(\alpha=\omega\)とし,
\(p\)を整数とする.\(C^\alpha_M\)を係数にもつ\(p\)次微分形式
の層を\(C_M^{\alpha,(p)}\)とおく.
また外微分を\(
    d\colon C_M^{\alpha,(p)}\to C_M^{\alpha,(p+1)}
\)で表す.

\((x_1,\dots,x_n)\)が\(M\)上の局所座標系であるとする.
このとき,\(p\)形式\(f\)は次の形にただ一通りに表されるのであった.
\[
    f=\sum_{\lvert I\rvert=p}^{}f_Idx_I,
\]ここに,\(
    I=\left\{i_1,\dots,i_p\right\}\subset\{1,\dots,n\}
\), \((i_1<i_2<\dots<i_p)\), 
\(dx_I=dx_{i_1}\wedge\dots\wedge dx_{i_p}\)で,
\(f_I\)は\(C^\alpha_M\)の切断である.
このとき,\[
    df=\sum_{i=1}^{n}\sum_{\lvert I\rvert=p}^{}\frac{\partial f_I}{\partial x_i}dx_i\wedge dx_I
\]となるのであった.
もうひとつ層を導入する.
\[
    \scV_M^\alpha
    \coloneqq
    C^{\alpha,(n)}_M\otimes\ori_M
%    \quad
 %   \text{}
\](\(\alpha=\infty\)または\(\alpha=\omega\))とおき,
\(M\)上の\(C^\alpha\)密度の層とよぶ.

コンパクト台をもつ\(C^\infty\)密度は積分することができる.
\(\int_M\cdot\)で積分写像
\begin{equation}
    \int_M\cdot\colon\Gamma_c(M;\scV_M^\infty)\to\cc \label{eq:int}
\end{equation}を表す.
\(C^{\alpha,(p)}_M\)と\(\scV_M^\alpha\)は
\(C^\alpha_M\)加群の層である.

「1の分割」の存在から,
層\(C^\alpha_M\),\(C^{\alpha,(p)}\),\(\scV_M^\alpha\)は
\(\alpha\neq\omega\)に対してはc柔軟であることが従う.
層\(C^\omega_M\),\(C^{\omega,(p)}\),\(\scV_M^\omega\)は
関手\(\Gamma(M;\cdot)\)に対し非輪状,
すなわち\(j>0\)に対し\(H^j(M;C^\omega_M)=0\)である.
Grauert\cite{G58}を参照.

\subsection{分布と超関数}
\(C^\infty\)多様体\(M\)上には
シュワルツ分布の層\(\Db_M\)が自然に定まる
(Schwartz\cite{S66},de Rham\cite{R55}を参照).
\(\Db_M\)はc柔軟層であり,
\(\Gamma_c(M;\Db_M)\)は\(\Gamma(M;\scV_M^\infty)\)の
双対位相線形空間である.
ただし,\(\Gamma(M;\scV_M^\infty)\)には
フレシェ空間としての自然な位相を入れている.

\(C^\omega\)多様体\(M\)上にも同様に
佐藤超関数の層\(\scB_M\)が自然に定まる(佐藤\cite{Sa59}を参照).
\(\scB_M\)は脆弱層であり,\(\Gamma_c(M;\scB_M)\)は
\(\Gamma(M;\scV_M^\omega)\)の
双対位相線形空間である.
ただし,\(\Gamma(M;\scV_M^\omega)\)には
DFS空間としての自然な位相を入れている
(MartineauとSchapiraに詳細な解説がある).
しかし,佐藤による構成は純粋にコホモロジーによるものである.
後ほど\ref{ssec:sato}項で復習する.

積分写像\eqref{eq:int}はペアリング
\begin{equation}
    \begin{array}{ccc}
        {\Gamma(M;C_M^\infty)\times\Gamma_c(M;\scV_M^\infty)}&\longrightarrow&\cc\\
        \rotatebox{90}{$\in$}&&\rotatebox{90}{$\in$}\\
        (f,g)&\longmapsto&\dip\int_Mfg
    \end{array}
\end{equation}
を定める.
このペアリングから\(C^\infty_M\)から\(\Db_M\)への層の射がひきおこされ,
この射が単射であることも示せる.
さらに,実解析多様体\(M\)の上では,
単射\(\Gamma(M;\scV_M^\omega)\to\Gamma(M;\scV_M^\infty)\)から
射\(\Db_M\to\scB_M\)が引き起こされ,こちらも単射であることがわかる.

分布係数の\(p\)形式の層\(
    \Db_M^{(p)}
    \coloneqq 
    C^{\infty,(p)}_M\otimes_{C^\infty_M}\Db_M
\)や超関数係数の\(p\)形式の層\(
    \scB_M^{(p)}
    \coloneqq 
    C^{\omega,(p)}_M\otimes_{C^\omega_M}\scB_M
\)も定義することができる.
\(\Db_M^{(p)}\)はc柔軟層,\(\scB_M^{(p)}\)は脆弱層である.

\section{向きづけと双対性{\cite[3.3]{KS90}}}

\begin{leftbar}
\begin{PRP}[{\cite[Prop3.3.6]{KS90}}]
    \(X\)を\(n\)次元\(C^0\)多様体とする.
    \begin{enumerate}[(i)]
        \item \(\ori_X\)は前層\(U\mapsto\Hom(H^n_c(U;A_X),A)\)から誘導された層である.
        \item \(x\in X\)に対し,
        標準的な同型\(
            \ori_{X,x}
            \cong\Hom\left(H^n_{\{x\}}(X;A_X),A\right)
            \cong H^n_{\{x\}}(X;A_X)
        \)が存在する.
        \item \(X\)が向きづけられた可微分多様体であるとする.このとき,同型\(\ori_X\cong A_X\)が存在する.この同型は\(X\)の向きをとりかえることで符号が変わる.
    \end{enumerate}
\end{PRP}
\end{leftbar}

\begin{leftbar}
\begin{LMM}[{\cite[Prop3.3.7]{KS90}}]
    \(E\)をユークリッド空間\(\rr^n\)とし,
    同型\(\ori_E\cong A_E\)を固定する.
    \(U\)と\(V\)を\(E\)の開集合とし,\(f\colon U\to V\)を
    微分同相写像とする.このとき,次の図式は可換である.
    \[    
        \vcenter{\xymatrix
        @C=22pt@R=32pt
        {
        \ori_U
        \ar@{-}[r]^-{\sim}
        &
        \ori_{E/U}\ar@{-}[r]^-{\sim}
        &
        A_U
        \\
        f^{-1}(\ori_V)\ar@{-}[r]^-{\sim}
        \ar[u]^-{f_{\ori}^\sharp}
        &
        f^{-1}(\ori_{E/V})\ar@{-}[r]^-{\sim}
        &
        f^{-1}(A_V)\ar[u]_-{f_{A}^\sharp} 
        }}    
    \]
    (ただし,射\(f_{\ori}^\sharp\)と\(f_{A}^\sharp\)は
    次のように定義する.
    \(a_U\)と\(a_V\)を\(U\)と\(V\)から\(\{\pt\}\)への射影とする.
    このとき\(f_A^\sharp\)は同型\(
        f^{-1}\circ a_{V}^{-1}\simar a_{U}^{-1}
    \)で\(f_{\ori}^\sharp\)は同型\(
        f^{-1}\circ a_{V}^{-1}[-n]
        \simar f^!\circ a^!_{V}[-n]
        \simar  \circ a_{U}^{-1}
    \)である.\(f^{-1}\cong f^!\)である.)
\end{LMM}
\end{leftbar}

\(f\colon Y\to X\)を\(C^0\)多様体の間の射とする.
\(f\)を閉うめ込み\(j\colon Y\hookrightarrow Y\times X\)と
しずめ込み\(p\colon Y\times X\to X\)の合成に分解して
\begin{equation}
    f\colon Y\overset{j}{\hookrightarrow}
    Y\times X\overset{p}{\longrightarrow}X\tag{3.3.5}\label{eq:decompose}
\end{equation}とかくことができる.
\(j(y)=(y,f(y))\)で\(p(y,x)=x\)である.
%\cite[Proposition 3.1.8, 3.1.12, 3.3.2]{KS90}
命題\ref{PRP318}--\ref{PRP332}を適用すると,
\(F\in\Dompl(A_X)\)に対し,
\begin{align*}
    f^!{F}\cong(p\circ j)^!{F}
    &\underset{\text{命題\ref{PRP318}}}{\cong} j^! p^!{F}\\
    &\underset{\text{命題\ref{PRP3112}}}{\cong} j^{-1} \RG_{j(Y)}\left(p^!{F}\right)\\
    &\underset{\text{命題\ref{PRP332} \eqref{332-2}}}{\cong} j^{-1} \RG_{j(Y)}p^{-1}{F}\otimes\omega_{(Y\times X)/X}\\
    &\cong j^{-1} \RG_{j(Y)}p^{-1}{F}\otimes\ori_{{(Y\times X)}/X}[\dim Y\times X-\dim X]\\
    &\cong j^{-1} \RG_{j(Y)}p^{-1}{F}\otimes\ori_Y[\dim Y],
\end{align*}
すなわち
\begin{equation}
    f^!{F}
    \cong j^{-1} \RG_{j(Y)}p^{-1}{F}\otimes\ori_Y[\dim Y]\tag{3.3.6}
\end{equation}
である.
\(F=A_X\)のとき,
\begin{equation}
    \ori_{Y/X}\cong
    j^{-1} \left(H^{\dim Y}_{j(Y)}(A_{Y\times X})\right)\otimes\ori_Y\tag{3.3.7}
\end{equation}
である.
さらに\(f=\id_X\)のときは,
\begin{equation}
    \ori_{X}\cong
    H^{\dim X}_{X}(A_{X\times X})\rvert_{X}\tag{3.3.8}
\end{equation}
である(\(X\)と\(X\times X\)の対角集合を同一視した).

\begin{leftbar}
\begin{NTN}[{\cite[Notation 3.3.8]{KS90}}]
    本書ではX章\S3をのぞき,実多様体\(X\)の次元を\(\dim X\)で表す.
    \(f\colon Y\to X\)を\(C^0\)多様体の射とするとき,
    次のようにおく.
    \begin{equation}
        \dim{Y/X}\coloneqq \dim Y-\dim{X}.\tag{3.3.9}
    \end{equation}
    \(Y\)が部分多様体のとき,
    \[
        \codim_{X}Y\coloneqq-\dim{Y/X}
    \]
    のようにもかく.
    混同する恐れがないときは\(\codim_X Y\)を\(\codim Y\)と略記する.
    次が成り立つ.
    \begin{equation}
        \omega_{Y/X}\cong
        \ori_{Y/X}[\dim{Y/X}].\tag{3.3.10}
    \end{equation}
    このとき,次のようにおくと自然.
    \begin{align}
            \omega_{Y/X}^{\otimes{-1}}&\coloneqq
            \RHOM(\omega_{Y/X},A_Y),\tag{3.3.11}\\
            &\cong\ori_{Y/X}[-\dim{Y/X}].\notag
    \end{align}
\end{NTN}
\end{leftbar}
\begin{leftbar}
\begin{PRP}[{\cite[Proposition 3.3.9]{KS90}}]\label{339}
    \(f\colon Y\to X\)を局所コンパクト空間の間の連続写像とする.
    次の条件が成りたつとする.
    \begin{enumerate}[(i)]
        \item \(f\)は位相的しずめ込みである.
        \item \(\Rder f_!f^!\zz_X\to\zz_X\)は同型である.\label{339-2}
    \end{enumerate}
    このとき,\(F\in\Dompl(\zz_X)\)に対し,射\(F\to \Rder f_\ast f^{-1}\)は同型である.
\end{PRP}
\end{leftbar}


\begin{RMK}
    \(f\colon Y\to X\)を局所コンパクト空間の間の連続写像とし,
    \(f\)はファイバー次元\(l\)の位相的しずめ込みであるとする.
    このとき,命題\ref{339}の条件\eqref{339-2}が成りたつためには
    任意の\(x\in X\)に対し,次の同型が成り立つことが必要十分である.
    \begin{equation}
        \RG_c(f^{-1}(x);\omega_{f^{-1}(x)})\simar\zz.\tag{3.3.12}
    \end{equation}
    この同型は次の同型と同値である.
    \begin{equation}
        \zz\simar
        \RG(f^{-1}(x);\zz_{f^{-1}(x)}).\tag{3.3.13}
    \end{equation}
\end{RMK}
実際,\(M\coloneqq\RG_c(f^{-1}(x);\omega_{f^{-1}(x)})\)とおき,
\(M^\ast\RHom(M,\zz)\)とおく.
このとき,\(M^\ast\cong \RG(f^{-1}(x);\zz_{f^{-1}(x)})\)
であり,\(\Dompb(\Mod(\zz))\)における同型\(M\simar\zz\)は
同型\(\zz\simar M^\ast\)と同じである.

\bigskip

いま,\(X\)を\(n\)次元\(C^0\)多様体とし,
\(\mathrm{a}_X\)で写像\(X\to\{\pt\}\)を表す.
射\(
    {\Rder\mathrm{a}_X}_!\mathrm{a}_X^!A_{\{\pt\}}
    \to A_{\{\pt\}}
\)から射
\begin{equation}
    {\Rder\mathrm{a}_X}_!\omega_X\to A\tag{3.3.14}
\end{equation}
が定まる.
\(0\)次コホモロジーをとることで,「積分射」
\begin{equation}\label{eq:int-mor-or}
    \int_X\colon H^n_c(X;\ori_X)\to A\tag{3.3.15}
\end{equation}
が定まる.
他方で,\(A=\cc\)かつ\(X\)が\(C^\infty\)多様体であるとき,
よく知られた射\(H^n_c(X;\ori_X)\to \cc\)が次のようにして得られる.
\(\ori_X\)はド・ラーム複体と擬同形である:
\[
    0\to C_X^{\infty,(0)}\otimes\ori_X\underset{d}{\to}
    \cdots
    \to
    C_X^{\infty,(n)}\otimes\ori_X
    \to0.
\]
\(
    C_X^{\infty,(j)}\otimes\ori_X
\)はc柔軟なので,
\[
    H^n_c(X;\ori_X)
    \cong
    \Gamma_c(X;C^{\infty,(n)}\otimes\ori_X)/d\Gamma_c(X;C^{\infty,(n-1)}\otimes\ori_X)
\]である.
\(\phi\)をコンパクト台をもつ密度,
すなわち\(\Gamma_c(X;C^{\infty,(n)}\otimes\ori_X)\)の元とすると,
\(\int_X\phi\)が意味をもち,ストークスの定理から
\(\psi\in\Gamma_c(X;C^{\infty,(n-1)}\otimes\ori_X)\)で
\(\phi=d\psi\)となるものが存在するとき\(\int_X\phi=0\)となる.
したがって,\(\int_X\)は射
\begin{equation}\label{eq:int-mor-dens}
    \int_X\colon
    \Gamma_c(X;C^{\infty,(n)}\otimes\ori_X)/d\Gamma_c(X;C^{\infty,(n-1)}\otimes\ori_X)
    \to\cc\tag{3.3.16}
\end{equation}
を定める.
この射\eqref{eq:int-mor-dens}は\eqref{eq:int-mor-or}と
0でない定数倍を除いて一致する.

\begin{PRP}
    \(X\)を\(C^0\)多様体とし,
\end{PRP}










%===============================================
% 参考文献スペース
%===============================================
\begin{thebibliography}{20} 
    \bibitem[Le13]{Le13} John M. Lee, 
    \textit{Introduction to Smooth Manifolds}, Second Edition,
    Graduate Texts in Mathematics, \textbf{218}, Springer, 2013.
    \bibitem[Sp65]{Sp65} Michael Spivak, 
    \textit{Calculus on Manifolds}, 
    Benjamin, 1965.

    \bibitem[B+84]{B+84} Borel, 
    \textit{Intersection Cohomology}, 
    Progress in Mathematics, 50, Birkh\"auser, 1984.
\bibitem[G58]{G58} Grauert, 
    \textit{On Levi's problem and the embedding of real analytic manifolds}, 
    Ann. Math. 68, 460--472 (1958).
\bibitem[GP74]{GP74} Victor Guillemin, Alan Pollack, 
    \textit{Differential Topology}, 
    Prentice-Hall, 1974.
\bibitem[KS90]{KS90} Masaki Kashiwara, Pierre Schapira, 
    \textit{Sheaves on Manifolds}, 
    Grundlehren der Mathematischen Wissenschaften, 292, Springer, 1990.
\bibitem[KS06]{KS06} Masaki Kashiwara, Pierre Schapira, 
    \textit{Categories and Sheaves}, 
    Grundlehren der Mathematischen Wissenschaften, 332, Springer, 2006.
\bibitem[R55]{R55} de Rham, 
    \textit{Vari'et'es diff\'erentiables}, 
    Hermann, Paris, 1955.
\bibitem[Sa59]{Sa59} Mikio Sato, 
    \textit{Theory of Hyperfunctions}, 
    1959--60.
\bibitem[S66]{S66} Schwartz, 
    \textit{Th\'eorie de distributions}, 
    Hermann, Paris, 1966.
\bibitem[Sh16]{Sh16} 志甫淳, 層とホモロジー代数, 共立出版, 2016.
\bibitem[Ike21]{Ike21} 池祐一, 超局所層理論概説, 2021.
\bibitem[Tak17]{Tak17} 竹内潔, \(\D\)加群, 共立出版, 2017.

\end{thebibliography}

%===============================================


\end{document}
