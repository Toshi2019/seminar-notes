\documentclass[dvipdfmx,12pt,aspectratio=169,leqno]{beamer}% dvipdfmxしたい
\usepackage{bxdpx-beamer}% dvipdfmxなので必要
\usepackage{pxjahyper}% 日本語で'しおり'したい
\usepackage{minijs}% min10ヤダ
\renewcommand{\kanjifamilydefault}{\gtdefault}
\renewcommand{\emph}[1]{{\upshape\bfseries #1}}

\usepackage{amsmath}
\usepackage{amsthm}
\usepackage{tikz}
\usepackage{color}
\usepackage{ascmac}
\usepackage{amsfonts}
\usepackage{mathrsfs}
\usepackage{mathtools}
\usepackage{amssymb}
\usepackage{graphicx}
\usepackage{fancybox}
\usepackage{enumerate}
\usepackage{verbatim}
\usepackage{subfigure}
\usepackage{proof}
\usepackage{listings}
\usepackage{otf}
\usepackage[all]{xy}
\usepackage{amscd}
%\usepackage[dvipdfmx]{hyperref}

\usepackage{xcolor}
\definecolor{darkgreen}{rgb}{0,0.45,0} 
\definecolor{darkred}{rgb}{0.75,0,0}
\definecolor{darkblue}{rgb}{0,0,0.6} 
\hypersetup{
    colorlinks=true,
    citecolor=darkgreen,
    linkcolor=darkblue,
    urlcolor=darkblue,
}

\usetikzlibrary{positioning}


\usepackage{latexsym}
\usepackage{wrapfig}
\usepackage{layout}
\usepackage{url}

\usepackage{okumacro}

\usepackage{comment}
%\usepackage{pxjahyper}
% ----------------------------
% commmand
% ----------------------------
% 執筆に便利なコマンド集です. 
% コマンドを追加する場合は下のスペースへ. 

% 集合の記号 (黒板文字)
\newcommand{\NN}{\mathbb{N}}
\newcommand{\ZZ}{\mathbb{Z}}
\newcommand{\QQ}{\mathbb{Q}}
\newcommand{\RR}{\mathbb{R}}
\newcommand{\CC}{\mathbb{C}}
\newcommand{\PP}{\mathbb{P}}
\newcommand{\KK}{\mathbb{K}}


% 集合の記号 (太文字)
\newcommand{\nn}{\mathbf{N}}
\newcommand{\zz}{\mathbf{Z}}
\newcommand{\qq}{\mathbf{Q}}
\newcommand{\rr}{\mathbf{R}}
\newcommand{\cc}{\mathbf{C}}
\newcommand{\pp}{\mathbf{P}}
\newcommand{\kk}{\mathbf{K}}

% 特殊な写像の記号
\newcommand{\ev}{\mathop{\mathrm{ev}}\nolimits} % 値写像
\newcommand{\pr}{\mathop{\mathrm{pr}}\nolimits} % 射影

% スクリプト体にするコマンド
%   例えば {\mcal C} のように用いる
\newcommand{\mcal}{\mathcal}

% 花文字にするコマンド 
%   例えば {\h C} のように用いる
\newcommand{\h}{\mathscr}

% ヒルベルト空間などの記号
\newcommand{\F}{\mcal{F}}
\newcommand{\X}{\mcal{X}}
\newcommand{\Y}{\mcal{Y}}
\newcommand{\Hil}{\mcal{H}}
\newcommand{\RKHS}{\Hil_{k}}
\newcommand{\Loss}{\mcal{L}_{D}}
\newcommand{\MLsp}{(\X, \Y, D, \Hil, \Loss)}

% 偏微分作用素の記号
\newcommand{\p}{\partial}

% 角カッコの記号 (内積は下にマクロがあります)
\newcommand{\lan}{\langle}
\newcommand{\ran}{\rangle}



% 圏の記号など
\newcommand{\Set}{{\bf Set}}
\newcommand{\Vect}{{\bf Vect}}
\newcommand{\FDVect}{{\bf FDVect}}
%\newcommand{\Ring}{{\bf Ring}}
\newcommand{\Ab}{{\bf Ab}}
\newcommand{\Mod}{\mathop{\mathrm{Mod}}\nolimits}
\newcommand{\Modf}{\mathop{\mathrm{Mod}^\mathrm{f}}\nolimits}
\newcommand{\CGA}{{\bf CGA}}
\newcommand{\GVect}{{\bf GVect}}
\newcommand{\Lie}{{\bf Lie}}
\newcommand{\dLie}{{\bf Liec}}



% 射の集合など
\newcommand{\Map}{\mathop{\mathrm{Map}}\nolimits} % 写像の集合
\newcommand{\Hom}{\mathop{\mathrm{Hom}}\nolimits} % 射集合
\newcommand{\End}{\mathop{\mathrm{End}}\nolimits} % 自己準同型の集合
\newcommand{\Aut}{\mathop{\mathrm{Aut}}\nolimits} % 自己同型の集合
\newcommand{\Mor}{\mathop{\mathrm{Mor}}\nolimits} % 射集合
\newcommand{\Ker}{\mathop{\mathrm{Ker}}\nolimits} % 核
\newcommand{\Img}{\mathop{\mathrm{Im}}\nolimits} % 像
\newcommand{\Cok}{\mathop{\mathrm{Coker}}\nolimits} % 余核
\newcommand{\Cim}{\mathop{\mathrm{Coim}}\nolimits} % 余像

% その他便利なコマンド
\newcommand{\dip}{\displaystyle} % 本文中で数式モード
\newcommand{\e}{\varepsilon} % イプシロン
\newcommand{\dl}{\delta} % デルタ
\newcommand{\pphi}{\varphi} % ファイ
\newcommand{\ti}{\tilde} % チルダ
\newcommand{\pal}{\parallel} % 平行
\newcommand{\op}{{\rm op}} % 双対を取る記号
\newcommand{\lcm}{\mathop{\mathrm{lcm}}\nolimits} % 最小公倍数の記号
\newcommand{\Probsp}{(\Omega, \F, \P)} 
\newcommand{\argmax}{\mathop{\rm arg~max}\limits}
\newcommand{\argmin}{\mathop{\rm arg~min}\limits}





% ================================
% コマンドを追加する場合のスペース 
%\newcommand{\OO}{\mcal{O}}



\makeatletter
\renewenvironment{proof}[1][\proofname]{\par
  \pushQED{\qed}%
  \normalfont \topsep6\p@\@plus6\p@\relax
  \trivlist
  \item[\hskip\labelsep
%        \itshape
         \bfseries
%    #1\@addpunct{.}]\ignorespaces
    {#1}]\ignorespaces
}{%
  \popQED\endtrivlist\@endpefalse
}
\makeatother

\renewcommand{\proofname}{\textrm{Proof.}}



%\renewcommand\proofname{\bf 証明} % 証明
\numberwithin{equation}{subsection}
\newcommand{\cTop}{\textsf{Top}}
%\newcommand{\cOpen}{\textsf{Open}}
\newcommand{\Op}{\mathop{\textsf{Op}}\nolimits}
\newcommand{\Ob}{\mathop{\textrm{Ob}}\nolimits}
\newcommand{\id}{\mathop{\mathrm{id}}\nolimits}
\newcommand{\pt}{\mathop{\mathrm{pt}}\nolimits}
\newcommand{\res}{\mathop{\rho}\nolimits}
\newcommand{\A}{\mcal{A}}
\newcommand{\B}{\mcal{B}}
\newcommand{\C}{\mcal{C}}
\newcommand{\D}{\mcal{D}}
\newcommand{\E}{\mcal{E}}
\newcommand{\G}{\mcal{G}}
%\newcommand{\H}{\mcal{H}}
\newcommand{\I}{\mcal{I}}
\newcommand{\J}{\mcal{J}}
\newcommand{\OO}{\mcal{O}}
\newcommand{\Ring}{\mathop{\textsf{Ring}}\nolimits}
\newcommand{\cAb}{\mathop{\textsf{Ab}}\nolimits}
%\newcommand{\Ker}{\mathop{\mathrm{Ker}}\nolimits}
\newcommand{\im}{\mathop{\mathrm{Im}}\nolimits}
\newcommand{\Coker}{\mathop{\mathrm{Coker}}\nolimits}
\newcommand{\Coim}{\mathop{\mathrm{Coim}}\nolimits}
\newcommand{\rank}{\mathop{\mathrm{rank}}\nolimits}
\newcommand{\Ht}{\mathop{\mathrm{Ht}}\nolimits}
\newcommand{\supp}{\mathop{\mathrm{supp}}\nolimits}
\newcommand{\colim}{\mathop{\mathrm{colim}}}
\newcommand{\Tor}{\mathop{\mathrm{Tor}}\nolimits}

\newcommand{\cat}{\mathscr{C}}

%筆記体
\newcommand{\cA}{\mcal{A}}
\newcommand{\cB}{\mcal{B}}
\newcommand{\cC}{\mcal{C}}
\newcommand{\cD}{\mcal{D}}
\newcommand{\cE}{\mcal{E}}
\newcommand{\cF}{\mcal{F}}
\newcommand{\cG}{\mcal{G}}
\newcommand{\cH}{\mcal{H}}
\newcommand{\cI}{\mcal{I}}
\newcommand{\cJ}{\mcal{J}}
\newcommand{\cK}{\mcal{K}}
\newcommand{\cL}{\mcal{L}}
\newcommand{\cM}{\mcal{M}}
\newcommand{\cN}{\mcal{N}}
\newcommand{\cO}{\mcal{O}}
\newcommand{\cP}{\mcal{P}}
\newcommand{\cQ}{\mcal{Q}}
\newcommand{\cR}{\mcal{R}}
\newcommand{\cS}{\mcal{S}}
\newcommand{\cT}{\mcal{T}}
\newcommand{\cU}{\mcal{U}}
\newcommand{\cV}{\mcal{V}}
\newcommand{\cW}{\mcal{W}}
\newcommand{\cX}{\mcal{X}}
\newcommand{\cY}{\mcal{Y}}
\newcommand{\cZ}{\mcal{Z}}


\newcommand{\scA}{\mathscr{A}}
\newcommand{\scB}{\mathscr{B}}
\newcommand{\scC}{\mathscr{C}}
\newcommand{\scD}{\mathscr{D}}
\newcommand{\scE}{\mathscr{E}}
\newcommand{\scF}{\mathscr{F}}
\newcommand{\scN}{\mathscr{N}}
\newcommand{\scO}{\mathscr{O}}
\newcommand{\scV}{\mathscr{V}}
\newcommand{\scU}{\mathscr{U}}


\newcommand{\ibA}{\mathop{\text{\textit{\textbf{A}}}}}
\newcommand{\ibB}{\mathop{\text{\textit{\textbf{B}}}}}
\newcommand{\ibC}{\mathop{\text{\textit{\textbf{C}}}}}
\newcommand{\ibD}{\mathop{\text{\textit{\textbf{D}}}}}
\newcommand{\ibE}{\mathop{\text{\textit{\textbf{E}}}}}
\newcommand{\ibF}{\mathop{\text{\textit{\textbf{F}}}}}
\newcommand{\ibG}{\mathop{\text{\textit{\textbf{G}}}}}
\newcommand{\ibH}{\mathop{\text{\textit{\textbf{H}}}}}
\newcommand{\ibI}{\mathop{\text{\textit{\textbf{I}}}}}
\newcommand{\ibJ}{\mathop{\text{\textit{\textbf{J}}}}}
\newcommand{\ibK}{\mathop{\text{\textit{\textbf{K}}}}}
\newcommand{\ibL}{\mathop{\text{\textit{\textbf{L}}}}}
\newcommand{\ibM}{\mathop{\text{\textit{\textbf{M}}}}}
\newcommand{\ibN}{\mathop{\text{\textit{\textbf{N}}}}}
\newcommand{\ibO}{\mathop{\text{\textit{\textbf{O}}}}}
\newcommand{\ibP}{\mathop{\text{\textit{\textbf{P}}}}}
\newcommand{\ibQ}{\mathop{\text{\textit{\textbf{Q}}}}}
\newcommand{\ibR}{\mathop{\text{\textit{\textbf{R}}}}}
\newcommand{\ibS}{\mathop{\text{\textit{\textbf{S}}}}}
\newcommand{\ibT}{\mathop{\text{\textit{\textbf{T}}}}}
\newcommand{\ibU}{\mathop{\text{\textit{\textbf{U}}}}}
\newcommand{\ibV}{\mathop{\text{\textit{\textbf{V}}}}}
\newcommand{\ibW}{\mathop{\text{\textit{\textbf{W}}}}}
\newcommand{\ibX}{\mathop{\text{\textit{\textbf{X}}}}}
\newcommand{\ibY}{\mathop{\text{\textit{\textbf{Y}}}}}
\newcommand{\ibZ}{\mathop{\text{\textit{\textbf{Z}}}}}

\newcommand{\ibx}{\mathop{\text{\textit{\textbf{x}}}}}

%\newcommand{\Comp}{\mathop{\mathrm{C}}\nolimits}
%\newcommand{\Komp}{\mathop{\mathrm{K}}\nolimits}
%\newcommand{\Domp}{\mathop{\mathsf{D}}\nolimits}%複体のホモトピー圏
%\newcommand{\Comp}{\mathrm{C}}
%\newcommand{\Komp}{\mathrm{K}}
%\newcommand{\Domp}{\mathsf{D}}%複体のホモトピー圏

\newcommand{\Comp}{\mathop{\mathrm{C}}\nolimits}
\newcommand{\Komp}{\mathop{\mathsf{K}}\nolimits}
\newcommand{\Domp}{\mathop{\mathsf{D}}\nolimits}
\newcommand{\Kompl}{\mathop{\mathsf{K}^\mathrm{+}}\nolimits}
\newcommand{\Kompu}{\mathop{\mathsf{K}^\mathrm{-}}\nolimits}
\newcommand{\Kompb}{\mathop{\mathsf{K}^\mathrm{b}}\nolimits}
\newcommand{\Dompl}{\mathop{\mathsf{D}^\mathrm{+}}\nolimits}
\newcommand{\Dompu}{\mathop{\mathsf{D}^\mathrm{-}}\nolimits}
\newcommand{\Dompb}{\mathop{\mathsf{D}^\mathrm{b}}\nolimits}
\newcommand{\Dompbf}{\mathop{\mathsf{D}_\mathrm{f}^\mathrm{b}}\nolimits}




\newcommand{\CCat}{\Comp(\cat)}
\newcommand{\KCat}{\Komp(\cat)}
\newcommand{\DCat}{\Domp(\cat)}%圏Cの複体のホモトピー圏
\newcommand{\HOM}{\mathop{\mathscr{H}\hspace{-2pt}om}\nolimits}%内部Hom
\newcommand{\RHOM}{\mathop{\mathrm{R}\hspace{-1.5pt}\HOM}\nolimits}

\newcommand{\muS}{\mathop{\mathrm{SS}}\nolimits}
\newcommand{\RG}{\mathop{\mathrm{R}\hspace{-0pt}\Gamma}\nolimits}
\newcommand{\RHom}{\mathop{\mathrm{R}\hspace{-1.5pt}\Hom}\nolimits}
\newcommand{\Rder}{\mathrm{R}}

\newcommand{\simar}{\mathrel{\overset{\sim}{\rightarrow}}}%同型右矢印
\newcommand{\simarr}{\mathrel{\overset{\sim}{\longrightarrow}}}%同型右矢印
\newcommand{\simra}{\mathrel{\overset{\sim}{\leftarrow}}}%同型左矢印
\newcommand{\simrra}{\mathrel{\overset{\sim}{\longleftarrow}}}%同型左矢印

\newcommand{\hocolim}{{\mathrm{hocolim}}}
\newcommand{\indlim}[1][]{\mathop{\varinjlim}\limits_{#1}}
\newcommand{\sindlim}[1][]{\smash{\mathop{\varinjlim}\limits_{#1}}\,}
\newcommand{\Pro}{\mathrm{Pro}}
\newcommand{\Ind}{\mathrm{Ind}}
\newcommand{\prolim}[1][]{\mathop{\varprojlim}\limits_{#1}}
\newcommand{\sprolim}[1][]{\smash{\mathop{\varprojlim}\limits_{#1}}\,}

\newcommand{\Sh}{\mathrm{Sh}}
\newcommand{\PSh}{\mathrm{PSh}}

\newcommand{\rmD}{\mathrm{D}}

\newcommand{\Lloc}[1][]{\mathord{\mathcal{L}^1_{\mathrm{loc},{#1}}}}
\newcommand{\ori}{\mathord{\mathrm{or}}}
\newcommand{\Db}{\mathord{\mathscr{D}b}}

\newcommand{\codim}{\mathop{\mathrm{codim}}\nolimits}



\newcommand{\gld}{\mathop{\mathrm{gld}}\nolimits}
\newcommand{\wgld}{\mathop{\mathrm{wgld}}\nolimits}


\newcommand{\tens}[1][]{\mathbin{\otimes_{\raise1.5ex\hbox to-.1em{}{#1}}}}
\newcommand{\etens}{\mathbin{\boxtimes}}
\newcommand{\ltens}[1][]{\mathbin{\overset{\mathrm{L}}\tens}_{#1}}
\newcommand{\mtens}[1][]{\mathbin{\overset{\mathrm{\mu}}\tens}_{#1}}
\newcommand{\lltens}[1][]{{\mathop{\tens}\limits^{\mathrm{L}}_{#1}}}
\newcommand{\letens}{\overset{\mathrm{L}}{\etens}}
\newcommand{\detens}{\underline{\etens}}
\newcommand{\ldetens}{\overset{\mathrm{L}}{\underline{\etens}}}
\newcommand{\dtens}[1][]{{\overset{\mathrm{L}}{\underline{\otimes}}}_{#1}}

\newcommand{\blk}{\mathord{\ \cdot\ }}
\newcommand{\mres}[2][]{{\left.{#1}\right\rvert}_{#2}}


%\newcommand{\hocolim}{{\mathrm{hocolim}}}
%\newcommand{\indlim}[1][]{\mathop{\varinjlim}\limits_{#1}}
%\newcommand{\sindlim}[1][]{\smash{\mathop{\varinjlim}\limits_{#1}}\,}
%\newcommand{\Pro}{\mathrm{Pro}}
%\newcommand{\Ind}{\mathrm{Ind}}
%\newcommand{\prolim}[1][]{\mathop{\varprojlim}\limits_{#1}}
%\newcommand{\sprolim}[1][]{\smash{\mathop{\varprojlim}\limits_{#1}}\,}
\newcommand{\proolim}[1][]{\mathop{\text{\rm``{$\varprojlim$}''}}\limits_{#1}}
\newcommand{\sproolim}[1][]{\smash{\mathop{\rm``{\varprojlim}''}\limits_{#1}}}
\newcommand{\inddlim}[1][]{\mathop{\text{\rm``{$\varinjlim$}''}}\limits_{#1}}
\newcommand{\sinddlim}[1][]{\smash{\mathop{\text{\rm``{$\varinjlim$}''}}\limits_{#1}}\,}
\newcommand{\ooplus}{\mathop{\text{\rm``{$\oplus$}''}}\limits}
\newcommand{\bbigsqcup}{\mathop{``\bigsqcup''}\limits}
\newcommand{\bsqcup}{\mathop{``\sqcup''}\limits}
\newcommand{\dsum}[1][]{\mathbin{\oplus_{#1}}}

\newcommand{\Fct}{\mathop{\mathsf{Fct}}\nolimits}





%================================================
% 自前の定理環境
%   https://mathlandscape.com/latex-amsthm/
% を参考にした
\newtheoremstyle{mystyle}%   % スタイル名
    {5pt}%                   % 上部スペース
    {5pt}%                   % 下部スペース
    {}%              % 本文フォント
    {}%                  % 1行目のインデント量
    {\bfseries}%                      % 見出しフォント
    {.}%                     % 見出し後の句読点
    {12pt}%                     % 見出し後のスペース
    {\thmname{#1}\thmnumber{ #2}\thmnote{{\hspace{2pt}\normalfont (#3)}}}% % 見出しの書式

\theoremstyle{mystyle}
\newtheorem{AXM}{公理}%[section]
\newtheorem{DFN}[AXM]{定義}
\newtheorem{THM}[AXM]{定理}
\newtheorem*{THM*}{定理}
\newtheorem{PRP}[AXM]{命題}
\newtheorem{LMM}[AXM]{補題}
\newtheorem{CRL}[AXM]{系}
\newtheorem{EG}[AXM]{例}
\newtheorem*{EG*}{例}
\newtheorem{RMK}[AXM]{注意}
\newtheorem{CNV}[AXM]{約束}
\newtheorem{CMT}[AXM]{コメント}
\newtheorem*{CMT*}{コメント}
\newtheorem{NTN}[AXM]{記号}
\newtheorem{CLM}[AXM]{Claim}
\newtheorem{Prop}[AXM]{Proposition}

% 定理環境ここまで
%====================================================

\usepackage{framed}
\definecolor{lightgray}{rgb}{0.75,0.75,0.75}
\renewenvironment{leftbar}{%
  \def\FrameCommand{\textcolor{lightgray}{\vrule width 4pt} \hspace{10pt}}% 
  \MakeFramed {\advance\hsize-\width \FrameRestore}}%
{\endMakeFramed}
\newenvironment{redleftbar}{%
  \def\FrameCommand{\textcolor{lightgray}{\vrule width 1pt} \hspace{10pt}}% 
  \MakeFramed {\advance\hsize-\width \FrameRestore}}%
 {\endMakeFramed}



\renewcommand{\Re}{\mathop{\mathrm{Re}}\nolimits}

% =================================





% ---------------------------
% new definition macro
% ---------------------------
% 便利なマクロ集です

% 内積のマクロ
%   例えば \inner<\pphi | \psi> のように用いる
\def\inner<#1>{\langle #1 \rangle}

% ================================
% マクロを追加する場合のスペース 

%=================================






\usetheme{Singapore}
\usecolortheme{rose}
\usefonttheme{professionalfonts}
\setbeamertemplate{navigaton symbols}{\false}
\setbeamertemplate{footline}[frame number]
%\usepackage{graphicx,xcolor}

\def\fracinline#1/#2{\mbox{\raise0.5ex\hbox{\footnotesize$#1$}{\hskip-.1em$/$\hskip-.1em}\raise-0.5ex\hbox{\footnotesize$#2$}}}

\begin{document}
    
\title{Normal Deformation and Normal Cones\subtitle{本多研 院生ゼミ}}
\date{2024年7月8日}
\author{大柴 寿浩}




\begin{frame}
    \titlepage
\end{frame}
\begin{comment}
    \begin{frame}\frametitle{書くこと}
    \begin{itemize}
        \item 特に興味を持った定理や理論の背景
        \item それを記述するための記号や概念の導入
        \item 正確な主張の紹介
        \item 証明の基本的なアイデアやアウトラインの紹介
        \item または定理の過程を満たす具体例や定理の応用例の紹介
    \end{itemize}
    \end{frame}
\end{comment}
\setcounter{section}{3}
\section[sect.4.1]{Normal Deformation and Normal Cones\cite[sect.4.1]{KS90}}
\subsection{normal deformation}
\begin{frame}
    \frametitle{Normal Deformation}

    \begin{itemize}
        \item \(X\): a manifold of \(\dim{M}=n\)
        \item \(M\subset X\): a closed submanifold of\(\codim{M}=l\)
        \item \(T_{M}X\): the normal bundle to \(M\) in \(X\)
    \end{itemize}

    We defined the \textbf{normal deformation} of \(M\) in \(X\):
    \begin{itemize}
        \item \(\widetilde{X}_{M}\)
        \item \(p\colon \widetilde{X}_{M}\to X\)
        \item \(t\colon \widetilde{X}_{M}\to\rr\)
    \end{itemize}
\end{frame}

\begin{frame}
    \frametitle{Normal Deformation}

    \(p\) and \(t\) satisfy the following conditions:
    \begin{equation}\tag{4.1.3}\label{4.1.3}
        \begin{cases}
            p^{-1}(X-M)\cong (X-M)\times (\rr-\{0\}),\\
            t^{-1}(\rr-\{0\})\cong X\times (\rr-\{0\}),\\
            t^{-1}(0)\cong T_{M}X.
        \end{cases}
    \end{equation}
\end{frame}


\begin{frame}
    \frametitle{Normal Deformation}

    \begin{itemize}
        \item \(\varOmega\coloneqq t^{-1}\big({\left]0,+\infty\right[}\big)\)
        \item \(j\colon\varOmega\hookrightarrow \widetilde{X}_{M}\)
        \item \(\widetilde{p}\coloneqq p\circ j\)
    \end{itemize}
    \begin{equation}\tag{4.1.5}\label{4.1.5}
        \vcenter{\xymatrix@C=36pt@R=36pt{
        T_{M}X
        \ar[d]_-{\tau}
        \ar@{^{(}->}[r]_-{s}
        &
        \widetilde{X}_{M}
        \ar[d]^-{p}
        &
        \varOmega
        \ar@{_{(}->}[l]_-{j}
        \ar[ld]^-{\widetilde{p}}
        \\
        M
        \ar@{^{(}->}[r]_-{i}
        &
        X
      }}
    \end{equation}
\end{frame}

\begin{frame}
    \frametitle{Normal Deformation}
    \begin{CLM}
        \(\widetilde{p}\) is smooth 
        and 
        \(\varOmega\) is isomorphic to \(X\times \rr^{+}\) 
        by the map \(\left(\widetilde{p},t\right)\).
    \end{CLM}
    \begin{proof}
        We have \(
            \widetilde{p}^{-1}(X)
            =j^{-1}p^{-1}(X)
            =\varOmega
        \) by the definition of \(\widetilde{p}\) and 
        the surjectivity of \(p\). 
        The condition about tangent maps is a local property, 
        and the claim follows.

        We have \(t^{-1}(\rr^{+})\cong\varOmega\). Therfore
        \begin{align*}
            \left(\widetilde{p},t\right)(\varOmega)
            &\cong \widetilde{p}(\varOmega)\times t(\varOmega)\\
            &\cong X\times \rr^{+}.
        \end{align*}
        The inverse morphism is induced by \eqref{4.1.3}.
    \end{proof}
\end{frame}

\begin{frame}
    \frametitle{Normal Deformation}
    \begin{CLM}
        \(p^{-1}(M)\) is the union of \(T_{M}X\) 
        and \(M\times \rr\).
    \end{CLM}
    \begin{proof}
        We can see locally
        \begin{align*}
            p^{-1}(M)
            &=\left\{(x,t)\in\widetilde{X}_{M};~(tx',x'')\in M\right\}\\
            &=\left\{(x,t)\in\widetilde{X}_{M};~tx'=0\right\}\\
            &=\left\{(x,t)\in\widetilde{X}_{M};~t=0,\text{ or }~x'=0\right\}\\
            &=T_{M}X\cup (M\times \rr).
        \end{align*}
    \end{proof}
\end{frame}

\begin{frame}
    \frametitle{Normal Deformation}
    \begin{CLM}
        \(T_{M}X\cap (M\times \rr)=M\times\{0\}\) coincides 
        with the zero-section of \(T_{M}X\).
    \end{CLM}
    \begin{proof}
        As how we consider above,
        \begin{align*}
            T_{M}X\cap (M\times \rr)
            &=t^{-1}(0)\cap (M\times \rr)\\
            &=\left\{(x,t)\in\widetilde{X}_{M};~t=0,\text{ and }x'=0\right\}\\
            &=M\times\{0\},
        \end{align*}
        and \(
            M\times\{0\}\cong M \subset T_{M}X.
        \)
    \end{proof}
\end{frame}

\subsection{normal cones}
\begin{frame}
    \frametitle{Normal Cones}
    \begin{definition}[{\cite[Def.4.1.1]{KS90}}]
        \begin{enumerate}[(i)]
            \item For \(S\subset X\), the \emph{normal cone to} \(S\) \emph{along} \(M\) is\[
                C_{M}(S)=T_{M}X\cap\overline{\widetilde{p}^{-1}(S)}.
            \]
            \item For \(S_1, S_2\subset X\), 
            we define \(C(S_1,S_2)\coloneqq C_{\Delta_{X}}(S_1\times S_2)\subset TX\).
        \end{enumerate}
        \[
            \begin{array}{ccc}
                T_{\Delta_X}(X\times X)&\longrightarrow&TX\\
                \rotatebox{90}{$\in$}&&\rotatebox{90}{$\in$}\\
                \left((x,x),(\xi,0)\right)&\longmapsto&\left(x, \xi\right)        
            \end{array}
        \]    
    \end{definition}
\end{frame}

\begin{frame}
    \frametitle{Normal Cones}

    We can write an element of \(T_{\Delta_X}(X\times X)\) as
    \[
        \left((x,x),(\xi,0)\right).
    \]
    Indeed, for \((\xi,\eta) \in T_{(x,x)}(X\times X)\), 
    we have
    \begin{align*}
        (\xi,\eta) + T_{(x,x)}\Delta_X
        &=\{(\xi+\zeta,\eta+\zeta);\zeta\in T_{(x,x)}\Delta_X\}\\
        &=\{(\xi+(\zeta-\eta),\eta+(\zeta-\eta));\zeta\in T_{(x,x)}\Delta_X\}\\
        &=\{(\xi-\eta,0)+\zeta;\zeta\in T_{(x,x)}\Delta_X\}\\
        &=(\xi-\eta,0)+ T_{(x,x)}\Delta_X.
    \end{align*}
    That is to say, we can make any vectors \(
        (\xi,\eta)
    \) in \(T_{\Delta_X}(X\times X)\) in the form \(
        (\zeta,0)
    \) preserving the difference between the first and second entries.
\end{frame}

\begin{frame}
    \frametitle{Normal Cones}

    \begin{CLM}
        \begin{enumerate}
            \item \(C_{M}(S)\) is a closed conic subset of \(T_{M}X\).
            \item its projection onto \(M\) is 
            the set \(M\cap\overline{S}\).
        \end{enumerate}
    \end{CLM}
    \begin{proof}
        1. 
        The claim follows from
        \begin{align*}
            C_{M}(S)
            &=T_{M}X\cap\overline{\widetilde{p}^{-1}(S)}\\
            &=T_{M}X\cap\overline{\left\{(x'/t,x'')\in\varOmega; x\in S, t\in\rr^{+}\right\}}\\
            &=T_{M}X\cap\overline{\rr^{+}\left\{(x',x'')\in\varOmega; x\in S\right\}}.
        \end{align*}
    \end{proof}
\end{frame}

\begin{frame}
    \frametitle{Normal Cones}

    \begin{proof}
        2. ????
        \[\tau(C_{M}(S))\overset{?}{=}M\cap\overline{S}\]
        \begin{align*}
            \tau(C_{M}(S))
            &=\tau(T_{M}X\cap\overline{\widetilde{p}^{-1}(S)})\\
            &=\tau\left(s^{-1}\left(\overline{\widetilde{p}^{-1}(S)}\right)\right)\\
            &=\tau\left(s^{-1}\left(\overline{j^{-1}(p^{-1}(S))}\right)\right)\\
            &=\tau\left(\overline{s^{-1}\left(j^{-1}(p^{-1}(S))\right)}\right)
        \end{align*}
    \end{proof}
\end{frame}

\begin{frame}
    \frametitle{Normal Cones}

    As a consequence of the claim, 
    \begin{center}
        \(C(S_1,S_2)\subset TX\): closed conic.
    \end{center}
    If \(M\) is a closed submanifold, then \(C(S,M)\) is 
    the inverse image of \(C_M(S)\) by the projection \(
        M\times_{X}TX\to T_{M}X
    \) by the next proposition.

\end{frame}


\begin{comment}
    \begin{frame}\frametitle{同値関係の幾何的イメージ}
    %\scalebox{0.8}{
    \begin{tikzpicture}
        \draw[>=stealth,semithick] (-2,0)--(3,0); %x軸
        \draw[>=stealth,semithick] (0,-2)--(0,4); %y軸
        \draw (0,0)node[below right]{O}; %原点
        \draw[thick, domain=-2:3] plot(\x,{0.5*\x});
        \draw[thick, domain=-1.:2] plot(\x,2*\x);
        \fill[black!20] (1.5,3) rectangle (2.5,1.25); %四角2U
        \draw[line width = 0.5pt] (1.5,3) rectangle (2.5,1.25); %四角2U
        \draw (2.5,1.25)node[below]{$2U$}; %点(2.5,1.25)
        \fill[black!20] (1,2) rectangle (1.66,0.83); %四角Uずらし
        \draw[line width = 0.5pt] (1,2) rectangle (1.66,0.83); %四角Uずらし
        \fill[black!40] (0.75,1.5) rectangle (1.25,0.625); %四角U
        \draw[line width = 0.5pt] (0.75,1.5) rectangle (1.25,0.625); %四角U
        \draw (0.75,1.5)node[left]{$U$}; %点(2.5,1.25)\draw[line width=1pt] (0.75,1.5) rectangle (1.25,0.625); %四角
        \fill[black!20] (-0.75,-1.5) rectangle (-1.25,-0.625); %四角-U
        \draw[line width = 0.5pt] (-0.75,-1.5) rectangle (-1.25,-0.625); %四角-U
        \fill[black!20] (-1,-2) rectangle (-1.66,-0.83); %四角-U
        \draw[line width = 0.5pt] (-1,-2) rectangle (-1.66,-0.83); %四角-U
        \draw (-0.75,-1.5)node[right]{$-U$}; %点(2.5,1.25)
        \fill[black!20] (0.375,0.75) rectangle (0.625,0.3125); %四角(1/2)U
        \draw[line width = 0.5pt] (0.375,0.75) rectangle (0.625,0.3125); %四角(1/2)U
        \draw (0.625,0.2)node[right]{$(1/2)U$}; %点(2.5,1.25)
        \draw[very thick, domain=-2.:3] plot(\x,{0.8*\x});
        %\draw[very thick, domain=-1.7:2.7] plot(\x,{1.3*\x});
        \draw (1.8,3.6)node[right]{$U$を通る直線の上端};
        \draw (3,2.4)node[right]{$U$を通る直線};
        \draw (3,1.5)node[right]{$U$を通る直線の下端};
    \end{tikzpicture}
    \centering
    %}
        %\caption{商写像の逆像1}
\end{frame}

\end{comment}

\section{References}

\begin{frame}[allowframebreaks]{References}
    \begin{thebibliography}{KS90}\beamertemplatetextbibitems
        \bibitem[KS90]{KS90} Kashiwara, Schapira 
        \textit{Sheaves on Manifolds}, Springer, 1990.
    \end{thebibliography}
\end{frame}

\begin{comment}
\begin{frame}[noframenumbering]
aaa
\end{frame}
\end{comment}

\end{document}