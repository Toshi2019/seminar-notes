\section{関数概念の一般化について}

\subparagraph{1.}
L.シュヴァルツは\(C^\infty\)多様体上の関数概念を
\textbf{分布} (distribution) という概念の導入によって一般化した.
そしてこの概念は,解析学における様々な分野の間で,
広く用いることのできるものであることが明らかになった.\footnote{
    L.シュヴァルツ, 超関数の理論, Hermann (1950--1951).
}
ここでは,定義域の多様体が (\(C^\infty\)ではなく) 
\(C^\omega\)である場合に,関数概念の別の一般化の仕方を提案する.
ここでの一般化の仕方は,解析関数の境界値を用いるものである.
この新概念は\(C^\omega\)多様体の場合に,
シュワルツの分布を含む,より広い概念になっている.
この一般化 ``関数'' を\textbf{超関数} (hyperfunction) と
呼ぶことにする.正確には次のように定義する.
(議論を簡略化するため,ここでは1次元の場合の定義を述べるが,
\(C^\omega\)多様体上の超関数も定義できる.)\footnote{
    このノートの原稿を完成させた後,
    ``超関数''はG.ケーテ教授によって``Die...''において,
    すでに導入されていることを
    彌永教授を通じてA.ヴェイユ氏からご教示いただいた.
}
